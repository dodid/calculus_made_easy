%%%%%%%%%%%%%%%%%%%%%%%%%%%%%%%%%%%%%%%%%%%%%%%%%%%%%%%%%%%%%%%%%%%%%%
%%                                                                  %%
%% Packages and substitutions:                                      %%
%%                                                                  %%
%% ctex:     Required.                                              %%
%%                                                                  %%
%% ifthen:   Logical conditionals. Required.                        %%
%%                                                                  %%
%% amsmath:  AMS mathematics enhancements. Required.                %%
%% amssymb:  Additional mathematical symbols. Required.             %%
%%                                                                  %%
%% alltt:    Fixed-width font environment. Required.                %%
%% array:    Enhanced tabular features. Required.                   %%
%% dcolumn:  Decimal-aligned data columns. Required.                %%
%%                                                                  %%
%% footmisc: Extended footnote capabilities. Required.              %%
%% perpage:  Start footnote numbering on each page. Required.       %%
%%                                                                  %%
%% indentfirst: Indent first word of each sectional unit. Optional. %%
%%                                                                  %%
%% calc:     Length calculations. Required.                         %%
%%                                                                  %%
%% fancyhdr: Enhanced running headers and footers. Required.        %%
%%                                                                  %%
%% graphicx: Standard interface for graphics inclusion. Required.   %%
%% wrapfig:  Illustrations surrounded by text. Required.            %%
%%                                                                  %%
%% geometry: Enhanced page layout package. Required.                %%
%% hyperref: Hypertext embellishments for pdf output. Required.     %%
%%                                                                  %%
%%                                                                  %%
%% Producer's Comments:                                             %%
%%                                                                  %%
%%   Changes are noted in this file in three ways.                  %%
%%   1. \DPnote{} for in-line `placeholder' notes.                  %%
%%   2. \DPtypo{}{} for typographical corrections, showing original %%
%%      and replacement text side-by-side.                          %%
%%   3. [** PP: Note]s for lengthier or stylistic comments.         %%
%%                                                                  %%
%%                                                                  %%
%% Compilation Flags:                                               %%
%%                                                                  %%
%%   The following behavior may be controlled by boolean flags.     %%
%%                                                                  %%
%%   ForPrinting (false by default):                                %%
%%   Compile a screen-optimized PDF file. Set to false for print-   %%
%%   optimized file (pages cropped, one-sided, blue hyperlinks).    %%
%%                                                                  %%
%%   Compile this project with:                                     %%
%%   xelatex calculus_made_easy_zh_cn.tex ..... THREE times         %%
%%                                                                  %%
%%%%%%%%%%%%%%%%%%%%%%%%%%%%%%%%%%%%%%%%%%%%%%%%%%%%%%%%%%%%%%%%%%%%%%
\listfiles
\documentclass{ctexbook}
% \setCJKmainfont{Songti SC}

%%%%%%%%%%%%%%%%%%%%%%%%%%%%% PACKAGES %%%%%%%%%%%%%%%%%%%%%%%%%%%%%%%
\usepackage{fontspec}[2006/05/05]

\usepackage{ifthen}[2001/05/26]  %% Logical conditionals

\usepackage{amsmath}[2000/07/18] %% Displayed equations
\usepackage{amssymb}[2002/01/22] %% and additional symbols

\usepackage{alltt}[1997/06/16]   %% boilerplate, credits, license

\usepackage{array}[2005/08/23]   %% extended array/tabular features
\usepackage{dcolumn}

                                 %% extended footnote capabilities
\usepackage[symbol,perpage]{footmisc}[2005/03/17]
\usepackage{perpage}[2006/07/15]

\usepackage{multicol}

\usepackage{graphicx}[1999/02/16]%% For diagrams
\usepackage{wrapfig}%% For diagram on 219.png

\IfFileExists{indentfirst.sty}{%
  \usepackage{indentfirst}[1995/11/23]
}{}

\usepackage{calc}[2005/08/06]

% for running heads
\usepackage{fancyhdr}

%%%%%%%%%%%%%%%%%%%%%%%%%%%%%%%%%%%%%%%%%%%%%%%%%%%%%%%%%%%%%%%%%
%%%% Interlude:  Set up PRINTING (default) or SCREEN VIEWING %%%%
%%%%%%%%%%%%%%%%%%%%%%%%%%%%%%%%%%%%%%%%%%%%%%%%%%%%%%%%%%%%%%%%%

% ForPrinting=true (default)           false
% Asymmetric margins                   Symmetric margins
% Black hyperlinks                     Blue hyperlinks
% Start Preface, ToC, etc. recto       No blank verso pages
%
% Chapter-like ``Sections'' start both recto and verso in the scanned
% book. This behavior has been retained.
\newboolean{ForPrinting}

%% UNCOMMENT the next line for a PRINT-OPTIMIZED VERSION of the text %%
% \setboolean{ForPrinting}{true}

%% Initialize values to ForPrinting=false
\newcommand{\Margins}{hmarginratio=1:1}     % Symmetric margins
\newcommand{\HLinkColor}{blue}              % Hyperlink color
\newcommand{\PDFPageLayout}{SinglePage}
\newcommand{\TransNote}{Transcriber's Note}
\newcommand{\TransNoteCommon}{%
  Minor presentational changes, and minor typographical and numerical
  corrections, have been made without comment. All textual changes are
  detailed in the \LaTeX\ source file.
  \bigskip
}

\newcommand{\TransNoteText}{%
  \TransNoteCommon

  This PDF file is optimized for screen viewing, but may easily be
  recompiled for printing. Please see the preamble of the \LaTeX\
  source file for instructions.
}
%% Re-set if ForPrinting=true
\ifthenelse{\boolean{ForPrinting}}{%
  \renewcommand{\Margins}{hmarginratio=2:3} % Asymmetric margins
  \renewcommand{\HLinkColor}{black}         % Hyperlink color
  \renewcommand{\PDFPageLayout}{TwoPageRight}
  \renewcommand{\TransNote}{Transcriber's Note}
  \renewcommand{\TransNoteText}{%
    \TransNoteCommon

    This PDF file is optimized for printing, but may easily be
    recompiled for screen viewing. Please see the preamble of the
    \LaTeX\ source file for instructions.
  }
}{% If ForPrinting=false, don't skip to recto
  \renewcommand{\cleardoublepage}{\clearpage}
}
%%%%%%%%%%%%%%%%%%%%%%%%%%%%%%%%%%%%%%%%%%%%%%%%%%%%%%%%%%%%%%%%%
%%%%  End of PRINTING/SCREEN VIEWING code; back to packages  %%%%
%%%%%%%%%%%%%%%%%%%%%%%%%%%%%%%%%%%%%%%%%%%%%%%%%%%%%%%%%%%%%%%%%

\ifthenelse{\boolean{ForPrinting}}{%
  \setlength{\paperwidth}{8.27in}%
  \setlength{\paperheight}{11.69in}%
  \usepackage[body={5in,7in},\Margins]{geometry}[2002/07/08]
}{%
  \setlength{\paperwidth}{6in}%
  \setlength{\paperheight}{8in}%
  \raggedbottom
  \usepackage[body={5in,7in},\Margins,includeheadfoot]{geometry}[2002/07/08]
}
% \usepackage[body={5.25in,8.4in},\Margins]{geometry}[2002/07/08]

\providecommand{\ebook}{00000}    % Overridden during white-washing
\usepackage{hyperref}

% Re-crop screen-formatted version, accommodating wide displays
\ifthenelse{\boolean{ForPrinting}}
  {}
  {\hypersetup{pdfpagescrop= 20 0 412 580}}


%%%% Fixed-width environment to format PG boilerplate %%%%
\newenvironment{PGtext}{%
\begin{alltt}
\fontsize{9.2}{10.5}\ttfamily\selectfont}%
{\end{alltt}}

\newcounter{mytnote}
\newcommand{\TNote}[1]{%
  \refstepcounter{mytnote}%
  \begin{minipage}{0.85\textwidth}
    \small
    \phantomsection
    \pdfbookmark[0]{Transcriber's Note}{TransNote\themytnote}
    \subsection*{\centering\normalfont\scshape%
      \normalsize\MakeLowercase{\TransNote}}%

    \raggedright
    #1
  \end{minipage}
}

\newcommand{\Corrections}{%
  The diagrams have been re-created, using accompanying formulas or
  descriptions from the text where possible.
  \bigskip

  In Chapter~XIV, pages \Pageref[]{erratum0}--\Pageref[]{erratum0a},
  numerical values of $\left(1+\frac{1}{n}\right)^n$, $\epsilon^x$,
  and related quantities of British currency have been verified and
  rounded to the nearest digit.
  \bigskip

  On \Pageref[page]{erratum1} (page~146 in the original), the graphs
  of the natural logarithm and exponential functions, Figures
  38~and~39, have been interchanged to match the surrounding text.
  \medskip

  The vertical dashed lines in the natural logarithm graph, Figure~39
  (Figure~38 in the original), have been moved to match the data in
  the corresponding table.
  \bigskip

  On \Pageref[page]{erratum2} (page~167 in the original), the graphs
  of the sine and cosine functions, Figures 44~and~45, have been
  interchanged to match the surrounding text.
}

% Misc text formatting tricks
\newlength{\TmpLen}
\newlength{\MySkip}
\newcommand{\PadTxt}[3][c]{%
  \settowidth{\TmpLen}{#2}%
  \makebox[\TmpLen][#1]{#3}%
}

\newcommand{\PadTo}[3][c]{\PadTxt[#1]{$#2$}{$#3$}}

% Style-setting
\renewcommand{\footnoterule}{}
\newcommand{\loosen}{\spaceskip0.5em plus 0.5em minus 0.25em}
\setlength{\emergencystretch}{1.5em}

\renewcommand{\topfraction}{0.6}
\renewcommand{\bottomfraction}{0.6}
\renewcommand{\textfraction}{0.1}
\DeclareMathSizes{12}{12}{8}{7}
\newcommand{\ds}{\displaystyle}

%% No hrule in page header
\renewcommand{\headrulewidth}{0pt}
\renewcommand{\baselinestretch}{1.33}

% Top-level footnote numbers restart on each page
% \MakePerPage{footnote}

% Stretch-fit argument to a box as wide as \TmpLen
\newcommand{\SetBox}[2][s]{%
  \ifthenelse{\equal{#1}{s}}{%
    \makebox[\TmpLen][#1]{\loosen #2}%
  }{%
    \makebox[\TmpLen][#1]{#2}%
  }%
}

% For Table of Standard Forms
\newcommand{\ColumnHead}[1]{%
  \multicolumn{3}{|c|}{\textbf{#1}}
}

%% Macro for reducing space between adjacent display environments
\newcommand{\CancelMathSkip}{\vspace{-\belowdisplayskip}}
\newcommand{\BindMath}[1]{%
{\vspace{\abovedisplayskip}%
{\setlength{\abovedisplayskip}{0pt}%
\setlength{\belowdisplayskip}{0pt}%
#1}\vspace{\belowdisplayskip}}%
}

% Running heads
\newcommand{\Heading}[1]{\textbf{\footnotesize\MakeUppercase{#1}}}

\newcommand{\SetOddHead}[1]{%
  \fancyhead{}%
  \setlength{\headheight}{15pt}%
  \fancyhead[CE]{\Heading{Calculus Made Easy}}%
  \fancyhead[CO]{\Heading{#1}}%
  \ifthenelse{\boolean{ForPrinting}}%
             {\fancyhead[RO,LE]{\thepage}}%
             {\fancyhead[R]{\thepage}}%
}

\newcommand{\SetBothHeads}[1]{%
  \fancyhead{}%
  \fancyhead[C]{\Heading{#1}}%
}

% Table of Contents
% Redefine headings for table of contents & index
\renewcommand{\contentsname}{%
  \protect\Large\protect\normalfont\protect\centering%
  CONTENTS.\\
}

\newcommand{\ToCBox}[1]{%
  \PadTxt[r]{XVIII.}{#1}\quad%
}

\newcommand{\ToCAnchor}{}

%\ToCLine[font style]{Chapter}{Title}{label}
\newcommand{\ToCLine}[4][\scshape]{%
  \par\noindent\label{toc:#4}%
  \ifthenelse{\not\equal{\pageref{toc:#4}}{\ToCAnchor}}{%
    % \noindent\textsc{\scriptsize Chapter\hfill Page}\medskip%
    \renewcommand{\ToCAnchor}{\pageref{toc:#4}}\par%
  }{}%
  \settowidth{\TmpLen}{9999}%
  \noindent\parbox[b]{\linewidth-\TmpLen}{\ToCBox{#2}%
    \rule{0pt}{16pt}#1\hangindent6em #3\dotfill}% [** TN: Hard-coded spaces]
  \PadTxt[r]{9999}{\pageref{#4}}\smallskip%
}

\newcommand{\ChapterSkip}{\vspace*{0.1\linewidth}}
\newcommand{\flushpage}{\clearpage\fancyhf{}\cleardoublepage}

% Numbered: \Chapter[Running Head]{number}{Title}
\newcommand{\Chapter}[3][]{%
  \flushpage
  \phantomsection\label{chap:#2}\pdfbookmark[0]{Chapter #2}{Chapter #2}%
  \ifthenelse{\not\equal{#1}{}}{%
    \SetOddHead{#1}%
  }{%
    \SetOddHead{#3}%
  }%

  \addtocontents{toc}{\protect\ToCLine{#2}{#3}{chap:#2}}
  \thispagestyle{empty}
  \ChapterSkip%
  \section*{\centering\normalfont\Large #2\ #3}
  \subsection*{\centering\normalfont\large\MakeUppercase{}}
  % \centerline{\normalfont\Large #2\ #3.}
}


% Unnumbered, but has a ToC entry: \AltChapter[Running Head]{Title}
%[** TN: Uses the book's chapter structure to select ToC entry font]
\newcommand{\AltChapter}[2][]{%
  \ifthenelse{\equal{#2}{Prologue}}{%
    \pagestyle{fancy}
  }{}
  \flushpage
  \phantomsection\label{#2}\pdfbookmark[0]{#2}{#2}%
  \ifthenelse{\not\equal{#1}{}}{%
    \SetOddHead{#1}%
    \addtocontents{toc}{\protect\ToCLine{}{#1}{#2}}%
  }{%
    \SetOddHead{#2}%
    \ifthenelse{\equal{#2}{Table of Standard Forms}}{%
      \addtocontents{toc}{\protect\ToCLine[\bfseries]{}{#2}{#2}}%
    }{%
      \addtocontents{toc}{\protect\ToCLine{}{#2}{#2}}%
      \ChapterSkip%
    }%
  }%

  \thispagestyle{empty}
  \section*{\centering\normalfont\Large\MakeUppercase{#2.}}
}


% For the Preface and Epilogue; no ToC entry
\newcommand\ChapterStar[2][]{%
  \flushpage
  \phantomsection

  \ifthenelse{\not\equal{#1}{}}{%
    \SetOddHead{#1}%
    \pdfbookmark[0]{#1}{#1}%
  }{%
    \SetOddHead{#2}%
    \pdfbookmark[0]{#2}{#2}%
  }%
  \thispagestyle{empty}
  \ChapterSkip%
  \section*{\centering\normalfont\Large\MakeUppercase{#2.}}
}

\newcounter{SectionNo}
\setcounter{SectionNo}{0}

\newcommand\Section[2][]{%
  \section*{\centering\normalfont\normalsize\bfseries#2}%
  \refstepcounter{SectionNo}%
  \phantomsection\pdfbookmark[1]{#2}{#2}\label{section:\theSectionNo}%
  \ifthenelse{\not\equal{#1}{}}{\SetOddHead{#1}}{}%
}

%tweaking \subsection - added italics & indent
\newcommand\Subsection[1]{%
  \medskip\pagebreak[1]\par\textit{#1}\pagebreak[0]\par%
}

%tweaking \paragraph - added italics & indent
\newcommand{\Paragraph}[1]{\medskip\pagebreak[1]\par\textit{#1}}

\newcommand{\Note}[1]{%
  \clearpage
  \section*{\centering\normalfont\normalsize #1}
}

% Dedicated structural macros
\newcommand{\Case}[1]{\Subsection{示例~\upshape{#1}.}}

% "Example." goes on its own line; otherwise, run-in heading
\newcommand{\Example}[1]{%
  \ifthenelse{\equal{#1}{.}}{%
    \Subsection{\textit{例题}}%
  }{%
    \Paragraph{\textit{例题}~\upshape{#1}}%
  }%
}

\newcommand{\Examples}[1]{%
  \ifthenelse{\equal{#1}{.}}{%
    \Subsection{\textit{例题}}%
  }{%
    \Subsection{\textit{#1}}%
  }
}


% Exercises section heading
% \Exercises[Running Head]{I} (See p.~254 for Answers.)
\newcommand{\Exercises}[2][]{%
  \tb\textit{练习~#2.}\quad
  \phantomsection\pdfbookmark[1]{练习 #2}{练习 #2}\label{Ex:#2}%
  \ifthenelse{\not\equal{#1}{}}{\SetOddHead{#1}}{}%
}

% Loosen up page spacing
%\setlength{\parsep}{1ex plus 0.5ex minus 1ex}
%\setlength{\partopsep}{0.5ex plus 1.5ex minus 0.25ex}
%\setlength{\itemsep}{1ex plus 1ex minus 1ex}

\newboolean{InMulticols}% true iff we're in a multicolumn envt

% List item formatting
\newcommand{\ListInit}{%
  \setlength{\leftmargin}{0pt}%
  \setlength{\labelwidth}{\parindent}%
  \setlength{\labelsep}{0.5em}%
  \setlength{\itemsep}{0pt}%
  \setlength{\listparindent}{\parindent}
  \setlength{\itemindent}{2\parindent}%
}

\newcommand{\SublistInit}{%
  \setlength{\leftmargin}{\parindent}%
  \setlength{\rightmargin}{3em}%
  \setlength{\labelwidth}{1em}%
  \setlength{\labelsep}{0.5em}%
  \setlength{\itemsep}{0pt}%
  \setlength{\listparindent}{\parindent}
  \setlength{\itemindent}{2.5em}%
}


% List environment initializer for Answers section
\newcommand{\ListInitAns}{%
  \setlength{\leftmargin}{\parindent}%0pt
  \setlength{\labelwidth}{\parindent}%
  \setlength{\labelsep}{0.5em}%
  \setlength{\itemsep}{2pt}%
  \setlength{\listparindent}{\parindent}
  \setlength{\itemindent}{0pt}%
}

% Reset number of columns *within a Problems or Answers environment*
\newcommand{\ResetCols}[1]{%
  \ifthenelse{\boolean{InMulticols}}{%
  \end{multicols}%
}{}
\ifthenelse{\equal{#1}{1}}{%
  \setboolean{InMulticols}{false}%
}{%
  \setboolean{InMulticols}{true}%
  \begin{multicols}{#1}[\raggedcolumns]%
  }%
}

% #1 = number of columns
\newenvironment{Problems}[1][1]{%
  \begin{list}{}{\ListInit}%
  \ifthenelse{\equal{#1}{1}}{%
    \setboolean{InMulticols}{false}%
  }{%
    \setboolean{InMulticols}{true}%
    \begin{multicols}{#1}
    }%
  }{% End of envt code
    \ifthenelse{\boolean{InMulticols}}{%
    \end{multicols}%
  }{}%
  \setboolean{InMulticols}{false}
\end{list}%
}

\newenvironment{SubProbs}{%
  \begin{list}{}{\SublistInit}%
  }{%
  \end{list}%
}

\newenvironment{Answers}[4][1]{%
  \ifthenelse{\not\equal{#2}{I}}{\vspace{12pt plus 24pt minus 12pt}\tb#4}{}%
%
  \section*{\centering\normalsize 练习~#2.\quad\normalfont#3\label{AnsEx:#2}}%
  \begin{list}{}{\ListInitAns}%
    \ifthenelse{\equal{#1}{1}}{%
      \setboolean{InMulticols}{false}% Update state; else
    }{%
      \setboolean{InMulticols}{true}%
      \begin{multicols}{#1}[\raggedcolumns]%
      }%
    }{% End of envt code
      \ifthenelse{\boolean{InMulticols}}{\end{multicols}}{}%
  \end{list}%
  \setboolean{InMulticols}{false}%
}

% Exercise and answer numbers
\newcommand\Item[2][]{%
   \item[#2]%
   \ifthenelse{\not\equal{#1}{}}{\phantomsection\label{#1}}{}%
}

% Table entries
\newcolumntype{.}[1]{D{.}{.}{#1}}

\newcommand{\Td}[2][r]{%
  \settowidth{\TmpLen}{9999}%
  \ifthenelse{\equal{#1}{r}}{%
    \makebox[\TmpLen][#1]{$#2$\;}%
  }{%
    \ifthenelse{\equal{#1}{l}}{%
      \makebox[\TmpLen][#1]{\;$#2$}%
    }{%
      \makebox[\TmpLen][#1]{$#2$}%
    }%
  }%
}

% polynomial division on 051.png
\newcommand\TmpColA{%
  \begin{tabular}{@{}c|}$v + dv$\DStrut\\\hline\end{tabular}%
}

\newcommand\TmpColB{%
  \;\begin{tabular}{@{}l@{}}$u + \dfrac{u· dv}{v}\DStrut$ \\\hline\end{tabular}%
}

% Catalogue macros
\newcommand{\Catalogue}{%
  \flushpage
  \thispagestyle{empty}
  \phantomsection\pdfbookmark[0]{Catalogue}{Catalogue}
  \SetBothHeads{Catalogue}
  \begin{center}
    \large A SELECTION OF \\
    % [** TN: Fake boldface smallcaps to avoid loading fontenc package]
    \bfseries \Huge M\Large ATHEMATICAL \Huge W\Large ORKS
    \tb
  \end{center}
}

\newcommand{\License}{%
  \flushpage
  \thispagestyle{empty}
  \phantomsection\pdfbookmark[0]{License}{License}
  \SetBothHeads{License}
}

\newcommand{\Title}[1]{{《#1》}}
\newcommand{\Author}[1]{\textsc{#1}}

\newcommand{\Book}[1]
  {\smallskip\par\noindent\hangindent\parindent #1}


% Illustrations
\newcommand{\Graphic}[2][]{%
  \ifthenelse{\equal{#1}{}}{%
    \includegraphics{./images/#2.pdf}%
  }{%
    \includegraphics[width=#1]{./images/#2.pdf}%
  }%
}

% Usage: \Figure[optional width]{File name}{Fig number}
\newcommand{\Figure}[3][2.25in]{%
  \begin{figure}[hbt]
    \centering%
    \Graphic[#1]{#2}%
    \phantomsection\label{fig:#3}%
  \end{figure}%
}

% For pairs of side-by-side images
\newcommand{\Figures}[5][2.25in]{%
  \begin{figure}[hbt]
    \centering%
    \makebox[0pt][c]{% Force centering even if sum of widths is large
      \Graphic[#1]{#2}\qquad\Graphic[#1]{#3}%
    }%
    \phantomsection\label{fig:#4}\label{fig:#5}%
  \end{figure}%
}

% Cross-referencing: anchors
\newcommand{\DPPageSep}[2]{\Pagelabel{#2}}

\newcommand{\Pagelabel}[1]
  {\phantomsection\label{#1}}

% and links
\newcommand{\Pageref}[2][p.]{%
  \ifthenelse{\not\equal{#1}{}}{%
    \hyperref[#2]{#1~\pageref{#2}}%
  }{%
    \hyperref[#2]{\pageref{#2}}%
  }%
}

\newcommand{\Pagerange}[2]{%
  \ifthenelse{\equal{\pageref{#1}}{\pageref{#2}}}{%
    \hyperref[#1]{p.~\pageref{#1}}%
  }{%
%[** TN: Formatting as pp. m--n instead of pp. n, n+1.]
    pp.~\hyperref[#1]{\pageref{#1}}--\hyperref[#2]{\pageref{#2}}%
  }%
}

\newcommand{\Fig}[1]{\hyperref[fig:#1]{Fig.~#1}}

\newcommand{\Figs}[3]{%
  Figs.\ \hyperref[fig:#1]{#1}~#2~\hyperref[fig:#3]{#3}}

\newcommand{\NB}{\textit{N.B.}}
\newcommand{\IE}{\textit{i.e.}}

% Thought break
\newcommand{\tb}[1][1.5in]{%
%  \pagebreak[0]\begin{center}\rule{#1}{0.5pt}\end{center}\pagebreak[3]%
  \pagebreak[0]\par{\centering\rule{#1}{0.5pt}\pagebreak[3]\par}%
}


\newcommand{\DPtypo}[2]{#2}% Corrections.
\newcommand{\DPchg}[2]{#2}%  Stylistic tweaks
\newcommand{\DPnote}[1]{}%   Notes to posterity

% \DeclareUnicodeCharacter{00A3}{\pounds}
% \DeclareUnicodeCharacter{00B0}{{}^\circ}
% \DeclareUnicodeCharacter{00B1}{\pm}
% \DeclareUnicodeCharacter{00B7}{\cdot}
% \DeclareUnicodeCharacter{00D7}{\times}
% \DeclareUnicodeCharacter{00F7}{\div}


\newcommand{\Z}{\phantom{0}}
\newcommand{\DStrut}{\rule[-12pt]{0pt}{32pt}}
\newcommand{\Strut}{\rule{0pt}{16pt}}

\newcommand{\First}[1]{\noindent\textsc{#1}}
\renewcommand{\First}[1]{#1}

% "exponent fraction" factored out for special handling, if desired
\newcommand{\efrac}[2]{\frac{#1}{#2}}

\DeclareMathOperator{\Arccos}{arc\,cos}
\renewcommand{\arccos}{\Arccos}

\DeclareMathOperator{\Arcsin}{arc\,sin}
\renewcommand{\arcsin}{\Arcsin}

\DeclareMathOperator{\Arctan}{arc\,tan}
\renewcommand{\arctan}{\Arctan}

\DeclareMathOperator{\arcsec}{arc\,sec}
\DeclareMathOperator{\cosec}{cosec}
\DeclareMathOperator{\cotan}{cotan}
\DeclareMathOperator{\sech}{sech}

% DPalign
\makeatletter
\providecommand\shortintertext\intertext
\newcount\DP@lign@no
\newtoks\DP@lignb@dy
\newif\ifDP@cr
\newif\ifbr@ce
\def\f@@zl@bar{\null}
\def\addto@DPbody#1{\global\DP@lignb@dy\@xp{\the\DP@lignb@dy#1}}
\def\parseb@dy#1{\ifx\f@@zl@bar#1\f@@zl@bar
    \addto@DPbody{{}}\let\@next\parseb@dy
  \else\ifx\end#1
    \let\@next\process@DPb@dy
    \ifDP@cr\else\addto@DPbody{\DPh@@kr&\DP@rint}\@xp\addto@DPbody\@xp{\@xp{\the\DP@lign@no}&}\fi
    \addto@DPbody{\end}
  \else\ifx\intertext#1
    \def\@next{\eat@command0}%
  \else\ifx\shortintertext#1
    \def\@next{\eat@command1}%
  \else\ifDP@cr\addto@DPbody{&\DP@lint}\@xp\addto@DPbody\@xp{\@xp{\the\DP@lign@no}&\DPh@@kl}
          \DP@crfalse\fi
    \ifx\begin#1\def\begin@stack{b}
      \let\@next\eat@environment
  \else\ifx\lintertext#1
    \let\@next\linter@text
  \else\ifx\rintertext#1
    \let\@next\rinter@text
  \else\ifx\\#1
    \addto@DPbody{\DPh@@kr&\DP@rint}\@xp\addto@DPbody\@xp{\@xp{\the\DP@lign@no}&\\}\DP@crtrue
    \global\advance\DP@lign@no\@ne
    \let\@next\parse@cr
  \else\check@braces#1!Q!Q!Q!\ifbr@ce\addto@DPbody{{#1}}\else
    \addto@DPbody{#1}\fi
    \let\@next\parseb@dy
  \fi\fi\fi\fi\fi\fi\fi\fi\@next}
\def\process@DPb@dy{\let\lintertext\@gobble\let\rintertext\@gobble
  \@xp\start@align\@xp\tw@\@xp\st@rredtrue\@xp\m@ne\the\DP@lignb@dy}
\def\linter@text#1{\@xp\DPlint\@xp{\the\DP@lign@no}{#1}\parseb@dy}
\def\rinter@text#1{\@xp\DPrint\@xp{\the\DP@lign@no}{#1}\parseb@dy}
\def\DPlint#1#2{\@xp\def\csname DP@lint:#1\endcsname{\text{#2}}}
\def\DPrint#1#2{\@xp\def\csname DP@rint:#1\endcsname{\text{#2}}}
\def\DP@lint#1{\ifbalancedlrint\@xp\ifx\csname
DP@lint:#1\endcsname\relax\phantom
  {\csname DP@rint:#1\endcsname}\else\csname DP@lint:#1\endcsname\fi
  \else\csname DP@lint:#1\endcsname\fi}
\def\DP@rint#1{\ifbalancedlrint\@xp\ifx\csname
DP@rint:#1\endcsname\relax\phantom
  {\csname DP@lint:#1\endcsname}\else\csname DP@rint:#1\endcsname\fi
  \else\csname DP@rint:#1\endcsname\fi}
\def\eat@command#1#2{\ifcase#1\addto@DPbody{\intertext{#2}}\or
  \addto@DPbody{\shortintertext{#2}}\fi\DP@crtrue
  \global\advance\DP@lign@no\@ne\parseb@dy}
\def\parse@cr{\new@ifnextchar*{\parse@crst}{\parse@crst{}}}
\def\parse@crst#1{\addto@DPbody{#1}\new@ifnextchar[{\parse@crb}{\parseb@dy}}
\def\parse@crb[#1]{\addto@DPbody{[#1]}\parseb@dy}
\def\check@braces#1#2!Q!Q!Q!{\def\dp@lignt@stm@cro{#2}\ifx
  \empty\dp@lignt@stm@cro\br@cefalse\else\br@cetrue\fi}
\def\eat@environment#1{\addto@DPbody{\begin{#1}}\begingroup
  \def\@currenvir{#1}\let\@next\digest@env\@next}
\def\digest@env#1\end#2{%
  \edef\begin@stack{\push@begins#1\begin\end \@xp\@gobble\begin@stack}%
  \ifx\@empty\begin@stack
    \@checkend{#2}
    \endgroup\let\@next\parseb@dy\fi
    \addto@DPbody{#1\end{#2}}
    \@next}
\def\lintertext{lint}\def\rintertext{rint}
\newif\ifbalancedlrint
\let\DPh@@kl\empty\let\DPh@@kr\empty
\def\DPg@therl{&\omit\hfil$\displaystyle}
\def\DPg@therr{$\hfil}

\newenvironment{DPalign*}[1][a]{%
  \if m#1\balancedlrintfalse\else\balancedlrinttrue\fi
  \global\DP@lign@no\z@\DP@crfalse
  \DP@lignb@dy{&\DP@lint0&}\parseb@dy
}{%
  \endalign
}
\newenvironment{DPgather*}[1][a]{%
  \if m#1\balancedlrintfalse\else\balancedlrinttrue\fi
  \global\DP@lign@no\z@\DP@crfalse
  \let\DPh@@kl\DPg@therl
  \let\DPh@@kr\DPg@therr
  \DP@lignb@dy{&\DP@lint0&\DPh@@kl}\parseb@dy
}{%
  \endalign
}
\makeatother

%%%%%%%%%%%%%%%%%%%%%%%% START OF DOCUMENT %%%%%%%%%%%%%%%%%%%%%%%%%%

\begin{document}

\pagestyle{empty}
\pagenumbering{Alph}

\phantomsection
\pdfbookmark[-1]{前言}{前言}

%%%%%%%%%%%%%%%%%%%%%%%%%%% 前言 %%%%%%%%%%%%%%%%%%%%%%%%%%

\frontmatter
\pagenumbering{roman}
\DPPageSep{001.png}{i}%

% 半标题页
\cleardoublepage
\null\vfill
\thispagestyle{empty}
\begin{center}
\Huge 微积分轻松学
\end{center}
\vfill
\DPPageSep{002.png}{ii}%

% 出版商页面
\newpage
\null\vfill
\thispagestyle{empty}
\begin{center}
\setlength{\MySkip}{4pt}%
\Graphic[2in]{pubmark} \\[3\MySkip]
%
\small 麦克米伦有限公司 \\[\MySkip]
\scriptsize 伦敦 : 孟买 : 加尔各答 \\[\MySkip]
墨尔本 \\[3\MySkip]
%
\small 麦克米伦公司 \\[\MySkip]
\scriptsize 纽约 : 波士顿 : 芝加哥 \\[\MySkip]
达拉斯 : 旧金山 \\[3\MySkip]
%
\small 麦克米伦加拿大有限公司 \\[\MySkip]
\scriptsize 多伦多 \\[\MySkip]
\end{center}
\vfill
\DPPageSep{003.png}{iii}%

% 标题页
\cleardoublepage
\thispagestyle{empty}
\begin{center}
\setlength{\MySkip}{12pt}%
\Huge \textbf{微积分轻松学}
\vspace{2\MySkip}

\small 这是一本非常简单的入门书,介绍那些美丽的计算方法
\vspace{2\MySkip}

\large 微分学 \\[\MySkip]
\small 和 \\[\MySkip]
\large 积分学
\vfill

\small 作者 \\
\Large F.~R.~S.
\vfill\vfill

\textit{\large 第二版,增订}
\vfill\vfill\vfill\vfill

\large\settowidth{\TmpLen}{圣马丁街,伦敦} %
\SetBox{麦克米伦有限公司} \\
\SetBox{圣马丁街,伦敦} \\
\SetBox[c]{\oldstylenums{1914}}
\end{center}
\DPPageSep{004.png}{iv}%

% 版权页
\newpage
\null\vfill
\thispagestyle{empty}
\begin{center}
\footnotesize
\textit{版权声明}
\bigskip

第一版 1910年.\\
重印 1911年(两次),1912年,1913年.\\
第二版 1914年。
\end{center}
\vfill
\DPPageSep{005.png}{v}%

\cleardoublepage
\thispagestyle{empty}
\null\vfill
\begin{quote}
一个傻瓜能做的事,另一个傻瓜也能做。 \\
\raggedleft(\textit{古老的猿猴谚语。})
\end{quote}
\vfill
\DPPageSep{006.png}{vi}%
%[空白页]
\DPPageSep{007.png}{vii}%

\ChapterStar[前言]{第二版前言}

\First{这本}书的惊人成功促使作者增加了大量例题和练习。同时,作者也根据经验在某些部分进行了扩展,以提供更有用的解释。

作者对许多教师、学生和批评者提出的宝贵建议和来信表示感谢。

\medskip
\emph{十月}, 1914.
\DPPageSep{008.png}{viii}%
%[空白页]
\DPPageSep{009.png}{ix}%

\cleardoublepage
\tableofcontents
\iffalse %%%% 原目录文本 %%%%

目录

章节       页码

序言  xi

一. 摆脱初步的恐惧   1

二. 关于不同程度的小  3

三. 关于相对的增长   9

四. 最简单的情况 18

五. 如何处理常数  26

六. 和、差、积与商  35

七. 逐次微分  49

八. 当时间变化时   52

九. 引入一个有用的技巧  67

十. 微分的几何意义 76

十一. 极大值与极小值   93

十二. 曲线的曲率   112

十三. 其他有用的技巧   121

十四. 关于真正的复利与有机增长的规律   134

十五. 如何处理正弦与余弦  165

十六. 偏微分  175

十七. 积分     182

\DPPageSep{010.png}{x}%

十八. 积分作为微分的逆运算  191

十九. 通过积分求面积  206

二十. 技巧、陷阱与成功   226

二十一. 寻找一些解  234

二十二. 尾声与寓言  249

标准形式表    252

习题答案  254

\fi%%%% 原目录结束 %%%%
\DPPageSep{011.png}{xi}%

\AltChapter{序言}

\First{考虑到}有多少傻瓜能计算,令人惊讶的是,对于其他傻瓜来说,学习掌握同样的技巧竟然被认为是一项困难或乏味的任务。

一些微积分技巧非常简单。一些则极其困难。编写高等数学教科书的傻瓜们——他们大多是聪明的傻瓜——很少费心向你展示简单的计算有多么容易。相反,他们似乎希望通过最困难的方式来完成任务,以向你展示他们惊人的聪明才智。

作为一个极其愚蠢的人,我不得不重新学习如何摆脱这些困难,现在我恳请向我的同类傻瓜们展示那些并不难的部分。彻底掌握这些内容,其余的将随之而来。一个傻瓜能做到的,另一个也能。
\DPPageSep{012.png}{xii}%
\DPPageSep{013.png}{1}%
\mainmatter
\phantomsection\pdfbookmark[-1]{正文}{正文}

\Chapter{第一章}{摆脱初步的恐惧}

\First{初步的恐惧},这种恐惧让大多数五年级的男孩甚至不敢尝试学习如何计算,可以通过简单地说明在常识术语中两个主要符号的含义来一劳永逸地消除。

这些可怕的符号是:

(1) $d$,它仅仅意味着“一小部分”。

因此,$dx$ 表示 $x$ 的一小部分;或者 $du$ 表示 $u$ 的一小部分。普通的数学家认为说“一个元素”比“一小部分”更有礼貌。随你喜欢。但你会发现,这些小部分(或元素)可以被认为是无限小的。

(2) $\ds\int$,它只是一个长长的 $S$,可以称为(如果你愿意)“总和”。

因此,$\ds\int dx$ 表示所有 $x$ 的小部分的总和;或者 $\ds\int dt$ 表示所有 $t$ 的小部分的总和。普通的数学家称这个符号为“积分”。现在,任何傻瓜都能明白,如果 $x$ 被认为是许多小部分组成的,每个小部分称为 $dx$,如果你把它们全部加起来,你将得到所有 $dx$ 的总和(这与整个 $x$ 是相同的)。“积分”这个词仅仅意味着“整体”。如果你想到一小时的时间,你可以(如果你愿意)把它看作被分割成 3600 个小部分,称为秒。所有 3600 个小部分加在一起就是一小时。

当你看到一个以这个可怕符号开头的表达式时,你从现在起就会知道,它放在那里只是为了指示你,你现在要执行的操作(如果你能的话)是将所有由后续符号指示的小部分加总。

就是这样。
\DPPageSep{015.png}{3}%

\Chapter[不同程度的小]{第二章}{关于不同程度的小}

\First{我们}会发现,在我们的计算过程中,我们必须处理各种程度的小量。

我们还需要学习在什么情况下可以将小量视为如此微小,以至于可以忽略它们。一切都取决于相对的微小程度。

在确定任何规则之前,让我们思考一些熟悉的例子。每小时有$60$分钟,每天有$24$小时,每周有$7$天。因此,每天有$1440$分钟,每周有$10080$分钟。

显然,$1$分钟与一整周相比是非常短的时间。事实上,我们的祖先认为它与一小时相比也很小,称之为“一分”,意思是小时的微小部分——即六十分之一。当他们需要更小的时间划分时,他们将每分钟分成$60$个更小的部分,在伊丽莎白女王时代,他们称之为“第二分”\DPnote{** TN: [sic]}(即第二阶微小量的微小部分)。如今,我们称这些第二阶微小量的微小部分为“秒”。但很少有人知道它们为何如此称呼。

现在,如果一分钟与一整天相比已经很小,那么一秒与之相比又小了多少呢?

再想想一个便士与一英镑的比较:它几乎只值$\frac{1}{1000}$。多一个或少一个便士与一英镑相比几乎无关紧要:它当然可以被视为一个\emph{小}量。但将一个便士与£$1000$相比:相对于这笔更大的金额,一个便士的重要性不比$\frac{1}{1000}$个便士对一英镑的重要性更大。即使是一枚金币,相对于百万富翁的财富,也是微不足道的。

现在,如果我们确定任何数值分数作为我们称之为相对小的比例,我们可以很容易地说明其他更高阶的小量。因此,如果出于时间的目的,$\frac{1}{60}$被称为\emph{小}分数,那么$\frac{1}{60}$的$\frac{1}{60}$(即\emph{小}分数的\emph{小}分数)可以被视为\emph{第二阶微小量的}\Pagelabel{smallness}小量。\footnote
  {数学家们谈论第二阶“量级”(即伟大),而他们实际上指的是第二阶\emph{微小量}。这对初学者来说非常令人困惑。}

或者,如果出于某种目的,我们将$1\%$(即$\frac{1}{100}$)视为\emph{小}分数,那么$1\%$的$1\%$(即$\frac{1}{10,000}$)将是第二阶微小量的小分数;而$\frac{1}{1,000,000}$将是第三阶微小量的小分数,即$1\%$的$1\%$的$1\%$。

最后,假设出于某种非常精确的目的,我们将$\frac{1}{1,000,000}$视为“小”。因此,如果一个一流的计时器在一年内不能快或慢超过半分钟,它必须保持$1$分之$1,051,200$的精度。现在,如果出于这样的目的,我们将$\frac{1}{1,000,000}$(或百万分之一)视为小量,那么$\frac{1}{1,000,000}$的$\frac{1}{1,000,000}$,即$\frac{1}{1,000,000,000,000}$(或十亿分之一)将是第二阶微小量的小量,与之相比可以完全忽略。

然后我们看到,小量本身越小,相应的第二阶小量就越微不足道。因此,我们知道\emph{在所有情况下,只要我们将第一阶小量本身取得足够小,我们就有理由忽略第二阶或第三阶(或更高阶)的小量}。

但必须记住,如果小量在我们的表达式中作为因子与其他因子相乘,那么当其他因子本身很大时,这些小量可能会变得重要。即使是一个便士,只要乘以几百,也会变得重要。

现在在微积分中,我们用$dx$表示$x$的一小部分。这些诸如$dx$、$du$和$dy$的量被称为“微分”,即$x$、$u$或$y$的微分,视情况而定。[你\emph{读}它们为\emph{dee-eks}、\emph{dee-you}或\emph{dee-wy}。]如果$dx$是$x$的一小部分,并且相对于其本身较小,这并不意味着诸如$x \cdot dx$、$x^2 \cdot dx$或$a^x \cdot dx$这样的量可以忽略不计。但$dx \times dx$将是可忽略的,因为它是一个二阶小量。

一个非常简单的例子可以作为说明。

让我们将$x$视为一个可以通过增长一小部分而变为$x + dx$的量,其中$dx$是增长所增加的小增量。这个量的平方是$x^2 + 2x \cdot dx + (dx)^2$。第二项是不可忽略的,因为它是第一阶的量;而第三项是第二阶的小量,是$x^2$的一小部分。因此,如果我们取$dx$在数值上表示,例如,$x$的$\frac{1}{60}$,那么第二项将是$x^2$的$\frac{2}{60}$,而第三项将是$x^2$的$\frac{1}{3600}$。显然,最后一项比第二项重要性低。但如果我们进一步取$dx$仅为$x$的$\frac{1}{1000}$,那么第二项将是$x^2$的$\frac{2}{1000}$,而第三项将仅为$x^2$的$\frac{1}{1,000,000}$。

\Figure[1.5in]{018a}{1}

几何上,这可以描述如下:画一个正方形(\Fig{1}),其边长我们取为$x$。现在假设这个正方形通过在每边增加一小部分$dx$而增长。增大的正方形由原来的正方形$x^2$、顶部和右侧的两个矩形组成,每个矩形的面积为$x \cdot dx$(或总共$2x \cdot dx$),以及右上角的小正方形,其面积为$(dx)^2$。在\Fig{2}中,我们取$dx$为$x$的相当大的一部分——大约$\frac{1}{5}$。但如果我们取$dx$仅为$\frac{1}{100}$——大约是一支细笔画出的墨线厚度。那么小角落正方形的面积将仅为$x^2$的$\frac{1}{10,000}$,实际上是看不见的。显然,$(dx)^2$是可忽略的,只要我们考虑增量$dx$本身足够小。

让我们考虑一个比喻。

假设一个百万富翁对他的秘书说:下周我将给你我收到的任何钱的一小部分。假设秘书对他的男孩说:我将给你我得到的一小部分。假设每种情况下的分数都是$\frac{1}{100}$。现在如果百万富翁在下一周收到£$1000$,秘书将收到£$10$,而男孩将收到$2$先令。十英镑与£$1000$相比是一个小量;但两先令确实是一个非常小的量,属于次要的阶次。但如果分数不是$\frac{1}{100}$,而是定为$\frac{1}{1000}$,那么比例会是怎样的呢?那么,当百万富翁得到他的£$1000$时,秘书将只得到£$1$,而男孩将不到一个便士!

机智的斯威夫特院长\footnote{\textit{On Poetry: a Rhapsody} (p.~20), printed 1733---usually misquoted.}曾写道:
\begin{center}\small%
\settowidth{\TmpLen}{``And these have smaller Fleas to bite 'em,}%
\begin{minipage}{\TmpLen}\raggedright
``So, Nat'ralists observe, a Flea\\
``Hath smaller Fleas that on him prey.\\
``And these have smaller Fleas to bite 'em,\\
``And so proceed \textit{ad infinitum}.''
\end{minipage}
\end{center}

一头牛可能会担心一只普通大小的跳蚤——一个属于第一级小型的生物。但它大概不会为跳蚤身上的跳蚤操心;作为第二级小型生物,它是可以忽略不计的。即使是一打跳蚤身上的跳蚤,对牛来说也不算什么。
\DPPageSep{021.png}{9}%

\Chapter{第三章}{关于相对增长}

\First{在}整个微积分中,我们处理的是增长的量和增长的速度。我们将所有量分为两类:\emph{常量}和\emph{变量}。那些我们认为具有固定值并称为\emph{常量}的量,我们通常用字母表开头的字母来表示,如$a$、$b$或$c$;而那些我们认为可以增长或(如数学家所说)“变化”的量,我们用字母表末尾的字母来表示,如$x$、$y$、$z$、$u$、$v$、$w$,有时也用$t$。

此外,我们通常同时处理多个变量,并思考其中一个变量如何依赖于另一个变量:例如,我们思考抛射体达到某一高度所需时间的方式。或者,我们被要求考虑一个给定面积的矩形,并探讨其长度增加时,宽度如何相应减少。或者,我们思考梯子倾斜度变化时,其达到的高度如何变化。

假设我们有两个这样的变量,它们相互依赖。一个变量的变化会导致另一个变量的变化,\emph{因为}这种依赖关系。我们称其中一个变量为$x$,另一个依赖于它的变量为$y$。

假设我们使$x$变化,也就是说,我们改变它或想象它被改变,通过给它加上一个我们称为$dx$的小量。这样,$x$就变成了$x + dx$。然后,因为$x$被改变了,$y$也会随之改变,变成$y + dy$。这里,$dy$在某些情况下可能是正的,在其他情况下可能是负的;并且它不会(除非奇迹发生)与$dx$大小相同。

\Subsection{举两个例子。}
(1) 设$x$和$y$分别为直角三角形的底和高(\Fig{4}),其中另一边的斜率固定为$30°$。如果我们假设这个三角形扩展但保持其角度不变,那么当底边增长到$x + dx$时,高度变为$y + dy$。这里,增加$x$会导致$y$的增加。小三角形的高度为$dy$,底为$dx$,与原三角形相似;显然,比率$\dfrac{dy}{dx}$的值与比率$\dfrac{y}{x}$相同。由于角度为$30°$,可以看出这里
\[
\frac{dy}{dx} = \frac{1}{1.73}.
\]

(2) 设$x$在\Fig{5}中表示从墙到梯子底部端点的水平距离,$AB$为固定长度的梯子,$y$为梯子达到的高度。现在,$y$显然依赖于$x$。很容易看出,如果我们把底部端点$A$拉离墙一点,顶部端点$B$会稍微下降。让我们用科学语言来表述这一点。如果我们把$x$增加到$x + dx$,那么$y$将变为$y - dy$;也就是说,当$x$得到一个正增量时,$y$得到的增量是负的。

是的,但具体是多少呢?假设梯子很长,当底部端点$A$离墙19英寸时,顶部端点$B$正好达到离地面15英尺的高度。现在,如果你把底部端点拉出1英寸,顶部端点会下降多少?把所有单位换算成英寸:$x = 19$英寸,$y = 180$英寸。现在,我们称之为$dx$的$x$的增量是1英寸:即$x + dx = 20$英寸。

$y$ 会减少多少?新的高度将是 $y - dy$。如果我们根据欧几里得第一卷第47命题计算高度,那么我们就能找出 $dy$ 的大小。梯子的长度为
\[
\sqrt{ (180)^2 + (19)^2 } = 181 \text{ 英寸}。
\]
显然,新的高度 $y - dy$ 将满足
\begin{align*}
(y - dy)^2 &= (181)^2 - (20)^2 = 32761 - 400 = 32361,   \\
y - dy     &= \sqrt{32361} = 179.89 \text{ 英寸}。
\end{align*}
现在 $y$ 是 $180$,所以 $dy$ 是 $180 - 179.89 = 0.11$ 英寸。

由此可见,使 $dx$ 增加 $1$ 英寸,导致 $dy$ 减少了 $0.11$ 英寸。

而 $dy$ 与 $dx$ 的比率可以表示为:
\[
\frac{dy}{dx} = - \frac{0.11}{1}。
\]

同样容易看出,(除了一种特定位置外)$dy$ 的大小将与 $dx$ 不同。

现在,在整个微分学中,我们一直在寻找,寻找,寻找一个奇妙的东西,一个简单的比率,即当 $dy$ 和 $dx$ 都无限小时,$dy$ 与 $dx$ 的比例。

这里需要注意的是,只有当 $y$ 和 $x$ 以某种方式相互关联时,我们才能找到这个比率 $\dfrac{dy}{dx}$,即每当 $x$ 变化时,$y$ 也会随之变化。例如,在第一个例子中,如果三角形的底边 $x$ 变长,三角形的高度 $y$ 也会变大;在第二个例子中,如果梯子脚与墙的距离 $x$ 增加,梯子达到的高度 $y$ 会相应地减少,起初缓慢,但随着 $x$ 的增大,减少得越来越快。在这些情况下,$x$ 和 $y$ 之间的关系是确定的,可以用数学表达,分别为 $\dfrac{y}{x} = \tan 30°$ 和 $x^2 + y^2 = l^2$(其中 $l$ 是梯子的长度),并且 $\dfrac{dy}{dx}$ 在每种情况下都有我们找到的意义。

如果,像之前一样,$x$ 是梯子脚与墙的距离,而 $y$ 不是达到的高度,而是墙的水平长度,或者是墙上的砖块数量,或者是墙建造以来的年数,那么 $x$ 的任何变化自然不会引起 $y$ 的任何变化;在这种情况下,$\dfrac{dy}{dx}$ 没有任何意义,也不可能找到它的表达式。每当我们使用微分 $dx$、$dy$、$dz$ 等时,都隐含着 $x$、$y$、$z$ 等之间存在某种关系,这种关系被称为 $x$、$y$、$z$ 等的“函数”;例如,上面给出的两个表达式,即 $\dfrac{y}{x} = \tan 30°$ 和 $x^2 + y^2 = l^2$,就是 $x$ 和 $y$ 的函数。这些表达式隐含地(即,包含但不明显显示)表达了用 $y$ 表示 $x$ 或用 $x$ 表示 $y$ 的方法,因此它们被称为 $x$ 和 $y$ 的隐函数;它们可以分别写成
\begin{DPalign*}
y &= x \tan 30° \quad\text{或}\quad x = \frac{y}{\tan 30°} \\
\lintertext{和}
y &= \sqrt{ l^2 - x^2} \quad\text{或}\quad x = \sqrt{ l^2 - y^2}。
\end{DPalign*}

这些最后的表达式明确地(即清晰地)表示了$x$关于$y$的值,或$y$关于$x$的值,因此它们被称为$x$或$y$的\emph{显函数}。例如,$x^2 + 3 = 2y - 7$是$x$和$y$的隐函数;它可以写成$y = \dfrac{x^2 + 10}{2}$($x$的显函数)或$x = \sqrt{2y - 10}$($y$的显函数)。我们看到,$x$、$y$、$z$等的显函数,其值简单地随着$x$、$y$、$z$等的变化而变化,无论是逐个变化还是多个一起变化。正因为如此,显函数的值被称为\emph{因变量},因为它依赖于函数中其他变量的值;\DPPageSep{027.png}{15}%
这些其他变量被称为\emph{自变量}\Pagelabel{indvar},因为它们的值不是由函数所取的值决定的。例如,如果$u = x^2 \sin \theta$,$x$和$\theta$是自变量,而$u$是因变量。

有时,几个量$x$、$y$、$z$之间的确切关系要么未知,要么不便于说明;只知道或便于说明这些变量之间存在某种关系,因此不能单独改变$x$、$y$或$z$中的任何一个而不影响其他量;这时,$x$、$y$、$z$之间的函数关系用符号\Pagelabel{notation} $F(x, y, z)$(隐函数)或$x = F(y, z)$、$y = F(x, z)$或$z = F(x, y)$(显函数)表示。有时用字母$f$或$\phi$代替$F$,因此$y = F(x)$、$y = f(x)$和$y = \phi(x)$都表示相同的意思,即$y$的值以某种未说明的方式依赖于$x$的值。

我们称比率$\dfrac{dy}{dx}$为“$y$关于$x$的\emph{导数}”。这是一个非常简单的科学名称。但我们不会被这些庄严的名字吓倒,因为事物本身是如此简单。我们不会被吓倒,而是简单地对给长而拗口的名字的愚蠢行为发出一声诅咒;然后,心绪平复后,我们将继续探讨简单的事物本身,即比率$\dfrac{dy}{dx}$。
\DPPageSep{028.png}{16}%

在你所学的普通代数中,你总是在寻找某个未知的量,称为$x$或$y$;有时甚至有两个未知的量需要同时寻找。你现在需要学习一种新的寻找方法;现在的目标既不是$x$也不是$y$。相反,你必须寻找这个被称为$\dfrac{dy}{dx}$的奇特小家伙。寻找$\dfrac{dy}{dx}$的值的过程称为“求导”。但请记住,我们所需要的是当$dy$和$dx$本身都无限小时,这个比率的值。导数的真正值是当它们都被视为无限小时,它所趋近的极限值。

现在让我们学习如何去寻找$\dfrac{dy}{dx}$。
\DPPageSep{029.png}{17}%

\Note{第三章注释}

\Section{如何阅读微分符号}

千万不要陷入学生常犯的错误,认为$dx$表示$d$乘以$x$,因为$d$不是一个因子——它意味着“……的元素”或“……的一部分”。我们这样读$dx$:“迪-艾克斯”。

如果读者在这些事情上没有人指导,这里可以简单地说,微分系数是这样读的。微分系数
\begin{DPgather*}%[** 语义滥用]
\dfrac{dy}{dx}
\text{ 读作“迪-威-艾克斯,”或“迪-威-艾克斯分之迪-艾克斯。”} \\
\lintertext{\rlap{同样}}
\dfrac{du}{dt} \text{ 读作“迪-优-迪-提。”}
\end{DPgather*}

二次微分系数将在后面遇到。它们是这样的:
\[
\dfrac{d^2 y}{dx^2};
\text{ 读作“\emph{dee-two-wy 除以 dee-eks-squared}”,}
\]
它表示对 $y$ 关于 $x$ 的微分操作已经(或需要)执行了两次。

另一种表示函数已被微分的方法是在函数符号上加一个撇号。因此,如果 $y=F(x)$,这意味着 $y$ 是 $x$ 的某个未指定函数(见 \Pageref{function}),我们可以写成 $F'(x)$ 而不是 $\dfrac{d\bigl(F(x)\bigr)}{dx}$。同样,$F''(x)$ 表示原函数 $F(x)$ 已经关于 $x$ 微分了两次。
\DPPageSep{030.png}{18}%

\Chapter{第四章}{最简单的情况}

\First{现在}让我们看看如何从基本原理出发,对一些简单的代数表达式进行微分。

\Case{1}
让我们从简单的表达式 $y=x^2$ 开始。
现在请记住,微积分的基本概念是\emph{增长}。数学家称之为\emph{变化}。既然 $y$ 和 $x^2$ 相等,显然如果 $x$ 增长,$x^2$ 也会增长。而如果 $x^2$ 增长,那么 $y$ 也会增长。我们需要找出的是 $y$ 的增长与 $x$ 的增长之间的比例。换句话说,我们的任务是找出 $dy$ 和 $dx$ 之间的比率,简而言之,就是求出 $\dfrac{dy}{dx}$ 的值。

那么,让 $x$ 稍微变大一点,变成 $x + dx$;同样,$y$ 也会稍微变大一点,变成 $y + dy$。显然,增大的 $y$ 仍然等于增大的 $x$ 的平方。写下来就是:
\begin{align*}
y + dy &= (x + dx)^2.
\intertext{\indent 进行平方运算后得到:}
y + dy &= x^2 + 2x · dx+(dx)^2.
\end{align*}
\DPPageSep{031.png}{19}%

$(dx)^2$ 是什么意思?记住 $dx$ 表示 $x$ 的一小部分——一点点 $x$。那么 $(dx)^2$ 将表示 $x$ 的一小部分的很小一部分;也就是说,如上所述(\Pageref{smallness}),它是第二阶小量的一个小量。因此,与其它项相比,它可以被忽略不计。去掉它,我们得到:\Pagelabel{diffexample}%
\begin{align*}
y + dy &= x^2 + 2x · dx. \displaybreak[1] \\
\intertext{\indent 现在 $y=x^2$;所以让我们从方程中减去这个,剩下的是}
dy &= 2x · dx. \displaybreak[1] \\
\intertext{\indent 两边除以 $dx$,我们得到}
\frac{dy}{dx} &= 2x.
\end{align*}

现在\emph{这个}\footnote
  {\NB---这个比率 $\dfrac{dy}{dx}$ 是 $y$ 对 $x$ 微分的结果。微分意味着求导数。假设我们有一些其他的 $x$ 的函数,例如 $u = 7x^2 + 3$。那么如果我们被告知要对 $x$ 微分这个函数,我们就必须求 $\dfrac{du}{dx}$,或者换句话说,$\dfrac{d(7x^2 + 3)}{dx}$。另一方面,我们可能有一个时间作为自变量的情况(见 \Pageref{indvar}),例如 $y = b + \frac{1}{2} at^2$。那么,如果我们被告知要微分它,这意味着我们必须求它对 $t$ 的导数。因此,我们的任务将是尝试求 $\dfrac{dy}{dt}$,即求 $\dfrac{d(b + \frac{1}{2} at^2)}{dt}$。}
就是我们一开始要找的。在我们讨论的情况下,$y$ 的增长与 $x$ 的增长的比率被发现是 $2x$。
\DPPageSep{032.png}{20}%

\Subsection{数值例子。}
假设 $x=100$ 且因此 $y=10,000$。然后让 $x$ 增长到 $101$(即 $dx=1$)。那么增大的 $y$ 将是 $101 × 101 = 10,201$。但如果我们同意可以忽略第二阶小量,$1$ 与 $10,000$ 相比可以忽略不计;所以我们可以将增大的 $y$ 近似为 $10,200$。$y$ 从 $10,000$ 增长到 $10,200$;增加的部分是 $dy$,因此是 $200$。

$\dfrac{dy}{dx} = \dfrac{200}{1} = 200$。根据前一段的代数推导,我们得到$\dfrac{dy}{dx} = 2x$。确实如此;因为当$x=100$时,$2x=200$。

但是,你会说,我们忽略了一个完整的单位。

好吧,再试一次,让$dx$变得更小。

设$dx=\frac{1}{10}$。那么$x+dx=100.1$,且
\[
(x+dx)^2 = 100.1 × 100.1 = 10,020.01.
\]

现在最后一位数字$1$仅是$10,000$的百万分之一,完全可以忽略不计;因此我们可以取$10,020$而忽略末尾的小数。这使得$dy=20$;且$\dfrac{dy}{dx} = \dfrac{20}{0.1} = 200$,这与$2x$仍然相同。

\Case{2}
尝试用同样的方法求导$y = x^3$。

我们让$y$增长到$y+dy$,同时$x$增长到$x+dx$。

于是我们有
\[
y + dy = (x + dx)^3.
\]
\DPPageSep{033.png}{21}%

展开立方后得到
\[
y + dy = x^3 + 3x^2 · dx + 3x(dx)^2+(dx)^3.
\]

现在我们知道可以忽略二阶和三阶的小量;因为当$dy$和$dx$都变得无限小时,$(dx)^2$和$(dx)^3$相比之下会变得更小。因此,将它们视为可忽略,我们得到:
\[
y + dy=x^3+3x^2 · dx.
\]

但$y=x^3$;减去这一项,我们得到:
\begin{DPalign*}
dy &= 3x^2 · dx, \\
\lintertext{且}
\frac{dy}{dx} &= 3x^2.
\end{DPalign*}

\Case{3}
尝试求导$y=x^4$。像之前一样,让$y$和$x$都增长一点,我们有:
\begin{DPalign*}
y + dy &= (x+dx)^4. \displaybreak[1] \\
%
\intertext{\indent 展开四次方后,我们得到}
y + dy &= x^4 + 4x^3\, dx + 6x^2(dx)^2 + 4x(dx)^3+(dx)^4. \displaybreak[1] \\
%
\intertext{\indent 然后去掉包含$dx$高次幂的项,因为它们相比可以忽略,我们得到}
y + dy &= x^4+4x^3\, dx. \displaybreak[1] \\
%
\intertext{\indent 减去原式$y=x^4$,我们得到}
dy &= 4x^3\, dx, \\
\lintertext{且}
\frac{dy}{dx} &= 4x^3.
\end{DPalign*}

\tb
\DPPageSep{034.png}{22}%

现在这些例子都很简单。让我们总结一下结果,看看是否能推导出一般规律。将它们列成两列,$y$的值在一列,对应的$\dfrac{dy}{dx}$的值在另一列:
\[
\begin{array}{|@{\quad}c@{\quad}|@{\quad}l@{\quad}|}
\hline
y   & \DStrut\dfrac{dy}{dx} \\
\hline
x^2 & 2x \Strut \\
x^3 & 3x^2  \\
x^4 & 4x^3  \\
\hline
\end{array}
\]

\Pagelabel{diffrule1}%
仔细观察这些结果:求导操作似乎使得$x$的幂次减1(例如在最后一个例子中,将$x^4$降为$x^3$),同时乘以一个数(实际上是原幂次)。现在,一旦你发现了这一点,你很容易推测其他情况。你会预期求导$x^5$会得到$5x^4$,或者求导$x^6$会得到$6x^5$。如果你犹豫,可以尝试其中一个,看看推测是否正确。

尝试$y = x^5$。
\begin{DPalign*}
\lintertext{\indent 那么}
y+dy &= (x+dx)^5     \\
     &= x^5 + 5x^4\, dx + 10x^3(dx)^2  + 10x^2(dx)^3 \\
     &\phantom{{}= x^5 + 5x^4\, dx} + 5x(dx)^4 + (dx)^5.
\end{DPalign*}

忽略所有包含高阶小量的项,我们得到
\begin{DPalign*}
y + dy &= x^5 + 5x^4\, dx, \displaybreak[1] \\
\DPPageSep{035.png}{23}%
\lintertext{\rlap{减去}}
y &= x^5 \text{ 后,我们得到} \\
dy &= 5x^4\, dx, \displaybreak[1] \\
\lintertext{因此}
\frac{dy}{dx} &= 5x^4, \text{ 正如我们所推测的那样。}
\end{DPalign*}

\tb

逻辑上遵循我们的观察,我们可以得出结论:如果我们想处理任意高次幂——设为$n$——我们可以用同样的方法。
\begin{DPalign*}
\lintertext{\indent 设}
y &= x^n, \\
\intertext{那么,我们预期会得到}
\frac{dy}{dx} &= nx^{(n-1)}.
\end{DPalign*}

例如,设$n=8$,则$y=x^8$;求导后得到$\dfrac{dy}{dx} = 8x^7$。

事实上,关于对$x^n$求导得到$nx^{n-1}$的规则,对于所有$n$为正整数的情况都是成立的。[通过二项式定理展开$(x + dx)^n$可以立即证明这一点。] 但当$n$为负数或分数时,这一规则是否仍然成立,则需要进一步考虑。

\Subsection{负幂次的情况。}
设$y = x^{-2}$。然后按照之前的方法进行:
\begin{align*}
y+dy &= (x+dx)^{-2} \\
     &= x^{-2} \left(1 + \frac{dx}{x}\right)^{-2}.
\end{align*}
\DPPageSep{036.png}{24}%
通过二项式定理展开(见\Pageref{binomtheo}),我们得到
\begin{align*}
&=x^{-2} \left[1 - \frac{2\, dx}{x} +
    \frac{2(2+1)}{1×2} \left(\frac{dx}{x}\right)^2 -
    \text{等等}\right]  \\
&=x^{-2} - 2x^{-3} · dx + 3x^{-4}(dx)^2 - 4x^{-5}(dx)^3 + \text{等等} \\
\intertext{%
\indent 因此,忽略高阶小量,我们有:}
       y + dy &= x^{-2} - 2x^{-3} · dx.
\intertext{减去原始的$y = x^{-2}$,我们得到}
           dy &= -2x^{-3}dx,   \\
\frac{dy}{dx} &= -2x^{-3}.
\end{align*}
这与上述推导的规则仍然一致。

\Subsection{分数幂次的情况。}
设$y= x^{\efrac{1}{2}}$。然后,同样地,
\settowidth{\TmpLen}{高阶项}%
\begin{align*}
y+dy &= (x+dx)^{\efrac{1}{2}} = x^{\efrac{1}{2}}
        \left(1 + \frac{dx}{x} \right)^{\efrac{1}{2}} \\
     &= \sqrt{x} + \frac{1}{2} \frac{dx}{\sqrt{x}} - \frac{1}{8}
        \frac{(dx)^2}{x\sqrt{x}} +
        \raisebox{-1.5ex}{\parbox[c]{\TmpLen}{\begin{center}
          高阶项\\
          关于$dx$的幂次。\end{center}}}\DPnote{[ **\raisebox 可选]}
\end{align*}

减去原始的$y = x^{\efrac{1}{2}}$,并忽略高阶项,我们得到:
\[
dy = \frac{1}{2} \frac{dx}{\sqrt{x}} = \frac{1}{2} x^{-\efrac{1}{2}} · dx,
\]
且$\dfrac{dy}{dx} = \dfrac{1}{2} x^{-\efrac{1}{2}}$。这与一般规则一致。
\DPPageSep{037.png}{25}%

\Paragraph{总结。}让我们看看我们已经得到了什么。我们得出了以下规则:\Pagelabel{multipow} 对$x^n$求导,乘以幂次并减少幂次1,结果为$nx^{n-1}$。

\Exercises{I}(答案见\Pageref{AnsEx:I})

求下列函数的导数:
\begin{Problems}[2]
\Item{(1)} $y = x^{13}$
\Item{(2)} $y = x^{-\efrac{3}{2}}$
\ResetCols{2}

\Item{(3)} $y = x^{2a}$
\Item{(4)} $u = t^{2.4}$
\ResetCols{2}

\Item{(5)} $z = \sqrt[3]{u}$
\Item{(6)} $y = \sqrt[3]{x^{-5}}$
\ResetCols{2}

\Item{(7)} $u = \sqrt[5]{\dfrac{1}{x^8}}$
\Item{(8)} $y = 2x^a$\DPtypo{.}{}
\ResetCols{2}

\Item{(9)} $y = \sqrt[q]{x^3}$
\Item{(10)} $y = \sqrt[n]{\dfrac{1}{x^m}}$
\end{Problems}

\textit{你现在学会了如何对$x$的幂次求导。多么简单啊!}
\DPPageSep{038.png}{26}%

\Chapter[常数的处理]{第五章}{如何处理常数}

\First{在}我们的方程中,我们将$x$视为可变的,并且由于$x$的变化,$y$也改变了其值并增长。我们通常认为$x$是一个我们可以改变的量;并且,将$x$的变化视为一种\emph{原因},我们将$y$的变化视为\emph{结果}。换句话说,我们认为$y$的值取决于$x$的值。$x$和$y$都是变量,但$x$是我们操作的对象,而$y$是“因变量”。在前一章中,我们一直在尝试找出因变量$y$的变化与独立变化的$x$之间的比例关系。

\Pagelabel{diffrule2}%
我们的下一步是找出\emph{常数}的存在对求导过程的影响,即当$x$或$y$改变其值时,这些常数不发生变化。

\Subsection{添加的常数}\Pagelabel{addconst}
让我们从一个添加常数的简单情况开始,如下所示:
\begin{DPalign*}
\lintertext{\indent 设}
y=x^3+5。
\end{DPalign*}
与之前一样,假设 $x$ 增长到 $x+dx$,
$y$ 增长到 $y+dy$。
\DPPageSep{039.png}{27}%
\begin{DPalign*}
\lintertext{\indent 那么:}
y + dy &= (x + dx)^3 + 5 \\
       &= x^3 + 3x^2\, dx + 3x(dx)^2 + (dx)^3 + 5。
\intertext{忽略高阶小量,这变为}
y + dy &= x^3 + 3x^2·dx + 5。 \\
\intertext{减去原式 $y = x^3 + 5$,我们得到:}
dy &= 3x^2\, dx。 \\
\frac{dy}{dx} &= 3x^2。
\end{DPalign*}

因此,$5$ 完全消失了。它没有增加 $x$ 的增长,也不进入微分系数。如果我们用 $7$ 或 $700$ 或其他任何数字代替 $5$,它也会消失。所以如果我们用字母 $a$、$b$ 或 $c$ 表示任何常数,它在微分时会简单地消失。

如果附加的常数是负值,例如 $-5$ 或 $-b$,它同样会消失。

\Subsection{乘以常数}
以这个简单的情况为例:

设 $y = 7x^2$。 \\
然后按照之前的方法进行,我们得到:
\begin{align*}
y + dy &= 7(x+dx)^2 \\
       &= 7\{x^2 + 2x·dx + (dx)^2\} \\
       &= 7x^2 + 14x·dx + 7(dx)^2。 \\
\intertext{然后,减去原式 $y = 7x^2$,并忽略最后一项,我们得到}
dy &= 14x·dx。\\
\frac{dy}{dx} &= 14x。
\end{align*}
\DPPageSep{040.png}{28}%

让我们通过给 $x$ 赋一系列递增值,如 $0$、$1$、$2$、$3$ 等,并找出相应的 $y$ 和 $\dfrac{dy}{dx}$ 的值,来绘制方程 $y = 7x^2$ 和 $\dfrac{dy}{dx} = 14x$ 的图形。

我们将这些值列成表格如下:
\[
\begin{array}{|c||*{6}{r|}|*{3}{r|}}
\hline
\Strut
\Td[c]{x} & \Td[c]{0} & \Td{1} & \Td{2} & \Td{3} & \Td{4} & \Td{5} & \Td{-1} & \Td{-2} & \Td{-3} \\
\hline
\Strut
\Td[c]{y} & \Td[c]{0} & \Td{7} & \Td{28} & \Td{63} & \Td{112} & \Td{175} & \Td{7} & \Td{28} & \Td{63} \\
\hline
\DStrut
\Td[c]{\dfrac{dy}{dx}}
  & \Td[c]{0} & \Td{14} & \Td{28} & \Td{42} & \Td{56} & \Td{70} & \Td{-14} & \Td{-28} & \Td{-42} \\
\hline
\end{array}
\]

%[** TN: Include manually in order to adjust caption heights to match]
\begin{figure}[hbt]
  \centering%
  \makebox[0pt][c]{%
    \raisebox{5pt}{\Graphic[2.5in]{040a}}\qquad\Graphic[2.5in]{040b}%
  }%
  \phantomsection\label{fig:6}\label{fig:6a}%
\end{figure}%

现在以某种方便的比例绘制这些值,我们得到两条曲线,\Figs{6}{和}{6a}。
\DPPageSep{041.png}{29}%

仔细比较这两幅图,并通过检查验证导数曲线 \Fig{6a} 的纵坐标高度与原曲线 \Fig{6} 在相应 $x$ 值处的 \emph{斜率} 成正比。\footnote
  {关于曲线的 \emph{斜率},请参见 \Pageref{slope}。}
在原点左侧,原曲线负斜率(即从左到右向下倾斜),导数曲线的相应纵坐标为负。

\Pagelabel{differ}%
现在如果我们回顾 \Pageref{diffexample},我们会看到简单地微分 $x^2$ 得到 $2x$。因此,$7x^2$ 的微分系数只是 $x^2$ 的微分系数的 7 倍。如果我们取 $8x^2$,微分系数将是 $x^2$ 的微分系数的 8 倍。如果我们设 $y = ax^2$,我们将得到
\[
\frac{dy}{dx} = a × 2x。
\]

如果我们从 $y = ax^n$ 开始,我们将得到
$\dfrac{dy}{dx} = a×nx^{n-1}$。因此,任何仅通过常数乘法的操作,在微分时都会以简单的乘法形式重新出现。而且,关于乘法的规则同样适用于 \emph{除法}:例如,在上面的例子中,如果我们取常数 $\frac{1}{7}$ 而不是 $7$,我们将在微分后的结果中得到相同的 $\frac{1}{7}$。
\DPPageSep{042.png}{30}%

\Examples{一些进一步的例子。}
以下是一些完全展开的进一步例子,
这将使你能够完全掌握
应用于普通代数表达式的微分过程,
并使你能够独立完成
本章末尾给出的例子。

(1) 求导 $y = \dfrac{x^5}{7} - \dfrac{3}{5}$。

$\dfrac{3}{5}$ 是一个附加常数,会消失(见 \Pageref{addconst})。

我们可以立即写出
\begin{DPgather*}
\frac{dy}{dx} = \frac{1}{7} \times 5 \times x^{5-1}, \\
\lintertext{或}
\frac{dy}{dx} = \frac{5}{7} x^4.
\end{DPgather*}

(2) 求导 $y = a\sqrt{x} - \dfrac{1}{2}\sqrt{a}$。

项~$\dfrac{1}{2}\sqrt{a}$ 消失,因为它是一个附加常数;
并且由于~$a\sqrt{x}$ 在指数形式中写作~$ax^{\efrac{1}{2}}$,我们有
\begin{DPgather*}
\frac{dy}{dx}
  = a \times \frac{1}{2} \times x^{\efrac{1}{2}-1}
  =     \frac{a}{2} \times x^{-\efrac{1}{2}}, \\
\lintertext{或}
\frac{dy}{dx} = \frac{a}{2\sqrt{x}}.
\end{DPgather*}

(3) 如果 $ay + bx = by - ax + (x+y)\sqrt{a^2 - b^2}$, \\ %[** TN: 原文中的换行符]
求~$y$ 关于~$x$ 的导数。

通常,这种表达式需要比我们目前所掌握的
更多的知识;
\DPPageSep{043.png}{31}%
然而,总是值得尝试表达式是否可以
简化为更简单的形式。

首先,我们必须尝试将其转化为 $y = {}$ 仅涉及 $x$ 的
某种表达式。

该表达式可以写为
\[
(a-b)y + (a + b)x = (x+y) \sqrt{a^2 - b^2}.
\]

平方后,我们得到%[** TN: 不中断下一个方程]
\[
(a-b)^2 y^2 + (a + b)^2 x^2 + 2(a+b)(a-b)xy = (x^2+y^2+2xy)(a^2-b^2),
\]
简化为
\BindMath{%
\begin{DPalign*}
(a-b)^2y^2 + (a+b)^2 x^2 &= x^2(a^2 - b^2) + y^2(a^2 - b^2); \\
\lintertext{或}
[(a-b)^2 - (a^2 - b^2)]y^2 &= [(a^2 - b^2) - (a+b)^2]x^2, \\
\lintertext{即}
2b(b-a)y^2 &= -2b(b+a)x^2;
\end{DPalign*}
\begin{DPgather*}
\lintertext{因此}
y = \sqrt{\frac{a+b}{a-b}} x \quad\text{且}\quad \frac{dy}{dx} = \sqrt{\frac{a+b}{a-b}}.
\end{DPgather*}}

(4) 半径为~$r$ 和高度为~$h$ 的圆柱体的体积
由公式 $V = \pi r^2 h$ 给出。求当 $r = 5.5$~英寸
且 $h=20$~英寸时,体积随半径变化的速率。如果 $r = h$,
求圆柱体的尺寸,使得半径变化 $1$~英寸导致体积变化 $400$~立方英寸。

体积~$V$ 关于~$r$ 的变化率为
\[
\frac{dV}{dr} = 2 \pi r h.
\]

如果 $r = 5.5$~英寸且 $h=20$~英寸,则变为 $690.8$。
这意味着半径变化 $1$~英寸将导致体积变化 $690.8$~立方英寸。
这可以很容易地验证,因为当 $r = 5$ 和 $r = 6$ 时,
体积分别为 $1570$~立方英寸和 $2260.8$~立方英寸,
且 $2260.8 - 1570 = 690.8$。

此外,如果
\[
r=h,\quad \dfrac{dV}{dr} = 2\pi r^2 = 400\quad \text{且}\quad r = h = \sqrt{\dfrac{400}{2\pi}} = 7.98~\text{英寸}。
\]

(5) Féry 辐射高温计的读数~$\theta$ 与
被观测物体的摄氏温度~$t$ 之间的关系为
\[
\dfrac{\theta}{\theta_1} = \left(\dfrac{t}{t_1}\right)^4,
\]
其中 $\theta_1$ 是对应于已知温度~$t_1$ 的读数。

比较高温计在温度 $800°$~C、$1000°$~C、$1200°$~C 时的灵敏度,
已知当温度为 $1000°$~C 时读数为 $25$。

灵敏度是读数随温度变化的速率,即~$\dfrac{d\theta}{dt}$。公式可以写为
\[
\theta = \dfrac{\theta_1}{t_1^4} t^4 = \dfrac{25t^4}{1000^4},
\]
我们有
\[
\dfrac{d\theta}{dt} = \dfrac{100t^3}{1000^4} = \dfrac{t^3}{10,000,000,000}.
\]

当 $t=800$、$1000$ 和~$1200$ 时,我们得到 $\dfrac{d\theta}{dt} = 0.0512$、$0.1$~和 $0.1728$ 分别。

灵敏度从 $800°$~到 $1000°$ 大约增加了一倍,
并且在 $1200°$~时再次增加了四分之三。
\DPPageSep{045.png}{33}%

\Exercises{第二章练习} (答案见 \Pageref{AnsEx:II}。)



区分以下各项:
\begin{Problems}[2]
\Item{(1)} $y = ax^3 + 6$。
\Item{(2)} $y = 13x^{\efrac{3}{2}} - c$。
\ResetCols{2}

\Item{(3)} $y = 12x^{\efrac{1}{2}} + c^{\efrac{1}{2}}$。
\Item{(4)} $y = c^{\efrac{1}{2}} x^{\efrac{1}{2}}$。
\ResetCols{2}

\Item{(5)} $u = \dfrac{az^n - 1}{c}$。
\Item{(6)} $y = 1.18t^2 + 22.4$。
\end{Problems}

为自己编写一些其他例子,并尝试
自己进行区分。

\begin{Problems}
\Item{(7)} 若 $l_t$ 和 $l_0$ 分别为铁棒在温度 $t°$C 和 $0°$C 时的长度,则
$l_t = l_0(1 + 0.000012t)$。求铁棒每摄氏度的长度变化。

\Item{(8)} 已知若 $c$ 为白炽电灯的烛光强度,$V$ 为电压,
$c = aV^b$,其中 $a$ 和 $b$ 为常数。

求烛光强度随电压变化的速率,并计算在电压为 $80$、$100$ 和 $120$ 伏时,
对于 $a = 0.5×10^{-10}$ 且 $b=6$ 的灯泡,每伏特的烛光强度变化。

\Item{(9)} 一根直径为 $D$、长度为 $L$、比重为 $\sigma$ 的弦,在张力 $T$ 作用下,其振动频率 $n$ 由下式给出:
\[
n = \dfrac{1}{DL} \sqrt{\dfrac{gT}{\pi\sigma}}。
\]

求当 $D$、$L$、$\sigma$ 和 $T$ 分别单独变化时,频率的变化率。

\DPPageSep{046.png}{34}%

\Item{(10)} 管子在不塌陷的情况下所能承受的最大外部压力 $P$ 由下式给出:
\[
  P = \left(\dfrac{2E}{1-\sigma^2}\right) \dfrac{t^3}{D^3},
\]
其中 $E$ 和 $\sigma$ 为常数,$t$ 为管壁厚度,$D$ 为其直径。(此公式假设 $4t$ 相对于 $D$ 较小。)

比较 $P$ 在厚度单独变化和直径单独变化时的小变化率。

\Item{(11)} 从基本原理出发,求以下各项随半径变化的变化率:
\begin{SubProbs}
\item[(\textit{a})] 半径为 $r$ 的圆的周长;

\item[(\textit{b})] 半径为 $r$ 的圆的面积;

\item[(\textit{c})] 斜高为 $l$ 的圆锥的侧面积;

\item[(\textit{d})] 半径为 $r$ 且高为 $h$ 的圆锥的体积;

\item[(\textit{e})] 半径为 $r$ 的球的表面积;

\item[(\textit{f})] 半径为 $r$ 的球的体积。
\end{SubProbs}

\Item{(12)} 铁棒在温度 $T$ 时的长度 $L$ 由 $L = l_t\bigl[1 + 0.000012(T-t)\bigr]$ 给出,其中 $l_t$ 为温度 $t$ 时的长度,求适合套在轮子上的铁箍直径 $D$ 随温度 $T$ 变化的变化率。
\end{Problems}
\DPPageSep{047.png}{35}%

\Chapter[求和、差、积]{第六章}{求和、差、积和商}

\First{我们}已经学会了如何区分简单的代数函数,如 $x^2 + c$ 或 $ax^4$,现在我们需要考虑如何处理两个或更多函数的\emph{和}\Pagelabel{sumdiffer}。

例如,设
\[
y = (x^2+c) + (ax^4+b);
\]
它的 $\dfrac{dy}{dx}$ 是什么?我们该如何处理这个新任务?

这个问题的答案非常简单:只需依次区分它们,如下所示:
\[
\dfrac{dy}{dx} = 2x + 4ax^3。\quad (\textit{答案})
\]

如果你对此有任何疑问,尝试一个更一般的情况,通过基本原理来解决。以下是方法。

设 $y = u+v$,其中 $u$ 是 $x$ 的任意函数,$v$ 是 $x$ 的另一个函数。然后,让 $x$ 增加到 $x+dx$,$y$ 将增加到 $y+dy$;$u$ 将增加到 $u+du$;$v$ 将增加到 $v+dv$。
\DPPageSep{048.png}{36}%

我们将有:
\begin{align*}
  y+dy &= u+du + v+dv。 \\
\intertext{\indent 减去原始的 $y = u+v$,我们得到}
dy &= du+dv, \\
\intertext{并除以 $dx$,我们得到:}
\dfrac{dy}{dx} &= \dfrac{du}{dx} + \dfrac{dv}{dx}。
\end{align*}

这证明了该过程的合理性。你分别对每个函数求导并将结果相加。因此,如果我们采用前一段的例子,并代入两个函数的值,我们将得到,使用所示的符号(\Pageref{section:1}),
\begin{alignat*}{2}
\frac{dy}{dx}
  & = \frac{d(x^2+c)}{dx} &&+ \frac{d(ax^4+b)}{dx} \\
  & = 2x                  &&+ 4ax^3,
\end{alignat*}
与之前完全一致。

如果有三个函数,我们称之为$u$、$v$和$w$,使得
\begin{DPalign*}
y &= u+v+w; \\
\lintertext{那么}
\frac{dy}{dx} &= \frac{du}{dx} + \frac{dv}{dx} + \frac{dw}{dx}.
\end{DPalign*}

至于\emph{减法},它立即得出;因为如果函数$v$本身带有负号,其导数也将为负;因此,通过对
\begin{DPalign*}
y &= u-v, \\
\lintertext{我们应得到}
\frac{dy}{dx} &= \frac{du}{dx} - \frac{dv}{dx}.
\end{DPalign*}
\DPPageSep{049.png}{37}%

但当我们处理\emph{乘积}时,事情就不那么简单了。

假设我们被要求对表达式
\[
y = (x^2+c) × (ax^4+b),
\]
求导,我们该怎么做?结果肯定\emph{不会}是$2x × 4ax^3$;因为很容易看出,$c × ax^4$和$x^2 × b$都不会被包含在这个乘积中。

现在有两种方法可以进行。

\Paragraph{第一种方法。}先进行乘法运算,然后求导。

因此,我们将$x^2 + c$和$ax^4 + b$相乘。

得到$ax^6 + acx^4 + bx^2 + bc$。

现在求导,得到:
\[
\dfrac{dy}{dx} = 6ax^5 + 4acx^3 + 2bx.
\]

\Paragraph{第二种方法。}回到基本原理,考虑方程
\[
y = u × v;
\]
其中$u$是$x$的一个函数,$v$是$x$的另一个函数。然后,如果$x$变为$x+dx$;$y$变为$y+dy$;$u$变为$u+du$,$v$变为$v+dv$,我们将有:
\begin{align*}
 y + dy &= (u + du) × (v + dv) \\
        &= u · v + u · dv + v · du + du · dv.
\end{align*}

现在$du · dv$是一个二阶小量,因此在极限情况下可以忽略,剩下
\[
y + dy = u · v + u · dv + v · du.
\]
\DPPageSep{050.png}{38}%

然后,减去原来的$y = u· v$,我们得到
\[
dy = u · dv + v · du;
\]
并且,两边除以$dx$,我们得到结果:
\[
\dfrac{dy}{dx} = u\, \dfrac{dv}{dx} + v\, \dfrac{du}{dx}.
\]

这表明我们的指导如下:\Pagelabel{differprod}
\emph{要区分两个函数的乘积,将每个函数乘以另一个函数的导数,并将两个结果相加。}

你应该注意到,这个过程相当于以下内容:将$u$视为常数,同时对$v$求导;然后将$v$视为常数,同时对$u$求导;整个导数$\dfrac{dy}{dx}$将是这两种处理方法的和。

现在,既然已经找到了这个规则,将其应用于前面考虑的具体例子。

我们想要对乘积
\[
(x^2 + c) × (ax^4 + b).
\]
求导。

设$(x^2 + c) = u$;$(ax^4 + b) = v$。

然后,根据刚刚建立的一般规则,我们可以写成:
\begin{alignat*}{2}
\dfrac{dy}{dx}
&= (x^2 + c)\, \frac{d(ax^4 + b)}{dx} &&+ (ax^4 + b)\, \frac{d(x^2 + c)}{dx} \\
&= (x^2 + c)\, 4ax^3                  &&+ (ax^4 + b)\, 2x \\
&= 4ax^5 + 4acx^3                     &&+ 2ax^5 + 2bx,  \displaybreak[1] \\
%
\dfrac{dy}{dx}
&= 6ax^5 + 4acx^3                     &&+ 2bx,
\end{alignat*}
与之前完全一致。
\DPPageSep{051.png}{39}%

最后,我们需要对\emph{商}求导。
\SetOddHead{商}

考虑这个例子,$y = \dfrac{bx^5 + c}{x^2 + a}$。在这种情况下,预先进行除法运算是没有用的,因为$x^2 + a$不能整除$bx^5 + c$,它们也没有任何公因数。因此,我们只能回到基本原理,寻找一个规则。
%[** TN: 假装一个新的段落;后续实例不再注明。]
\begin{DPgather*}
\lintertext{\indent 所以我们设}
y = \frac{u}{v};
\end{DPgather*}
其中$u$和$v$是两个不同的自变量$x$的函数。然后,当$x$变为$x + dx$时,$y$将变为$y + dy$;$u$将变为$u + du$;$v$将变为$v + dv$。因此
\[
y + dy = \dfrac{u + du}{v + dv}.
\]

现在进行代数除法,如下:
\begin{align*}
& \TmpColA \;
  \PadTo[l]{\TmpColB}{u + du}
  \begin{tabular}{|c@{}}
    $\dfrac{u}{v} + \dfrac{du}{v} - \dfrac{u· dv}{v^2}\DStrut$\\\hline
  \end{tabular} \\[-2.ex]
& \phantom{\TmpColA}
  \TmpColB \\
& \phantom{\TmpColA \;u + {}}
  \begin{tabular}{@{}l@{}}
    $du - \dfrac{u· dv}{v}$ \\
    $du + \dfrac{du· dv}{v}\DStrut$ \\\hline
  \end{tabular}\\
& \phantom{\TmpColA \;u + du}
  \begin{tabular}{@{}l@{}}
    ${} - \dfrac{u· dv}{v} - \dfrac{du· dv}{v}\DStrut$ \\
    ${} - \dfrac{u· dv}{v} - \dfrac{u· dv· dv}{v^2}\DStrut$ \\\hline
  \end{tabular}\\
& \phantom{\TmpColA \;u + du - \dfrac{u· dv}{v}}
  {} - \dfrac{du· dv}{v} + \dfrac{u· dv· dv}{v^2}.
\end{align*}
\DPPageSep{052.png}{40}%

由于这两个余数都是二阶小量,可以忽略不计,除法可以在此停止,因为任何进一步的余数都会更小。

因此我们得到:
\begin{DPalign*}
y + dy &= \dfrac{u}{v} + \dfrac{du}{v} - \dfrac{u· dv}{v^2}; \\
\intertext{可以写成}
%
&= \dfrac{u}{v} + \dfrac{v· du - u· dv}{v^2}. \\
\intertext{\indent 现在减去原式$y = \dfrac{u}{v}$,我们得到:}
%
dy &= \dfrac{v· du - u· dv}{v^2}; \\
%
\lintertext{由此}
\dfrac{dy}{dx}
&= \dfrac{v\, \dfrac{du}{dx} - u\, \dfrac{dv}{dx}}{v^2}.
\end{DPalign*}

这为我们提供了如何微分两个函数之商的指导:将除数函数乘以被除数函数的导数;然后将除数函数乘以被除数函数的导数;并相减。最后除以除数函数的平方。\SetOddHead{微分}%

回到我们的例子$y = \dfrac{bx^5 + c}{x^2 + a}$,
\begin{DPalign*}
\lintertext{设}
bx^5 + c &= u; \\
\lintertext{和}
x^2  + a &= v.
\end{DPalign*}
\DPPageSep{053.png}{41}%

然后
\begin{align*}
\frac{dy}{dx}
&= \frac{(x^2 + a)\, \dfrac{d(bx^5 + c)}{dx} - (bx^5 + c)\, \dfrac{d(x^2 + a)}{dx}}{(x^2 + a)^2} \\
&= \frac{(x^2 + a)(5bx^4) - (bx^5 + c)(2x)}{(x^2 + a)^2}, \\
\frac{dy}{dx}
&= \frac{3bx^6 + 5abx^4 - 2cx}{(x^2 + a)^2}.\quad\text{(\textit{答案}.)}
\end{align*}

商的运算过程往往繁琐,但并不困难。

以下给出了一些完全展开的进一步例子。\Pagelabel{例子3}

(1) 微分$y = \dfrac{a}{b^2} x^3 - \dfrac{a^2}{b} x + \dfrac{a^2}{b^2}$。

由于$\dfrac{a^2}{b^2}$是常数,其导数为零,
因此我们得到
\[
\frac{dy}{dx} = \frac{a}{b^2} × 3 × x^{3-1} - \frac{a^2}{b} × 1 × x^{1-1}.
\]

但$x^{1-1} = x^0 = 1$;所以得到:
\[
\frac{dy}{dx} = \frac{3a}{b^2} x^2 - \frac{a^2}{b}.
\]

(2) 微分$y = 2a\sqrt{bx^3} - \dfrac{3b \sqrt[3]{a}}{x} - 2\sqrt{ab}$。

将$x$写成指数形式,得到
\[
y = 2a\sqrt{b} x^{\efrac{3}{2}} - 3b \sqrt[3]{a} x^{-1} - 2\sqrt{ab}.
\]

现在
\begin{DPgather*}
\frac{dy}{dx} = 2a\sqrt{b} × \tfrac{3}{2} × x^{\efrac{3}{2}-1} - 3b\sqrt[3]{a} × (-1) × x^{-1-1}; \\
\lintertext{或}
\frac{dy}{dx} = 3a\sqrt{bx} + \frac{3b\sqrt[3]{a}}{x^2}.
\end{DPgather*}
\DPPageSep{054.png}{42}%

(3) 微分$z = 1.8 \sqrt[3]{\dfrac{1}{\theta^2}} - \dfrac{4.4}{\sqrt[5]{\theta}} - 27°$。

这可以写成:$z= 1.8\, \theta^{-\efrac{2}{3}} - 4.4\, \theta^{-\efrac{1}{5}} - 27°$。

$27°$消失,我们得到
\begin{DPgather*}
\frac{dz}{d\theta}
  = 1.8 × -\tfrac{2}{3} × \theta^{-\efrac{2}{3}-1}
  - 4.4 × \left(-\tfrac{1}{5}\right)\theta^{-\efrac{1}{5}-1}; \\
\lintertext{或者,}
\frac{dz}{d\theta}
  = -1.2\, \theta^{-\efrac{5}{3}} + 0.88\, \theta^{-\efrac{6}{5}}; \\
\lintertext{或者,}
\frac{dz}{d\theta} = \frac{0.88}{\sqrt[5]{\theta^6}}
  - \frac{1.2}{\sqrt[3]{\theta^5}}.
\end{DPgather*}

(4) 求导 $v = (3t^2 - 1.2 t + 1)^3$。

直接求导的方法将在后面解释(见 \Pageref{dodge});但我们现在可以毫不费力地处理它。

展开立方,我们得到
\[
v = 27t^6 - 32.4t^5 + 39.96t^4 - 23.328t^3 + 13.32t^2 - 3.6t + 1; %% 修正 "- 13.32t^2" 为 "+ 13.32t^2"
\]
因此
\[
\frac{dv}{dt} = 162t^5 - 162t^4 + 159.84t^3 - 69.984t^2 + 26.64t - 3.6.
\]

(5) 求导 $y = (2x - 3)(x + 1)^2$。
\begin{alignat*}{2}
\frac{dy}{dx}
  &= (2x - 3)\, \frac{d\bigl[(x + 1)(x + 1)\bigr]}{dx}
     &&+ (x + 1)^2\, \frac{d(2x - 3)}{dx} \\
  &= (2x - 3) \left[(x + 1)\, \frac{d(x + 1)}{dx}\right.
     &&+ \left.(x + 1)\, \frac{d(x + 1)}{dx}\right] \\
  &  &&+ (x + 1)^2\, \frac{d(2x - 3)}{dx} \\
% [** TN: 下一行没有自然的第二个对齐点;使用假象]
  &= \rlap{$2(x + 1)\bigl[(2x - 3) + (x + 1)\bigr] = 2(x + 1)(3x - 2)$;}&&
\end{alignat*}
或者,更简单地,先展开再求导。
\DPPageSep{055.png}{43}%

(6) 求导 $y = 0.5 x^3(x-3)$。
\begin{align*}
\frac{dy}{dx}
  &= 0.5\left[x^3 \frac{d(x-3)}{dx} + (x-3) \frac{d(x^3)}{dx}\right] \\
  &= 0.5\left[x^3 + (x-3) × 3x^2\right] = 2x^3 - 4.5x^2.
\end{align*}

与前例相同的注释。

(7) 求导 $w = \left(\theta + \dfrac{1}{\theta}\right)
   \left(\sqrt{\theta} + \dfrac{1}{\sqrt{\theta}}\right)$。

这可以写成
\begin{gather*}
w = (\theta + \theta^{-1})(\theta^{\efrac{1}{2}} + \theta^{-\efrac{1}{2}}). \\
\begin{aligned}
\frac{dw}{d\theta}
  &= (\theta + \theta^{-1})
     \frac{d(\theta^{\efrac{1}{2}} + \theta^{-\efrac{1}{2}})}{d\theta}
   + (\theta^{\efrac{1}{2}} + \theta^{-\efrac{1}{2}})
     \frac{d(\theta+\theta^{-1})}{d\theta} \\
%
  &= (\theta + \theta^{-1})(\tfrac{1}{2}\theta^{-\efrac{1}{2}}
                          - \tfrac{1}{2}\theta^{-\efrac{3}{2}})
   + (\theta^{\efrac{1}{2}} + \theta^{-\efrac{1}{2}})(1 - \theta^{-2}) \\
%
  &= \tfrac{1}{2}(\theta^{ \efrac{1}{2}} + \theta^{-\efrac{3}{2}}
                - \theta^{-\efrac{1}{2}} - \theta^{-\efrac{5}{2}})
   + (\theta^{ \efrac{1}{2}} + \theta^{-\efrac{1}{2}}
    - \theta^{-\efrac{3}{2}} - \theta^{-\efrac{5}{2}}) \\%
%
  &= \tfrac{3}{2} \left(\sqrt{\theta} - \frac{1}{\sqrt{\theta^5}}\right)
   + \tfrac{1}{2} \left(\frac{1}{\sqrt{\theta}} - \frac{1}{\sqrt{\theta^3}}\right).
\end{aligned}
\end{gather*}

同样,这可以通过先乘以两个因子,然后求导来更简单地获得。然而,这并不总是可能的;例如,参见 \Pageref{example1},例 8,其中必须使用求导乘积的规则。

(8) 求导 $y =\dfrac{a}{1 + a\sqrt{x} + a^2x}$。
\begin{align*}
\frac{dy}{dx}
  &= \frac{(1 + ax^{\efrac{1}{2}} + a^2x) × 0 - a\dfrac{d(1 + ax^{\efrac{1}{2}} + a^2x)}{dx}}
          {(1 + a\sqrt{x} + a^2x)^2} \\
  &= - \frac{a(\frac{1}{2}ax^{-\efrac{1}{2}} + a^2)}
            {(1 + ax^{\efrac{1}{2}} + a^2x)^2}.
\end{align*}
\DPPageSep{056.png}{44}%

(9) 求导 $y = \dfrac{x^2}{x^2 + 1}$。
\[
\dfrac{dy}{dx} = \dfrac{(x^2 + 1)\, 2x - x^2 × 2x}{(x^2 + 1)^2} = \dfrac{2x}{(x^2 + 1)^2}.
\]

(10) 求导 $y = \dfrac{a + \sqrt{x}}{a - \sqrt{x}}$。

在指标形式中,\( y = \dfrac{a + x^{\efrac{1}{2}}}{a - x^{\efrac{1}{2}}} \)。
\begin{DPgather*}
\frac{dy}{dx}
  = \frac{(a - x^{\efrac{1}{2}})( \tfrac{1}{2} x^{-\efrac{1}{2}})
        - (a + x^{\efrac{1}{2}})(-\tfrac{1}{2} x^{-\efrac{1}{2}})}
         {(a - x^{\efrac{1}{2}})^2}
  = \frac{ a - x^{\efrac{1}{2}}
         + a + x^{\efrac{1}{2}}}
        {2(a - x^{\efrac{1}{2}})^2\, x^{\efrac{1}{2}}}; \\
\lintertext{\rlap{因此}}
\frac{dy}{dx} = \frac{a}{(a - \sqrt{x})^2\, \sqrt{x}}.
\end{DPgather*}

\begin{DPalign*}
\lintertext{\indent (11) 求导}
\theta &= \frac{1 - a \sqrt[3]{t^2}}{1 + a \sqrt[2]{t^3}}. \\
\lintertext{\indent 现在}
\theta &= \frac{1 - at^{\efrac{2}{3}}}{1 + at^{\efrac{3}{2}}}.
\end{DPalign*}
\begin{align*}
\frac{d\theta}{dt}
  &= \frac{(1 + at^{\efrac{3}{2}}) (-\tfrac{2}{3} at^{-\efrac{1}{3}})
         - (1 - at^{\efrac{2}{3}}) × \tfrac{3}{2} at^{\efrac{1}{2}}}
          {(1 + at^{\efrac{3}{2}})^2} \\
  &= \frac{5a^2 \sqrt[6]{t^7} - \dfrac{4a}{\sqrt[3]{t}} - 9a \sqrt[2]{t}}
          {6(1 + a \sqrt[2]{\DPtypo{3}{t^3}})^2}.
\end{align*}

(12) 一个横截面为正方形的水库,其侧面与垂直方向成 \(45^\circ\) 角倾斜。底边长为 \(200\) 英尺。当水深变化 \(1\) 英尺时,求流入或流出的水量表达式;进而求出当水深从 \(14\) 英尺降至 \(10\) 英尺时,每小时抽出的水量(以加仑计)。

一个高度为 \(H\),底面积为 \(A\) 和 \(a\) 的截头锥体的体积为 \(V = \dfrac{H}{3} (A + a + \sqrt{Aa} )\)。显然,当坡度为 \(45^\circ\) 时,若水深为 \(h\),则水面正方形的边长为 \(200 + 2h\) 英尺,因此水的体积为
\[
\dfrac{h}{3} [200^2 + (200 + 2h)^2 + 200(200 + 2h)]
= 40,000h + 400h^2 + \dfrac{4h^3}{3}.
\]

\(\dfrac{dV}{dh} = 40,000 + 800h + 4h^2 = \) 每英尺深度变化的立方英尺数。从 \(14\) 英尺到 \(10\) 英尺的平均水深为 \(12\) 英尺,当 \(h = 12\) 时,\(\dfrac{dV}{dh} = 50,176\) 立方英尺。

对应于 \(24\) 小时内水深变化 \(4\) 英尺的每小时加仑数为
\[
\dfrac{4 × 50,176 × 6.25}{24} = 52,267 \text{加仑}。
\]

(13) 饱和蒸汽在温度 \(t^\circ\) C 时的绝对压力 \(P\)(以大气压计),根据杜隆的公式,当 \(t\) 高于 \(80^\circ\) 时,\(P = \left( \dfrac{40 + t}{140} \right)^5\)。求压力随温度变化在 \(100^\circ\) C 时的变化率。
\DPPageSep{058.png}{46}%

利用二项式定理展开分子(见 \Pageref{binomtheo})。
\[
P = \frac{1}{140^5} (40^5 + 5×40^4 t + 10 × 40^3 t^2 + 10 × 40^2 t^3
                            + 5 × 40t^4 + t^5);
\]
\CancelMathSkip%[** TN: \BindMath gives bad line spacing after DPalign]
\begin{DPalign*}
\lintertext{因此} \dfrac{dP}{dt} = &\dfrac{1}{537,824 × 10^5}\\
      &(5 × 40^4 + 20 × 40^3 t + 30 × 40^2 t^2 + 20 × 40t^3 + 5t^4),
\end{DPalign*}%
当 \(t = 100\) 时,这变为每摄氏度温度变化 \(0.036\) 大气压。

\Exercises{III}(答案见 \Pageref{AnsEx:III})

\begin{Problems}
\Item{(1)} 求导\Pagelabel{examples2}
\begin{SubProbs}
\item[(\textit{a})] \( u = 1 + x + \dfrac{x^2}{1 × 2} + \dfrac{x^3}{1 × 2 × 3} + \dotsb \)。

\item[(\textit{b})] \( y = ax^2 + bx + c \)。 \hfil (\textit{c}) \( y = (x + a)^2 \)。

\item[(\textit{d})] \( y = (x + a)^3 \)。
\end{SubProbs}

\Item{(2)} 若 \( w = at - \frac{1}{2}bt^2 \),求 \(\dfrac{dw}{dt}\)。

\Item{(3)} 求
\[
y = (x + \sqrt{-1}) × (x - \sqrt{-1})
\]
的导数。

\Item{(4)} 求导
\[
y = (197x - 34x^2) × (7 + 22x - 83x^3)。
\]

\Item{(5)} 若 \( x = (y + 3) × (y + 5) \),求 \(\dfrac{dx}{dy}\)。

\Item{(6)} 求导 \( y = 1.3709x × (112.6 + 45.202x^2) \)。
\end{Problems}
\DPPageSep{059.png}{47}%

求下列函数的导数:
\begin{Problems}[2]
\Item{(7)} \( y = \dfrac{2x + 3}{3x + 2} \)。
\Item{(8)} \( y = \dfrac{1 + x + 2x^2 + 3x^3}{1 + x + 2x^2} \)。
\ResetCols{2}

\Item{(9)} \( y = \dfrac{ax + b}{cx + d} \)。
\Item{(10)} \( y = \dfrac{x^n + a}{x^{-n} + b} \)。
\ResetCols{1}

\Item{(11)} 白炽灯丝的温度~$t$ 与通过灯的电流之间的关系由以下关系式给出:
\[
C = a + bt + ct^2.
\]

求对应于温度变化的电流变化表达式。

\Item{(12)} 以下公式被提出用于表示电阻 $R$ 在温度 $t°$\;C. 时的关系,以及该导线在 $0°$ 摄氏度时的电阻 $R_0$,其中 $a$, $b$, $c$ 为常数。
\begin{align*}
R &= R_0(1 + at + bt^2). \\
R &= R_0(1 + at + b\sqrt{t}). \\
R &= R_0(1 + at + bt^2)^{-1}.
\end{align*}

求每个公式给出的电阻随温度变化的速率。

\Item{(13)} 某种标准电池的电动势~$E$ 随温度~$t$ 的变化关系为:
\[
E = 1.4340 \bigl[1 - 0.000814(t-15)
                 + 0.000007(t-15)^2\bigr] \text{ 伏特}.
\]

求在 $15°$、$20°$ 和~$25°$ 时每度电动势的变化。
\DPPageSep{060.png}{48}%

\Item{(14)} 阿恩顿夫人发现,维持长度为~$l$ 的电弧所需的电动势与电流强度~$i$ 的关系为:
\[
E = a + bl + \frac{c + kl}{i},
\]
其中 $a$, $b$, $c$, $k$ 为常数。

求电动势的变化表达式:
(\textit{a})~关于电弧长度的变化;
(\textit{b})~关于电流强度的变化。
\end{Problems}
\DPPageSep{061.png}{49}%

\Chapter{第七章}{逐次微分法}

\First{让我们}尝试多次重复微分一个函数的效果(参见 \Pageref{function})。
从一个具体的例子开始。

设 $y = x^5$。
\begin{alignat*}{3}
&\text{第一次微分, }  &&5x^4.              &&         \\
&\text{第二次微分, } &&5 × 4x^3           &&= 20x^3. \\
&\text{第三次微分, }  &&5 × 4 × 3x^2       &&= 60x^2. \\
&\text{第四次微分, } &&5 × 4 × 3 × 2x     &&= 120x.  \\
&\text{第五次微分, }  &&5 × 4 × 3 × 2 × 1  &&= 120.   \\
&\text{第六次微分, }  &&                   &&= 0.
\end{alignat*}

有一种我们已经熟悉(参见 \Pageref{notation})的符号,一些作者使用它非常方便。这就是使用一般符号~$f(x)$ 表示任何函数~$x$。这里符号~$f(~)$ 读作“函数”,而不说明是什么特定的函数。因此,陈述 $y=f(x)$ 仅告诉我们 $y$ 是~$x$ 的函数,它可能是 $x^2$ 或 $ax^n$,或 $\cos x$ 或任何其他复杂的~$x$ 函数。

相应的微分系数符号是~$f'(x)$,它比 $\dfrac{dy}{dx}$ 更简洁。这被称为~$x$ 的“导函数”。
\DPPageSep{062.png}{50}%

假设我们再次微分,我们将得到“第二导函数”或第二微分系数,记为~$f''(x)$;依此类推。

现在让我们一般化。

设 $y = f(x) = x^n$。
\begin{DPalign*}[m]
\lintertext{\indent 第一次微分,}     f'(x) &= nx^{n-1}. \\
\lintertext{\indent 第二次微分,}   f''(x) &= n(n-1)x^{n-2}. \\
\lintertext{\indent 第三次微分,}   f'''(x) &= n(n-1)(n-2)x^{n-3}. \\
\lintertext{\indent 第四次微分,} f''''(x) &= n(n-1)(n-2)(n-3)x^{n-4}. \\
    &\llap{\text{等等,}} \text{ 等等。}
\end{DPalign*}

但这种表示逐次微分的方法并非唯一。因为,
\begin{DPalign*}
\lintertext{如果原函数为 }             y &= f(x);  \\
\lintertext{一次微分得到 }  \frac{dy}{dx} &= f'(x); \\
\lintertext{二次微分得到 } \frac{d\left(\dfrac{dy}{dx}\right)}{dx} &= f''(x);
\end{DPalign*}
这更方便地写作~$\dfrac{d^2y}{(dx)^2}$,或更通常地写作~$\dfrac{d^2y}{dx^2}$。同样,我们可以写成三次微分的结果,$\dfrac{d^3y}{dx^3} = f'''(x)$。
\tb
\DPPageSep{063.png}{51}%

\Examples.
现在让我们尝试 $y = f(x) = 7x^4 + 3.5x^3 - \frac{1}{2}x^2 + x - 2$。
\begin{align*}
\frac{dy}{dx}     &= f'(x) = 28x^3 + 10.5x^2 - x + 1, \\
\frac{d^2y}{dx^2} &= f''(x) = 84x^2 + 21x - 1,        \\
\frac{d^3y}{dx^3} &= f'''(x) = 168x + 21,             \\
\frac{d^4y}{dx^4} &= f''''(x) = 168,                  \\
\frac{d^5y}{dx^5} &= f'''''(x) = 0.
\end{align*}
同样地,如果 $y = \phi(x) = 3x(x^2 - 4)$,
\begin{align*}
\phi'(x)    &= \frac{dy}{dx} = 3\bigl[x × 2x + (x^2 - 4) × 1\bigr] = 3(3x^2 - 4), \\
\phi''(x)   &= \frac{d^2y}{dx^2} = 3 × 6x = 18x, \\
\phi'''(x)  &= \frac{d^3y}{dx^3} = 18, \\
\phi''''(x) &= \frac{d^4y}{dx^4} = 0.
\end{align*}

%[** TN: Heading indented in the original]
\Exercises{IV} (答案见 \Pageref[page]{AnsEx:III}。)

求下列表达式的 $\dfrac{dy}{dx}$ 和 $\dfrac{d^2y}{dx^2}$:
\begin{Problems}[2]
\Item{(1)} $y = 17x + 12x^2$.
\Item{(2)} $y = \dfrac{x^2 + a}{x + a}$.
\ResetCols{1}

\Item{(3)} $y = 1 + \dfrac{x}{1} + \dfrac{x^2}{1×2} + \dfrac{x^3}{1×2×3} + \dfrac{x^4}{1×2×3×4}$.

\Item{(4)} 求练习 III (\Pageref{examples2}) 中第 1 至第 7 题,以及例题 (\Pageref{examples3}) 中第 1 至第 7 题的二阶和三阶导函数。
\end{Problems}
\DPPageSep{064.png}{52}%

\Chapter{第八章}{当时间变化时}

\First{一些}微积分中最重要的问题是那些时间作为自变量,而我们必须考虑其他随时间变化的量的值。有些事物随着时间的推移而变大;有些则变小。火车从起点出发的距离随着时间的推移不断增加。树木随着年份的增长而长高。哪种增长得更快;一株12英寸高,一个月后长到14英寸高的植物,还是一棵12英尺高,一年后长到14英尺高的树?

在本章中,我们将大量使用“速率”这个词。与贫民税或水费无关(尽管这里这个词暗示了一种比例——一种比率——每磅多少便士)。甚至与出生率或死亡率无关,尽管这些词暗示了每千人中的出生或死亡人数。当一辆汽车飞驰而过时,我们会说:多么惊人的速度!当一个挥霍无度的人乱花钱时,我们会评论说,那个年轻人过着奢侈的生活。
\DPPageSep{065.png}{53}%
我们所说的“速率”是什么意思?在这两种情况下,我们都在对某件事情的发生与它发生所需的时间长度进行心理上的比较。如果一辆汽车以每秒10码的速度从我们身边飞驰而过,简单的算术会告诉我们,这相当于——在它持续的时间内——每分钟600码,或每小时超过20英里的速度。

现在,在什么意义上,速度为每秒10码与每分钟600码是相同的?10码并不等于600码,一秒也不等于一分钟。我们所说的“速率”相同,是指在两种情况下,所经过的距离与所花费的时间之间的比例是相同的。

再举一个例子。一个人可能只有几英镑的财产,但如果他能在几分钟内以每年百万英镑的速度花钱,他就能做到这一点。假设你用一先令在柜台上买一些商品;假设这个操作持续了正好一秒钟。那么,在这短暂的瞬间,你花钱的速度是每秒1先令,相当于每分钟3英镑,每小时180英镑,每天4320英镑,或每年1,576,800英镑!如果你口袋里有10英镑,你可以在5又1/4分钟内以每年百万英镑的速度花钱。

据说,桑迪到伦敦不到五分钟,“砰”的一声就花掉了六便士。如果他整天都以这个速度花钱,比如说12小时,他每小时将花费6先令,或每天3英镑12先令,或每周21英镑12先令,不包括安息日。

现在试着把这些想法用微分符号表示出来。

在这种情况下,设 \( y \) 代表钱,\( t \) 代表时间。

如果你在花钱,并且在短时间内 \( dt \) 花费的金额称为 \( dy \),那么花钱的速度将是 \( \dfrac{dy}{dt} \),或者更准确地说,应该写成负号,即 \( -\dfrac{dy}{dt} \),因为 \( dy \) 是一个减量,而不是增量。但钱并不是微积分的好例子,因为它通常是跳跃式地进出,而不是连续流动——你可能每年赚200英镑,但它不会整天以细流的方式流入;它每周、每月或每季度以块状形式流入:而你的支出也是以突然支付的形式流出。

一个更恰当的速率概念的例子是由运动物体的速度提供的。从伦敦(尤斯顿车站)到利物浦是200英里。如果一列火车在7点离开伦敦,并在11点到达利物浦,你知道,因为它在4小时内行驶了200英里,它的平均速度一定是每小时50英里;因为 \( \frac{200}{4} = \frac{50}{1} \)。这里,你实际上是在将行驶的距离和花费的时间进行心理比较。你将其中一个除以另一个。如果 \( y \) 是总距离,\( t \) 是总时间,显然平均速度是 \( \dfrac{y}{t} \)。现在,速度实际上并不是全程保持不变的:在开始和旅程结束时减速时,速度较低。可能在某些部分,当行驶下坡时,速度超过了每小时60英里。如果在任何特定的时间元素 \( dt \) 内,相应的行驶距离元素是 \( dy \),那么在旅程的那部分,速度是 \( \dfrac{dy}{dt} \)。因此,一个量(在本例中,距离)相对于另一个量(在本例中,时间)的变化率,应该通过声明一个量相对于另一个量的微分系数来正确表达。科学上表达的速度,是任何给定方向上非常小的距离被覆盖的速率;因此可以写成
\[
v = \dfrac{dy}{dt}.
\]

但如果速度~$v$ 不均匀,则它必定在增加或减少。速度增加的速率称为\emph{加速度}。如果一个运动的物体在某一特定时刻在时间元素~$dt$ 内获得额外的速度~$dv$,则该时刻的加速度~$a$ 可以写为
\[
a = \dfrac{dv}{dt};
\]
\DPPageSep{068.png}{56}%
但 $dv$ 本身是~$d\left( \dfrac{dy}{dt} \right)$。因此我们可以写成
\[
a = \frac{d\left( \dfrac{dy}{dt} \right)}{dt};
\]
这通常写作 $a = \dfrac{d^2y}{dt^2}$; \\
即加速度是距离对时间的二阶导数。加速度表示单位时间内速度的变化,例如,以每秒若干英尺表示;使用的符号为 $\text{英尺} ÷ \text{秒}^2$。

当火车刚开始移动时,其速度~$v$ 很小;但它正在迅速加速——它被引擎的推力加速。因此其 $\dfrac{d^2y}{dt^2}$ 很大。当它达到最高速度时,不再被加速,因此此时 $\dfrac{d^2y}{dt^2}$ 降至零。但当它接近停止点时,速度开始减慢;如果刹车,速度可能会迅速减慢,在\emph{减速}或速度减缓期间,$\dfrac{dv}{dt}$ 的值,即 $\dfrac{d^2y}{dt^2}$ 将为负。

要使质量~$m$ 加速,需要持续施加力。加速质量所需的力与质量成正比,也与所施加的加速度成正比。因此我们可以写出力~$f$ 的表达式
\begin{DPalign*}
f &= ma;\\
\DPPageSep{069.png}{57}%
\lintertext{或} f &= m \frac{dv}{dt}; \\
\lintertext{或} f &= m \frac{d^2y}{dt^2}.
\end{DPalign*}

质量与其运动速度的乘积称为其\emph{动量},符号为~$mv$。如果我们对动量关于时间求导,将得到~$\dfrac{d(mv)}{dt}$,即动量变化率。但由于 $m$ 是一个常数,这可以写成 $m \dfrac{dv}{dt}$,我们上面看到这与~$f$ 相同。也就是说,力可以表示为质量乘以加速度,或动量变化率。

此外,如果一个力用于移动某物(对抗相等且相反的反作用力),它就做了\emph{功};所做功的量度是力与其作用点沿其方向移动的距离的乘积。因此,如果力~$f$ 移动了距离~$y$,则所做的功(我们可以称为~$w$)将是
\[
w = f × y;
\]
这里我们假设 $f$ 是一个恒定的力。如果力在范围~$y$ 的不同部分变化,则我们必须找到其逐点变化的表达式。如果 $f$ 是沿长度元素~$dy$ 的力,则所做的功将是 $f × dy$。但由于 $dy$ 只是一个长度元素,所做的功也只是一个元素。如果我们用 $w$ 表示功,则一个功元素将是~$dw$;我们有
\begin{DPalign*}
dw &= f × dy; \\
\intertext{可以写成}
dw &= ma·dy; \\
\lintertext{或}
dw &= m \frac{d^2y}{dt^2}· dy; \\
\lintertext{或}
dw &= m \frac{dv}{dt}· dy.     \\
\intertext{\indent 进一步,我们可以对表达式进行转置并写成}
\frac{dw}{dy} &= f.
\end{DPalign*}

这为我们提供了\emph{力}的第三个定义;即如果它用于产生任何方向的位移,则该方向上的力等于在该方向上单位长度内所做功的速率。在最后这句话中,\emph{速率}一词显然不是在其时间意义上使用,而是在其比率或比例的意义上使用。

艾萨克·牛顿爵士(与莱布尼茨一起)\DPnote{** [原文如此,应为Leibniz]}
是微积分方法的发明者之一,他认为所有变化的量都是“流动的”;而我们今天称之为导数的比率,
他认为是对应量的流动速率,或称为“流数”。他没有使用$dy$、$dx$和$dt$的记号(这是莱布尼茨的贡献),
而是采用了自己的记号。如果$y$是一个变化的量,或“流动的”量,那么他用$\dot{y}$表示其变化率(或“流数”)。
如果$x$是变量,那么它的流数称为$\dot{x}$。字母上的点表示它已经被求导。然而,这种记号并没有告诉我们
是相对于哪个自变量进行的求导。当我们看到$\dfrac{dy}{dt}$时,我们知道$y$是相对于$t$求导的。
当我们看到$\dfrac{dy}{dx}$时,我们知道$y$是相对于$x$求导的。但如果我们只看到$\dot{y}$,
我们无法在不查看上下文的情况下判断这是指$\dfrac{dy}{dx}$、$\dfrac{dy}{dt}$还是$\dfrac{dy}{dz}$,或者其他变量。
因此,这种流数记号比微分记号提供的信息更少,因此在很大程度上已经不再使用。
但它的简洁性在某些情况下具有优势,特别是当我们约定仅在时间作为自变量时使用它。
在这种情况下,$\dot{y}$将表示$\dfrac{dy}{dt}$,$\dot{u}$将表示$\dfrac{du}{dt}$;而$\ddot{x}$将表示$\dfrac{d^2x}{dt^2}$。

采用这种流数记号,我们可以将上述段落中考虑的力学方程写成如下形式:
\begin{center}
\begin{tabular}{l@{\qquad\qquad}l}
距离     &  $x$\DPtypo{}{,} \\
速度     &  $v = \dot{x}$, \\
加速度   &  $a = \dot{v} = \ddot{x}$, \\
力       &  $f = m\dot{v} = m\ddot{x}$, \\
功       &  $w = x × m \ddot{x}$.
\end{tabular}
\end{center}

\tb
\DPPageSep{072.png}{60}%

\Examples.
(1) 一个物体运动时,它从某点$O$出发所经过的距离$x$(单位:英尺)由关系式$x = 0.2t^2 + 10.4$给出,
其中$t$是自某一时刻起经过的时间(单位:秒)。求物体开始运动后5秒时的速度和加速度,
以及当物体经过100英尺时的相应值。还要求物体在运动前10秒内的平均速度。
(假设距离和向右的运动为正。)
\begin{DPgather*}
\lintertext{\rlap{\indent 现在}}
x = 0.2t^2 + 10.4 \\
v = \dot{x}  = \frac{dx}{dt} = 0.4t;\quad\text{且}\quad
a = \ddot{x} = \frac{d^2x}{dt^2} = 0.4 = \text{常数.}
\end{DPgather*}

当$t = 0$时,$x = 10.4$且$v = 0$。物体从点$O$右侧10.4英尺处开始运动;
时间从物体开始运动的那一刻起计算。

当$t = 5$时,$v = 0.4 × 5 = 2~\text{英尺/秒}$;$a = 0.4~\text{英尺/秒}^2$。

%[ ** 原文中部分方程对齐。]
当$x = 100$时,$100 = 0.2t^2 + 10.4$,即$t^2 = 448$,
且$t = 21.17~\text{秒}$;$v = 0.4 × 21.17 = 8.468~\text{英尺/秒}$。

当$t = 10$时,
\begin{gather*}
\text{行驶距离} = 0.2 × 10^2 + 10.4 - 10.4 = 20~\text{英尺} \\
\text{平均速度} = \tfrac{20}{10} = 2~\text{英尺/秒}
\end{gather*}

(这与区间中点$t = 5$时的速度相同;因为加速度恒定,速度从$t = 0$时的零均匀增加到$t = 10$时的4英尺/秒。)
\DPPageSep{073.png}{61}%

(2) 在上面的题目中,假设
\begin{gather*}
x = 0.2t^2 + 3t + 10.4.\\
v = \dot{x} = \dfrac{dx}{dt} = 0.4t + 3;\quad a = \ddot{x} = \frac{d^2x}{dt^2} = 0.4 = \text{常数}.
\end{gather*}

当 $t = 0$ 时,$x = 10.4$ 且 $v = 3$~英尺/秒,时间从物体经过距离点 $O$ 10.4 英尺的时刻开始计算,此时其速度已为 3 英尺/秒。为了求出物体开始移动以来经过的时间,设 $v = 0$;则 $0.4t + 3 = 0$,$t= -\frac{3}{\DPtypo{4}{.4}} = -7.5$~秒。
物体在时间开始被观测前 7.5 秒开始移动;此后的 5 秒给出 $t = -2.5$ 且 $v = 0.4 × -2.5 + 3 = 2$~英尺/秒。

当 $x = 100$~英尺时,
\[
100 = 0.2t^2 + 3t + 10.4;\quad \text{或 } t^2 + 15t - 448 = 0;
\]
因此 $t = 14.95$~秒,$v = 0.4 × 14.95 + 3 = 8.98$~英尺/秒。

为了求出运动前 10 秒内物体行驶的距离,必须知道物体开始时距离点 $O$ 有多远。

当 $t = -7.5$ 时,
\[
x = 0.2 × (-7.5)^2 - 3 × 7.5 + 10.4 = -0.85~\text{英尺},
\]
即在点 $O$ 左侧 0.85 英尺处。

现在,当 $t = 2.5$ 时,
\[
x = 0.2 × 2.5^2 + 3 × 2.5 + 10.4 = 19.15。
\]

因此,在 10 秒内,行驶的距离为 $19.15 + 0.85 = 20$~英尺,且
\[
\text{平均速度 } = \tfrac{20}{10} = 2 \text{ 英尺/秒}。
\]

(3) 考虑一个类似的问题,当距离由 $x = 0.2t^2 - 3t + 10.4$ 给出时。则 $v = 0.4t - 3$,$a = 0.4 = \text{常数}$。当 $t = 0$ 时,$x = 10.4$ 如前,且
\DPPageSep{074.png}{62}%
$v = -3$;因此物体正在向与其在前述情况中运动方向相反的方向移动。然而,由于加速度为正,我们看到随着时间的推移,速度将减小,直到变为零,此时 $v = 0$ 或 $0.4t - 3 = 0$;即 $t = 7.5$~秒。此后,速度变为正;物体开始运动后 5 秒,$t = 12.5$,且
\[
v = 0.4 × 12.5 - 3 = 2 \text{ 英尺/秒}。
\]

当 $x = 100$ 时,
\begin{DPgather*}
100 = 0.2t^2 - 3t + 10.4,\quad \text{或 } t^2 - 15t - 448 = 0, \\
\lintertext{且}
t = 29.95;\ v = 0.4 × 29.95 - 3 = 8.98~\text{英尺/秒}。
\end{DPgather*}

当 $v$ 为零时,$x = 0.2 × 7.5^2 - 3 × 7.5 + 10.4 = -0.85$,
表明物体在停止前向点 $O$ 后方移动了 0.85 英尺。十秒后
\[
t = 17.5 \text{ 且 } x = 0.2 × 17.5^2 - 3 × 17.5 + 10.4 = 19.15。
\]
$\text{行驶的距离} = .85 + 19.15 = 20.0$,且平均速度再次为 2 英尺/秒。

(4) 考虑另一个类似的问题,其中 $x = 0.2t^3 - 3t^2 + 10.4$;$v = 0.6t^2 - 6t$;$a = 1.2t - 6$。加速度不再是常数。

当 $t = 0$ 时,$x = 10.4$,$v = 0$,$a = -6$。物体处于静止状态,但即将以负加速度开始移动,即向点 $O$ 方向获得速度。

(5) 如果我们有 $x = 0.2t^3 - 3t + 10.4$,则 $v = 0.6t^2 - 3$,且 $a = 1.2t$。

当 $t = 0$ 时,$x = 10.4$;$v = -3$;$a = 0$。

物体正以 3 英尺/秒的速度向点 $O$ 移动,且在那一瞬间速度是均匀的。

我们看到,运动的条件总是可以从时间-距离方程及其一阶和二阶导数中立即确定。在最后两种情况下,前 10 秒内的平均速度和开始后 5 秒的速度将不再相同,因为速度不是均匀增加的,加速度不再是常数。

(6) 轮子转过的角度 $\theta$(以弧度计)由 $\theta = 3 + 2t - 0.1t^3$ 给出,其中 $t$ 是从某一时刻开始的时间(以秒计);求角速度 $\omega$ 和角加速度 $\alpha$,
(\textit{a}) 1 秒后;(\textit{b}) 完成一次旋转后。轮子在何时静止,且到那时为止它完成了多少次旋转?

写成加速度的形式
\[
\omega =  \dot{\theta} = \dfrac{d\theta}{dt} = 2 - 0.3t^2,\quad
\alpha = \ddot{\theta} = \dfrac{d^2\theta}{dt^2} = -0.6t。
\]

当 $t = 0$ 时,$\theta = 3$;$\omega = 2$~弧度/秒;$\alpha = 0$。

当 $t = 1$ 时,
\[
\omega = 2 - 0.3 = 1.7~\text{弧度/秒};\quad \alpha = -0.6~\text{弧度/秒}^2。
\]

这是一个减速过程;车轮正在减速。

经过$1$圈后
\[
\theta = 2\pi = 6.28;\quad 6.28 = 3 + 2t - 0.1t^3.
\]

通过绘制图像,$\theta = 3 + 2t - 0.1t^3$,我们可以得到
$\theta = 6.28$时的$t$值;这些值是$2.11$和$3.03$(还有一个负值)。
\DPPageSep{076.png}{64}%

当$t = 2.11$时,
\begin{gather*}
\theta = 6.28;\quad\omega = 2 - 1.34 = 0.66 \text{ 弧度/秒}; \\
\alpha = -1.27 \text{ 弧度/秒}^2. \\
\intertext{\indent 当$t = 3.03$时,}
\theta = 6.28;\quad \omega = 2 - 2.754 = -0.754 \text{ 弧度/秒}; \\
\alpha = -1.82 \text{ 弧度/秒}^2.
\end{gather*}

速度方向发生了改变。显然,车轮在这两个时刻之间是静止的;当$\omega = 0$时,即当$0 = 2 - 0.3t^3$,或当$t = 2.58~\text{秒}$时,车轮静止,此时它已经完成了
\[
\frac{\theta}{2\pi}
  = \frac{3 + 2 × 2.58 - 0.1 × 2.58^3}{6.28} = 1.025 \text{ 圈}.
\]

\Exercises{V}(答案见\Pageref[page]{AnsEx:V})

\begin{Problems}
\Item{(1)} 若$y = a + bt^2 + ct^4$;求$\dfrac{dy}{dt}$和$\dfrac{d^2y}{dt^2}$。
\[
\text{\textit{答案.} $\dfrac{dy}{dt} = 2bt + 4ct^3$;\quad $\dfrac{d^2y}{dt^2} = 2b + 12ct^2$.}
\]

\Item{(2)} 一个在太空中自由下落的物体在$t$秒内描述的空间$s$(以英尺为单位)由方程$s = 16t^2$表示。绘制一条曲线显示$s$与$t$之间的关系。并确定物体在以下时刻的速度:$t = 2$秒;$t = 4.6$秒;$t = 0.01$秒。

\Item{(3)} 若$x = at - \frac{1}{2}gt^2$;求$\dot{x}$和$\ddot{x}$。

\Item{(4)} 若一个物体按照规律
\[
s = 12 - 4.5t + 6.2t^2,
\]
求$t = 4$秒时的速度;$s$以英尺为单位。
\DPPageSep{077.png}{65}%

\Item{(5)} 求上例中物体的加速度。加速度对于所有$t$值是否相同?

\Item{(6)} 一个旋转的轮子转过的角度$\theta$(以弧度为单位)与自开始以来经过的时间$t$(以秒为单位)由规律
\[
\theta = 2.1 - 3.2t + 4.8t^2.
\]

求$1\frac{1}{2}$秒后该轮子的角速度(以弧度每秒为单位)。并求其角加速度。

\Item{(7)} 一个滑块在运动过程中,其距离$s$(以英寸为单位)从起点开始由表达式
\[
s = 6.8t^3 - 10.8t;\quad\text{$t$以秒为单位}。
\]

求任意时刻的速度和加速度的表达式;并由此求$3$秒后的速度和加速度。

\Item{(8)} 一个上升的气球的运动规律使得其高度$h$(以英里为单位)在任意时刻由表达式$h = 0.5 + \frac{1}{10}\sqrt[3]{t-125}$给出;$t$以秒为单位。

求任意时刻的速度和加速度的表达式。绘制曲线显示上升前十分钟内高度、速度和加速度的变化。

\Item{(9)} 一块石头被向下扔入水中,其深度$p$(以米为单位)在到达水面后的任意时刻$t$秒由表达式
\[
p = \frac{4}{4+t^2} + 0.8t - 1.
\]
\DPPageSep{078.png}{66}%

求任意时刻的速度和加速度的表达式。求$10$秒后的速度和加速度。

\Item{(10)} 一个物体以这样的方式运动,使得从开始运动的时间$t$内描述的空间由$s = t^n$给出,其中$n$为常数。当速度从第$5$秒到第$10$秒翻倍时,求$n$的值;并求当第$10$秒末速度在数值上等于加速度时的$n$值。
\end{Problems}
\DPPageSep{079.png}{67}%

\Chapter{第九章}{引入一个有用的技巧}

\First{有时},人们会发现自己被一个过于复杂、难以直接处理的表达式所困扰。

例如,方程\Pagelabel{dodge}
\[
y = (x^2+a^2)^{\efrac{3}{2}}
\]
对于初学者来说显得棘手。

现在,解决这个难题的方法是这样的:用一个符号,比如~$u$,来表示表达式 $x^2 + a^2$;
那么方程变为
\[
y = u^{\efrac{3}{2}},
\]
这对你来说很容易处理;因为
\[
\frac{dy}{du} = \frac{3}{2} u^{\efrac{1}{2}}.
\]
然后处理表达式
\[
u = x^2 + a^2,
\]
并对其关于~$x$求导,
\[
\frac{du}{dx} = 2x.
\]
\DPPageSep{080.png}{68}%
然后剩下的就是简单的计算了;
\begin{DPalign*}
\lintertext{因为}
\frac{dy}{dx} &= \frac{dy}{du} × \frac{du}{dx}; \\
\lintertext{即}
\frac{dy}{dx}
  &= \frac{3}{2} u^{\efrac{1}{2}} × 2x \\
  &= \tfrac{3}{2} (x^2 + a^2)^{\efrac{1}{2}} × 2x \\
  &= 3x(x^2 + a^2)^{\efrac{1}{2}};
\end{DPalign*}
于是这个技巧就完成了。

不久之后,\DPnote{** TN: [sic], 古拼写}当你学会了如何处理正弦、余弦和指数函数时,你会发现这个技巧越来越有用。

\tb

\Examples.
让我们通过几个例子来练习这个技巧。

\Pagelabel{ExNo1}%
(1) 求 $y = \sqrt{a+x}$ 的导数。

设 $a+x = u$。
\begin{align*}
\frac{du}{dx} &= 1;\quad y=u^{\efrac{1}{2}};\quad
  \frac{dy}{du} = \tfrac{1}{2} u^{-\efrac{1}{2}}
                = \tfrac{1}{2} (a+x)^{-\efrac{1}{2}}.\\
%
\frac{dy}{dx} &= \frac{dy}{du} × \frac{du}{dx} = \frac{1}{2\sqrt{a+x}}.
\end{align*}

(2) 求 $y = \dfrac{1}{\sqrt{a+x^2}}$ 的导数。

设 $a + x^2 = u$。
\begin{align*}
\frac{du}{dx} &= 2x;\quad y=u^{-\efrac{1}{2}};\quad
  \frac{dy}{du} = -\tfrac{1}{2}u^{-\efrac{3}{2}}.\\
%
\frac{dy}{dx} &= \frac{dy}{du}×\frac{du}{dx} = - \frac{x}{\sqrt{(a+x^2)^3}}.
\end{align*}
\DPPageSep{081.png}{69}%

(3) 求 $y = \left(m - nx^{\efrac{2}{3}} + \dfrac{p}{x^{\efrac{4}{3}}}\right)^a$ 的导数。

设 $m - nx^{\efrac{2}{3}} + px^{-\efrac{4}{3}} = u$。
\begin{gather*}
\frac{du}{dx} = -\tfrac{2}{3} nx^{-\efrac{1}{3}} - \tfrac{4}{3} px^{-\efrac{7}{3}};\\
%
y = u^a;\quad \frac{dy}{du} = a u^{a-1}. \\
%
\frac{dy}{dx} = \frac{dy}{du}×\frac{du}{dx}
  = -a\left(m -nx^{\efrac{2}{3}} + \frac{p}{x^{\efrac{4}{3}}}\right)^{a-1}
     (\tfrac{2}{3} nx^{-\efrac{1}{3}} + \tfrac{4}{3} px^{-\efrac{7}{3}}).
\end{gather*}

(4) 求 $y=\dfrac{1}{\sqrt{x^3 - a^2}}$ 的导数。

设 $u = x^3 - a^2$。
\begin{align*}
\frac{du}{dx} &= 3x^2;\quad y = u^{-\efrac{1}{2}};\quad
  \frac{dy}{du}=-\frac{1}{2}(x^3 - a^2)^{-\efrac{3}{2}}. \\
\frac{dy}{dx} &= \frac{dy}{du} × \frac{du}{dx} = -\frac{3x^2}{2\sqrt{(x^3 - a^2)^3}}.
\end{align*}

\Pagelabel{examples4}
(5) 求 $y=\sqrt{\dfrac{1-x}{1+x}}$ 的导数。

将此写为 $y=\dfrac{(1-x)^{\efrac{1}{2}}}{(1+x)^{\efrac{1}{2}}}$。
\[
\frac{dy}{dx}
  = \frac{(1+x)^{\efrac{1}{2}}\, \dfrac{d(1-x)^{\efrac{1}{2}}}{dx}
        - (1-x)^{\efrac{1}{2}}\, \dfrac{d(1+x)^{\efrac{1}{2}}}{dx}}{1+x}.
\]

(我们也可以写成 $y = (1-x)^{\efrac{1}{2}} (1+x)^{-\efrac{1}{2}}$ 并作为乘积求导。)
\DPPageSep{082.png}{70}%

按照上面\DPtypo{exercise}{example}~(1) 的方法进行,我们得到
\[
\frac{d(1-x)^{\efrac{1}{2}}}{dx} = -\frac{1}{2\sqrt{1-x}};
\quad\text{且}\quad
\frac{d(1+x)^{\efrac{1}{2}}}{dx} = \frac{1}{2\sqrt{1+x}}.
\]

因此
\begin{DPalign*}
\frac{dy}{dx}
  &= - \frac{(1 + x)^{\efrac{1}{2}}}{2(1 + x)\sqrt{1-x}}
     - \frac{(1 - x)^{\efrac{1}{2}}}{2(1 + x)\sqrt{1+x}} \\
  &= - \frac{1}{2\sqrt{1+x}\sqrt{1-x}} - \frac{\sqrt{1-x}}{2 \sqrt{(1+x)^3}};\\
\lintertext{或}
\frac{dy}{dx}
  &= - \frac{1}{(1+x)\sqrt{1-x^2}}.
\end{DPalign*}

(6) 求 $y = \sqrt{\dfrac{x^3}{1+x^2}}$ 的导数。

我们可以将其写为
\begin{gather*}
y = x^{\efrac{3}{2}}(1+x^2)^{-\efrac{1}{2}}; \\
\frac{dy}{dx}
  = \tfrac{3}{2} x^{\efrac{1}{2}}(1 + x^2)^{-\efrac{1}{2}}
  + x^{\efrac{3}{2}} × \frac{d\bigl[(1+x^2)^{-\efrac{1}{2}}\bigr]}{dx}.
\end{gather*}

按照上面\DPtypo{exercise}{example}~(2) 的方法,对 $(1+x^2)^{-\efrac{1}{2}}$ 求导,我们得到
\[
\frac{d\bigl[(1+x^2)^{-\efrac{1}{2}}\bigr]}{dx} = - \frac{x}{\sqrt{(1+x^2)^3}};
\]
因此
\[
\frac{dy}{dx}
  = \frac{3\sqrt{x}}{2\sqrt{1+x^2}} - \frac{\sqrt{x^5}}{\sqrt{(1+x^2)^3}}
  = \frac{\sqrt{x}(3+x^2)}{2\sqrt{(1+x^2)^3}}.
\]
\DPPageSep{083.png}{71}%

(7) 求导 $y=(x+\sqrt{x^2+x+a})^3$。

设 $x+\sqrt{x^2+x+a}=u$。
\begin{gather*}
\frac{du}{dx} = 1 + \frac{d\bigl[(x^2+x+a)^{\efrac{1}{2}}\bigr]}{dx}. \\
y = u^3;\quad\text{且}\quad \frac{dy}{du} = 3u^2= 3\left(x+\sqrt{x^2+x+a}\right)^2.
\end{gather*}

现在设 $(x^2+x+a)^{\efrac{1}{2}}=v$ 且 $(x^2+x+a) = w$。
\begin{DPalign*}
\frac{dw}{dx}
  &= 2x+1;\quad v = w^{\efrac{1}{2}};\quad \frac{dv}{dw} = \tfrac{1}{2}w^{-\efrac{1}{2}}. \\
\frac{dv}{dx}
  &= \frac{dv}{dw} × \frac{dw}{dx} = \tfrac{1}{2}(x^2+x+a)^{-\efrac{1}{2}}(2x+1). \\
\lintertext{\indent 因此}
\frac{du}{dx}
  &= 1 + \frac{2x+1}{2\sqrt{x^2+x+a}}, \\
\frac{dy}{dx}
  &= \frac{dy}{du} × \frac{du}{dx}\\
  &= 3\left(x+\sqrt{x^2+x+a}\right)^2
      \left(1 +\frac{2x+1}{2\sqrt{x^2+x+a}}\right).
\end{DPalign*}

(8) 求导 $y=\sqrt{\dfrac{a^2+x^2}{a^2-x^2}} \sqrt[3]{\dfrac{a^2-x^2}{a^2+x^2}}$。

我们得到
\begin{align*}
y &= \frac{(a^2+x^2)^{\efrac{1}{2}} (a^2-x^2)^{\efrac{1}{3}}}
          {(a^2-x^2)^{\efrac{1}{2}} (a^2+x^2)^{\efrac{1}{3}}}
  = (a^2+x^2)^{\efrac{1}{6}} (a^2-x^2)^{-\efrac{1}{6}}. \\
\frac{dy}{dx}
  &= (a^2+x^2)^{\efrac{1}{6}} \frac{d\bigl[(a^2-x^2)^{-\efrac{1}{6}}\bigr]}{dx}
   + \frac{d\bigl[(a^2+x^2)^{\efrac{1}{6}}\bigr]}{(a^2-x^2)^{\efrac{1}{6}}\, dx}.
\end{align*}
\DPPageSep{084.png}{72}%

设 $u = (a^2-x^2)^{-\efrac{1}{6}}$ 且 $v = (a^2 - x^2)$。
\begin{align*}
u &= v^{-\efrac{1}{6}};\quad
  \frac{du}{dv} = -\frac{1}{6}v^{-\efrac{7}{6}};\quad
  \frac{dv}{dx} = -2x. \\
%
\frac{du}{dx} &= \frac{du}{dv} × \frac{dv}{dx} = \frac{1}{3}x(a^2-x^2)^{-\efrac{7}{6}}.
\end{align*}

设 $w = (a^2 + x^2)^{\efrac{1}{6}}$ 且 $z = (a^2 + x^2)$。
\begin{align*}
w &= z^{\efrac{1}{6}};\quad
  \frac{dw}{dz} = \frac{1}{6}z^{-\efrac{5}{6}};\quad
  \frac{dz}{dx} = 2x. \\
%
\frac{dw}{dx} &= \frac{dw}{dz} × \frac{dz}{dx} = \frac{1}{3} x(a^2 + x^2)^{-\efrac{5}{6}}.
\end{align*}

因此
\begin{DPalign*}
\frac{dy}{dx}
  &= (a^2+x^2)^{\efrac{1}{6}} \frac{x}{3(a^2-x^2)^{\efrac{7}{6}}}
   + \frac{x}{3(a^2-x^2)^{\efrac{1}{6}} (a^2+x^2)^{\efrac{5}{6}}}; \\
%
\lintertext{或}
\frac{dy}{dx}
  &= \frac{x}{3}
     \left[\sqrt[6]{\frac{a^2+x^2}{(a^2-x^2)^7}}
           + \frac{1}{\sqrt[6]{(a^2-x^2)(a^2+x^2)^5]}} \right].
\end{DPalign*}

(9) 对 $y^n$ 关于 $y^5$ 求导。
\[
\frac{d(y^n)}{d(y^5)} = \frac{ny^{n-1}}{5y^{5-1}} = \frac{n}{5} y^{n-5}.
\]

%[** TN: Manual linebreak improves typeset appearance]
(10)\Pagelabel{Example10} 求 $y = \dfrac{x}{b} \sqrt{(a-x)x}$ 的一阶和二阶导数。
\[
\frac{dy}{dx}
  = \frac{x}{b}\,
    \frac{d\bigl\{\bigl[(a-x)x\bigr]^{\efrac{1}{2}}\bigr\}}{dx}
  + \frac{\sqrt{(a-x)x}}{b}.
\]

设 $\bigl[(a-x)x\bigr]^{\efrac{1}{2}} = u$ 且设 $(a-x)x = w$;则 $u = w^{\efrac{1}{2}}$。
\BindMath{\[
\frac{du}{dw}
  = \frac{1}{2} w^{-\efrac{1}{2}}
  = \frac{1}{2w^{\efrac{1}{2}}} = \frac{1}{2\sqrt{(a-x)x}}.
\]
\DPPageSep{085.png}{73}%
\begin{align*}
&\frac{dw}{dx} = a-2x.\\
&\frac{du}{dw} × \frac{dw}{dx} = \frac{du}{dx} = \frac{a-2x}{2\sqrt{(a-x)x}}.
\end{align*}}%

因此
\[
\frac{dy}{dx}
  = \frac{x(a-2x)}{2b\sqrt{(a-x)x}} + \frac{\sqrt{(a-x)x}}{b}
  = \frac{x(3a-4x)}{2b\sqrt{(a-x)x}}.
\]

现在
\begin{align*}
\frac{d^2y}{dx^2}
  &= \frac{2b \sqrt{(a-x)x}\, (3a-8x)
           - \dfrac{(3ax-4x^2)b(a-2x)}{\sqrt{(a-x)x}}}
          {4b^2(a-x)x} \\
  &= \frac{3a^2-12ax+8x^2}{4b(a-x)\sqrt{(a-x)x}}.
\end{align*}

(我们将在后面需要这两个最后的导数。参见 \hyperref[Ex:X11]{习题 X. 第 11 题}。)

\Exercises{VI} (答案见 \Pageref[page]{AnsEx:VI}。)

求下列函数的导数:

\begin{Problems}[2]
\Item{(1)} $y = \sqrt{x^2 + 1}$。
\Item{(2)} $y = \sqrt{x^2+a^2}$。
\ResetCols{2}

\Item{(3)} $y = \dfrac{1}{\sqrt{a+x}}$。
\Item{(4)} $y = \dfrac{a}{\sqrt{a-x^2}}$。
\ResetCols{2}

\Item{(5)} $y = \dfrac{\sqrt{x^2-a^2}}{x^2}$。
\Item{(6)} $y = \dfrac{\sqrt[3]{x^4+a}}{\sqrt[2]{x^3+a}}$。
\ResetCols{1}

\Item{(7)} $y = \dfrac{a^2+x^2}{(a+x)^2}$。
\DPPageSep{086.png}{74}%

\Item{(8)} 对 $y^5$ 关于 $y^2$ 求导。

\Item{(9)} 求导 $y = \dfrac{\sqrt{1 - \theta^2}}{1 - \theta}$。
\end{Problems}
\tb

该过程可以扩展到三个或更多微分系数,因此 $\dfrac{dy}{dx} = \dfrac{dy}{dz} × \dfrac{dz}{dv} × \dfrac{dv}{dx}$。

\Examples.
(1) 若 $z = 3x^4$;\quad $v = \dfrac{7}{z^2}$;\quad $y =\sqrt{1+v}$,求~$\dfrac{dv}{dx}$。

我们有
\begin{align*}
\frac{dy}{dv} &= \frac{1}{2\sqrt{1+v}};\quad
\frac{dv}{dz} = -\frac{14}{z^3};\quad
\frac{dz}{dx} = 12x^3。 \\
%
\frac{dy}{dx} &= -\frac{168x^3}{(2\sqrt{1+v})z^3}
               = -\frac{28}{3x^5\sqrt{9x^8+7}}。
\end{align*}

(2) 若 $t = \dfrac{1}{5\sqrt{\theta}}$;\quad $x = t^3 + \dfrac{t}{2}$;\quad $v = \dfrac{7x^2}{\sqrt[3]{x-1}}$,求~$\dfrac{dv}{d\theta}$。
\begin{DPgather*}
\frac{dv}{dx} = \frac{7x(5x-6)}{3\sqrt[3]{(x-1)^4}};\quad
\frac{dx}{dt} = 3t^2 + \tfrac{1}{2};\quad
\frac{dt}{d\theta} = -\frac{1}{10\sqrt{\theta^3}}。 \\
\lintertext{因此}
\frac{dv}{d\theta}
  = -\frac{7x(5x-6)(3t^2+\frac{1}{2})}
          {30\sqrt[3]{(x-1)^4} \sqrt{\theta^3}},
\end{DPgather*}
其中 $x$ 必须以其值替换,$t$ 以其关于~$\theta$ 的值替换。

(3) 若 $\theta = \dfrac{3a^2x}{\sqrt{x^3}}$;\quad $\omega = \dfrac{\sqrt{1-\theta^2}}{1+\theta}$;\quad 且 $\phi = \sqrt{3} - \dfrac{1}{\omega\sqrt{2}}$,
求~$\dfrac{d\phi}{dx}$。
\DPPageSep{087.png}{75}%

我们得到
\begin{gather*}
\theta = 3a^2x^{-\efrac{1}{2}};\quad
\omega = \sqrt{\frac{1-\theta}{1+\theta}};\quad \text{且}\quad
\phi = \sqrt{3} \DPtypo{=}{-} \frac{1}{\sqrt{2}} \omega^{-1}。 \\
\frac{d\theta}{dx} = -\frac{3a^2}{2\sqrt{x^3}};\quad
\frac{d\omega}{d\theta} = -\frac{1}{(1+\theta)\sqrt{1-\theta^2}}
\end{gather*}
(见例 5,\Pageref{examples4});且
\[
\frac{d\phi}{d\omega} = \frac{1}{\sqrt{2}\omega^2}。
\]

因此 $\dfrac{d\theta}{dx} = \dfrac{1}{\sqrt{2} × \omega^2}
  × \dfrac{1}{(1+\theta) \sqrt{1-\theta^2}}
  × \dfrac{3a^2}{2\sqrt{x^3}}$。

现在首先替换~$\omega$,然后替换~$\theta$ 以其值。

\Exercises{VII}
%[ ** TN: 原文格式不一致;标题单独一行]
你现在可以成功尝试以下题目。(见\Pageref[page]{AnsEx:VII} 获取答案。)
\begin{Problems}
\Item{(1)} 若 $u = \frac{1}{2}x^3$;\quad $v = 3(u+u^2)$;\quad 且 $w = \dfrac{1}{v^2}$,求~$\dfrac{dw}{dx}$。

\Item{(2)} 若 $y = 3x^2 + \sqrt{2}$;\quad $z = \sqrt{1+y}$;\quad 且 $v = \dfrac{1}{\sqrt{3}+4z}$,
求~$\dfrac{dv}{dx}$。

\Item{(3)} 若 $y = \dfrac{x^3}{\sqrt{3}}$;\quad $z = (1+y)^2$;\quad 且 $u = \dfrac{1}{\sqrt{1+z}}$,求~$\dfrac{du}{dx}$。
\end{Problems}
\DPPageSep{088.png}{76}%

\Chapter[微分意义]{第十章}{微分的几何意义}

\First{首先},考虑微分系数的几何意义是有益的。

首先,任何关于~$x$ 的函数,例如~$x^2$、~$\sqrt{x}$ 或~$ax+b$,都可以绘制成曲线;如今,每个学生都熟悉绘制曲线的过程。

\Figure{088a}{7}

设在图7中,$PQR$ 为一条相对于坐标轴 $OX$ 和 $OY$ 绘制的曲线的一部分。考虑曲线上任意一点 $Q$,其横坐标为 $x$,纵坐标为 $y$。现在观察当 $x$ 变化时,$y$ 如何变化。如果 $x$ 向右增加一个微小的增量 $dx$,可以观察到 $y$ 也会(在**这条**特定的曲线上)增加一个微小的增量 $dy$(因为这条特定的曲线恰好是一条**上升**曲线)。那么,$dy$ 与 $dx$ 的比值是衡量曲线在两点 $Q$ 和 $T$ 之间上升程度的指标。事实上,从图中可以看出,曲线在 $Q$ 和 $T$ 之间有多个不同的斜率,因此我们很难准确地谈论曲线在 $Q$ 和 $T$ 之间的斜率。然而,如果 $Q$ 和 $T$ 非常接近,以至于曲线的小部分 $QT$ 几乎是一条直线,那么可以说比值 $\dfrac{dy}{dx}$ 是曲线沿 $QT$ 的斜率。直线 $QT$ 向两侧延伸,仅在 $QT$ 部分与曲线接触,如果这部分无限小,直线将几乎只在一个点接触曲线,因此成为曲线的**切线**。

这条切线显然与 $QT$ 具有相同的斜率,因此 $\dfrac{dy}{dx}$ 是曲线在点 $Q$ 处的切线的斜率,其中 $\dfrac{dy}{dx}$ 的值被求出。

我们已经看到,简短的表达“曲线的斜率”没有精确的含义,因为曲线有多个斜率——事实上,曲线的每一小部分都有不同的斜率。然而,“曲线在某点的斜率”是一个完全定义明确的概念;它是曲线在该点附近非常小部分的斜率;我们已经看到,这与“曲线在该点的切线的斜率”相同。

注意,$dx$ 是向右的一个小步,$dy$ 是相应向上的小步。这些步长必须被视为尽可能短——实际上是无限短——尽管在图中我们必须用非无穷小的部分来表示它们,否则它们将无法被看到。

**我们今后将大量利用这一事实,即 $\dfrac{dy}{dx}$ 表示曲线在任意点的斜率。**

\Figure{090a}{8}

如果一条曲线在某点以 $45°$ 的斜率上升,如图8所示,$dy$ 和 $dx$ 将相等,且 $\dfrac{dy}{dx} = 1$。
\DPPageSep{091.png}{79}%

如果曲线以比 $45°$ 更陡的斜率上升(如图9所示),$\dfrac{dy}{dx}$ 将大于 $1$。
\Figures{091a}{091b}{9}{10}

如果曲线上升得非常平缓,如图10所示,$\dfrac{dy}{dx}$ 将是一个小于 $1$ 的分数。

对于一条水平线,或曲线中的水平部分,$dy=0$,因此 $\dfrac{dy}{dx}=0$。
\Figure{091c}{11}

如果一条曲线**向下**倾斜,如图11所示,$dy$ 将是一个向下的步长,因此必须被视为负值;因此 $\dfrac{dy}{dx}$ 也将具有负号。

如果“曲线”恰好是一条直线,如图12所示,$\dfrac{dy}{dx}$ 的值将在其上的所有点保持不变。换句话说,它的**斜率**是恒定的。

\Figure{092a}{12}

如果一条曲线随着向右延伸而变得更加向上弯曲,$\dfrac{dy}{dx}$ 的值将随着斜率的增加而越来越大,如图13所示。
\Figure{092b}{13}
\DPPageSep{093.png}{81}%

如果一条曲线随着向右延伸而变得越来越平缓,$\dfrac{dy}{dx}$ 的值将随着平缓部分的到达而变得越来越小,如图14所示。

\Figures{093a}{093b}{14}{15}

如果一条曲线首先下降,然后再次上升,
如图\Fig{15}所示,呈现向上凹的形状,那么
显然$\dfrac{dy}{dx}$将首先为负,随着曲线变平,其值逐渐减小,然后在曲线底部到达最低点时为零;
从这一点开始,$\dfrac{dy}{dx}$将具有正值,并持续增加。在这种情况下,我们称$y$经过一个\emph{最小值}。$y$的最小值并不一定是$y$的最小值,
它是与曲线底部相对应的$y$值;例如,在图\Fig{28}(\Pageref{fig:28})中,与曲线底部对应的$y$值为$1$,而$y$在其他地方的值则小于此值。最小值的特征是$y$在其两侧都必须增加。
\DPPageSep{094.png}{82}%

\NB---对于使$y$为\emph{最小值}的特定$x$值,$\dfrac{dy}{dx} = 0$。

如果一条曲线首先上升然后下降,$\dfrac{dy}{dx}$的值将首先为正;然后在到达顶点时为零;然后为负,随着曲线向下倾斜,如图\Fig{16}所示。在这种情况下,我们称$y$经过一个\emph{最大值},但$y$的最大值并不一定是$y$的最大值。在图\Fig{28}中,$y$的最大值为$2\frac{1}{3}$,但这绝不是$y$在曲线其他点的最大值。

\Figures{094a}{094b}{16}{17}

\NB---对于使$y$为\emph{最大值}的特定$x$值,$\dfrac{dy}{dx}= 0$。

如果一条曲线具有图\Fig{17}所示的特殊形状,$\dfrac{dy}{dx}$的值将始终为正;但会有一个特定位置,其斜率最不陡峭,$\dfrac{dy}{dx}$的值将是最小值;即,小于曲线其他部分的任何值。
\DPPageSep{095.png}{83}%

如果一条曲线具有图\Fig{18}所示的形状,$\dfrac{dy}{dx}$的值在上部为负,在下部为正;而在曲线鼻部实际垂直的位置,$\dfrac{dy}{dx}$的值将无限大。

\Figure{095a}{18}

现在我们理解了$\dfrac{dy}{dx}$在任何点上测量曲线陡峭度的含义,让我们转向我们已经学会如何微分的一些方程。

(1)作为最简单的例子,取这个方程:
\[
y=x+b.
\]

它在图\Fig{19}中绘制出来,使用$x$和$y$的相同刻度。如果我们设$x = 0$,那么对应的纵坐标将是$y = b$;也就是说,“曲线”在高度$b$处与$y$轴相交。从这里开始,它以$45°$的斜率上升;因为无论我们给$x$赋予什么正值,$y$都会相应上升。这条线的斜率为$1$比$1$。

现在根据我们已经学过的规则(第\pageref{diffrule1}和\pageref{diffrule2}页)对$y = x+b$进行微分,我们得到$\dfrac{dy}{dx} = 1$。

这条线的斜率是这样的:每向右移动一小步$dx$,我们就向上移动相等的一小步$dy$。并且这个斜率是恒定的——始终保持相同的斜率。

(2)再举一个例子:\Pagelabel{Case2}
\[
y = ax+b.
\]
我们知道这条曲线与前一条一样,将从$y$轴上的高度$b$开始。但在绘制曲线之前,让我们通过微分来求其斜率;得到$\dfrac{dy}{dx} = a$。斜率将是恒定的,其角度正切值在这里称为$a$。让我们给$a$赋予一些数值——比如说$\frac{1}{3}$。那么我们必须给它一个$1$比$3$的斜率;即$dx$将是$dy$的三倍;如图\Fig{21}中放大所示。因此,在图\Fig{20}中以这个斜率绘制这条线。

\Figure[1.75in]{097a}{21}
(3)现在来看一个稍微难一点的例子。
\begin{DPalign*}
\lintertext{\indent 设}
y= ax^2 + b.
\end{DPalign*}

同样,曲线将从$y$轴上的高度$b$开始,高于原点。

现在进行求导。[如果你忘记了,翻回到\Pageref{diffrule2};或者,更确切地说,\emph{不要}翻回去,而是思考一下如何进行求导。]
\[
\frac{dy}{dx} = 2ax.
\]

\Figure[3.25in]{097b}{22}

这表明陡度并不是恒定的:随着$x$的增加,陡度也会增加。在起点$P$,
\DPPageSep{098.png}{86}%
当$x = 0$时,曲线(\Fig{22})没有陡度——也就是说,它是水平的。在原点的左侧,当$x$取负值时,$\dfrac{dy}{dx}$也会有负值,或者从左向右下降,如图所示。

让我们通过一个具体的例子来说明这一点。取方程
\[
y = \tfrac{1}{4}x^2 + 3,
\]
并对其求导,我们得到
\[
\dfrac{dy}{dx} = \tfrac{1}{2}x.
\]
现在给$x$赋几个连续的值,比如从$0$到$5$,通过第一个方程计算相应的$y$值,并通过第二个方程计算$\dfrac{dy}{dx}$的值。将结果列成表格,我们得到:
\[
\begin{array}{|c||*{6}{c|}}
\hline
\Strut
\Td[c]{x} & \Td[c]{0} & \Td{1\Z} & \Td[c]{2} & \Td{3\Z} & \Td[c]{4} & \Td{5\Z} \\
\hline
\Strut
\Td[c]{y} & \Td[c]{3} & \Td{3\frac{1}{4}} & \Td[c]{4} & \Td{5\frac{1}{4}} & \Td[c]{7} & \Td{9\frac{1}{4}} \\
\hline
\DStrut
\Td[c]{\dfrac{dy}{dx}} &
\Td[c]{0} & \Td{\frac{1}{2}} & \Td[c]{1} & \Td{1\frac{1}{2}} & \Td[c]{2} & \Td{2\frac{1}{2}} \\
\hline
\end{array}
\]
然后,将它们绘制在两条曲线上,\Figs{23}{and}{24}\DPtypo{}{,},在\Fig{23}中绘制$y$相对于$x$的值,在\Fig{24}中绘制$\dfrac{dy}{dx}$相对于$x$的值。对于任何给定的$x$值,第二条曲线中纵坐标的高度与第一条曲线的斜率成正比。

\Figures[2.5in]{099a}{099b}{23}{24}

如果一条曲线突然出现尖点,如图\Fig{25}所示,该点的斜率会突然从向上的斜率变为向下的斜率。在这种情况下,$\dfrac{dy}{dx}$显然会从正值突然变为负值。
\DPPageSep{100.png}{88}%

以下示例进一步展示了刚刚解释的原理的应用。

(4) 求曲线
\[
y = \frac{1}{2x} + 3,
\]
在$x = -1$处的切线斜率。求这条切线与曲线$y = 2x^2 + 2$所成的角度。

切线的斜率是曲线在它们相切点的斜率(见\Pageref{slope});也就是说,它是曲线在该点的$\dfrac{dy}{dx}$。这里
\[
\dfrac{dy}{dx} = -\dfrac{1}{2x^2}
\]
对于$x = -1$,$\dfrac{dy}{dx} = -\dfrac{1}{2}$,这是切线和曲线在该点的斜率。切线是一条直线,其方程为$y = ax + b$,其斜率为$\dfrac{dy}{dx} = a$,因此$a = -\dfrac{1}{2}$。同时,当$x = -1$时,$y = \dfrac{1}{2(-1)} + 3 = 2\frac{1}{2}$;由于切线经过该点,该点的坐标必须满足切线的方程,即
\[
y = -\dfrac{1}{2} x + b,
\]
因此$2\frac{1}{2} = -\dfrac{1}{2} \times (-1) + b$,得到$b = 2$;因此切线的方程为$y = -\dfrac{1}{2} x + 2$。

现在,当两条曲线相交时,交点是两条曲线共有的点,其坐标必须满足每条曲线的方程;
\DPPageSep{101.png}{89}%
也就是说,它必须是将两条曲线的方程联立形成的方程组的解。这里,曲线在由以下方程组的解给出的点处相交:
\begin{DPgather*}
\left\{
\begin{aligned}
y &= 2x^2 + 2, \\
y &= -\tfrac{1}{2} x + 2 \quad\text{或}\quad 2x^2 + 2 = -\tfrac{1}{2} x + 2;
\end{aligned}
\right. \displaybreak[1] \\
\lintertext{即}
x(2x + \tfrac{1}{2}) = 0.
\end{DPgather*}

这个方程的解为$x = 0$和$x = -\tfrac{1}{4}$。曲线$y = 2x^2 + 2$在任意点的斜率为
\[
\dfrac{dy}{dx} = 4x.
\]

对于点 \( x = 0 \),该斜率为零;曲线在该点是水平的。对于点
\[
x = -\dfrac{1}{4},\quad \dfrac{dy}{dx} = -1;
\]
因此,曲线在该点向右下方倾斜,与水平线成角度 \(\theta\),使得
\(\tan \theta = 1\);即与水平线成 \(45^\circ\)。

直线的斜率为 \(-\tfrac{1}{2}\);即它向右下方倾斜,与水平线成角度 \(\phi\),使得 \(\tan \phi = \tfrac{1}{2}\);即与水平线成 \(26^\circ 34'\)。由此可知,在第一个点,曲线与直线相交成 \(26^\circ 34'\) 的角度,而在第二个点,它与直线相交成 \(45^\circ - 26^\circ 34' = 18^\circ 26'\) 的角度。

(5) 一条直线要通过坐标为 \( x = 2 \)、\( y = -1 \) 的点,作为曲线 \( y = x^2 - 5x + 6 \) 的切线。求切点的坐标。
\DPPageSep{102.png}{90}%

切线的斜率必须与曲线的 \(\dfrac{dy}{dx}\) 相同;即 \(2x - 5\)。

直线的方程为 \( y = ax + b \),并且由于它满足 \( x = 2 \)、\( y = -1 \) 的值,因此有
\(-1 = a \times 2 + b\);同时,其 \(\dfrac{dy}{dx} = a = 2x - 5\)。

切点的 \( x \) 和 \( y \) 也必须同时满足切线方程和曲线方程。

于是我们有
%[** TN: Omitting dot leaders.]
\begin{align*}
  y &= x^2 - 5x + 6, \tag*{(i)}   \\
  y &= ax + b,       \tag*{(ii)}  \\[-0.6\baselineskip]%[** Hard-coded dim]
% [** TN: Now set the left brace]
\smash{\left\{\begin{aligned} & \\ & \\ & \\ & \\ \end{aligned} \right.} \phantom{-1} & \\[-1.5ex]
 -1 &= 2a + b,       \tag*{(iii)} \\
  a &= 2x - 5,       \tag*{(iv)}
\end{align*}
四个方程,涉及 \( a \)、\( b \)、\( x \)、\( y \)。

方程 (i) 和 (ii) 给出 \( x^2 - 5x + 6 = ax + b \)。

将 \( a \) 和 \( b \) 的值代入其中,我们得到
\[
  x^2 - 5x + 6 = (2x - 5)x - 1 - 2(2x - 5),
\]
简化为 \( x^2 - 4x + 3 = 0 \),其解为:\( x = 3 \) 和 \( x = 1 \)。代入 (i),我们分别得到 \( y = 0 \) 和 \( y = 2 \);因此,两个切点分别为 \( x = 1 \)、\( y = 2 \),以及 \( x = 3 \)、\( y = 0 \)。

\Paragraph{注意。}---在所有涉及曲线的练习中,学生会发现通过实际绘制曲线来验证推导结果极为有益。
\DPPageSep{103.png}{91}%

\Exercises{VIII}(答案见 \Pageref[page]{AnsEx:VIII})

\begin{Problems}
\Item{(1)} 绘制曲线 \( y = \tfrac{3}{4} x^2 - 5 \),使用毫米刻度。在对应于不同 \( x \) 值的点上测量其斜角。

通过对方程求导,求出斜率的表达式;并通过自然正切表查看是否与测量的角度一致。

\Item{(2)} 求曲线
\[
y = 0.12x^3 - 2,
\]
在 \( x = 2 \) 的特定点的斜率。

\Item{(3)} 若 \( y = (x - a)(x - b) \),证明在 \(\dfrac{dy}{dx} = 0\) 的特定点,\( x \) 的值为 \(\tfrac{1}{2} (a + b)\)。

\Item{(4)} 求方程 \( y = x^3 + 3x \) 的 \(\dfrac{dy}{dx}\);并计算对应于 \( x = 0 \)、\( x = \tfrac{1}{2} \)、\( x = 1 \)、\( x = 2 \) 的点的 \(\dfrac{dy}{dx}\) 的数值。

\Item{(5)} 在方程为 \( x^2 + y^2 = 4 \) 的曲线上,求斜率 \( = 1 \) 的点的 \( x \) 值。

\Item{(6)} 求方程为 \(\dfrac{x^2 }{3^2} + \dfrac{y^2}{2^2} = 1\) 的曲线在任意点的斜率;并给出 \( x = 0 \) 和 \( x = 1 \) 处的斜率的数值。

\Item{(7)} 曲线 \( y = 5 - 2x + 0.5x^3 \) 的切线方程为 \( y = mx + n \),其中 \( m \) 和 \( n \) 为常数,求 \( m \) 和 \( n \) 的值,如果切点以 \( x = 2 \) 为横坐标。

\Item{(8)} 两条曲线
\[
y = 3.5x^2 + 2 \quad \text{和} \quad y = x^2 - 5x + 9.5
\]
相交成什么角度?

\Item{(9)} 在曲线 $y = ± \sqrt{25-x^2}$ 上,当 $x = 3$ 和 $x = 4$ 时分别画出切线。求这两条切线的交点坐标及其相互夹角。

\Item{(10)} 直线 $y = 2x - b$ 与曲线 $y = 3x^2 + 2$ 在某一点相切。求切点的坐标以及 $b$ 的值。
\end{Problems}
\DPPageSep{105.png}{93}%

\Chapter{第十一章}{极大值与极小值}

\First{微分过程的主要用途之一}是找出在什么条件下,被微分的事物的值达到最大或最小。这在工程问题中极为重要,因为了解哪些条件能使工作成本最小化或效率最大化是非常有价值的。

现在,让我们从一个具体的例子开始,考虑方程
\[
y = x^2 - 4x + 7.
\]
\Figure{105a}{26}

通过给 $x$ 赋予一系列连续的值,并求出相应的 $y$ 值,我们可以很容易地看出这个方程表示一条具有最小值的曲线。
\[
\begin{array}{|c||c|c|c|c|c|c|}
\hline
\Strut
\Td[c]{x} & \Td[c]{0} & \Td[c]{1} & \Td[c]{2} & \Td[c]{3} & \Td[c]{4} & \Td[c]{\Z5} \\
\hline
\Strut
\Td[c]{y} & \Td[c]{7} & \Td[c]{4} & \Td[c]{3} & \Td[c]{4} & \Td[c]{7} & \Td[c]{12} \\
\hline
\end{array}
\]

这些值在图 \Fig{26} 中绘制,显示当 $x$ 等于 $2$ 时,$y$ 显然有一个最小值 $3$。但你确定最小值发生在 $2$ 而不是 $2 \tfrac{1}{4}$ 或 $1 \tfrac{3}{4}$ 吗?

当然,对于任何代数表达式,都可以计算出许多值,并以此逐步找到可能是最大值或最小值的特定值。

\Figure{106a}{27}

这里还有一个例子:

设 \hfill $y = 3x - x^2$。 \hfill\null

计算一些值如下:
\[
\begin{array}{|c||c|c|c|c|c|c|c|}
\hline
\Strut
\Td[c]{x} & \Td[c]{-1} & \Td[c]{0} & \Td[c]{1} & \Td[c]{2} & \Td[c]{3} & \Td{4} & \Td{5} \\
\hline
\Strut
\Td[c]{y} & \Td[c]{-4} & \Td[c]{0} & \Td[c]{2} & \Td[c]{2} & \Td[c]{0} & \Td{-4} & \Td{-10} \\
\hline
\end{array}
\]
\DPPageSep{107.png}{95}%

将这些值绘制在图 \Fig{27} 中。

显然,在 $x = 1$ 和 $x = 2$ 之间会有一个最大值;而且看起来 $y$ 的最大值应该大约是 $2 \tfrac{1}{4}$。尝试一些中间值。如果 $x = 1 \tfrac{1}{4}$,$y = 2.187$;如果 $x = 1 \tfrac{1}{2}$,$y = 2.25$;如果 $x = 1.6$,$y = 2.24$。我们怎么能确定 $2.25$ 是真正的最大值,或者它恰好发生在 $x = 1 \tfrac{1}{2}$ 时呢?

现在,这听起来可能像是在玩把戏,但可以肯定的是,有一种方法可以直接找到最大(或最小)值,而不需要进行大量的初步试验或猜测。这种方法依赖于微分。回顾前面的页面(\pageref{curve})关于图 \Figs{14}{和}{15} 的注释,你会看到当曲线达到其最大或最小高度时,在该点的 $\dfrac{dy}{dx} = 0$。现在,这给了我们所需的线索。当给出一个方程,并且你想找到使 $y$ 达到最小(或最大)的 $x$ 值时,\emph{首先对其进行微分},然后将其 $\dfrac{dy}{dx}$ 设为 \emph{等于零},接着解出 $x$。将这个特定的 $x$ 值代入原方程,你将得到所需的 $y$ 值。这个过程通常被称为“令其等于零”。

为了看看它有多简单,以本章开头的例子为例,即
\[
y = x^2 - 4x + 7.
\]
\DPPageSep{108.png}{96}%
%
微分后,我们得到:
\[
\dfrac{dy}{dx} = 2x - 4.
\]
现在令其等于零,即:
\[
2x - 4 = 0.
\]
解这个方程得到:
\begin{align*}
2x &= 4, \\
 x &= 2.
\end{align*}

现在,我们知道最大(或最小)值恰好发生在 $x=2$ 时。

将值 \( x = 2 \) 代入原方程,我们得到
\begin{align*}
y &= 2^2 - (4 \times 2) + 7 \\
  &= 4 - 8 + 7       \\
  &= 3.
\end{align*}

现在回头看图 \Fig{26},你会发现最小值出现在 \( x = 2 \) 时,且此时的最小值为 \( y = 3 \)。

尝试第二个例子(图 \Fig{24}),其方程为
\begin{DPalign*}
y &= 3x - x^2. \\
\lintertext{\indent 求导数,}
\frac{dy}{dx} &= 3 - 2x. \\
\intertext{\indent 令其等于零,}
3 - 2x &= 0, \\
\lintertext{由此得}
x &= 1 \tfrac{1}{2}; \\
\intertext{将此 \( x \) 值代入原方程,我们得到:}
y &= 4 \tfrac{1}{2} - (1 \tfrac{1}{2} \times 1 \tfrac{1}{2}), \\
y &= 2 \tfrac{1}{4}.
\end{DPalign*}
这为我们提供了关于该方法尝试大量值时所留下的不确定性的确切信息。

现在,在我们继续讨论其他情况之前,我们有两点需要说明。当你被告知将 \(\dfrac{dy}{dx}\) 设为零时,你起初(如果你有自己的思维)会感到一种反感,因为你明白 \(\dfrac{dy}{dx}\) 在曲线的不同部分有各种不同的值,取决于它是向上倾斜还是向下倾斜。因此,当你突然被告知写下
\[
\frac{dy}{dx} = 0,
\]
你会感到不满,并倾向于认为这是不可能的。现在你必须理解“方程”和“条件方程”之间的本质区别。通常你处理的是本身成立的方程,但在某些情况下,如当前的例子所示,你必须写下那些不一定成立,但只有在满足某些条件时才成立的方程;你写下它们是为了通过解方程来找到使它们成立的条件。现在我们想找到曲线既不向上倾斜也不向下倾斜时 \( x \) 的特定值,即在 \(\dfrac{dy}{dx} = 0\) 的特定位置。因此,写下 \(\dfrac{dy}{dx} = 0\) 并不意味着它总是等于零;你写下它作为条件,以查看如果 \(\dfrac{dy}{dx}\) 为零,\( x \) 会是多少。

第二点是(如果你有自己的思维)你可能已经注意到的:即这个备受赞誉的设为零的过程完全无法告诉你所找到的 \( x \) 是否会给 \( y \) 带来一个最大值或最小值。确实如此。它本身无法区分;它为你找到了 \( x \) 的正确值,但让你自己找出相应的 \( y \) 是最大值还是最小值。当然,如果你已经绘制了曲线,你会事先知道它是什么。

例如,考虑方程:
\[
y = 4x + \frac{1}{x}.
\]

在不考虑它对应的曲线的情况下,求导并设为零:
\begin{DPalign*}
\frac{dy}{dx}  &= 4 - x^{-2} = 4 - \frac{1}{x^2} = 0; \\
\lintertext{由此得}
x &= \tfrac{1}{2}; \\
\intertext{并代入此值,}
y &= 4
\end{DPalign*}
将是最大值或最小值。但到底是哪一个?你将在后面学到一种依赖于二次求导的方法(见第十二章,\Pageref{chap:XII})。但目前,如果你只需尝试与找到的值略有不同的 \( x \) 值,并查看在此改变后的值下,相应的 \( y \) 值是小于还是大于已找到的值,这就足够了。

让我们尝试另一个简单的极值问题。假设你被要求将任意一个数分成两部分,使得它们的乘积最大?如果你不知道通过令导数为零的技巧,你会如何着手解决这个问题呢?我想你可以通过反复尝试的规则来解决它。设这个数是$60$。你可以尝试将其分成两部分,然后将它们相乘。例如,$50$乘以$10$是$500$;$52$乘以$8$是$416$;$40$乘以$20$是$800$;$45$乘以$15$是$675$;$30$乘以$30$是$900$。这看起来像是最大值:尝试稍微调整一下。$31$乘以$29$是$899$,这不如之前的;$32$乘以$28$是$896$,这更差。因此,似乎最大的乘积是通过将数分成两个相等的部分得到的。

现在看看微积分告诉我们什么。设要分成两部分的数为$n$。那么如果$x$是一部分,另一部分将是$n-x$,乘积将是$x(n-x)$或$nx-x^2$。所以我们写成$y=nx-x^2$。现在求导并令其等于零;
\begin{DPalign*}
\dfrac{dy}{dx} = n - 2x = 0\\
\lintertext{解得} \dfrac{n}{2} = x.\\
\end{DPalign*}
所以现在我们知道了,无论$n$是多少,如果想要两部分的乘积最大,我们必须将其分成两个相等的部分;并且这个最大乘积的值总是等于$\tfrac{1}{4} n^2$。

这是一个非常有用的规则,适用于任意数量的因数,因此如果$m+n+p$是一个常数,$m×n×p$在$m=n=p$时达到最大值。
\DPPageSep{112.png}{100}%

\newpage%[** TN: Text block-dependent page break]
\Subsection{测试案例。}
让我们立即应用我们的知识到一个我们可以测试的案例。
%
\begin{DPalign*}
\lintertext{\indent 设} y &= x^2 - x;
\end{DPalign*}
让我们找出这个函数是否有最大值或最小值;如果有,测试它是最大值还是最小值。

求导,我们得到
\begin{DPalign*}
\frac{dy}{dx} &= 2x - 1. \\
\intertext{\indent 令其等于零,我们得到}
2x - 1 &= 0, \\
\lintertext{由此}
2x &= 1, \\
\lintertext{或}
x &= \tfrac{1}{2}.
\end{DPalign*}

也就是说,当$x$取$\frac{1}{2}$时,对应的$y$值将是最大值或最小值。因此,将$x=\frac{1}{2}$代入原方程,我们得到
\begin{DPalign*}
y &= (\tfrac{1}{2})^2 - \tfrac{1}{2}, \\
\lintertext{或}
y &= -\tfrac{1}{4}.
\end{DPalign*}

这是最大值还是最小值?为了测试它,尝试将$x$取稍大于$\frac{1}{2}$的值,比如设$x=0.6$。那么
\[
y = (0.6)^2 - 0.6 = 0.36 - 0.6 = -0.24,
\]
这比$-0.25$更高,表明$y = -0.25$是一个\emph{最小值}。

自己绘制曲线,并验证计算结果。
\DPPageSep{113.png}{101}%

\Examples{进一步的例子。}
一个非常有趣的例子是由一条既有最大值又有最小值的曲线提供的。它的方程是:
\begin{DPalign*}
y &=\tfrac{1}{3} x^3 - 2x^2 + 3x + 1. \\
\lintertext{\indent 现在} \dfrac{dy}{dx} &= x^2 - 4x +3.
\end{DPalign*}

\Figure[2.5in]{113a}{28}

令其等于零,我们得到二次方程,
\[
x^2 - 4x +3 = 0;
\]
解这个二次方程得到两个根,即
\[
\left\{
\begin{aligned}
x &= 3 \\
x &= 1.
\end{aligned}
\right.
\]

现在,当$x=3$时,$y=1$;当$x=1$时,$y=2\frac{1}{3}$。第一个是最小值,第二个是最大值。

这条曲线本身可以通过从原方程计算的值绘制出来(如图\Fig{28}),如下表所示。
\[
\begin{array}{|c||*{8}{c|}}
\hline
\Strut
\Td[c]{x} & \Td{-1\Z} & \Td[c]{0} & \Td[c]{1\Z} & \Td[c]{2\Z} & \Td[c]{3} & \Td{4\Z} & \Td{5\Z} & \Td{6} \\
\hline
\Strut
\Td[c]{y} & \Td{-4\tfrac{1}{3}} & \Td[c]{1} &  \Td[c]{2\tfrac{1}{3}} &  \Td[c]{1\tfrac{2}{3}} & \Td[c]{1} & \Td{2\tfrac{1}{3}} & \Td{7\tfrac{2}{3}} & \Td{19} \\
\hline
\end{array}
\]

进一步的极值练习由以下例子提供:



圆的半径为~$r$,其中心~$C$位于坐标为 $x=a$、$y=b$ 的点上,如\Fig{29}所示,其方程为:
\[
(y-b)^2 + (x-a)^2 = r^2.
\]
\Figure[2.5in]{114a}{29} %[** TN: 上移]

此式可变形为
\[
y = \sqrt{r^2-(x-a)^2} + b.
\]

现在,我们仅通过观察图形即可预知,当 $x=a$ 时,$y$ 将达到其最大值~$b+r$ 或最小值~$b-r$。但让我们不利用这一知识;而是通过微分并令其等于零的过程,来寻找使 $y$ 达到最大或最小的 $x$ 值。
\begin{align*}
\frac{dy}{dx} &= \frac{1}{2} \frac{1}{\sqrt{r^2-(x-a)^2}} × (2a-2x), \\
\intertext{化简后得到}
\frac{dy}{dx} &= \frac{a-x}{\sqrt{r^2-(x-a)^2}}.
\end{align*}

于是,$y$ 达到最大或最小的条件为:
\[
\frac{a-x}{\sqrt{r^2-(x-a)^2}} = 0.
\]

由于 $x$ 的任何值都不会使分母无穷大,因此唯一使结果为零的条件是
\begin{align*}
x &= a. \\
\intertext{\indent 将此值代入圆的原始方程,我们得到}
y &= \sqrt{r^2}+b;
\intertext{由于 $r^2$ 的平方根为 $+r$ 或~$-r$,我们得到 $y$ 的两个结果值,}
%[** TN: 希望在 = 上对齐,同时保持左大括号]
\left\{\begin{aligned}y \\ y\end{aligned}\right. &
\begin{aligned}= b & + r \\ = b & - r.\end{aligned}
\end{align*}

第一个值是最大值,位于顶部;第二个值是最小值,位于底部。
\DPPageSep{116.png}{104}%

如果曲线没有最大或最小值的位置,则通过令其等于零的过程将得到不可能的结果。例如:
\begin{DPalign*}
\lintertext{\indent 设}
y &= ax^3 + bx + c. \\ % [** TN: a, b 当然具有相同的符号]
\lintertext{\indent 则}
\frac{dy}{dx} &= 3ax^2 + b.
\end{DPalign*}

令其等于零,我们得到 $3ax^2 + b = 0$,
\[
x^2 = \frac{-b}{3a}, \quad\text{且}\quad x = \sqrt{\frac{-b}{3a}},\text{ 这是不可能的。}
\]
因此,$y$ 既无最大值也无最小值。

通过几个更多的实际例子,你将能够彻底掌握这一最有趣且有用的微积分应用。

(1) 内接于半径为~$R$ 的圆的矩形,其最大面积的边长是多少?

如果一边称为~$x$,
\[
\text{另一边} = \sqrt{(\text{对角线})^2 - x^2};
\]
由于矩形的对角线必然是直径,另一边~$ = \sqrt{4R^2 - x^2}$。

于是,矩形的面积 $S = x\sqrt{4R^2 - x^2}$,
\[
\frac{dS}{dx} = x × \dfrac{d\left(\sqrt{4R^2 - x^2}\,\right)}{dx} + \sqrt{4R^2 - x^2} × \dfrac{d(x)}{dx}.
\]

如果你忘记了如何对 $\sqrt{4R^2-x^2}$ 进行微分,这里有一个提示:设 $4R^2-x^2=w$ 且 $y=\sqrt{w}$,并求 $\dfrac{dy}{dw}$ 和 $\dfrac{dw}{dx}$;努力解决,只有在无法继续时才参考\Pageref[page]{dodge}。
\DPPageSep{117.png}{105}%

你将得到
\[
\dfrac{dS}{dx}
  = x × -\dfrac{x}{\sqrt{4R^2 - x^2}} + \sqrt{4R^2 - x^2}
  = \dfrac{4R^2 - 2x^2}{\sqrt{4R^2 - x^2}}.
\]

对于最大或最小值,我们必须有
\[
\dfrac{4R^2 - 2x^2}{\sqrt{4R^2 - x^2}} = 0;
\]
即 $4R^2 - 2x^2 = 0$ 且 $x = R\sqrt{2}$。

另一边 ${} = \sqrt{4R^2 - 2R^2} = R\sqrt{2}$;两边相等;该图形为正方形,其边长等于以半径为边长的正方形的对角线。在这种情况下,我们处理的是最大值。

(2) 当圆锥形容器的容量最大时,其开口的半径是多少?

设 $R$ 为半径,$H$ 为对应的高度,则 $H = \sqrt{l^2 - R^2}$。
\[
\text{体积 } V = \pi R^2 × \dfrac{H}{3} = \pi R^2 × \dfrac{\sqrt{l^2 - R^2}}{3}.
\]

按照前一问题的步骤,我们得到
\begin{align*}
\dfrac{dV}{dR}
  &= \pi R^2 × -\dfrac{R}{3\sqrt{l^2 - R^2}} + \dfrac{2\pi R}{3} \sqrt{l^2 - R^2} \\
  &= \dfrac{2\pi R(l^2 - R^2) - \pi R^3}{3\sqrt{l^2 - R^2}} = 0
\end{align*}
以确定最大值或最小值。

或者,$2\pi R(l^2 - R^2) - \pi R^2 = 0$,且$R = l\sqrt{\tfrac{2}{3}}$,显然为最大值。
\DPPageSep{118.png}{106}%

(3) 求函数
\[
y = \dfrac{x}{4-x} + \dfrac{4-x}{x}
\]
的极大值和极小值。

我们得到
\[
\dfrac{dy}{dx} = \dfrac{(4-x)-(-x)}{(4-x)^2} + \dfrac{-x - (4-x)}{x^2} = 0
\]
以确定最大值或最小值;或者
\[
\dfrac{4}{(4-x)^2} - \dfrac{4}{x^2} = 0 \quad\text{且}\quad x = 2。
\]

只有一个值,因此只有一个极大值或极小值。
\begin{align*}
\text{当}\quad x &= 2,\phantom{.5}\quad y = 2, \\
\text{当}\quad x &= 1.5,\quad y = 2.27,   \\
\text{当}\quad x &= 2.5,\quad y = 2.27;
\end{align*}
因此这是一个极小值。(绘制函数的图形是有益的。)

(4) 求函数
$y = \sqrt{1+x} + \sqrt{1-x}$
的极大值和极小值。(绘制图形会发现很有启发。)

微分后立即得到(参见例1,\Pageref{ExNo1})
\[
\dfrac{dy}{dx} = \dfrac{1}{2\sqrt{1+x}} - \dfrac{1}{2\sqrt{1-x}} = 0
\]
以确定最大值或最小值。

因此$\sqrt{1+x} = \sqrt{1-x}$且$x = 0$,这是唯一的解。

当$x=0$时,$y=2$。

当$x=±0.5$时,$y= 1.932$,所以这是一个极大值。
\DPPageSep{119.png}{107}%

(5) 求函数
\[
y = \dfrac{x^2-5}{2x-4}
\]
的极大值和极小值。

我们有
\[
\dfrac{dy}{dx} = \dfrac{(2x-4) × 2x - (x^2-5)2}{(2x-4)^2} = 0
\]
以确定最大值或最小值;或者
\[
\dfrac{2x^2 - 8x + 10}{\DPtypo{(2x - 5)^2}{(2x - 4)^2}} = 0;
\]
或者$x^2 - 4x + 5 = 0$;其解为
\[
x = \tfrac{5}{2} ± \sqrt{-1}。
\]

由于这些解为虚数,不存在使$\dfrac{dy}{dx} = 0$的实数$x$;因此既无极大值也无极小值。

(6) 求函数
\[
(y-x^2)^2 = x^5
\]
的极大值和极小值。

这可以写成$y = x^2 ± x^{\efrac{5}{2}}$。
\[
\dfrac{dy}{dx} = 2x ± \tfrac{5}{2} x^{\efrac{3}{2}} = 0 \quad\text{以确定最大值或最小值};
\]
即$x(2 ± \tfrac{5}{2} x^{\efrac{1}{2}}) = 0$,满足$x = 0$,
以及$2 ± \tfrac{5}{2} x^{\efrac{1}{2}} = 0$,即$x=\tfrac{16}{25}$。因此有两个解。

首先取$x = 0$。若$x = -0.5$,$y = 0.25 ± \sqrt[2]{-(.5)^5}$,
若$x = +0.5$,$y = 0.25 ± \sqrt[2]{\DPtypo{.55}{(.5)^5}}$。在一侧$y$为虚数;
即,不存在可由图形表示的$y$值;因此图形完全位于$y$轴的右侧(见\Fig{30})。

绘制图形时会发现,曲线趋向原点,似乎那里有一个极小值;
但并未继续超越,而是折返(形成所谓的“尖点”)。因此没有极小值,
尽管满足极小值的条件,即$\dfrac{dy}{dx} = 0$。因此,始终需要通过取两侧的值来验证。

\Figure[2.75in]{120a}{30}

现在,若取$x = \tfrac{16}{25} = 0.64$。若$x = 0.64$,$y = 0.7373$
且$y = 0.0819$;若$x = 0.6$,$y$变为$0.6389$和$0.0811$;
若$x = 0.7$,$y$变为$0.8996$和$0.0804$。

这表明曲线有两支;上支未经过极大值,但下支经过。

(7) 一个圆柱体,其高度是底面半径的两倍,体积在增加,因此其所有部分始终保持相同的相互比例;也就是说,在任何时刻,圆柱体与原始圆柱体都是\emph{相似}的。当底面半径为$r$~英尺时,表面积以每秒$20$~平方英寸的速度增加;此时其体积增加的速度是多少?
\begin{align*}
\text{面积}   &= S = 2(\pi r^2)+ 2 \pi r × 2r = 6 \pi r^2.\\
\text{体积} &= V = \pi r^2 × 2r=2 \pi r^3.\\
\frac{dS}{dr} &= 12\pi r,\quad \frac{dV}{dr}=6 \pi r^2,\\
dS            &= 12\pi r\, dr=20,\quad dr=\frac{20}{12 \pi r},\\
dV            &=  6\pi r^2\, dr = 6 \pi r^2 × \frac{20}{12 \pi r} = 10r.
\end{align*}

体积以每秒$10r$立方英寸的速度变化。

\tb

为自己设计其他例子。很少有学科能提供如此丰富的有趣例子。

\Exercises{IX} (答案见\Pageref[page]{AnsEx:IX}。)

\begin{Problems}
\Item{(1)} 如果$y=\dfrac{x^2}{x+1}$,$x$取何值时$y$达到最大值和最小值?

\Item{(2)} 在方程$y=\dfrac{x}{a^2+x^2}$中,$x$取何值时$y$达到最大值?
\DPPageSep{122.png}{110}%

\Item{(3)} 一根长度为$p$的线被分成4部分,并组成一个矩形。证明矩形的面积最大时,其每条边等于$\frac{1}{4}p$。

\Item[Ex9No4]{(4)} 一根30英寸长的绳子两端连接在一起,并通过3个钉子拉伸形成一个三角形。绳子能围成的最大三角形面积是多少?

\Item{(5)} 绘制对应于方程$y = \frac{10}{x} + \frac{10}{8-x}$的曲线;同时求出$\dfrac{dy}{dx}$,并推导出使$y$达到最小值的$x$值;以及$y$的最小值。

\Item{(6)} 如果$y = x^5-5x$,求出使$y$达到最大值或最小值的$x$值。

\Item{(7)} 在给定的正方形中,能内接的最小正方形是什么?

\Item{(8)} 在给定的圆锥中,其高度等于底面半径,内接一个圆柱体(\textit{a})体积最大;(\textit{b})侧面积最大;(\textit{c})总面积最大。

\Item{(9)} 在给定的球体中,内接一个圆柱体(\textit{a})体积最大;(\textit{b})侧面积最大;(\textit{c})总面积最大。
\DPPageSep{123.png}{111}%

\Item{(10)} 一个球形气球体积在增加。当其半径为$r$~英尺时,体积以每秒$4$~立方英尺的速度增加,此时其表面积增加的速度是多少?

\Item{(11)} 在给定的球体中,内接一个体积最大的圆锥。

\Item{(12)} 由$N$~个相似的伏打电池组成的电池组提供的电流为$C=\dfrac{n×E}{R+\dfrac{rn^2}{N}}$,其中$E$、$R$、$r$为常数,$n$为串联的电池数。求出使电流最大的$n$与$N$的比例。
\end{Problems}
\DPPageSep{124.png}{112}%

\Chapter{第十二章}{曲线的曲率}

\First{回到}逐次微分的过程,可能会问:为什么有人想要两次微分?我们知道,当变量是空间和时间时,通过两次微分可以得到运动物体的加速度,并且在几何解释中,应用于曲线时,$\dfrac{dy}{dx}$~表示曲线的\emph{斜率}。但在这个情况下,$\dfrac{d^2 y}{dx^2}$~意味着什么呢?显然,它表示斜率变化的速率(每单位长度$x$)——简而言之,它是\emph{斜率曲率的度量}。

\Figures{124a}{124b}{31}{32}

假设斜率恒定,如\Fig{31}所示。

这里,$\dfrac{dy}{dx}$是一个恒定值。
\DPPageSep{125.png}{113}%

然而,假设一个情况,如\Fig{32}所示,斜率本身向上增加,那么$\dfrac{d\left(\dfrac{dy}{dx}\right)}{dx}$,即$\dfrac{d^2y}{dx^2}$,将是\emph{正的}。

如果斜率在向右移动时逐渐减小(如\Fig{14},\Pageref{fig:14}所示,或如\Fig{33}所示),那么即使曲线可能向上延伸,由于变化导致斜率减小,其$\dfrac{d^2y}{dx^2}$将为\emph{负}。

\Figure{125a}{33}

现在是时候向你揭示另一个秘密了——如何判断通过“令其等于零”得到的结果是极大值还是极小值。诀窍是这样的:在你进行微分(以得到令其等于零的表达式)后,再进行第二次微分,并观察第二次微分的结果是\emph{正}还是\emph{负}。如果$\dfrac{d^2y}{dx^2}$为\emph{正},那么你得到的$y$值就是\emph{极小值};但如果$\dfrac{d^2y}{dx^2}$为\emph{负},那么你得到的$y$值必定是\emph{极大值}。这就是规则。

其原因应当相当明显。想象任何一条在某个点具有极小值的曲线(如\Fig{15},\Pageref{fig:15}所示,或如\Fig{34}所示,其中极小值点$y$标记为$M$,曲线呈\emph{向上凹})。在$M$的左侧,斜率为向下,即负值,且逐渐减少负值。在$M$的右侧,斜率变为向上,且越来越向上。显然,当曲线通过$M$时,斜率的变化使得$\dfrac{d^2y}{dx^2}$为\emph{正},因为随着$x$向右增加,其作用是将向下斜率转化为向上斜率。

同样地,考虑任何一条在某个点具有极大值的曲线(如\Fig{16},\Pageref{fig:16}所示,或如\Fig{35}所示,其中曲线呈\emph{向下凸},极大值点标记为$M$)。在这种情况下,当曲线从左向右通过$M$时,其向上斜率转化为向下或负斜率,因此在这种情况下,“斜率的斜率”$\dfrac{d^2y}{dx^2}$为\emph{负}。

现在回到上一章的例子,通过这种方式验证得出的结论,判断在任何特定情况下是极大值还是极小值。你会在下面找到一些已解决的例子。

\tb

(1) 求下列函数的极大值或极小值:
\begin{align*}
\text{(\textit{a})}\quad y &= 4x^2-9x-6; \qquad \text{(\textit{b})}\quad y = 6 + 9x-4x^2; \\
\intertext{并确定在每种情况下是极大值还是极小值。}
\text{(\textit{a})}\quad \dfrac{dy}{dx}
  &= 8x-9=0;\quad x=1\tfrac{1}{8},\quad \text{且 } y = -11.065.\\
%
\dfrac{d^2y}{dx^2}
  &= 8;\quad \text{为正;故为极小值。} \\
%
\text{(\textit{b})}\quad \DPtypo{\dfrac{dx}{dy}}{\dfrac{dy}{dx}}
  &= 9-8x=0;\quad x = 1\tfrac{1}{8};\quad \text{且 } y = +11.065.\\
%
\dfrac{d^2y}{dx^2}
  &= -8;\quad \text{为负;故为极大值。}
\end{align*}

(2) 求函数$y = x^3-3x+16$的极大值和极小值。
\begin{align*}
\dfrac{dy}{dx}
  &= 3x^2 - 3 = 0;\quad x^2 = 1;\quad \text{且 } x = ±1.\\
%
\dfrac{d^2y}{dx^2}
  &= 6x;\quad \text{对于 } x = 1; \text{ 为正;}
\end{align*}
因此,$x=1$对应于极小值$y=14$。对于$x=-1$,为负;因此$x=-1$对应于极大值$y=+18$。
\DPPageSep{128.png}{116}%

(3) 求$y=\dfrac{x-1}{x^2+2}$的极大值和极小值。
\[
\frac{dy}{dx} = \frac{(x^2+2) × 1 - (x-1) × 2x}{(x^2+2)^2}
  = \frac{2x - x^2 + 2}{(x^2 + 2)^2} = 0;
\]
即$x^2 - 2x - 2 = 0$,其解为$x =+2.73$和$x=-0.73$。
\begin{align*}
\dfrac{d^2y}{dx^2}
  &= - \frac{(x^2 + 2)^2 × (2x-2) - (x^2 - 2x - 2)(4x^3 + 8x)}{(x^2 + 2)^4} \\
  &= - \frac{2x^5 - 6x^4 - 8x^3 - 8x^2 - 24x + 8}{(x^2 + 2)^4}.
\end{align*}

分母始终为正,因此只需确定分子的符号。

若取$x = 2.73$,分子为负;故为极大值,$y = 0.183$。

若取$x=-0.73$,分子为正;故为极小值,$y=-0.683$。

(4) 某工厂处理产品所需的费用\( C \)与每周产量\( P \)的关系为\( C = aP + \dfrac{b}{c+P} + d \),其中\( a \)、\( b \)、\( c \)、\( d \)为正的常数。问产量为多少时,费用最小?
\[
\dfrac{dC}{dP} = a - \frac{b}{(c+P)^2} = 0\quad \text{求极大或极小值;}
\]
因此\( a = \dfrac{b}{(c+P)^2} \),且\( P = ±\sqrt{\dfrac{b}{a}} - c \)。

由于产量不能为负,故\( P=+\sqrt{\dfrac{b}{a}} - c \)。
\DPPageSep{129.png}{117}%
\begin{DPalign*}
\lintertext{\indent 现在}
\frac{d^2C}{dP^2} &= + \frac{b(2c + 2P)}{(c + P)^4},
\end{DPalign*}
此式对所有\( P \)的值均为正;因此\( P = +\sqrt{\dfrac{b}{a}} - c \)对应于最小值。

(5) 使用\( N \)盏某种灯具照亮一栋建筑的每小时总费用\( C \)为
\[
C = N\left(\frac{C_l}{t} + \frac{EPC_e}{1000}\right),
\]
其中\( E \)为商业效率(瓦特每烛光),
\begin{align*}
&\text{\( P \)为每盏灯的烛光功率,} \\
&\text{\( t \)为每盏灯的平均寿命(小时),} \\
&\text{\( C_l = \) 使用每小时的更换成本(便士),} \\
&\text{\( C_e = \) 每1000瓦每小时的能源成本(便士)。}
\end{align*}

此外,灯具的平均寿命与运行时的商业效率之间的关系近似为\( t = mE^n \),其中\( m \)和\( n \)是取决于灯具类型的常数。

求使照明总成本最小的商业效率。
%
\begin{DPalign*}
\lintertext{\indent 我们有}
C &= N\left(\frac{C_l}{m} E^{-n} + \frac{PC_e}{1000} E\right), \\
\dfrac{dC}{dE}
  &= \frac{PC_e}{1000} - \frac{nC_l}{m} E^{-(n+1)} = 0
\end{DPalign*}
求极大或极小值。
\[
E^{n+1} = \frac{1000 × nC_l}{mPC_e}\quad \text{且}\quad
E = \sqrt[n+1]{\frac{1000 × nC_l}{mPC_e}}.
\]
\DPPageSep{130.png}{118}%

显然这是最小值,因为
\[
\frac{d^2C}{dE^2} = (n + 1) \frac{nC_l}{m} E^{-(n+2)},
\]
对正的\( E \)值为正。

对于一种16烛光功率的灯具,\( C_l= 17 \)便士,\( C_e=5 \)便士;且已知\( m=10 \)和\( n=3.6 \)。
\[
E = \sqrt[4.6]{\frac{1000 × 3.6 × 17}{10 × 16 × 5}}
  = 2.6\text{ 瓦特每烛光功率}。
\]

\Exercises{X}(建议绘制任何数值例子的图形。)(答案见\Pageref{AnsEx:X}。)
\begin{Problems}
\Item{(1)} 求
\[
y = x^3 + x^2 - 10x + 8
\]
的极大值和极小值。

\Item{(2)} 已知\( y = \dfrac{b}{a}x - cx^2 \),求\( \dfrac{dy}{dx} \)和\( \dfrac{d^2y}{dx^2} \)的表达式,并求使\( y \)为极大或极小的\( x \)值,并说明是极大还是极小。

\Item{(3)} 求曲线方程为
\[
y = 1 - \frac{x^2}{2} + \frac{x^4}{24}
\]
和
\[
y = 1 - \frac{x^2}{2} + \frac{x^4}{24} - \frac{x^6}{720}
\]
的曲线中分别有多少个极大值和极小值。
\DPPageSep{131.png}{119}%

\Item{(4)} 求
\[
y=2x+1+\frac{5}{x^2}
\]
的极大值和极小值。

\Item{(5)} 求
\[
y=\frac{3}{x^2+x+1}
\]
的极大值和极小值。

\Item{(6)} 求
\[
y=\frac{5x}{2+x^2}
\]
的极大值和极小值。

\Item{(7)} 求
\[
y=\frac{3x}{x^2-3} + \frac{x}{2} + 5
\]
的极大值和极小值。

\Item{(8)} 将一个数\( N \)分成两部分,使得其中一部分的平方的三倍加上另一部分的平方的两倍为最小。

\Item{(9)} 电发电机在不同输出值\( x \)下的效率\( u \)由一般方程表示:
\[
u=\frac{x}{a+bx+cx^2};
\]
其中\( a \)是主要取决于铁损耗的常数,\( c \)是主要取决于铜部件电阻的常数。求使效率最大的输出值的表达式。
\DPPageSep{132.png}{120}%

\Item{(10)} 假设已知某艘轮船的煤炭消耗量可以用公式 $y = 0.3 + 0.001v^3$ 表示;其中 $y$ 是每小时燃烧的煤炭吨数,$v$ 是以海里每小时表示的速度。该船的工资、资本利息和折旧费用的总和,每小时等于 $1$ 吨煤炭的成本。问何种速度能使航行 $1000$ 海里的总成本最小?如果煤炭每吨成本为 $10$ 先令,那么该航行的最小成本是多少?

\Item{(11)} 求函数的极大值和极小值\Pagelabel{Ex:X11}%
\[
y = ±\frac{x}{6}\sqrt{x(10-x)}.
\]

\Item{(12)} 求函数的极大值和极小值
\[
y= 4x^3 - x^2 - 2x + 1.
\]
\end{Problems}
\DPPageSep{133.png}{121}%

\Chapter{第十三章}{其他有用的技巧}

\Section{部分分式。}

\First{我们}已经看到,当对一个分数进行微分时,我们需要执行一项相当复杂的操作;而且,如果这个分数本身不是一个简单的分数,结果必然是一个复杂的表达式。如果我们能够将这个分数拆分成两个或多个更简单的分数,使得它们的和等于原分数,那么我们就可以分别对这些更简单的表达式进行微分。微分的结果将是两个(或更多)微分的和,每个微分都相对简单;而最终的表达式,虽然当然与不采用此技巧时得到的结果相同,但通过这种方法可以更轻松地获得,并且以简化的形式呈现。

让我们看看如何实现这一结果。首先尝试将两个分数相加,形成一个结果分数。例如,取两个分数 $\dfrac{1}{x+1}$ 和 $\dfrac{2}{x-1}$。每个学童都能将它们相加,得到它们的和为 $\dfrac{3x+1}{x^2-1}$。同样地,他可以将三个或更多分数相加。现在,这个过程当然可以反过来:\Pagelabel{partfracs2} 也就是说,如果给出最后一个表达式,那么它肯定可以以某种方式拆分回其原始组成部分或部分分式。只是我们不知道在每个可能出现的情况下,如何进行这种拆分。为了找出这一点,我们首先考虑一个简单的例子。但重要的是要记住,以下所有内容仅适用于所谓的“真”代数分数,即分子次数低于分母的分数;也就是说,那些在分子中 $x$ 的最高指数比分母中的最高指数小的分数。如果我们必须处理像 $\dfrac{x^2+2}{x^2-1}$ 这样的表达式,我们可以通过除法简化它,因为它等价于 $1+\dfrac{3}{x^2-1}$;而 $\dfrac{3}{x^2-1}$ 是一个真代数分数,可以按照后面解释的方法拆分为部分分式。

% [** TN: 保留格式不一致;标题不单独成行]
\Paragraph{情况~I。} 如果我们进行多次两个或多个分数的相加,这些分数的分母只包含 $x$ 的项,而不包含 $x^2$、$x^3$ 或 $x$ 的任何其他幂的项,我们总是会发现最终结果分数的分母是相加分数分母的乘积。因此,通过因式分解这个最终分数的分母,我们可以找到我们正在寻找的部分分式的每个分母。
\DPPageSep{135.png}{123}%

假设我们希望从 $\dfrac{3x+1}{x^2-1}$ 回到已知的组成部分 $\dfrac{1}{x+1}$ 和 $\dfrac{2}{x-1}$。如果我们不知道这些组成部分是什么,我们仍然可以通过写下以下内容来准备:
\[
\frac{3x+1}{x^2-1} = \frac{3x+1}{(x+1)(x-1)} = \frac{}{x+1} + \frac{}{x-1},
\]
将分子的位置留空,直到我们知道该填入什么。我们总是可以假设部分分数之间的符号为\emph{加号},因为如果是\emph{减号},我们只需找到相应的分子为负数即可。现在,由于部分分数是\emph{真分数},分子只是不带$x$的纯数字,我们可以随意称它们为$A$、$B$、$C\dots$。因此,在这种情况下,我们有:
\[
\frac{3x+1}{x^2-1} = \frac{A}{x+1} + \frac{B}{x-1}.
\]

如果现在我们对这两个部分分数进行相加,我们得到 $\dfrac{A(x-1)+B(x+1)}{(x+1)(x-1)}$;这必须等于 $\dfrac{3x+1}{(x+1)(x-1)}$。由于这两个表达式中的分母相同,分子也必须相等,给出我们:
\[
3x + 1 = A(x-1) + B(x + 1).
\]

现在,这是一个含有两个未知量的方程,似乎我们需要另一个方程才能解出$A$和$B$。
\DPPageSep{136.png}{124}%
但有一种方法可以解决这个困难。这个方程必须对所有$x$的值都成立;因此,它必须对使$x-1$和$x+1$变为零的$x$值成立,即分别对$x=1$和$x=-1$成立。如果我们令$x=1$,我们得到$4 = (A \times 0) + (B \times 2)$,所以$B=2$;如果我们令$x=-1$,我们得到$-2 = (A \times -2) + (B \times 0)$,所以$A=1$。将部分分数中的$A$和$B$替换为这些新值,我们发现它们变为$\dfrac{1}{x+1}$和$\dfrac{2}{x-1}$;这样就完成了。

作为一个进一步的例子,让我们取分数$\dfrac{4x^2 + 2x - 14}{x^3 + 3x^2 - x - 3}$。当$x$取值$1$时,分母变为零;因此$x-1$是其一个因子,显然另一个因子将是$x^2 + 4x + 3$;这个因子又可以分解为$(x+1)(x+3)$。因此我们可以将分数写成:
\[
\frac{4x^2 + 2x - 14}{x^3 + 3x^2 - x - 3}
  = \frac{A}{x+1} + \frac{B}{x-1} + \frac{C}{x+3},
\]
形成三个部分因子。

按照之前的方法,我们发现
\[%[** TN: Set on two lines in the original]
4x^2 + 2x - 14 = A(x-1)(x+3) + B(x+1)(x+3) + C(x+1)(x-1).
\]

现在,如果我们令$x=1$,我们得到:
\[
-8 = (A \times 0) + B(2 \times 4) + (C \times 0);\quad \text{即,} B = -1.
\]

如果$x= -1$,我们得到:
\[
-12 = A(-2 \times 2) + (B \times 0) + (C \times 0);\quad \text{因此} A = 3.
\]
\DPPageSep{137.png}{125}%

如果$x = -3$,我们得到:
\[
16 = (A \times 0) + (B \times 0) + C(-2 \times -4);\quad \text{因此} C = 2.
\]

因此,部分分数为:
\[
\frac{3}{x+1} - \frac{1}{x-1} + \frac{2}{x+3},
\]
这比从其导出的复杂表达式更容易对$x$求导。

\Paragraph{情况二。}如果分母的某些因子包含$x^2$的项,并且不方便分解,那么相应的分子可能包含$x$的项,以及一个简单的数字;因此,有必要用$Ax + B$而不是符号$A$来表示这个未知分子;其余的计算与之前相同。
%
\begin{DPgather*}
\lintertext{\rlap{\indent 例如,尝试:}}
\frac{-x^2 - 3}{(x^2+1)(x+1)}. \\
\frac{-x^2 - 3}{(x^2+1)(x+1)} = \frac{Ax+B}{x^2+1} + \frac{C}{x+1};\\
-x^2 - 3 = (Ax + B)(x+1) + C(x^2+1).
\end{DPgather*}

令$x= -1$,我们得到$-4 = C \times 2$;因此$C = -2$;
\begin{DPalign*}
\lintertext{因此}
-x^2 - 3 &= (Ax + B)(x + 1) - 2x^2 - 2; \\
\lintertext{并且}
x^2 - 1 &= Ax(x+1) + B(x+1).
\end{DPalign*}

令$x = 0$,我们得到$-1 = B$;\\
因此
\begin{DPgather*}
x^2 - 1 = Ax(x + 1) - x - 1;\quad \text{即 } x^2 + x = Ax(x+1); \\
\lintertext{并且}
x+1 = A(x+1),
\end{DPgather*}
所以$A=1$,部分分数为:
\[
\frac{x-1}{x^2+1} - \frac{2}{x+1}.
\]

再举一个例子,考虑分数
\[
\frac{x^3-2}{(x^2+1)(x^2+2)}.
\]

我们得到
\begin{align*}
\frac{x^3-2}{(x^2+1)(x^2+2)}
  &= \frac{Ax+B}{x^2+1} + \frac{Cx+D}{x^2+2}\\
  &= \frac{(Ax+B)(x^2+2)+(Cx+D)(x^2+1)}{(x^2+1)(x^2+2)}.
\end{align*}

在这种情况下,确定$A$、$B$、$C$、$D$并不容易。按照以下方法进行会更简单:
由于给定的分数与通过部分分式相加得到的分数相等,并且具有\emph{相同的}分母,因此分子也必须完全相同。在这种情况下,对于我们正在处理的代数表达式,\emph{相同幂次的$x$的系数相等且符号相同}。

因此,由于
\begin{align*}
x^3-2
  &= (Ax+B)(x^2+2) + (Cx+D)(x^2+1) \\
  &= (A+C)x^3 + (B+D)x^2 + (2A+C)x + 2B+D,
\end{align*}
我们得到$1=A+C$;$0=B+D$(左边表达式中$x^2$的系数为零);$0=2A+C$;以及$-2=2B+D$。这里有四个方程,从中我们容易得到$A=-1$;$B=-2$;$C=2$;$D=0$;因此部分分式为$\dfrac{2(x+1)}{x^2+2} - \dfrac{x+2}{x^2+1}$。
\DPPageSep{139.png}{127}%
这种方法总是可以使用的;但在分母中只有$x$的因式时,第一种方法会更快。

\Paragraph{情况三。}当分母的因式中有一些被提升到某个幂次时,必须考虑到可能存在以该因式的各个幂次为分母的部分分式,直到最高幂次。例如,在分解分数$\dfrac{3x^2-2x+1}{(x+1)^2(x-2)}$时,必须考虑到可能存在分母为$x+1$、$(x+1)^2$和$(x-2)$的部分分式。

然而,可能会认为,由于分母为$(x+1)^2$的分数的分子可能包含$x$的项,因此必须在分子中写入$Ax+B$,以便
\[
\frac{3x^2 - 2x + 1}{(x+1)^2(x-2)}
  = \frac{Ax+B}{(x+1)^2} + \frac{C}{x+1} + \frac{D}{x-2}.
\]
然而,如果我们尝试在这种情况下找到$A$、$B$、$C$和$D$,我们会失败,因为我们得到四个未知数;而我们只有三个关系连接它们,然而
\[
\frac{3x^2 - 2x + 1}{(x+1)^2(x-2)}
  = \frac{x-1}{(x+1)^2} + \frac{1}{x+1} + \frac{1}{x-2}.
\]

但如果我们写
\[
\frac{3x^2 - 2x + 1}{(x+1)^2(x-2)}
  = \frac{A}{(x+1)^2} + \frac{B}{x+1} + \frac{C}{x-2},
\]
我们得到
\[
3x^2 - 2x+1 = A(x-2) + B(x+1)(x-2) + C(x+1)^2,
\]
\DPPageSep{140.png}{128}%
这给出$C=1$当$x=2$时。将$C$替换为其值,移项,合并同类项并除以$x-2$,我们得到$-2x= A+B(x+1)$,这给出$A=-2$当$x=-1$时。将$A$替换为其值,我们得到
\[
2x = -2+B(x+1).
\]

因此$B=2$;所以部分分式为:
\[
\frac{2}{x+1} - \frac{2}{(x+1)^2} + \frac{1}{x-2},
\]
而不是上面所说的$\dfrac{1}{x+1} + \dfrac{x-1}{(x+1)^2} + \dfrac{1}{x-2}$,这是从$\dfrac{3x^2-2x+1}{(x+1)^2(x-2)}$得到的分式。如果我们注意到$\dfrac{x-1}{(x+1)^2}$本身可以分解为两个分式$\dfrac{1}{x+1} - \dfrac{2}{(x+1)^2}$,那么这三个分式实际上等价于
\[
\frac{1}{x+1} + \frac{1}{x+1} - \frac{2}{(x+1)^2} + \frac{1}{x-2}
  = \frac{2}{x+1} - \frac{2}{(x+1)^2} + \frac{1}{x-2},
\]
这就是得到的部分分式。

我们看到,只需在每个分子中允许一个数值项,我们总是能得到最终的部分分式。

然而,当分母中有$x^2$的因式的幂次时,相应的分子必须具有$Ax+B$的形式;例如,
\[
\frac{3x-1}{(2x^2-1)^2(x+1)}
  = \frac{Ax+B}{(2x^2-1)^2} + \frac{Cx+D}{2x^2-1} + \frac{E}{x+1},
\]
\DPPageSep{141.png}{129}%
这给出
\[%[** TN: 原文中分两行设置]
3x - 1 = (Ax + B)(x + 1)
       + (Cx + D)(x + 1)(2x^2 - 1) + E(2x^2 - 1)^2.
\]

对于 $x = -1$,得到 $E = -4$。代入、移项、合并同类项并除以 $x + 1$,
我们得到
\[
16x^3 - 16x^2 + 3 = 2Cx^3 + 2Dx^2 + x(A - C) + (B - D).
\]

因此 $2C = 16$ 且 $C = 8$;$2D = -16$ 且 $D = -8$;
$A - C = 0$ 或 $A - 8 = 0$ 且 $A = 8$,最后,$B - D = 3$
或 $B = -5$。于是我们得到部分分式为:
\[
\frac{(8x - 5)}{(2x^2 - 1)^2} + \frac{8(x - 1)}{2x^2 - 1} - \frac{4}{x + 1}.
\]

检查所得结果是有用的。最简单的方法是用一个值,例如 $x = +1$,
同时代入原表达式和得到的部分分式中。

每当分母只包含一个因子的幂时,有一个非常快速的方法如下:

例如,取 $\dfrac{4x + 1}{(x + 1)^3}$,令 $x + 1 = z$;则
$x = z - 1$。

代入后,我们得到
\[
\frac{4(z - 1) + 1}{z^3} = \frac{4z - 3}{z^3} = \frac{4}{z^2} - \frac{3}{z^3}.
\]

因此,部分分式为
\[
\frac{4}{(x + 1)^2} - \frac{3}{(x + 1)^3}.
\]
\DPPageSep{142.png}{130}%

应用于微分。设需要对 $y = \dfrac{5-4x}{6x^2 + 7x - 3}$ 进行微分;我们有
\begin{align*}
\frac{dy}{dx}
  &= -\frac{(6x^2+7x-3) × 4 + (5 - 4x)(12x + 7)}{(6x^2 + 7x - 3)^2}\\
  &=  \frac{24x^2 - 60x - 23}{(6x^2 + 7x - 3)^2}.
\end{align*}

如果我们将给定的表达式分解为
\[
\frac{1}{3x-1} - \frac{2}{2x+3},
\]
我们得到,然而,
\[
\frac{dy}{dx} = -\frac{3}{(3x-1)^2} + \frac{4}{(2x+3)^2},
\]
这实际上与上述结果相同,只是分解为部分分式。但如果在微分后进行分解,会更加复杂,这一点很容易看出。当我们处理这类表达式的\emph{积分}时,我们会发现将它们分解为部分分式是一个宝贵的辅助工具(见 \Pageref{partfracs})。

\Exercises{XI}(答案见 \Pageref[page]{AnsEx:XI})

分解为分式:
\begin{Problems}[2]
\Item{(1)} $\dfrac{3x + 5}{(x - 3)(x + 4)}$.
\Item{(2)} $\dfrac{3x - 4}{(x - 1)(x - 2)}$.
\ResetCols{2}

\Item{(3)} $\dfrac{3x + 5}{x^2 + x - 12}$.
\Item{(4)} $\dfrac{x + 1}{x^2 - 7x + 12}$.
\ResetCols{2}

\Item{(5)} $\dfrac{x - 8}{(2x + 3)(3x - 2)}$.
\Item{(6)} $\dfrac{x^2 - 13x + 26}{(x - 2)(x - 3)(x - 4)}$.
\ResetCols{1}
\DPPageSep{143.png}{131}%

\Item{(7)} $\dfrac{x^2 - 3x + 1}{(x - 1)(x + 2)(x - 3)}$.

\Item{(8)} $\dfrac{5x^2 + 7x + 1}{(2x + 1)(3x - 2)(3x + 1)}$.
\ResetCols{2}

\Item{(9)} $\dfrac{x^2}{x^3 - 1}$.
\Item{(10)} $\dfrac{x^4 + 1}{x^3 + 1}$.
\ResetCols{2}

\Item{(11)} $\dfrac{5x^2 + 6x + 4}{(x +1)(x^2 + x + 1)}$.
\Item{(12)} $\dfrac{x}{(x - 1)(x - 2)^2}$.
\ResetCols{2}

\Item{(13)} $\dfrac{x}{(x^2 - 1)(x + 1)}$.
\Item{(14)} $\dfrac{x + 3}{ (x +2)^2(x - 1)}$.
\ResetCols{2}

\Item{(15)} $\dfrac{3x^2 + 2x + 1}{(x + 2)(x^2 + x + 1)^2}$.
\Item{(16)} $\dfrac{5x^2 + 8x - 12}{(x + 4)^3}$.
\ResetCols{2}

\Item{(17)} $\dfrac{7x^2 + 9x - 1}{(3x - 2)^4}$.
\Item{(18)} $\dfrac{x^2}{(x^3 - 8)(x - 2)}$.
\end{Problems}

\Section{逆函数的微分。}

考虑函数(见 \Pageref{function})$y = 3x$;它可以表示为 $x = \dfrac{y}{3}$;这种形式称为原函数的\emph{逆函数}。

如果 $y = 3x$,则 $\dfrac{dy}{dx} = 3$;如果 $x = \dfrac{y}{3}$,则 $\dfrac{dx}{dy} = \dfrac{1}{3}$,我们发现
\[
\frac{dy}{dx} = \frac{1}{\ \dfrac{dx}{dy}\ }\quad \text{或}\quad
\frac{dy}{dx} × \frac{dx}{dy} = 1.
\]

考虑 $y = 4x^2$,$\dfrac{dy}{dx} = 8x$;逆函数为
\[
x = \frac{y^{\efrac{1}{2}}}{2},\quad \text{且}\quad
\frac{dx}{dy} = \frac{1}{4\sqrt{y}} = \frac{1}{4 × 2x} = \frac{1}{8x}.
\]
\DPPageSep{144.png}{132}%
%
\begin{DPalign*}
\lintertext{\indent 这里再次}
\frac{dy}{dx}×\frac{dx}{dy} &= 1.
\end{DPalign*}

可以证明,对于所有可以表示为逆形式函数,总可以写成
\[
\frac{dy}{dx} × \frac{dx}{dy} = 1\quad \text{或}\quad
\frac{dy}{dx} = \frac{1}{\ \dfrac{dx}{dy}\ }.
\]

由此可见,给定一个函数,如果求其反函数的导数更为简便,则可以先求反函数的导数,再取其倒数,即可得到原函数的导数。

举例来说,假设我们希望求导函数 \( y = \sqrt[2]{\dfrac{3}{x} - 1} \)。我们已经知道一种方法,即设 \( u = \dfrac{3}{x} - 1 \),然后求 \( \dfrac{dy}{du} \) 和 \( \dfrac{du}{dx} \)。这样得到
\[
\frac{dy}{dx} = -\frac{3}{2x^2\sqrt{\dfrac{3}{x} -1}}.
\]

如果我们忘记了这种方法,或者希望通过其他途径验证结果,或者由于其他原因无法使用普通方法,我们可以按如下步骤进行:反函数为 \( x = \dfrac{3}{1 + y^2} \)。
\[
\frac{dx}{dy} = -\frac{3 \times 2y}{(1 + y^2)^2} = -\frac{6y}{(1 + y^2)^2};
\]
因此
\[
\frac{dy}{dx} = \frac{1}{\ \dfrac{dx}{dy}\ }
  = -\frac{(1 + y^2)^2}{6y}
  = -\frac{\left(1 + \dfrac{3}{x} - 1\right)^2}{6 \times \sqrt[2]{\dfrac{3}{x} - 1}}
  = -\frac{3}{2x^2\sqrt{\dfrac{3}{x} - 1}}.
\]

再举一个例子,设 \( y = \dfrac{1}{\sqrt[3]{\theta + 5}} \)。

反函数为 \( \theta = \dfrac{1}{y^3} - 5 \) 或 \( \theta = y^{-3} - 5 \),且
\[
\frac{d\theta}{dy} = -3y^{-4} = -3\sqrt[3]{(\theta + 5)^4}.
\]

由此可得 \( \dfrac{dy}{dx} = -\dfrac{1}{3\sqrt{(\theta + 5)^4}} \),这也可以通过其他方法求得。

我们将在后续发现这一技巧极为有用;同时,建议您通过验证练习 I(第 \Pageref{Ex:I} 页)中的第 5、6、7 题,例题(第 \Pageref{ExNo1} 页)中的第 1、2、4 题,以及练习 VI(第 \Pageref{Ex:VI} 页)中的第 1、2、3 和 4 题,来熟悉这一方法。

\tb

通过本章及前述内容,您定会意识到,微积分在许多方面更像是一门\emph{艺术}而非\emph{科学}:如同其他艺术一样,唯有通过实践方能掌握。因此,您应多做练习,并自行设置题目,检验是否能独立解答,直至通过反复使用熟悉各种技巧。

\Chapter[关于真实复利的计算]{第十四章}{关于真实复利与有机增长规律}

\First{假设}有一个量以如下方式增长:在给定时间内,其增长的增量始终与其自身的大小成正比。这类似于以固定利率计算利息的过程;因为资本越大,在给定时间内产生的利息也越多。

现在我们必须明确区分两种情况,计算时根据所采用的方式是算术书中所谓的“单利”还是“复利”。因为在单利情况下,资本保持不变,而在复利情况下,利息被加入资本,因此资本通过连续累加而增加。

\Paragraph{{\upshape(1)}~单利计算。}考虑一个具体例子。设初始资本为 £100,利率为每年 10\%。那么,资本所有者每年将获得 £10 的利息。假设他每年提取利息,并将其存放在袜子中或锁在保险箱里。那么,如果他持续 10 年,到第 10 年末,他将收到 10 次 £10 的利息,总计 £100,加上原始的 £100,总共为 £200。他的财产将在 10 年内翻倍。如果利率为 5\%,他需要存 20 年才能使财产翻倍。如果利率仅为 2\%,他则需要存 50 年。显然,如果年利息为资本的 \( \dfrac{1}{n} \),他必须存 \( n \) 年才能使财产翻倍。

或者,如果 \( y \) 是原始资本,而每年的利息是 \( \dfrac{y}{n} \),那么在 \( n \) 年后,他的财产将是
\[
y + n\dfrac{y}{n} = 2y.
\]

%[** TN: 以下有几个小数错误已修正]
\Paragraph{{\upshape(2)}~复利计算。} 与之前一样,假设所有者
\Pagelabel{erratum0}%
以 £100 的资本开始,年利率为 10\%;但这次不是将利息存起来,而是每年将利息加到资本中,使得资本逐年增长。那么,在第一年末,资本将增长到 £110;在第二年(仍为 10\%),这将产生 £11 的利息。他将以 £121 开始第三年,该资本的利息将是 £12.2s;因此他将以 £133.2s 开始第四年,依此类推。很容易计算出,在十年末,总资本将增长到 £259.7s.6d。事实上,我们看到在每年末,每英镑将产生 \( \tfrac{1}{10} \) 英镑的利息,因此,如果这总是被加到资本中,每年资本将乘以 \( \tfrac{11}{10} \);如果持续十年(这将使这个因子乘以十次),将使原始资本乘以 \( \DPtypo{2.59375}{2.59374} \)。让我们用符号表示这一点。设 \( y_0 \) 为原始资本;\( \dfrac{1}{n} \) 为每次操作中增加的部分;\( y_n \) 为第 \( n \) 次操作结束时的资本价值。那么
\[
y_n = y_0\left(1 + \frac{1}{n}\right)^n.
\]

但这种每年计算一次复利的方式实际上并不完全公平;因为在第一年期间,£100 应该已经在增长。在半年末,它至少应该是 £105,如果第二半年的利息是基于 £105 计算的,那会更公平。这相当于称之为 5\% 每半年;因此,有 20 次操作,每次资本乘以 \( \tfrac{21}{20} \)。如果这样计算,十年末资本将增长到
\DPtypo{£265.8s.}
       {£265.6s.7d.};因为
\[
(1 + \tfrac{1}{20})^{20} = \DPtypo{2.654}{2.653}.
\]

然而,即使如此,这个过程仍然不完全公平;因为在第一个月末,将会有一些利息产生;而半年一次的计算假设资本在六个月内保持不变。假设我们将一年分成 10 个部分,并计算每年每个十分之一的 1\% 利息。我们现在有 100 次操作持续十年;即
\[
y_n = £100 \left( 1 + \tfrac{1}{100} \right)^{100};
\]
这计算结果为
\DPtypo{£$270$.~$8$\textit{s}.}
       {£$270$.~$9$\textit{s}.~$7\frac{1}{2}$\textit{d}.}

这还不是最终结果。将十年分成 $1000$ 个时期,每个时期为 $\frac{1}{100}$~ 年;利息为每个时期 $\frac{1}{10}$~\%;那么
\[
y_n = £100 \left( 1 + \tfrac{1}{1000} \right)^{1000};
\]
这计算结果为
\DPtypo{£$271$.~$14$\textit{s}.~$2\frac{1}{2}$\textit{d}.}
       {£$271$.~$13$\textit{s}.~$10$\textit{d}.}

更细致地划分,将十年分成 10,000 个部分,每个部分为 \( \frac{1}{1000} \) 年,利息为 \( \frac{1}{100} \) 的 1\%。那么
\[
y_n = £100 \left( 1 + \tfrac{1}{10,000} \right)^{10,000};
\]
这相当于
\DPtypo{£$271$.~$16$\textit{s}.~$4$\textit{d}.}
       {£$271$.~$16$\textit{s}.~$3\frac{1}{2}$\textit{d}.}

最后,我们会发现,我们试图寻找的实际上是表达式
$\left(1 + \dfrac{1}{n}\right)^n$ 的极限值,正如我们所见,它大于 $2$;并且,随着我们取越来越大的 $n$,这个表达式会越来越接近一个特定的极限值。无论 $n$ 取多大,这个表达式的值都会越来越接近于
\[
2.71828\ldots
\]
这个数字是**永远不应被遗忘的**。

让我们通过几何图形来说明这些概念。在图36中,$OP$ 代表初始值。$OT$ 是值增长的总时间。它被分为10个周期,每个周期内有一个相等的增长步长。这里 $\dfrac{dy}{dx}$ 是一个常数;如果每个增长步长是初始 $OP$ 的 $\frac{1}{10}$,那么经过10个这样的步长后,高度将翻倍。如果我们取20个步长,每个步长是图中所示高度的一半,最终高度仍然会翻倍。或者取 $n$ 个步长,每个步长是初始高度 $OP$ 的 $\dfrac{1}{n}$,也足以使高度翻倍。这是简单利息的情况。这里,1增长到2。

在图37中,我们展示了相应的几何级数增长示意图。每个连续的纵坐标是 $1 + \dfrac{1}{n}$,即 $\dfrac{n+1}{n}$ 倍于其前一个纵坐标的高度。增长步长并不相等,因为每个增长步长现在是曲线在该部分的纵坐标的 $\dfrac{1}{n}$。如果我们严格地取10个步长,乘数因子为 $\left(1 + \frac{1}{10} \right)$,最终总量将是
$(1 + \tfrac{1}{10})^{10}$ 或 $\DPtypo{2.593}{2.594}$ 倍于原始的1。但如果我们取足够大的 $n$(以及相应的足够小的 $\dfrac{1}{n}$),那么最终值 $\left(1 + \dfrac{1}{n}\right)^n$ 将使1增长到 $2.71828$。

\Paragraph{Epsilon.} 对于这个神秘的数字 $2.7182818$ 等等,数学家们用希腊字母 $\epsilon$(发音为**epsilon**)作为符号。所有学童都知道希腊字母 $\pi$(称为**pi**)代表 $3.141592$ 等等;但有多少人知道**epsilon**代表 $2.71828$ 呢?然而,它是一个比 $\pi$ 更重要的数字!

那么,**epsilon** 是什么?

假设我们让1以简单利息增长到2;然后,如果在相同的利率和相同的时间内,我们让1以真正的复利增长,而不是简单利息,它将增长到**epsilon**的值。

这种在每一瞬间按当前量成比例增长的过程,有些人称之为**对数增长率**。单位对数增长率是指在单位时间内使1增长到 $2.718281$ 的增长率。它也可以称为**有机增长率**:因为在某些情况下,有机体的增长特征是,在给定时间内有机体的增量与有机体本身的大小成正比。

如果我们以100\%作为利率单位,并以任何固定周期作为时间单位,那么让1以单位利率**算术增长**,在单位时间内,结果将是2,而让1以单位利率**对数增长**,在相同时间内,结果将是 $2.71828\ldots$。

\Paragraph{关于Epsilon的更多内容} 我们已经知道,
\SetOddHead{有机增长的规律}
我们需要知道当$n$变得无限大时,表达式$\left(1 + \dfrac{1}{n}\right)^n$的值是多少。从算术上讲,这里列出了许多值(任何人都可以借助对数表计算出来),假设$n = 2$;$n = 5$;$n = 10$;等等,直到$n = 10,000$。
\begin{alignat*}{2}
&(1 + \tfrac{1}{2})^2             &&= 2.25.    \\
&(1 + \tfrac{1}{5})^5             &&= \DPtypo{2.489}{2.488}.   \\
&(1 + \tfrac{1}{10})^{10}         &&= 2.594.   \\
&(1 + \tfrac{1}{20})^{20}         &&= 2.653.   \\
&(1 + \tfrac{1}{100})^{100}       &&= \DPtypo{2.704}{2.705}.   \\
&(1 + \tfrac{1}{1000})^{1000}     &&= \DPtypo{2.7171}{2.7169}.  \\
&(1 + \tfrac{1}{10,000})^{10,000} &&= \DPtypo{2.7182}{2.7181}.
\end{alignat*}
\DPPageSep{153.png}{141}%

然而,找到另一种计算这个极其重要的数值的方法是值得的。

因此,我们将利用二项式定理,以这种众所周知的方式展开表达式$\left(1 + \dfrac{1}{n}\right)^n$。

二项式定理\Pagelabel{binomtheo}给出了如下规则:
\begin{align*}
(a + b)^n &= a^n + n \dfrac{a^{n-1} b}{1!} + n(n - 1) \dfrac{a^{n-2} b^2}{2!} \\
  & \phantom{= a^n\ } + n(n -1)(n - 2) \dfrac{a^{n-3} b^3}{3!} + \text{等等}. \\
\intertext{设$a = 1$和$b = \dfrac{1}{n}$,我们得到}
\left(1 + \dfrac{1}{n}\right)^n
  &= 1 + 1 + \dfrac{1}{2!} \left(\dfrac{n - 1}{n}\right) + \dfrac{1}{3!} \dfrac{(n - 1)(n - 2)}{n^2} \\
  &\phantom{= 1 + 1\ } + \dfrac{1}{4!} \dfrac{(n - 1)(n - 2)(n - 3)}{n^3} + \text{等等}.
\end{align*}

现在,如果我们假设$n$变得无限大,比如十亿,或十亿个十亿,那么$n - 1$,$n - 2$,和$n - 3$,等等,都将近似等于$n$;然后级数变为
\[
\epsilon = 1 + 1 + \dfrac{1}{2!} + \dfrac{1}{3!} + \dfrac{1}{4!} + \text{等等}.\ldots
\]

通过取这个快速收敛的级数到我们想要的任意多项,我们可以将和计算到所需的任何精度。以下是计算十项的结果:
\DPPageSep{154.png}{142}%
\begin{center}
\begin{tabular}{@{}r<{\qquad}@{}l@{}}
                & $1.000000$ \\
除以~$1$ & $1.000000$ \\
除以~$2$ & $0.500000$ \\
除以~$3$ & $0.166667$ \\
除以~$4$ & $0.041667$ \\
除以~$5$ & $0.008333$ \\
除以~$6$ & $0.001389$ \\
除以~$7$ & $0.000198$ \\
除以~$8$ & $0.000025$ \\
除以~$9$ & $0.000002$ \\
\cline{2-2}
\multicolumn{2}{r@{}}{总计\quad $2.718281$} \\
\cline{2-2}
\end{tabular}
\end{center}

$\epsilon$与$1$不可公度,类似于$\pi$,是一个无止境的非循环小数。

\Paragraph{指数级数} 我们将需要另一个级数。

让我们再次利用二项式定理,展开表达式$\left(1 + \dfrac{1}{n}\right)^{nx}$,当$n$变得无限大时,这与$\epsilon^x$相同。
\begin{align*}
\epsilon^x
  &= 1^{nx} + nx \frac{1^{nx-1} \left(\dfrac{1}{n}\right)}{1!}
            + nx(nx - 1) \frac{1^{nx - 2} \left(\dfrac{1}{n}\right)^2}{2!} \\
  & \phantom{= 1^{nx}\ }
    + nx(nx - 1)(nx - 2) \frac{1^{nx-3} \left(\dfrac{1}{n}\right)^3}{3!}
    + \text{等等}.\\
  &= 1 + x + \frac{1}{2!} · \frac{n^2x^2 - nx}{n^2}
    + \frac{1}{3!} · \frac{n^3x^3 - 3n^2x^2 + 2nx}{n^3} + \text{等等}. \\
%\DPPageSep{155.png}{143}%
  &= 1 + x + \frac{x^2 -\dfrac{x}{n}}{2!}
    + \frac{x^3 - \dfrac{3x^2}{n} + \dfrac{2x}{n^2}}{3!} + \text{等等}.
\end{align*}

但是,当$n$变得无限大时,这简化为以下形式:
\[
\epsilon^x
  = 1 + x + \frac{x^2}{2!} + \frac{x^3}{3!} + \frac{x^4}{4!} + \text{等等.}\dots
\]

这个级数被称为\emph{指数级数}。

$\epsilon$ 之所以被认为重要,主要原因在于 $\epsilon^x$ 具有一个特性,即当对其求导时,其值保持不变\Pagelabel{unchanged};换句话说,其导数与自身相同。这一点可以通过对 $x$ 求导立即看出,如下所示:
\begin{DPalign*}
\frac{d(\epsilon^x)}{dx}
  &= 0 + 1 + \frac{2x}{1 · 2} + \frac{3x^2}{1 · 2 · 3} + \frac{4x^3}{1 · 2 · 3 · 4} \\
&\phantom{= 0 + 1 + \frac{2x}{1 · 2} + \frac{3x^2}{1 · 2 · 3}\ } + \frac{5x^4}{1 · 2 · 3 · 4 · 5} + \text{等等}.  \\
\lintertext{或者}
  &= 1 + x + \frac{x^2}{1 · 2} + \frac{x^3}{1 · 2 · 3} + \frac{x^4}{1 · 2 · 3 · 4} + \text{等等}.,
\end{DPalign*}
这与原级数完全相同。

现在,我们也可以从另一个角度出发,并说:去吧;\DPnote{** 译注:[原文如此],推测为古旧表达}让我们寻找一个函数 $x$,使得其导数与自身相同。或者,是否存在一个仅涉及 $x$ 的幂的表达式,其在求导后保持不变?因此,让我们假设一个一般表达式为
\begin{align*}
y &= A + Bx + Cx^2 + Dx^3 + Ex^4 + \text{等等}.,\\
\intertext{(其中系数 $A$, $B$, $C$ 等需要确定),并对其求导。}
\dfrac{dy}{dx} &= B + 2Cx + 3Dx^2 + 4Ex^3 + \text{等等}.
\end{align*}

现在,如果这个新表达式确实要与原表达式相同,显然有
$A$ 必须 $=B$;$C=\dfrac{B}{2}=\dfrac{A}{1· 2}$;$D = \dfrac{C}{3} = \dfrac{A}{1 · 2 · 3}$;
$E = \dfrac{D}{4} = \dfrac{A}{1 · 2 · 3 · 4}$,等等。

因此,变化规律为\Strut
\[
y = A\left(1 + \dfrac{x}{1} + \dfrac{x^2}{1 · 2} + \dfrac{x^3}{1 · 2 · 3} + \dfrac{x^4}{1 · 2 · 3 · 4} + \text{等等}.\right).
\]

如果现在我们取 $A = 1$ 以进一步简化,则有
\[
y = 1 + \dfrac{x}{1} + \dfrac{x^2}{1 · 2} + \dfrac{x^3}{1 · 2 · 3} + \dfrac{x^4}{1 · 2 · 3 · 4} + \text{等等}.
\]

对其进行任意次数的求导,总会得到相同的级数。

如果现在我们取 $A=1$ 的特例,并计算该级数,我们将得到
\begin{align*}
\text{当 } x &= 1,\quad & y &= 2.718281 \text{ 等等};    & \text{即 } y &= \epsilon;   \\
\text{当 } x &= 2,\quad & y &=(2.718281 \text{ 等等})^2; & \text{即 } y &= \epsilon^2; \\
\text{当 } x &= 3,\quad & y &=(2.718281 \text{ 等等})^3; & \text{即 } y &= \epsilon^3;
\end{align*}
\DPPageSep{157.png}{145}%
因此
\[
\text{当 } x=x,\quad y=(2.718281 \text{ 等等}.)^x;\quad\text{即 } y=\epsilon^x,
\]
从而最终证明了
\[
\epsilon^x = 1 + \dfrac{x}{1} + \dfrac{x^2}{1·2} + \dfrac{x^3}{1· 2· 3} + \dfrac{x^4}{1· 2· 3· 4} + \text{等等}.
\]

[\textsc{注}.---\textit{如何读指数}. 为了帮助那些没有导师在身边的人,说明 $\epsilon^x$ 读作“\emph{epsilon 的 eksth 次方}”;或者有些人读作“\emph{指数 eks}”。因此,$\epsilon^{pt}$ 读作“\emph{epsilon 的 pee-teeth-power}”或“\emph{指数 pee tee}”。举一些类似的例子:---例如,$\epsilon^{-2}$ 读作“\emph{epsilon 的负二次方}”或“\emph{指数负二}”。$\epsilon^{-ax}$ 读作“\emph{epsilon 的负 ay-eksth}”或“\emph{指数负 ay-eks}”。]

当然,$\epsilon^y$ 在对其关于 $y$ 求导时保持不变。同样,$\epsilon^{ax}$,即 $(\epsilon^a)^x$,当对其关于 $x$ 求导时,将得到 $a\epsilon^{ax}$,因为 $a$ 是一个常数。

\Subsection{自然对数或纳皮尔对数。}
$\epsilon$ 之所以重要的另一个原因是,它是由对数的发明者纳皮尔作为其系统的基底而制定的。如果 $y$ 是 $\epsilon^x$ 的值,那么 $x$ 就是以 $\epsilon$ 为底 $y$ 的\emph{对数}。或者,如果
\begin{DPalign*}
                  y &= \epsilon^x, \\
\lintertext{那么} x &= \log_\epsilon y.
\end{DPalign*}

图 \Figs{38}{和}{39} 中绘制的两条曲线代表了这些方程。
\DPPageSep{158.png}{146}%

计算的点如下:
\begin{align*}
\text{对于图 \textsc{图}~38} \left\{
\begin{array}{|c||*{5}{c|}}
\hline
\Strut
\Td[c]{x} & \Td[c]{0} & \Td[l]{0.5}  & \Td[l]{1}    & \Td[l]{1.5}  & \Td[l]{2} \\
\hline
\Strut
\Td[c]{y} & \Td[c]{1} & \Td[l]{1.65} & \Td[l]{2.71} & \Td[l]{4.50} & \Td[l]{\DPtypo{7.69}{7.39}} \\
\hline
\end{array}
\right. \\
\text{对于图 \textsc{图}~39} \left\{
\begin{array}{|c||*{5}{c|}}
\hline
\Strut
\Td[c]{y} & \Td[c]{1} & \Td[l]{2}    & \Td[l]{3}    & \Td[l]{4}    & \Td[l]{8} \\
\hline
\Strut
\Td[c]{x} & \Td[c]{0} & \Td[l]{0.69} & \Td[l]{1.10} & \Td[l]{1.39} & \Td[l]{2.08} \\
\hline
\end{array}
\right.
\end{align*}%
%[** TN: 原文中的图表顺序颠倒;已交换。]
\Figures{158b}{158a}{38}{39}\Pagelabel{erratum1}

可以看出,尽管计算得出的绘图点不同,但结果是相同的。这两个方程实际上表示的是同一个意思。

由于许多使用以 $10$ 为底的对数(而不是以 $\epsilon$ 为底)的人对“自然”对数不熟悉,因此有必要简要说明一下。普通对数的加法规则仍然适用,即
\[
\log_\epsilon a + \log_\epsilon b = \log_\epsilon ab.
\]
幂的规则也适用;
\[
n × \log_\epsilon a = \log_\epsilon a^n.
\]
\DPPageSep{159.png}{147}%
但由于 $10$ 不再是基底,因此不能通过简单地在对数上加 $2$ 或 $3$ 来乘以 $100$ 或 $1000$。可以通过将自然对数乘以 $0.4343$ 来将其转换为普通对数;或者
\begin{DPalign*}
\log_{10} x &= 0.4343 × \log_{\epsilon} x, \\
\lintertext{反之,}
\log_{\epsilon} x &= 2.3026 × \log_{10} x.
\end{DPalign*}

%[** TN: 允许此材料浮动以改善页面布局。]
\begin{table}[hp]
\centering
\textsc{“纳皮尔对数”实用表} \\
(也称为自然对数或双曲对数)
\[
\begin{array}{c<{\quad}|>{\ }c>{\qquad\qquad}cr<{\quad}|>{\ }c}
\multicolumn{1}{c}{\text{\footnotesize 数字}} &
\multicolumn{1}{c}{\footnotesize\log_{\epsilon}} &&
\multicolumn{1}{c}{\text{\footnotesize 数字}} &
\multicolumn{1}{c}{\footnotesize\log_{\epsilon}}\DPnote{** TN: 原文使用 "Log"} \\
\Strut\PadTo[l]{1.1}{1}
    & 0.0000 &&      6 & 1.7918 \\
1.1 & 0.0953 &&      7 & 1.9459 \\
1.2 & 0.1823 &&      8 & 2.0794 \\
1.5 & 0.4055 &&      9 & 2.1972 \\
1.7 & 0.5306 &&     10 & 2.3026 \\
2.0 & 0.6931 &&     20 & 2.9957 \\
2.2 & 0.7885 &&     50 & 3.9120 \\
2.5 & 0.9163 &&    100 & 4.6052 \\
2.7 & 0.9933 &&    200 & 5.2983 \\
2.8 & 1.0296 &&    500 & 6.2146 \\
3.0 & 1.0986 &&  1,000 & 6.9078 \\
3.5 & 1.2528 &&  2,000 & \DPtypo{7.6010}{7.6009} \\
4.0 & 1.3863 &&  5,000 & 8.5172 \\
4.5 & 1.5041 && 10,000 & \DPtypo{9.2104}{9.2103} \\
5.0 & 1.6094 && 20,000 & 9.9035 \\
\end{array}
\]
\end{table}

\Subsection{指数和对数方程。}\Pagelabel{expolo}
现在让我们尝试对包含对数或指数的某些表达式进行微分。

取方程:
\[
y = \log_\epsilon x.
\]
首先将其转换为
\[
\epsilon^y = x,
\]
\DPPageSep{160.png}{148}%
因此,由于 $\epsilon^y$ 关于 $y$ 的微分是其原函数不变(见 \Pageref{unchanged}),
\[
\frac{dx}{dy} = \epsilon^y,
\]
并且,从反函数回到原函数,
\[
\frac{dy}{dx}
  = \frac{1}{\ \dfrac{dx}{dy}\ }
  = \frac{1}{\epsilon^y}
  = \frac{1}{x}.
\]

这是一个非常有趣的结果。可以写成\Pagelabel{differlog}
\[
\frac{d(\log_\epsilon x)}{dx} = x^{-1}.
\]

注意,$x^{-1}$ 是一个我们无法通过幂函数的求导规则得到的结果。该规则(见\Pageref[page]{multipow}页)是将幂次乘以该幂次,并将幂次减去1。因此,对 $x^3$ 求导得到 $3x^2$;对 $x^2$ 求导得到 $2x^1$。但对 $x^0$ 求导并不会得到 $x^{-1}$ 或 $0 \times x^{-1}$,因为 $x^0$ 本身等于1,是一个常数。我们将在关于积分的章节中回到这个有趣的事实,即对 $\log_\epsilon x$ 求导得到 $\dfrac{1}{x}$。

\tb

现在,尝试对以下函数求导
\begin{DPalign*}
                              y &= \log_\epsilon(x+a),\\
\lintertext{即} \epsilon^y &= x+a;
\end{DPalign*}
我们有 $\dfrac{d(x+a)}{dy} = \epsilon^y$,因为 $\epsilon^y$ 的微分仍然是 $\epsilon^y$。
\DPPageSep{161.png}{149}%
%
\BindMath{\begin{DPalign*}
\lintertext{\indent 这给出}
\frac{dx}{dy} &= \epsilon^y = x+a; \\
\intertext{因此,回到原函数(见\Pageref{section:3}),我们得到}
\frac{dy}{dx} &= \frac{1}{\;\dfrac{dx}{dy}\;} = \frac{1}{x+a}.
\end{DPalign*}\Pagelabel{differ2}%
\tb
\begin{DPalign*}
\lintertext{\indent 接下来尝试}
y &= \log_{10} x.
\end{DPalign*}}

首先通过乘以模数 $0.4343$ 转换为自然对数。这给出
\begin{DPalign*}
y &= 0.4343 \log_\epsilon x; \\
\lintertext{因此}
\frac{dy}{dx} &= \frac{0.4343}{x}.
\end{DPalign*}

\tb

下一个问题并不那么简单。尝试这个:\Pagelabel{diffexp}
\[
y = a^x.
\]

对两边取对数,我们得到
\begin{DPalign*}
\log_\epsilon y &= x \log_\epsilon a, \\
\lintertext{或}
x  = \frac{\log_\epsilon y}{\log_\epsilon a}
  &= \frac{1}{\log_\epsilon a} × \log_\epsilon y.
\end{DPalign*}

由于 $\dfrac{1}{\log_\epsilon a}$ 是一个常数,我们得到
\[
\frac{dx}{dy}
  = \frac{1}{\log_\epsilon a} × \frac{1}{y}
  = \frac{1}{a^x × \log_\epsilon a};
\]
因此,回到原函数。
\[
\frac{dy}{dx} = \frac{1}{\;\dfrac{dx}{dy}\;} = a^x × \log_\epsilon a.
\]
\DPPageSep{162.png}{150}%

我们看到,由于
\[
\frac{dx}{dy} × \frac{dy}{dx} =1\quad\text{且}\quad
\frac{dx}{dy} = \frac{1}{y} × \frac{1}{\log_\epsilon a},\quad
\frac{1}{y} × \frac{dy}{dx} = \log_\epsilon a.
\]

我们将发现,每当有一个表达式如 $\log_\epsilon y =$ 某个 $x$ 的函数时,我们总是有
$\dfrac{1}{y}\, \dfrac{dy}{dx} =$ 该 $x$ 函数的导数,因此我们可以直接从 $\log_\epsilon y = x \log_\epsilon a$ 写出
\[
\frac{1}{y}\, \frac{dy}{dx}
  = \log_\epsilon a\quad\text{且}\quad
\frac{dy}{dx} = a^x \log_\epsilon a.
%[ **"/" 假定][ **F1 - 此注释位于 1/y 之后]
\]

\tb

现在让我们尝试更多的例子。

\Examples.
(1) $y=\epsilon^{-ax}$。设 $-ax=z$;则 $y=\epsilon^z$。
\[
\frac{dy}{dx} = \epsilon^z;\quad
\frac{dz}{dx} = -a;\quad\text{因此}\quad
\frac{dy}{dx} = -a\epsilon^{-ax}.
\]

或者这样:
\[
\log_\epsilon y = -ax;\quad
\frac{1}{y}\, \frac{dy}{dx} = -a;\quad
\frac{dy}{dx} = -ay = -a\epsilon^{-ax}.
\]

(2) $y=\epsilon^{\efrac{x^2}{3}}$。设 $\dfrac{x^2}{3}=z$;则 $y=\epsilon^z$。
\[
\frac{dy}{dz} = \epsilon^z;\quad
\frac{dz}{dx} = \frac{2x}{3};\quad
\frac{dy}{dx} = \frac{2x}{3}\, \epsilon^{\efrac{x^2}{3}}.
\]

或者这样:
\[
\log_\epsilon y = \frac{x^2}{3};\quad
\frac{1}{y}\, \frac{dy}{dx} = \frac{2x}{3};\quad
\frac{dy}{dx} = \frac{2x}{3}\, \epsilon^{\efrac{x^2}{3}}.
\]
\DPPageSep{163.png}{151}%

(3) $y = \epsilon^{\efrac{2x}{x+1}}$。
\begin{DPalign*}
\log_\epsilon y &= \frac{2x}{x+1},\quad
\frac{1}{y}\, \frac{dy}{dx} = \frac{2(x+1)-2x}{(x+1)^2}; \\
\lintertext{因此}
\frac{dy}{dx} &= \frac{2}{(x+1)^2} \epsilon^{\efrac{2x}{x+1}}.
\end{DPalign*}

通过设 $\dfrac{2x}{x+1}=z$ 进行验证。

(4) $y=\epsilon^{\sqrt{x^2+a}}$.\quad $\log_\epsilon y=(x^2+a)^{\efrac{1}{2}}$.
\[
\frac{1}{y}\, \frac{dy}{dx} = \frac{x}{(x^2+a)^{\efrac{1}{2}}}\quad\text{且}\quad
\frac{dy}{dx} = \frac{x × \epsilon^{\sqrt{x^2+a}}}{(x^2+a)^{\efrac{1}{2}}}.
\]
\DPchg{(}{}因为如果 $(x^2+a)^{\efrac{1}{2}}=u$ 且 $x^2+a=v$,$u=v^{\efrac{1}{2}}$,
\[
\frac{du}{dv} = \frac{1}{{2v}^{\efrac{1}{2}}};\quad
\frac{dv}{dx} = 2x;\quad
\frac{du}{dx} = \frac{x}{\DPtypo{(x^2+)a}{(x^2+a)}^{\efrac{1}{2}}}.\DPchg{)}{}
\]

通过设 $\sqrt{x^2+a}=z$ 来验证。

(5) $y=\log(a+x^3)$。设 $(a+x^3)=z$;则 $y=\log_\epsilon z$。
\[
\frac{dy}{dz} = \frac{1}{z};\quad
\frac{dz}{dx} = 3x^2;\quad\text{因此}\quad
\frac{dy}{dx} = \frac{3x^2}{a+x^3}.
\]

(6) $y=\log_\epsilon\{{3x^2+\sqrt{a+x^2}}\}$。设 $3x^2 + \sqrt{a+x^2}=z$;
则 $y=\log_\epsilon z$。
\begin{align*}
\frac{dy}{dz}
  &= \frac{1}{z};\quad \frac{dz}{dx} = 6x + \frac{x}{\sqrt{x^2+a}}; \\
\frac{dy}{dx}
  &= \frac{6x + \dfrac{x}{\sqrt{x^2+a}}}{3x^2 + \sqrt{a+x^2}}
   = \frac{x(1 + 6\sqrt{x^2+a})}{(3x^2 + \sqrt{x^2+a}) \sqrt{x^2+a}}.
\end{align*}
\DPPageSep{164.png}{152}%

(7) $y=(x+3)^2 \sqrt{x-2}$。
\begin{align*}
\log_\epsilon y
  &= 2 \log_\epsilon(x+3)+ \tfrac{1}{2} \log_\epsilon(x-2). \\
\frac{1}{y}\, \frac{dy}{dx}
  &= \frac{2}{(x+3)} + \frac{1}{2(x-2)}; \\
\frac{dy}{dx}
  &= (x+3)^2 \sqrt{x-2} \left\{\frac{2}{x+3} + \frac{1}{2(x-2)}\right\}.
\end{align*}

(8) $y=(x^2+3)^3(x^3-2)^{\efrac{2}{3}}$。
\begin{align*}
\log_\epsilon y
  &= 3 \log_\epsilon(x^2+3) + \tfrac{2}{3} \log_\epsilon(x^3-2); \\
\frac{1}{y}\, \frac{dy}{dx}
  &= 3 \frac{2x}{(x^2+3)} + \frac{2}{3} \frac{3x^2}{x^3-2}
   = \frac{6x}{x^2+3} + \frac{2x^2}{x^3-2}.
\end{align*}
\DPchg{(}{}因为如果 $y=\log_\epsilon(x^2+3)$,设 $x^2+3=z$ 且 $u=\log_\epsilon z$。
\[
\frac{du}{dz} = \frac{1}{z};\quad
\frac{dz}{dx} = 2x;\quad
\frac{du}{dx} = \frac{2x}{x^2+3}.
\]
类似地,如果 $v=\log_\epsilon(x^3-2)$,$\dfrac{dv}{dx} = \dfrac{3x^2}{x^3-2}$\DPchg{)}{} 且
\[
\frac{dy}{dx}
  = (x^2+3)^3(x^3-2)^{\efrac{2}{3}}
    \left\{ \frac{6x}{x^2+3} + \frac{2x^2}{x^3-2} \right\}.
\]

(9) $y=\dfrac{\sqrt[2]{x^2+a}}{\sqrt[3]{x^3-a}}$。
\begin{DPalign*}
\log_\epsilon y
  &= \frac{1}{2} \log_\epsilon(x^2+a) - \frac{1}{3} \log_\epsilon(x^3-a). \\
\frac{1}{y}\, \frac{dy}{dx}
  &= \frac{1}{2}\, \frac{2x}{x^2+a} - \frac{1}{3}\, \frac{3x^2}{x^3-a}
   = \frac{x}{x^2+a} - \frac{x^2}{x^3-a} \\
\lintertext{且}
\frac{dy}{dx}
  &= \frac{\sqrt[2]{x^2+a}}{\sqrt[3]{x^3-a}}
     \left\{ \frac{x}{x^2+a} - \frac{x^2}{x^3-a} \right\}.
\end{DPalign*}
\DPPageSep{165.png}{153}%

(10) $y=\dfrac{1}{\log_\epsilon x}$
\[
\frac{dy}{dx}
  = \frac{\log_\epsilon x × 0 - 1 × \dfrac{1}{x}}
         {\log_\epsilon^2 x}
  = -\frac{1}{x \log_\epsilon^2x}.
\]

(11) $y=\sqrt[3]{\log_\epsilon x} = (\log_\epsilon x)^{\efrac{1}{3}}$。设 $z=\log_\epsilon x$;$y=z^{\efrac{1}{3}}$。
\[
\frac{dy}{dz} = \frac{1}{3} z^{-\efrac{2}{3}};\quad
\frac{dz}{dx} = \frac{1}{x};\quad
\frac{dy}{dx} = \frac{1}{3x \sqrt[3]{\log_\epsilon^2 x}}.
\]

(12) $y=\left(\dfrac{1}{a^x}\right)^{ax}$。
\begin{DPalign*}
\log_\epsilon y
  &= ax(\log_\epsilon 1 - \log_\epsilon a^x) = -ax \log_\epsilon a^x. \\
\frac{1}{y}\, \frac{dy}{dx}
  &= -ax × a^x \log_\epsilon a - a \log_\epsilon a^x. \displaybreak[1] \\
\lintertext{且}
\frac{dy}{dx}
  &= -\left(\frac{1}{a^x}\right)^{ax}
      (x × a^{x+1} \log_\epsilon a + a \log_\epsilon a^x).
\end{DPalign*}

现在尝试以下练习。

\Exercises{XII} (答案见第\Pageref[page]{AnsEx:XII}页)
\begin{Problems}
\Item{(1)} 求导 $y=b(\epsilon^{ax} -\epsilon^{-ax})$。

\Item{(2)} 求表达式 $u=at^2+2\log_\epsilon t$ 关于 $t$ 的导数。

\Item{(3)} 若 $y=n^t$,求 $\dfrac{d(\log_\epsilon y)}{dt}$。

\Item{(4)} 证明若 $y=\dfrac{1}{b}·\dfrac{a^{bx}}{\log_\epsilon a}$,则 $\dfrac{dy}{dx}=a^{bx}$。

\Item{(5)} 若 $w=pv^n$,求 $\dfrac{dw}{dv}$。
\end{Problems}
\DPPageSep{166.png}{154}%



\begin{Problems}[2]
\Item{(6)} $y=\log_\epsilon x^n$。
\Item{(7)} $y=3\epsilon^{-\efrac{x}{x-1}}$。

\ResetCols{2}
\Item{(8)} $y=(3x^2+1)\epsilon^{-5x}$。
\Item{(9)} $y=\log_\epsilon(x^a+a)$。

\ResetCols{1}
\Item{(10)} $y=(3x^2-1)(\sqrt{x}+1)$。

\ResetCols{2}
\Item{(11)} $y=\dfrac{\log_\epsilon(x+3)}{x+3}$。
\Item{(12)} $y=a^x × x^a$。

\ResetCols{1}
\Item{(13)} 开尔文勋爵曾指出,通过海底电缆的信号传输速度取决于电缆芯的外径与其中铜线直径之比的值。如果将此比值称为~$y$,则每分钟可以发送的信号数~$s$ 可以用公式表示为
\[
s=ay^2 \log_\epsilon \frac{1}{y};
\]
其中 $a$ 是一个依赖于电缆长度和材料质量的常数。证明如果这些参数已知,$s$~将在 $y=1 ÷ \sqrt{\epsilon}$ 时达到最大值。

\Item{(14)} 求
\[
y=x^3-\log_\epsilon x
\]
的最大值或最小值。

\Item{(15)} 求导 $y=\log_\epsilon(ax\epsilon^x)$。

\Item{(16)} 求导 $y=(\log_\epsilon ax)^3$。
\end{Problems}
\tb

\Section{对数曲线。}

让我们回到那些纵坐标成几何级数的曲线,例如由方程 $y=bp^x$ 表示的曲线。

通过令 $x=0$,我们可以看出 $b$ 是~$y$ 的初始高度。

然后当
\[
x=1,\quad y=bp;\qquad
x=2,\quad y=bp^2;\qquad
x=3,\quad y=bp^3,\quad \text{等等}
\]
\DPPageSep{167.png}{155}%

此外,我们看到 $p$ 是任意纵坐标与前一个纵坐标高度之比的数值。在\Fig{40}中,我们取 $p$ 为~$\frac{6}{5}$;每个纵坐标的高度都是前一个的 $\frac{6}{5}$ 倍。

\Figures{167a}{167b}{40}{41}

如果两个连续的纵坐标以恒定比例相关联,它们的对数将具有恒定的差值;因此,如果我们绘制一条新的曲线,\Fig{41},以~$\log_\epsilon y$ 为纵坐标,它将是一条以等步长上升的直线。事实上,从方程可以得出
\begin{DPalign*}
\log_\epsilon y &= \log_\epsilon b + x · \log_\epsilon p, \\
\lintertext{因此}
\log_\epsilon y &- \log_\epsilon b = x · \log_\epsilon p。
\end{DPalign*}

现在,由于 $\log_\epsilon p$ 只是一个数字,可以写成 $\log_\epsilon p=a$,因此
\[
\log_\epsilon \frac{y}{b}=ax,
\]
方程变为新的形式
\[
y = b\epsilon^{ax}。
\]
\DPPageSep{168.png}{156}%

\Section{衰减曲线。}

如果我们取 $p$ 为一个真分数(小于1),曲线显然会向下趋近,如\Fig{42}所示,其中每个连续的纵坐标都是前一个高度的 $\frac{3}{4}$。

方程仍然是
\[
y=bp^x;
\]
\Figure[2.5in]{168a}{42}
但由于 $p$ 小于1,$\log_\epsilon p$ 将是一个负数,可以写成~$-a$;因此 $p=\epsilon^{-a}$,现在曲线的方程变为
\[
y=b\epsilon^{-ax}。
\]

这个表达式的重要性在于,在自变量为\emph{时间}的情况下,该方程表示了许多物理过程中某种量\emph{逐渐衰减}的过程。例如,热体的冷却(在牛顿著名的“冷却定律”中)由方程表示为
\[
\theta_t=\theta_0 \epsilon^{-at};
\]
\DPPageSep{169.png}{157}%
其中 $\theta_0$ 是热体相对于周围环境的初始温度超量,$\theta_t$~是时间~$t$ 结束时的温度超量,$a$~是一个常数——即衰减常数,取决于物体暴露的表面积及其导热系数和辐射系数等。

类似的公式,
\[
Q_t=Q_0 \epsilon^{-at},
\]
用于表示一个最初带有电荷~$Q_0$ 的带电体,在衰减常数~$a$ 的作用下逐渐漏电的情况;该常数在这种情况下取决于物体的电容和漏电路径的电阻。

给柔性弹簧施加的振动会在一段时间后消失;振幅的衰减可以用类似的方式表示。

事实上,$\epsilon^{-at}$ 作为所有那些现象的**衰减因子**,其中衰减率与衰减量的幅度成正比;或者在我们通常的符号中,$\dfrac{dy}{dt}$ 在每一时刻都与 $y$ 在该时刻的值成正比。因为我们只需检查上图 \Fig{42},就可以看到,在曲线的每一点上,斜率 $\dfrac{dy}{dx}$ 与高度 $y$ 成正比;随着 $y$ 变小,曲线变得更平坦。用符号表示,即
\begin{DPgather*}
y=b\epsilon^{-ax}\\
\DPPageSep{170.png}{158}%
\lintertext{或}
\log_\epsilon y
  = \log_\epsilon b - ax \log_\epsilon \epsilon
  = \log_\epsilon b - ax,\\
\lintertext{\rlap{并且,微分后,}}
\frac{1}{y}\, \frac{dy}{dx} = -a;\\
\lintertext{因此} \frac{dy}{dx} = b\epsilon^{-ax} × (-a) = -ay;
\end{DPgather*}
或者,用文字表示,曲线的斜率向下,并且与 $y$ 和常数 $a$ 成正比。

如果我们以形式
\begin{DPalign*}
y &= bp^x; \\
\lintertext{那么}
\frac{dy}{dx}
  &= bp^x × \log_\epsilon p. \\
\lintertext{\indent 但}
\log_\epsilon p &= -a; \\
\lintertext{给出我们}
\frac{dy}{dx} &= y × (-a) = -ay,
\end{DPalign*}
如前所述。

\Paragraph{时间常数。} 在“衰减因子”$\epsilon^{-at}$ 的表达式中,量 $a$ 是另一个称为“**时间常数**”的量的倒数,我们可以用符号 $T$ 表示。那么衰减因子将写为 $\epsilon^{-\efrac{t}{T}}$;通过令 $t = T$ 可以看出,$T$(或 $\dfrac{1}{a}$)的含义是,这是原量(在前面的例子中称为 $\theta_0$ 或 $Q_0$)衰减到其原始值的 $\dfrac{1}{\epsilon}$ 分之一——即 $0.3678$——所需的时间长度。
\DPPageSep{171.png}{159}%

$\epsilon^x$ 和 $\epsilon^{-x}$ 的值在物理学的不同分支中经常需要,由于它们在很少的数学表格中给出,为了方便,这里列出了一些值。

%[** TN: 允许表格浮动;重新表述前面的句子]
\begin{table}[p]
\Pagelabel{littletable}%
\[
\setlength{\arraycolsep}{1.5em}% [** 硬编码长度]
\begin{array}{| .{2,2} | .{5,4} | .{1,6} | .{1,6} |}
\hline
\multicolumn{1}{|c|}{\Strut x} &
  \multicolumn{1}{c|}{\epsilon^x} &
  \multicolumn{1}{c|}{\epsilon^{-x}} &
  \multicolumn{1}{c|}{1-\epsilon^{-x}} \\
\hline
\Strut
0.00   &     1.0000 & 1.0000   & 0.0000   \\
0.10   &     1.1052 & 0.9048   & 0.0952   \\
0.20   &     1.2214 & 0.8187   & 0.1813   \\
0.50   &     1.6487 & 0.6065   & 0.3935   \\
0.75   &     2.1170 & 0.4724   & 0.5276   \\
0.90   &     2.4596 & 0.4066   & 0.5934   \\
1.00   &     2.7183 & 0.3679   & 0.6321   \\
1.10   &     3.0042 & 0.3329   & 0.6671   \\
1.20   &     3.3201 & 0.3012   & 0.6988   \\
1.25   &     3.4903 & 0.2865   & 0.7135   \\
1.50   &     4.4817 & 0.2231   & 0.7769   \\
1.75   &     \DPtypo{5.754}{5.755}  & 0.1738   & 0.8262   \\
2.00   &     7.389  & 0.1353   & 0.8647   \\
2.50   &    \DPtypo{12.183}{12.182}  & 0.0821   & 0.9179   \\
3.00   &    \DPtypo{20.085}{20.086}  & 0.0498   & 0.9502   \\
3.50   &    33.115  & 0.0302   & 0.9698   \\
4.00   &    54.598  & 0.0183   & 0.9817   \\
4.50   &    90.017  & 0.0111   & 0.9889   \\
5.00   &   148.41   & 0.0067   & 0.9933   \\
5.50   &   244.69   & 0.0041   & 0.9959   \\
6.00   &   403.43   & 0.00248  & 0.99752  \\
7.50   &  1808.04   & \DPtypo{0.00053}{0.00055}  & 0.99947  \\
10.00  & 22026.5    & 0.000045 & 0.999955 \\
\hline
\end{array}
\]
\end{table}

作为一个使用此表的示例,假设有一个正在冷却的热物体,实验开始时(即当 \( t = 0 \) 时),它比周围物体高出 \( 72^\circ \),并且其冷却的时间常数为 \( 20 \) 分钟(即,其温度超出部分在 \( 20 \) 分钟内降至 \( 72^\circ \) 的 \( \dfrac{1}{\epsilon} \) 部分),那么我们可以计算出在任意给定时间 \( t \) 时,温度超出部分会降至多少。例如,设 \( t \) 为 \( 60 \) 分钟。则 \( \dfrac{t}{T} = 60 ÷ 20 = 3 \),我们需要找到 \( \epsilon^{-3} \) 的值,然后将原始的 \( 72^\circ \) 乘以该值。表中显示 \( \epsilon^{-3} \) 为 \( 0.0498 \)。因此,在 \( 60 \) 分钟后,温度超出部分将降至 \( 72^\circ × 0.0498 = 3.586^\circ \)。

\tb

\clearpage%[** TN: 根据文本块大小决定分页]
\Examples{更多示例。}
(1) 在施加产生电流的电动势后,导体中电流的强度在时间 \( t \) 秒后的表达式为
\[
C = \dfrac{E}{R}\left\{1 - \epsilon^{-\efrac{Rt}{L}}\right\}。
\]
时间常数为 \( \dfrac{L}{R} \)。

若 \( E = 10 \),\( R = 1 \),\( L = 0.01 \);则当 \( t \) 非常大时,项 \( \epsilon^{-\efrac{Rt}{L}} \) 变为 \( 1 \),且 \( C = \dfrac{E}{R} = 10 \);同时
\[
\frac{L}{R} = T = 0.01。
\]

其在任意时间的值可写为:
\[
C = 10 - 10\epsilon^{-\efrac{t}{0.01}},
\]
时间常数为 \( 0.01 \)。这意味着变量项需要 \( 0.01 \) 秒才能下降其初始值 \( 10\epsilon^{-\efrac{0}{0.01}} = 10 \) 的 \( \dfrac{1}{\epsilon} = 0.3678 \) 部分。

要找出当 \( t = 0.001 \) 秒时电流的值,设 \( \dfrac{t}{T} = 0.1 \),\( \epsilon^{-0.1} = 0.9048 \)(根据表)。

因此,在 \( 0.001 \) 秒后,变量项为 \( 0.9048 × 10 = 9.048 \),实际电流为 \( 10 - 9.048 = 0.952 \)。

同样,在 \( 0.1 \) 秒结束时,
\[
\frac{t}{T} = 10;\quad \epsilon^{-10} = 0.000045;
\]
变量项为 \( 10 × 0.000045 = 0.00045 \),电流为 \( 9.9995 \)。

(2) 一束光穿过厚度为 \( l \) 厘米的某种透明介质后的强度 \( I \) 为 \( I = I_0\epsilon^{-Kl} \),其中 \( I_0 \) 是光束的初始强度,\( K \) 是“吸收常数”。

该常数通常通过实验确定。例如,若发现一束光在穿过 \( 10 \) 厘米的某种透明介质后,强度减少了 \( 18\% \),这意味着 \( 82 = 100 × \epsilon^{-K×10} \) 或 \( \epsilon^{-10K} = 0.82 \),根据表可知 \( 10K = 0.20 \) 近似;因此 \( K = 0.02 \)。

要找出使强度减半的厚度,必须找到满足等式 \( 50 = 100 × \epsilon^{-0.02l} \) 或 \( 0.5 = \epsilon^{-0.02l} \) 的 \( l \) 值。
\DPPageSep{174.png}{162}%
通过将其转化为对数形式,即
\[
\log 0.5 = -0.02 × l × \log \epsilon,
\]
得到
\[
l = \frac{\DPchg{\overset{-}{1}.6990}{-0.3010}}{-0.02 × 0.4343}
 = \DPtypo{34.5}{34.7}~\text{厘米近似}。
\]

(3) 已知未发生转化的放射性物质的量 \( Q \) 与初始量 \( Q_0 \) 之间的关系为 \( Q = Q_0 \epsilon^{-\lambda t} \),其中 \( \lambda \) 是常数,\( t \) 是自转化开始以来的时间(以秒计)。

对于“镭 \( A \)”,若时间以秒表示,实验表明 \( \lambda = 3.85 × 10^{-3} \)。求使物质转化一半所需的时间。(此时间称为物质的“平均寿命”。)

我们有 \( 0.5 = \epsilon^{-0.00385t} \)。\Pagelabel{erratum0a}%
\begin{DPalign*}
\log 0.5 &= -0.00385t × \log \epsilon; \\
\lintertext{且}
t &= 3\text{ 分钟近似}。
\end{DPalign*}

\Exercises{XIII} (答案见\Pageref[page]{AnsEx:XIII}。)
\begin{Problems}
\Item{(1)} 绘制曲线 $y = b \epsilon^{-\efrac{t}{T}}$;其中 $b = 12$,$T = 8$,
且 $t$ 取从 $0$ 到 $20$ 的各种值。

\Item{(2)} 若一热体冷却使得在 $24$ 分钟内其
温度超出量降至初始值的一半,推导时间常数,并求冷却至原始
超出量的 $1\%$ 所需的时间。
\DPPageSep{175.png}{163}%

\Item{(3)} 绘制曲线 $y = 100(1-\epsilon^{-2t})$。

\Item[XIII:4]{(4)} 以下方程给出非常相似的曲线:
\begin{align*}
\text{(i)}\   y &= \frac{ax}{x + b}; \\
\text{(ii)}\  y &= a(1 - \epsilon^{-\efrac{x}{b}}); \\
\text{(iii)}\ y &= \frac{a}{90°} \arctan \left(\frac{x}{b}\right).
\end{align*}

绘制所有三条曲线,取 $a= 100$ 毫米;
$b = 30$ 毫米。

\Item{(5)} 若
\[
(\textit{a})~y = x^x;\quad
(\textit{b})~y = (\epsilon^x)^x;\quad
(\textit{c})~y = \epsilon^{x^x}.
\]
求 $y$ 对 $x$ 的导数。

\Item{(6)} 对于“钍 $A$”,$\lambda$ 的值为 $5$;求“平均寿命”,即在表达式
\[
Q = Q_0 \epsilon^{-\lambda t};
\]
中,$Q$ 为初始量 $Q_0$ 的一半时所需的时间,$t$ 以秒计。

\Item{(7)} 容量 $K = 4 × 10^{-6}$ 的电容器,充电至电势 $V_0 = 20$,通过 $10,000$ 欧姆的电阻放电。求 (\textit{a}) $0.1$ 秒后;(\textit{b}) $0.01$ 秒后的电势 $V$,假设电势下降遵循规则 $V = V_0 \epsilon^{-\efrac{t}{KR}}$。

\Item{(8)} 一个绝缘金属球上的电荷 $Q$ 在 $10$ 分钟内从 $20$ 降至 $16$ 单位。若 $Q = Q_0 × \epsilon^{-\mu t}$;$Q_0$ 为初始电荷,$t$ 以秒计,求泄漏系数 $\mu$。进而求电荷泄漏一半所需的时间。
\DPPageSep{176.png}{164}%

\Item{(9)} 电话线路的阻尼可通过关系式 $i = i_0 \epsilon^{-\beta l}$ 确定,其中 $i$ 为初始强度为 $i_0$ 的电话电流在 $t$ 秒后的强度;$l$ 为线路长度,单位为公里,$\beta$ 为常数。对于 1910 年铺设的法英海底电缆,$\beta = 0.0114$。求电缆末端 ($40$ 公里) 的阻尼,以及 $i$ 仍为原始电流 $8\%$ 时的线路长度(良好听觉的极限值)。

\Item{(10)} 海拔 $h$ 公里处的大气压 $p$ 由 $p=p_0 \epsilon^{-kh}$ 给出;$p_0$ 为海平面压力 ($760$ 毫米)。

$10$、$20$ 和 $50$ 公里处的压力分别为 $199.2$、$42.2$、$0.32$,求各情况下的 $k$。使用 $k$ 的平均值,求各情况下的百分比误差。

\Item{(11)} 求 $y = x^x$ 的最小值或最大值。

\Item{(12)} 求 $y = x^{\efrac{1}{x}}$ 的最小值或最大值。

\Item{(13)} 求 $y = xa^{\efrac{1}{x}}$ 的最小值或最大值。
\end{Problems}
\DPPageSep{177.png}{165}%

\Chapter[正弦和余弦]{第十五章}{如何处理正弦和余弦}

\First{希腊}字母通常用于表示角度,我们将
使用字母 $\theta$(“theta”)作为任意变量角度的常用字母。

让我们考虑函数
\[
y= \sin \theta.
\]

\Figure{177a}{43}

我们需要研究的是 $\dfrac{d(\sin \theta)}{d \theta}$ 的值;
换句话说,如果角度~$\theta$ 发生变化,我们需要找出正弦的增量与角度增量之间的关系,这两个增量本身都是无限小的。观察图 \Fig{43},   %[ **","?]
其中,如果圆的半径为单位长度,高度~$y$ 就是正弦,$\theta$ 是角度。现在,如果 $\theta$ 被假设增加了一个小角度 $d \theta$——一个角度元素——高度~$y$,即正弦,将增加一个小的元素~$dy$。
新的高度~$y + dy$ 将是新角度 $\theta + d \theta$ 的正弦,或者用方程表示为
\[
y+dy = \sin(\theta + d \theta);
\]
从这减去第一个方程得到
\[
dy = \sin(\theta + d \theta)- \sin \theta.
\]

右边这个量是两个正弦之间的差,三角学书籍告诉我们如何计算这个差。它们告诉我们,如果 $M$ 和~$N$ 是两个不同的角度,
\[
\sin M - \sin N = 2 \cos\frac{M+N}{2}·\sin\frac{M-N}{2}.
\]

那么,如果我们设 $M= \theta + d \theta$ 为一个角度,$N= \theta$ 为另一个角度,我们可以写成
\begin{DPalign*}
dy &= 2 \cos\frac{\theta + d\theta + \theta}{2}
      · \sin\frac{\theta + d\theta - \theta}{2},\\
\lintertext{或者}
dy &= 2\cos(\theta + \tfrac{1}{2}d\theta)
     · \sin\tfrac{1}{2} d\theta.
\end{DPalign*}

但如果我们把 $d \theta$ 看作无限小,那么在极限情况下,我们可以忽略~$\frac{1}{2} d \theta$ 相对于~$\theta$,也可以认为 $\sin\frac{1}{2} d \theta$ 与~$\frac{1}{2} d \theta$ 相同。于是方程变为:\Pagelabel{differsin}
\begin{DPalign*}
dy &= 2 \cos \theta × \tfrac{1}{2} d \theta; \\
dy &= \cos \theta · d \theta, \\
\lintertext{最终得到}
\dfrac{dy}{d \theta} &= \cos \theta.
\end{DPalign*}
\DPPageSep{179.png}{167}%

附带的曲线,图 \Figs{44}{和}{45},按比例绘制了 $y=\sin \theta$ 和 $\dfrac{dy}{d\theta}=\cos\theta$ 的值,对应于~$\theta$ 的相应值。
%[** TN: 原文中的图表顺序颠倒;cos 标记为图 44,等等。]
\Figure[4in]{179a}{44}\Pagelabel{erratum2}
\Figure[4in]{179b}{45}
\tb
\DPPageSep{180.png}{168}%

接下来讨论余弦。\Pagelabel{differcos}

设 $y=\cos \theta$。

现在 $\cos \theta=\sin\left(\dfrac{\pi}{2}-\theta\right)$。

因此
\begin{align*}
&\begin{aligned}
dy = d\left(\sin\left(\frac{\pi}{2} - \theta\right)\right)
  &= \cos\left(\frac{\pi}{2} - \theta\right) × d(-\theta), \\
  &= \cos\left(\frac{\pi}{2} - \theta\right) × (-d\theta),
\end{aligned} \\
&\frac{dy}{d\theta} = -\cos\left(\frac{\pi}{2} - \theta\right).
\intertext{\indent 由此可得}
&\frac{dy}{d\theta} = -\sin \theta.
\end{align*}

\tb

最后,讨论正切。
%
\begin{DPalign*}
\lintertext{\indent 设}
y  &= \tan \theta, \\
dy &= \tan(\theta + d\theta) - \tan\theta. \\
\intertext{\indent 根据三角学书籍中的展开式,}
\tan(\theta + d\theta)
   &= \frac{\tan\theta + \tan d\theta}
           {1 - \tan\theta·\tan d\theta}; \\
\lintertext{因此}
dy &= \frac{\tan\theta + \tan d\theta}
           {1-\tan\theta·\tan d\theta} - \tan\theta \\
   &= \frac{(1 + \tan^2\theta)\tan d\theta}
           {1-\tan\theta·\tan d\theta}.
\end{DPalign*}
\DPPageSep{181.png}{169}%

现在记住,如果 $d\theta$ 无限减小,$\tan d\theta$ 的值与~$d\theta$ 相同,而 $\tan\theta · d\theta$ 相对于~$1$ 可以忽略不计,因此表达式简化为
\begin{DPalign*}
dy &= \frac{(1+\tan^2 \theta)\, d\theta}{1}, \\
\lintertext{所以}
\frac{dy}{d\theta} &= 1 + \tan^2\theta, \\
\lintertext{或者}
\frac{dy}{d\theta} &= \sec^2 \theta.
\end{DPalign*}

综合这些结果,我们得到:
\[
\begin{array}{|*{2}{>{\quad}c<{\quad}|}}
\hline
\DStrut y   & \dfrac{dy}{d\theta} \\
\hline
\Strut\sin\theta & \cos\theta \\
\cos\theta & -\sin\theta \\
\Strut\tan\theta & \sec^2 \theta\\
\hline
\end{array}
\]

有时,在机械和物理问题中,例如在简谐运动和波动中,我们需要处理与时间成正比增加的角度。因此,如果 \( T \) 是一个完整周期的时间,或者绕圆运动的时间,那么由于绕圆的角度是 \( 2\pi \) 弧度,或 \( 360^\circ \),在时间 \( t \) 内移动的角度量将是
\begin{DPalign*}
\theta &= 2\pi\frac{t}{T},\quad \text{以弧度表示,} \\
\lintertext{或}
\theta &= 360\frac{t}{T},\quad \text{以度表示。}
\end{DPalign*}
\DPPageSep{182.png}{170}%

如果频率,即每秒的周期数,用 \( n \) 表示,则 \( n = \dfrac{1}{T} \),我们可以写成:
\[
\theta=2\pi nt.
\]
那么我们将有
\[
y = \sin 2\pi nt.
\]

现在,如果我们想知道正弦函数随时间的变化情况,我们必须对时间 \( t \) 而不是对 \( \theta \) 进行微分。为此,我们必须采用第九章第\Pageref{chap:IX}页解释的技巧,并设
\[
\frac{dy}{dt} = \frac{dy}{d\theta} · \frac{d\theta}{dt}.
\]

显然,\( \dfrac{d\theta}{dt} \) 将是 \( 2\pi n \);因此
\begin{align*}
\frac{dy}{dt} &= \cos \theta × 2\pi n \\
              &= 2\pi n · \cos 2\pi nt. \\
\intertext{\indent 同样地,可以得出}
\frac{d(\cos 2\pi nt)}{dt} &= -2\pi n · \sin 2\pi nt.
\end{align*}

\Section{正弦或余弦的二阶导数。}

我们已经看到,当 \( \sin \theta \) 对 \( \theta \) 微分时,它变为 \( \cos \theta \);当 \( \cos \theta \) 对 \( \theta \) 微分时,它变为 \( -\sin \theta \);或者用符号表示,
\[
\frac{d^2(\DPtypo{\cos \theta}{\sin \theta})}{d\theta^2} = -\sin \theta.
\]
\DPPageSep{183.png}{171}%

因此,我们得到了一个有趣的结果:我们发现了一个函数,如果对其进行两次微分,我们得到的与开始时的函数相同,但符号从 \( +\) 变为 \( -\)。

同样的事情也适用于余弦函数;因为对 \( \cos\theta \) 微分得到 \( -\sin\theta \),而对 \( -\sin\theta \) 微分得到 \( -\cos\theta \);即:
\[
\frac{d^2(\cos\theta)}{d\theta^2} = -\cos\theta.
\]

\emph{正弦和余弦是唯一二阶导数等于原函数(且符号相反)的函数。}

\tb

\Examples.\Pagelabel{intex3}
通过我们目前所学的内容,我们现在可以对更复杂的表达式进行微分。

(1) \( y=\arcsin x \)。

如果 \( y \) 是正弦为 \( x \) 的弧,则 \( x = \sin y \)。
\[
\frac{dx}{dy}=\cos y.
\]

现在从反函数回到原函数,我们得到
\begin{DPalign*}
\frac{dy}{dx}
  &= \frac{1}{\;\dfrac{dx}{dy}\;} = \frac{1}{\cos y}. \\
\lintertext{\indent 现在}
\cos y
  &= \sqrt{1-\sin^2 y}=\sqrt{1-x^2}; \\
\lintertext{因此}
\frac{dy}{dx}
  &= \frac{1}{\sqrt{1-x^2}},
\end{DPalign*}
这是一个出乎意料的结果\DPtypo{}{。}
\DPPageSep{184.png}{172}%

(2) \( y=\cos^3 \theta \)。

这等同于 \( y=(\cos \theta)^3 \)。

设 \( \cos\theta=v \);则 \( y=v^3 \);\( \dfrac{dy}{dv}=3v^2 \)。
\begin{align*}
\frac{dv}{d\theta} &= -\sin\theta.\\
\frac{dy}{d\theta} &=  \frac{dy}{dv} × \frac{dv}{d\theta}
                    = -3 \cos^2 \theta \sin\theta.
\end{align*}

(3) \( y=\sin(x+a) \)。

设 \( x+a=v \);则 \( y=\sin v \)。
\[
\frac{dy}{dv}=\cos v;\qquad
\frac{dv}{dx}=1 \quad\text{且}\quad
\frac{dy}{dx}=\cos(x+a).
\]

(4) \( y=\log_\epsilon \sin \theta \)。

设 \( \sin\theta=v \);\( y=\log_\epsilon v \)。
\begin{align*}
\frac{dy}{dv}      &= \frac{1}{v};\quad \frac{dv}{d\theta}=\cos\theta;\\
\frac{dy}{d\theta} &= \frac{1}{\sin\theta} × \cos\theta = \cot\theta.
\end{align*}

(5) \( y = \cot \theta = \dfrac{\cos \theta}{\sin \theta} \)。
\begin{align*}
\frac{dy}{d\theta}
  &= \frac{-\sin^2\theta - \cos^2 \theta}{\sin^2 \theta}\\
  &= -(1+\cot^2 \theta) = -\cosec^2 \theta。
\end{align*}

(6) \( y = \tan 3\theta \)。

设 \( 3\theta = v \);\( y = \tan v \);\( \dfrac{dy}{dv} = \sec^2 v \)。
\[
\frac{dv}{d\theta} = 3;\quad
\frac{dy}{d\theta} = 3 \sec^2 3\theta。
\]
\DPPageSep{185.png}{173}%

(7) \( y = \sqrt{1+3\tan^2\theta} \);\( y = (1+3 \tan^2 \theta)^{\efrac{1}{2}} \)。

设 \( 3\tan^2\theta = v \)。
\begin{DPalign*}
y &= (1+v)^{\efrac{1}{2}};\quad
\frac{dy}{dv} = \frac{1}{2\sqrt{1+v}} \text{(见 \Pageref{ExNo1});} \\
%
\frac{dv}{d\theta}
  &= 6\tan\theta \sec^2 \theta \\
%
\intertext{(因为,若 \( \tan \theta = u \),}
v &= 3u^2;\quad \frac{dv}{du} = 6u;\quad \frac{du}{d\theta} = \sec^2 \theta; \\
%
\lintertext{因此}
\frac{dv}{d\theta}
  &= 6 \DPtypo{}{(}\tan \theta \sec^2 \theta) \\
\lintertext{因此}
%
\frac{dy}{d\theta}
  &= \frac{6\tan\theta \sec^2\theta}{2\sqrt{1 + 3\tan^2\theta}}。
\end{DPalign*}

(8) \( y = \sin x \cos x \)。\Pagelabel{example1}
\begin{align*}
\frac{dy}{dx}
  &= \sin x(-\sin x) + \cos x × \cos x \\
  &= \cos^2 x - \sin^2 x。
\end{align*}

\Exercises{XIV}(答案见 \Pageref[page]{AnsEx:XIV})
\begin{Problems}
\Item{(1)} 求下列函数的导数:
\begin{align*}
\text{(i)}\quad   y &= A \sin\left(\theta - \frac{\pi}{2}\right)。\\
\text{(ii)}\quad  y &= \sin^2 \theta;\quad \text{以及 } y = \sin 2\theta。\\
\text{(iii)}\quad y &= \sin^3 \theta;\quad \text{以及 } y = \sin 3\theta。
\end{align*}

\Item{(2)} 求使 \( \sin\theta × \cos\theta \) 达到最大值的 \( \theta \) 值。

\Item{(3)} 求 \( y = \dfrac{1}{2\pi} \cos 2\pi nt \) 的导数。
\DPPageSep{186.png}{174}%

\Item{(4)} 若 \( y = \sin a^x \),求 \( \dfrac{dy}{dx} \)。

\Item{(5)} 求 \( y = \log_\epsilon \cos x \) 的导数。

\Item{(6)} 求 \( y = 18.2 \sin(x+26°) \) 的导数。

\Item{(7)} 绘制曲线 \( y = 100 \sin(\theta-15°) \);并证明在 \( \theta = 75° \) 时,曲线的斜率为最大斜率的一半。

\Item{(8)} 若 \( y = \sin \theta·\sin 2\theta \),求 \( \dfrac{dy}{d\theta} \)。

\Item{(9)} 若 \( y = a·\tan^m(\theta^n) \),求 \( y \) 对 \( \theta \) 的导数。

\Item{(10)} 求 \( y = \epsilon^x \sin^2 x \) 的导数。

\Item{(11)} 求练习 XIII(\Pageref{XIII:4})第 4 题中三个方程的导数,并比较它们的导数,看在 \( x \) 非常小、非常大或接近 30 时是否相等或近似相等。

\Item{(12)} 求下列函数的导数:
\begin{align*}%[** TN: Reformatted in two columns]
\text{(i)}\quad   y &= \sec x。    &
\text{(ii)}\quad  y &= \arccos x。 \\
\text{(iii)}\quad y &= \arctan x。 &
\text{(iv)}\quad  y &= \arcsec x。 \\
\text{(v)}\quad   y &= \tan x × \sqrt{3 \sec x}。 &&
\end{align*}

\Item{(13)} 求 \( y = \sin(2\theta +3)^{2.3} \) 的导数。

\Item{(14)} 求 \( y = \theta^3 + 3 \sin(\theta+3) - 3^{\sin \theta} - 3^\theta \) 的导数。

\Item{(15)} 求 \( y = \theta \cos \theta \) 的极大值或极小值。
\end{Problems}
\DPPageSep{187.png}{175}%

\Chapter{第十六章}{偏导数}

\First{我们}有时会遇到依赖于多个自变量的函数。例如,我们可能会发现 \( y \) 依赖于另外两个变量,我们称之为 \( u \) 和 \( v \)。用符号表示为\Pagelabel{partialdiff}
\begin{DPalign*}
y &= f(u, v)。 \\
\intertext{取一个最简单的具体例子。}
\lintertext{设 }  y &= u×v。 \\
\intertext{我们该怎么做?如果我们把 \( v \) 当作常数,对 \( u \) 求导,我们会得到}
dy_v &= v\, du; \\
\intertext{或者如果我们把 \( u \) 当作常数,对 \( v \) 求导,我们会得到:}
dy_u &= u\, dv。
\end{DPalign*}

这里用作下标的小写字母是为了表明在运算中哪个量被视为常数。

另一种表示微分仅是\emph{部分}进行的方式,即仅对\emph{一个}自变量进行微分,是用希腊字母$\partial$代替小写字母$d$来书写微分系数。这样,
\begin{align*}
\DStrut
\frac{\partial y}{\partial u} &= v, \\
\frac{\partial y}{\partial v} &= u.
\end{align*}

如果我们分别代入$v$和$u$的这些值,我们将得到
\[
\left.
\begin{aligned}
\DStrut
dy_v &= \frac{\partial y}{\partial u}\, du, \\
dy_u &= \frac{\partial y}{\partial v}\, dv,
\end{aligned} \right\}
\quad\text{这些是\emph{偏微分}。}
\]

但是,如果你仔细思考,你会发现$y$的总变化同时依赖于\emph{这两者}。也就是说,如果两者都在变化,实际的$dy$应该写成
\[
dy = \frac{\partial y}{\partial u}\, du + \dfrac{\partial y}{\partial v}\, dv;
\]
这被称为\emph{全微分}。在某些书中,它被写成$dy = \left(\dfrac{dy}{du}\right)\, du + \left(\dfrac{dy}{dv}\right)\, dv$。

\textbf{例(1)}。求表达式$w = 2ax^2 + 3bxy + 4cy^3$的偏微分系数。答案是:
\[
\left.
\begin{aligned}
\frac{\partial w}{\partial x} &= 4ax + 3by. \\
\frac{\partial w}{\partial y} &= 3bx + 12cy^2.
\end{aligned} \right\}
\]
第一个是在假设$y$为常数的情况下得到的,第二个是在假设$x$为常数的情况下得到的;然后
\[
dw = (4ax+3by)\, dx + (3bx+12cy^2)\, dy.
\]

\textbf{例(2)}。设$z = x^y$。然后,首先将$y$视为常数,然后将$x$视为常数,我们以通常的方式得到
\[
\left.
\begin{aligned}
\dfrac{\partial z}{\partial x} &= yx^{y-1}, \\
\dfrac{\partial z}{\partial y} &= x^y × \log_\epsilon x,
\end{aligned}\right\}
\]
因此$dz = yx^{y-1}\, dx + x^y \log_\epsilon x \, dy$。

\textbf{例(3)}。一个高度为$h$、底面半径为$r$的圆锥的体积为$V=\frac{1}{3} \pi r^2 h$。如果高度保持不变,而$r$变化,体积相对于半径的变化率与高度变化而半径保持不变时体积相对于高度的变化率不同,因为
\[
\left.
\begin{aligned}
\frac{\partial V}{\partial r} &= \dfrac{2\pi}{3} rh, \\
\frac{\partial V}{\partial h} &= \dfrac{\pi}{3} r^2.
\end{aligned}\right\}
\]

当半径和高度同时变化时,变化量为$dV = \dfrac{2\pi}{3} rh\, dV + \dfrac{\pi}{3} r^2\, dh$。

\Example{(4).} \Pagelabel{Example4} 在下面的例子中,$F$ 和 $f$ 表示任意形式的两个函数。
例如,它们可以是正弦函数、指数函数,或者是两个独立变量 $t$ 和 $x$ 的简单代数函数。理解了这一点后,我们取表达式
\begin{DPalign*}
                    y &= F(x+at) + f(x-at), \\
\lintertext{或者,}   y &= F(w) + f(v); \\
\lintertext{其中} w &= x+at,\quad \text{且}\quad v = x-at. \\
\lintertext{\indent 那么} \frac{\partial y}{\partial x}
                      &= \frac{\partial F(w)}{\partial w} · \frac{\partial w}{\partial x}
                       + \frac{\partial f(v)}{\partial v} · \frac{\partial v}{\partial x} \\
                      &= F'(w) · 1 + f'(v) · 1
\intertext{(其中数字 $1$ 只是 $w$ 和 $v$ 中 $x$ 的系数);}
\lintertext{并且}     \frac{\partial^2 y}{\partial x^2}
                      &= F''(w) + f''(v). && \\
\lintertext{\indent 同样} \frac{\partial y}{\partial t}
                      &= \frac{\partial F(w)}{\partial w} · \frac{\partial w}{\partial t}
                      + \frac{\partial f(v)}{\partial v} · \frac{\partial v}{\partial t} \\
                      &= F'(w) · a - f'(v) a; \\
\lintertext{并且}     \frac{\partial^2 y}{\partial t^2}
                      &= F''(w)a^2 + f''(v)a^2; \\
\lintertext{因此}  \frac{\partial^2 y}{\partial t^2}
                      &= a^2\, \frac{\partial^2 y}{\partial x^2}.
\end{DPalign*}

这个微分方程在数学物理中具有极其重要的意义。

\Section{函数的极值与两个独立变量}

\Example{(5).} 让我们再次讨论练习 IX.,\Pageref{Ex9No4},第 4 题。

设 $x$ 和 $y$ 为绳子的两段长度。第三段为 $30-(x+y)$,三角形的面积为 $A = \sqrt{s(s-x)(s-y)(s-30+x+y)}$,其中 $s$ 是半周长,即 $15$,因此 $A = \sqrt{15P}$,其中
\begin{align*}%[** TN: 原文中此公式居中]
P &= (15-x)(15-y)(x+y-15) \\
  &= xy^2 + x^2y - 15x^2 - 15y^2 - 45xy + 450x + 450y - 3375.
\end{align*}

显然,当 $P$ 达到最大值时,$A$ 也达到最大值。
\[
dP = \dfrac{\partial P}{\partial x}\, dx + \dfrac{\partial P}{\partial y}\, dy.
\]
为了达到最大值(显然在此情况下不会是最小值),必须同时满足
\BindMath{%
\[
\dfrac{\partial P}{\partial x} = 0 \quad\text{且}\quad
\dfrac{\partial P}{\partial y} = 0;
\]
\begin{DPalign*}
%[** 无法在不扩大右大括号的情况下使方程顶部对齐]
\lintertext{\raisebox{0.5\baselineskip}{即,}}
\left.
\begin{aligned}
2xy - 30x + y^2 - 45y + 450 &= 0, \\
2xy - 30y + x^2 - 45x + 450 &= 0.
\end{aligned}
\right\}
\end{DPalign*}}%

一个直接的解是 $x=y$。

如果现在我们引入这个条件到 $P$ 的表达式中,我们得到
\[
P = (15-x)^2 (2x-15) = 2x^3 - 75x^2 + 900x - 3375.
\]
为了达到最大值或最小值,$\dfrac{dP}{dx} = 6x^2 - 150x + 900 = 0$,
这给出 $x=15$ 或 $x=10$。

显然,$x=15$ 给出最小面积;$x=10$ 给出最大面积,因为 $\dfrac{d^2 P}{dx^2} = 12x - 150$,当 $x=15$ 时为 $+30$,当 $x=10$ 时为 $-30$。

\Example{(6).} 求普通铁路煤车(矩形端面)的尺寸,使得在给定体积 $V$ 的情况下,侧面积与底面积之和最小。
\DPPageSep{192.png}{180}%

该车是一个顶部开口的矩形箱子。设 $x$ 为长度,$y$ 为宽度;则深度为 $\dfrac{V}{xy}$。表面积为 $S=xy + \dfrac{2V}{x} + \dfrac{2V}{y}$。
\[
dS = \frac{\partial S}{\partial x}\, dx
   + \frac{\partial S}{\partial y}\, dy
   = \left(y - \frac{2V}{x^2}\right) dx
   + \left(x - \frac{2V}{y^2}\right) dy.
\]
为了达到最小值(显然在此情况下不会是最大值),
\[
y - \frac{2V}{x^2} = 0,\quad
x - \frac{2V}{y^2} = 0.
\]

这里同样,一个直接的解是 $x = y$,因此
$S = x^2 + \dfrac{4V}{x}$,$\dfrac{dS}{dx}= 2x - \dfrac{4V}{x^2} =0$ 为最小值,且
\[
x = \sqrt[3]{2V}.
\]

\Exercises{XV}(答案见\Pageref[page]{AnsEx:XV}。)
\begin{Problems}
\Item{(1)} 对表达式 $\dfrac{x^3}{3} - 2x^3y - 2y^2x + \dfrac{y}{3}$ 分别关于 $x$ 和 $y$ 求偏导。

\Item{(2)} 对表达式
\[
x^2yz + xy^2z + xyz^2 + x^2y^2z^2
\]
分别关于 $x$、$y$ 和 $z$ 求偏导。

\Item{(3)} 设 $r^2 = (x-a)^2 + (y-b)^2 + (z-c)^2$。

求 $\dfrac{\partial r}{\partial x} + \dfrac{\partial r}{\partial y} + \dfrac{\partial r}{\partial z}$ 的值,以及
$\dfrac{\partial^2r}{\partial x^2} + \dfrac{\partial^2r}{\partial y^2} + \dfrac{\partial^2r}{\partial z^2}$ 的值。

\Item{(4)} 求 $y=u^v$ 的全微分。
\DPPageSep{193.png}{181}%

\Item{(5)} 求 $y=u^3 \sin v$;$y = (\sin x)^u$;以及 $y = \dfrac{\log_\epsilon u}{v}$ 的全微分。

\Item{(6)} 验证三个量 $x$、$y$、$z$ 的乘积为常数 $k$ 时,当这三个量相等时,它们的和最大。

\Item{(7)} 求函数
\[
u = x + 2xy + y
\]
的极大值或极小值。

\Item{(8)} 邮局规定,包裹的长度加围长不得超过 6 英尺。求在(a)矩形截面包裹;(b)圆形截面包裹的情况下,可以邮寄的最大体积。

\Item{(9)} 将 $\pi$ 分成 3 部分,使得其正弦的连乘积为极大或极小。

\Item{(10)} 求 $u = \dfrac{\epsilon^{x+y}}{xy}$ 的极大值或极小值。

\Item{(11)} 求
\[
u = y + 2x - 2 \log_\epsilon y - \log_\epsilon x
\]
的极大值和极小值。

\Item{(12)} 一个给定容量的缆车斗呈水平等腰三角形棱柱形状,顶点在下,对面开口。求其尺寸,使得在建造时使用最少的铁皮。
\end{Problems}
\DPPageSep{194.png}{182}%

\Chapter{第十七章}{积分}

\First{积}分这个伟大的秘密已经揭示,这个神秘的符号 $\ds\int$,毕竟只是一个长长的 $S$,仅仅意味着“求和”或“所有此类量的和”。因此,它类似于另一个符号 $\sum$(希腊字母 \emph{Sigma}),这也是求和的符号。然而,数学家在使用这些符号时有一个区别,即 $\sum$ 通常用于表示有限数量的量的和,而积分符号 $\ds\int$ 通常用于表示大量微小量的求和,这些量实际上是无限小的元素,构成了所需的总和。因此,$\ds\int dy = y$,$\ds\int dx = x$。

任何人都可以理解,任何事物都可以被看作是由许多小部分组成的;部分越小,它们的数量就越多。例如,一条一英寸长的线可以被看作是由 10 个部分组成的,每个部分长 $\frac{1}{10}$ 英寸;或者由 100 个部分组成,每个部分长 $\frac{1}{100}$ 英寸;\DPPageSep{195.png}{183}%
或者由 1,000,000 个部分组成,每个部分长 $\frac{1}{1,000,000}$ 英寸;或者,将思维推向可想象的极限,它可以被看作是由无限多个无限小的元素组成的。

是的,你会说,但这样思考有什么用呢?为什么不直接把它当作一个整体来思考呢?简单的理由是,有许多情况下,如果不计算许多小部分的和,就无法计算整个事物的大小。“积分”的过程是为了让我们能够计算出否则我们无法直接估计的总和。

让我们先通过一两个简单的例子来熟悉这个求和的概念。

考虑以下级数:
\[
1 + \tfrac{1}{2} + \tfrac{1}{4} + \tfrac{1}{8}
  + \tfrac{1}{16} + \tfrac{1}{32} + \tfrac{1}{64} + \text{等等}
\]

这里,级数的每一项都是前一项的一半。如果我们能够无限地继续下去,总和的值是多少?每个学童都知道答案是~$2$。可以将其想象为一条线。从一英寸开始;加上半英寸,加上四分之一英寸;加上八分之一英寸;依此类推。如果在操作的任何阶段停止,仍然会有一段缺失,以达到全长 $2$~英寸;而缺失的部分总是与最后添加的部分大小相同。因此,如果在添加了 $1$、$\frac{1}{2}$ 和 $\frac{1}{4}$ 之后停止,将会有 $\frac{1}{4}$~缺失。如果继续添加到 $\frac{1}{64}$,仍然会有 $\frac{1}{64}$~缺失。所需补充的部分总是等于最后添加的项。只有通过无限多次的操作,我们才能达到实际的 $2$~英寸。实际上,当我们达到无法绘制的微小部分时,我们就会接近这个值——大约在第 $10$~项之后,因为第 $11$~项是 $\frac{1}{1024}$。如果我们想要达到即使使用惠特沃斯测量机也无法检测到的程度,我们只需进行大约 $20$~项。甚至显微镜也无法显示第 $18$~项!因此,无限多次的操作实际上并不可怕。积分就是整个部分。但正如我们将看到的,在某些情况下,积分学使我们能够得到作为无限多次操作结果的精确总和。在这些情况下,积分学为我们提供了一种快速且简便的方法来获得结果,否则这将需要无休止的大量繁琐计算。因此,我们最好立即学习如何积分。

\Section{曲线的斜率及其本身}

让我们对曲线的斜率进行一些初步的探讨。因为我们已经看到,对曲线求导意味着找到其斜率(或在不同点的斜率)的表达式。如果我们已知斜率(或斜率),能否反向重建整个曲线呢?

回到第 \Pageref{Case2} 页的案例~(2)。这里我们有一个最简单的曲线,一条斜率为 $a$ 的直线,其方程为
\[
y = ax+b.
\]

我们知道,这里 $b$~表示当 $x=0$ 时 $y$~的初始高度,而 $a$,即 $\dfrac{dy}{dx}$,是这条直线的“斜率”。这条直线具有恒定的斜率。沿着它,所有基本三角形 \raisebox{-12pt}{\Graphic{197b}} 的高度与底边的比例都是相同的。假设我们取有限大小的 $dx$ 和 $dy$,使得 $10$ 个 $dx$ 组成一英寸,那么将会有十个像这样的小三角形
\begin{center}
  \Graphic{198a}
\end{center}

现在,假设我们被要求从仅有的信息$\dfrac{dy}{dx} = a$出发,重建这条“曲线”。我们该如何做呢?仍然将小~$d$视为有限大小,我们可以画出10个斜率相同的线段,然后将它们首尾相连,像这样:
\Figure[3.25in]{198b}{48}
由于所有线段的斜率相同,它们会连接起来,如\Fig{48}所示,形成一条具有正确斜率$\dfrac{dy}{dx} = a$的斜线。无论我们将$dy$和~$dx$视为有限还是无限小,由于它们都相同,显然$\dfrac{y}{x} = a$,如果我们把$y$视为所有~$dy$的总和,$x$视为所有~$dx$的总和。但这条斜线应该放在哪里呢?是从原点~$O$开始,还是更高一点?由于我们唯一的信息是关于斜率的,我们没有关于特定高度的指示;事实上,初始高度是不确定的。斜率将保持不变,无论初始高度如何。因此,我们不妨猜测可能需要从高度~$C$开始画这条斜线。即,我们有方程
\[
y = ax + C.
\]

现在显而易见\Pagelabel{constant},在这种情况下,添加的常数意味着当$x = 0$时$y$所具有的特定值。

现在让我们考虑一个更复杂的情况,即一条斜率不是恒定的,而是随着$x$的增长而不断增加的直线。我们假设向上的斜率随着$x$的增长而按比例增加。用符号表示就是:
\[
\frac{dy}{dx} = ax.
\]
或者,给出一个具体例子,设$a = \frac{1}{5}$,因此
\[
\frac{dy}{dx} = \tfrac{1}{5} x.
\]

那么我们最好先计算一些不同$x$值下的斜率,并绘制它们的小图。
\DPPageSep{200.png}{188}%
\begin{DPalign*}
\lintertext{\indent 当}
x &=0,\quad \frac{dy}{dx} = 0,\Z && \Graphic{200a1} \\
x &=1,\quad \frac{dy}{dx} = 0.2, && \Graphic{200a2} \\
x &=2,\quad \frac{dy}{dx} = 0.4, && \Graphic{200a3} \\
x &=3,\quad \frac{dy}{dx} = 0.6, && \Graphic{200a4} \\
x &=4,\quad \frac{dy}{dx} = 0.8, && \Graphic{200a5} \\
x &=5,\quad \frac{dy}{dx} = 1.0. && \Graphic{200a6}
\end{DPalign*}

现在尝试将这些片段组合起来,使每个片段的基部中点位于正确的距离右侧,并使它们在角点处连接;如图(\Fig{49})所示。当然,结果并不是一条平滑的曲线:但它是一个近似。如果我们取一半长度的片段,并且数量加倍,如图\Fig{50}所示,我们将得到一个更好的近似。但\DPPageSep{201.png}{189}%
为了得到完美的曲线,我们应该取每个~$dx$及其对应的~$dy$为无限小,并且数量无限多。

\Figure[3in]{200b}{49}
那么,任何~$y$的值应该是多少呢?显然,在曲线的任意点~$P$处,$y$的值将是所有从~$0$到该高度的小~$dy$的总和,即$\ds\int dy = y$。由于每个~$dy$等于$\frac{1}{5}x · dx$,因此整个~$y$将等于所有这样的$\frac{1}{5}x · dx$的总和,或者我们应写作$\ds\int \tfrac{1}{5}x · dx$。

现在,如果$x$是常数,$\ds\int \tfrac{1}{5}x · dx$将与$\frac{1}{5} x \ds\int dx$或~$\frac{1}{5}x^2$相同。但$x$从~$0$开始,并增加到点~$P$处的特定值,因此其从~$0$到该点的平均值是~$\frac{1}{2}x$。因此$\ds\int \tfrac{1}{5} x\, dx = \tfrac{1}{10} x^2$;即$y=\frac{1}{10}x^2$。

但与前一种情况一样,这需要加上一个未确定的常数~$C$,因为我们没有被告知曲线在$x = 0$时从原点上方多高开始。因此,我们写出如图\Fig{51}中绘制的曲线的方程:
\[
y = \tfrac{1}{10}x^2 + C.
\]
\Figure{202a}{51}

\Exercises{XVI}(答案见\Pageref[page]{AnsEx:XVI}。)
\begin{Problems}
\Item{(1)} 求 $\frac{2}{3} + \frac{1}{3} + \frac{1}{6} + \frac{1}{12} + \frac{1}{24} + \text{等等}$ 的最终和。

\Item{(2)} 证明级数 $1 - \frac{1}{2} + \frac{1}{3} - \frac{1}{4} + \frac{1}{5} - \frac{1}{6} + \frac{1}{7}$\DPnote{[** TN: [sic], no +]}~等等,
是收敛的,并求其前 $8$ 项的和。

\Item{(3)} 若 $\log_\epsilon(1+x) = x - \dfrac{x^2}{2} + \dfrac{x^3}{3} - \dfrac{x^4}{4} + \text{等等}$,求 $\log_\epsilon 1.3$。

\Item{(4)} 按照本章解释的类似推理,求~$y$,
\[
\text{(\textit{a}) 若 $\frac{dy}{dx} = \tfrac{1}{4} x$;\quad
(\textit{b}) 若 $\frac{dy}{dx} = \cos x$。}
\]

\Item{(5)} 若 $\dfrac{dy}{dx} = 2x + 3$,求~$y$。
\end{Problems}
\DPPageSep{203.png}{191}%

\Chapter[如何积分]{第十八章}{积分作为微分的逆运算}

\First{微分}是当我们已知 $y$(作为~$x$ 的函数)时,求~$\dfrac{dy}{dx}$ 的过程。

\Pagelabel{revdif}%
与其他数学运算一样,微分过程可以逆向进行;因此,如果对 $y = x^4$ 进行微分得到 $\dfrac{dy}{dx} = 4x^3$;如果从 $\dfrac{dy}{dx} = 4x^3$ 开始,可以说逆向过程将得到 $y = x^4$。但这里有一个有趣的问题。如果我们从以下任意一个开始:$x^4$、$x^4 + a$、$x^4 + c$ 或 $x^4$ 加上任意常数,我们都会得到 $\dfrac{dy}{dx} = 4x^3$。因此,很明显,在从 $\dfrac{dy}{dx}$ 反推到~$y$ 时,必须考虑到可能存在一个附加常数,其值在通过其他方式确定之前是未知的。因此,如果对 $x^n$ 微分得到~$nx^{n-1}$,从 $\dfrac{dy}{dx} = nx^{n-1}$ 反推将得到 $y = x^n + C$;其中 $C$ 代表尚未确定的常数。

显然,在处理~$x$ 的幂时,逆向运算的规则是:将幂次加 $1$,然后除以增加后的幂次,并加上未确定的常数。

因此,在 $\dfrac{dy}{dx} = x^n$ 的情况下,逆向运算得到
\[
y = \frac{1}{n + 1} x^{n+1} + C。
\]

如果对 $y = ax^n$ 微分得到
\[
\frac{dy}{dx} = anx^{n-1},
\]
那么从
\[
\frac{dy}{dx} = anx^{n-1}
\]
开始,逆向过程将得到
\[
y = ax^n。
\]
因此,当处理乘法常数时,我们必须简单地将常数作为积分结果的乘数。

因此,如果 $\dfrac{dy}{dx} = 4x^2$,逆向过程得到 $y = \frac{4}{3}x^3$。

但这并不完整。因为我们必须记住,如果我们从
\[
y = ax^n + C
\]
开始,其中 $C$ 是任意常数,我们同样会得到
\[
\frac{dy}{dx} = anx^{n-1}。
\]

因此,在逆向过程中,我们必须始终记住加上这个未确定的常数,即使我们还不知道它的值。

这个逆向微分的过程称为\emph{积分};因为它包括在已知~$dy$ 或~$\dfrac{dy}{dx}$ 的表达式时,求出整个量~$y$ 的值。迄今为止,我们尽可能将 $dy$ 和 $dx$ 保持在一起作为微分系数:今后我们将更频繁地分离它们。

如果我们从一个简单的情况开始,
\[
\frac{dy}{dx} = x^2。
\]

我们可以写成
\[
dy = x^2\, dx。
\]

现在这是一个“微分方程”,它告诉我们 $y$ 的一个元素等于 $x$ 的对应元素乘以 $x^2$。现在,我们想要的是积分;因此,用适当的符号写下对两边进行积分的指令,如下:
\[
\int dy = \int x^2\, dx。
\]

[关于阅读积分的说明:上述内容应这样读:
\begin{quote}
“\emph{积分 dee-wy \emph{等于} 积分 eks-平方 dee-eks}。”]
\end{quote}

我们尚未进行积分:我们只是写下了积分的指令——如果我们能做到的话。让我们试试。其他许多人都做到了——为什么我们不能呢?左边的部分非常简单。所有$y$的微小部分之和就是$y$本身。因此我们可以立即写成:
\[
y = \int x^2\, dx.
\]

但当我们处理方程的右边时,我们必须记住,我们需要加总的是所有$x^2\, dx$这样的项,而不是所有的$dx$。这\emph{不}等于$x^2 \ds\int dx$,因为$x^2$不是常数。有些$dx$会乘以较大的$x^2$值,有些则会乘以较小的$x^2$值,具体取决于$x$的取值。因此,我们必须思考一下关于积分作为微分逆过程的知识。现在,我们的规则是——见\Pageref{revdif} \textit{前文}——处理$x^n$时,“将幂次增加一,并\Pagelabel{diffrule}
除以增加后的幂次相同的数。”
\DPPageSep{207.png}{195}%
也就是说,$x^2\, dx$将变为$\frac{1}{3} x^3$。\footnote
  {你可能会问,末尾的小$dx$去哪儿了?其实,它原本是微分系数的一部分,当它被移到右边,如$x^2\, dx$中,它起到提醒作用,即$x$是进行该运算的自变量;而乘积被求和后,$x$的幂次增加了\emph{一}。你很快就会熟悉这一切。}
将此代入方程;但别忘了在末尾加上“积分常数”$C$。于是我们得到:
\[
y = \tfrac{1}{3} x^3 + C.
\]

你实际上已经完成了积分。多么简单!

让我们尝试另一个简单的例子。

\begin{DPalign*}
\lintertext{\indent 设}
\dfrac{dy}{dx} &= ax^{12},
\end{DPalign*}
其中$a$是任意常数乘数。我们在微分时发现(见\Pageref{differ}),$y$中的任何常数因子在$\dfrac{dy}{dx}$中保持不变。因此,在积分的逆过程中,它也会在$y$的值中重新出现。所以我们可以像之前一样进行,如下:
\begin{align*}
dy &= ax^{12} · dx,\\
\int dy &= \int ax^{12} · dx,\\
\int dy &= a \int x^{12}\, dx,\\
y &= a × \tfrac{1}{13} x^{13} + C.
\end{align*}

就这样完成了。多么简单!
\DPPageSep{208.png}{196}%

我们现在开始意识到,积分是一个与微分相比的\emph{逆向过程}。如果在微分时我们得到了某个特定的表达式——在这个例子中是$ax^{12}$——我们可以从它推导出原来的$y$。这两个过程的对比可以用一位著名教师的话来说明。如果一个陌生人被放在特拉法加广场,并被告知要找到尤斯顿车站,他可能会觉得任务无望。但如果他之前曾被人从尤斯顿车站带到特拉法加广场,那么他找到回尤斯顿车站的路就会相对容易。

\Section{两个函数的和或差的积分}

\begin{DPalign*}
\lintertext{\indent 设}
\frac{dy}{dx} &= x^2 + x^3, \\
\lintertext{则}
dy &= x^2\, dx + x^3\, dx.
\end{DPalign*}

我们没有理由不分别对每一项进行积分:因为正如在\Pageref{sumdiffer}中看到的,我们发现当对两个独立函数的和进行微分时,微分系数就是两个独立微分之和。因此,当我们逆向操作,进行积分时,积分也将是两个独立积分之和。
\DPPageSep{209.png}{197}%

我们的指令将是:
\begin{align*}
\int dy
  &= \int (x^2 + x^3)\, dx \\
  &= \int x^2\, dx + \int x^3\, dx   \\
y &= \tfrac{1}{3} x^3 + \tfrac{1}{4} x^4 + C.
\end{align*}

如果其中任何一个项是负数,
积分中的相应项也将
是负数。因此,差值与和值一样容易处理。

\Section{如何处理常数项。}

假设待积分的表达式中
有一个常数项——例如:
\[
\frac{dy}{dx} = x^n + b.
\]

这非常简单。你只需
记住当你对表达式
$y = ax$ 求导时,结果是 $\dfrac{dy}{dx} = a$。因此,当你反向操作并积分时,常数会重新出现并乘以~$x$。所以我们得到
\begin{align*}
dy &= x^n\, dx + b · dx,  \\
\int dy &= \int x^n\, dx + \int b\, dx, \\
y &= \frac{1}{n+1} x^{n+1} + bx + C.
\end{align*}

这里有许多例子可以用来尝试你
新获得的能力。

\tb
\DPPageSep{210.png}{198}%

\Examples.
(1) 已知 $\dfrac{dy}{dx} = 24x^{11}$。求~$y$。\qquad \textit{答案}。$y = 2x^{12} + C$。

(2) 求 $\ds\int (a + b)(x + 1)\, dx$。\qquad 它是 $(a + b) \ds\int (x + 1)\, dx$ \\
或\quad $(a + b) \left[\ds\int x\, dx + \ds\int dx\right]$ \quad 或\quad $(a + b) \left(\dfrac{x^2}{2} + x\right) + C$。

(3) 已知 $\dfrac{du}{dt} = gt^{\efrac{1}{2}}$。求~$u$。\qquad \textit{答案}。$u = \frac{2}{3} gt^{\efrac{3}{2}} + C$。

(4) $\dfrac{dy}{dx} = x^3 - x^2 + x$。求~$y$。
\BindMath{\begin{align*}
dy &= (x^3 - x^2 + x)\, dx\quad\text{或} \\
dy &= x^3\, dx - x^2\, dx + x\, dx;\quad
y = \int x^3\, dx - \int x^2\, dx + \int x\, dx;
\end{align*}
\begin{DPalign*}
\lintertext{并且}
y &= \tfrac{1}{4} x^4 - \tfrac{1}{3} x^3 + \tfrac{1}{2} x^2 + C.
\end{DPalign*}}%

(5) 积分 $9.75x^{2.25}\, dx$。\qquad \textit{答案}。$y = 3x^{3.25} + C$。

\tb

所有这些都足够简单。让我们尝试另一种情况。%
\SetOddHead{最简单的积分}%

\begin{DPalign*}
\lintertext{设}
\dfrac{dy}{dx} &= ax^{-1}.
\end{DPalign*}

按照之前的方法,我们将写成
\[
dy = a x^{-1} · dx,\quad \int dy = a \int x^{-1}\, dx.
\]

那么,$x^{-1}\, dx$ 的积分是什么?

如果你回顾一下对 $x^2$、$x^3$ 和 $x^n$ 等求导的结果,你会发现我们从未
从任何一个得到 $\dfrac{dy}{dx} = x^{-1}$。我们从 $x^3$ 得到 $3x^2$;从 $x^2$ 得到 $2x$;从 $x^1$(即 $x$ 本身)得到 $1$;但我们没有从 $x^0$ 得到 $x^{-1}$,原因有二。\emph{首先},$x^0$ 就是 $1$,是一个常数,不可能有
导数。\emph{其次},即使它可以
求导,其导数(通过机械地遵循通常的规则)将是 $0 × x^{-1}$,这个乘以零的结果是零值!
因此,当我们现在尝试积分
$x^{-1}\, dx$ 时,我们发现它不在由规则给出的 $x$ 的幂次中:
\[
\int x^n\, dx = \dfrac{1}{n+1} x^{n+1}.
\]
这是一个例外情况。

好吧;但再试一次。查看从 $x$ 的各种函数中获得的所有各种导数,并尝试在其中找到 $x^{-1}$。充分的搜索将表明,我们实际上确实通过微分函数 $y = \log_\epsilon x$ 得到了 $\dfrac{dy}{dx} = x^{-1}$(见
\Pageref{differlog})。 %[ ** 页码]

那么,当然,既然我们知道微分
$\log_\epsilon x$ 得到 $x^{-1}$,我们就知道,通过反向过程,积分 $dy = x^{-1}\, dx$ 将得到 $y = \log_\epsilon x$。
但我们不能忘记给定的常数因子 $a$,也不能省略未定常数 $C$。因此,这给出了当前问题的解:
\[
y = a \log_\epsilon x + C.
\]

\NB---这里要特别注意一个非常显著的事实,即如果我们事先不知道相应的微分,就无法在上面的例子中进行积分。如果没有人发现对$\log_\epsilon x$求导得到$x^{-1}$,我们在面对如何积分$x^{-1}\, dx$的问题时就会完全束手无策。事实上,应该坦率地承认,这是积分学中一个奇特的特点:在你通过微分其他函数得到你想要积分的表达式之前,你无法对任何东西进行积分。即使到了今天,也没有人能够找到表达式
\[
\frac{dy}{dx} = a^{-x^2},
\]
的一般积分,因为$a^{-x^2}$从未被发现是由其他函数的微分得出的。

\Subsection{另一个简单例子。}
求$\ds\int (x + 1)(x + 2)\, dx$。

在查看要积分的函数时,你会注意到它是两个不同函数的乘积。你可能会想,你可以单独积分$(x + 1)\, dx$或$(x + 2)\, dx$。当然你可以。但如何处理一个乘积呢?你所学过的任何微分都没有产生过这样的微分系数。既然没有这样的微分,最简单的方法就是将这两个函数相乘,然后进行积分。这给我们
\[
\int (x^2 + 3x + 2)\, dx.
\]
这等同于
\[
\int x^2\, dx + \int 3x\, dx + \int 2\, dx.
\]
进行积分后,我们得到
\[
\tfrac{1}{3} x^3 + \tfrac{3}{2} x^2 + 2x + C.
\]

\Section[其他一些积分]{其他一些积分。}

既然我们知道积分是微分的逆运算,我们可以立即查阅我们已经知道的微分系数,看看它们是由哪些函数导出的。这为我们提供了以下现成的积分:\Pagelabel{intex2}\Pagelabel{differ3}\Pagelabel{cosax}%
\begin{alignat*}{4}
&x^{-1} &&\text{(\Pageref{differlog});}\qquad &&
  \int x^{-1}\, dx      &&= \log_\epsilon x + C. \\
%
%\label{intex2}
&\frac{1}{x+a} && \text{(\Pageref{differ2});} &&
  \int \frac{1}{x+a}\, dx &&= \log_\epsilon (x+a) + C. \\
%
&\epsilon^x && \text{(\Pageref{unchanged});} &&
  \int \epsilon^x\, dx    &&= \epsilon ^x + C. \\
%
&\epsilon^{-x} &&&&
  \int \epsilon^{-x}\, dx &&= -\epsilon^{-x} + C \\
%
\intertext{(因为如果$y = - \dfrac{1}{\epsilon^x}$,\quad $\dfrac{dy}{dx} = -\dfrac{\epsilon^x × 0 - 1 × \epsilon^x}{\epsilon^{2x}} = \epsilon^{-x}$).}
%
&\sin x && \text{(\Pageref{differcos});} &&
  \int \sin x\, dx        &&= -\cos x + C. \\
%
&\cos x && \text{(\Pageref{differsin});} &&
  \int \cos x\, dx        &&= \sin x + C. \\
%
\intertext{\indent 我们还可以推导出以下结果:}
%
&\log_\epsilon x; &&&&
  \int\log_\epsilon x\, dx &&= x(\log_\epsilon x - 1) + C \\
%
\intertext{(因为如果$y = x \log_\epsilon x - x$,\quad $\dfrac{dy}{dx} = \dfrac{x}{x} + \log_\epsilon x - 1 = \log_\epsilon x$).}
%
&\log_{10} x;   &&&&
  \int\log_{10} x\, dx &&= 0.4343x (\log_\epsilon x - 1) + C. \\
%
&a^x && \text{(\Pageref{diffexp});}  &&
  \int a^x\, dx        &&= \dfrac{a^x}{\log_\epsilon a} + C. \\
%
% \label{cosax}
&\cos ax; &&&& \int\cos ax\, dx     &&= \frac{1}{a} \sin ax + C \\
\intertext{(因为如果$y = \sin ax$,$\dfrac{dy}{dx} = a \cos ax$;因此要得到$\cos ax$,必须对$y = \dfrac{1}{a} \sin ax$求导).}
%
&\sin ax; &&&& \int\sin ax\, dx     &&= -\frac{1}{a} \cos ax + C. \\
\end{alignat*}



尝试也使用 $\cos^2\theta$;稍作变通将简化问题:
\begin{DPgather*}
\cos 2\theta = \cos^2\theta - \sin^2\theta
             = \DPtypo{2\cos 2\theta - 1}{2\cos^2 \theta - 1}; \\
\lintertext{因此}
\cos^2\theta = \tfrac{1}{2}(\DPtypo{\cos^2 \theta + 1}{\cos 2\theta + 1}),
\end{DPgather*}
\begin{DPalign*}
\lintertext{并且}
\int\cos^2 \theta\, d\theta
  &= \tfrac{1}{2} \int (\cos 2\theta + 1)\, d\theta \\
  &= \tfrac{1}{2} \int \cos 2 \theta\, d\theta + \tfrac{1}{2} \int d\theta. \\
  &= \frac{\sin 2\theta}{4} + \frac{\theta}{2} + C.\text{ (另见 \Pageref{moreexamples}。)} %[ ** 页码]
\end{DPalign*}

另见 \Pagerange{stdforms1}{stdforms2} 的标准形式表。
你应该为自己制作这样一个表格,只放入
你已成功微分和积分的通用函数。确保它
不断增长!
\DPPageSep{215.png}{203}%

\Section[二重积分]{关于二重和三重积分。}

在许多情况下,需要对包含两个或多个变量的表达式进行积分;
在这种情况下,积分符号会出现多次。因此,
\[
\iint f(x,y,)\, dx\, dy
\]
意味着对变量 $x$ 和 $y$ 的某个函数进行积分。
积分的顺序并不重要。例如,取函数
$x^2 + y^2$。对其关于 $x$ 积分得到:
\[
\int (x^2+y^2)\, dx = \tfrac{1}{3} x^3 + xy^2.
\]

现在,对其关于 $y$ 积分:
\[
\int (\tfrac{1}{3} x^3 + xy^2)\, dy = \tfrac{1}{3} x^3y + \tfrac{1}{3} xy^3,
\]
当然,还需要加上一个常数。如果我们
颠倒操作顺序,结果将相同。

在处理表面和固体的面积时,我们
经常需要对长度和宽度进行积分,
从而得到如下形式的积分:
\[
\iint u · dx\, dy,
\]
其中 $u$ 是某种属性,在每一点上
取决于 $x$ 和 $y$。这被称为 \emph{面积积分}。
它表示所有这些
\DPPageSep{216.png}{204}%
元素 $u · dx · dy$(即 $u$ 在长为 $dx$、宽为 $dy$ 的小矩形上的值)
在整个长度和宽度上的总和。

同样,在处理固体时,涉及三个维度。
考虑任意体积元素,即边长为 $dx$、$dy$、$dz$ 的小立方体。
如果固体的形状由函数 $f(x, y, z)$ 表示,
则整个固体的 \emph{体积积分} 为:
\[
\text{体积} = \iiint f(x,y,z) · dx · dy · dz.
\]
当然,这种积分必须在每个维度上
取适当的极限\footnote
  {参见 \Pageref{limits} 关于在极限之间积分。};
并且除非知道表面边界如何依赖于
$x$、$y$ 和 $z$,否则无法进行积分。
如果 $x$ 的极限从 $x_1$ 到 $x_2$,
$y$ 的极限从 $y_1$ 到 $y_2$,
$z$ 的极限从 $z_1$ 到 $z_2$,
则显然有
\[
\text{体积} = \int_{z1}^{z2} \int_{y1}^{y2} \int_{x1}^{x2} f(x,y,z) · dx · dy · dz.
\]

当然,存在许多复杂和困难的情况;
但一般来说,符号的意义相当容易理解,
当它们旨在表示对给定表面或整个固体空间进行某种积分时。
\DPPageSep{217.png}{205}%

\Exercises[简单积分]{XVII}
(答案见 \Pageref{AnsEx:XVII}。)
\begin{Problems}
\Item{(1)} 当 $y^2 = 4 ax$ 时,求 $\ds\int y\, dx$。

\ResetCols{2}
\Item{(2)} 求 $\ds\int \frac{3}{x^4}\, dx$。
\Item{(3)} 求 $\ds\int \frac{1}{a} x^3\, dx$。

\ResetCols{2}
\Item{(4)} 求 $\ds\int (x^2 + a)\, dx$。
\Item{(5)} 积分 $5x^{-\efrac{7}{2}}$。

\ResetCols{1}
\Item{(6)} 求 $\ds\int (4x^3 + 3x^2 + 2x + 1)\, dx$。

\Item{(7)} 若 $\dfrac{dy}{dx} = \dfrac{ax}{2} + \dfrac{bx^2}{3} + \dfrac{cx^3}{4}$,求 $y$。

\ResetCols{2}
\Item{(8)} 求 $\ds\int \left(\frac{x^2 + a}{x + a}\right) dx$。
\Item{(9)} 求 $\ds\int (x + 3)^3\, dx$。

\ResetCols{1}
\Item{(10)} 求 $\ds\int (x + 2)(x - a)\, dx$。

\Item{(11)} 求 $\ds\int (\sqrt x + \sqrt[3] x) 3a^2\, dx$。

\Item{(12)} 求 $\ds\int (\sin \theta - \tfrac{1}{2})\, \frac{d\theta}{3}$。

\ResetCols{2}
\Item{(13)} 求 $\ds\int \cos^2 a \theta\, d\theta$。
\Item{(14)} 求 $\ds\int \sin^2 \theta\, d\theta$。

\ResetCols{2}
\Item{(15)} 求 $\ds\int \sin^2 a \theta\, d\theta$。
\Item{(16)} 求 $\ds\int \epsilon^{3x}\, dx$。

\ResetCols{2}
\Item{(17)} 求 $\ds\int \dfrac{dx}{1 + x}$。
\Item{(18)} 求 $\ds\int \dfrac{dx}{1 - x}$。
\end{Problems}
\DPPageSep{218.png}{206}%

\Chapter[通过积分求面积]{第十九章}{通过积分求面积}

\First{积分}微积分的一个用途是使我们能够确定由曲线界定的面积的值。

让我们逐步探讨这个问题。

\Figure[2.5in]{218a}{52}

设 $AB$(\Fig{52})为一条曲线,其方程已知。即,曲线中的 $y$ 是 $x$ 的某个已知函数。想象从点 $P$ 到点 $Q$ 的一段曲线。

从 $P$ 点垂直向下作 $PM$,从 $Q$ 点作 $QN$。设 $OM = x_1$,$ON = x_2$,纵坐标 $PM = y_1$,$QN = y_2$。这样我们就标出了位于 $PQ$ 段下方的区域 $PQNM$。问题是,\emph{如何计算这个面积的值}?

%[插图]

解决这个问题的关键是将面积想象为被分割成许多窄条,每条的宽度为 $dx$。$dx$ 越小,$x_1$ 和 $x_2$ 之间的条数就越多。显然,整个面积等于所有这些窄条面积的总和。我们的任务是找到任意一条窄条面积的表达式,并对其进行积分,以将所有窄条的面积相加。现在考虑任意一条窄条。它
\begin{wrapfigure}[7]{r}{0.5in}
  \hfill\smash[t]{\raisebox{-1.375in}{\Graphic[0.375in]{219a}}}
\end{wrapfigure}
将类似于这样:
被两个垂直边界定,底部平坦为 $dx$,顶部略微弯曲倾斜。假设我们将其\emph{平均}高度设为 $y$;那么,由于其宽度为 $dx$,其面积将为 $y\, dx$。而且,由于我们可以将宽度设得尽可能窄,只要足够窄,其平均高度将与其中间的高度相同。现在,我们设整个未知面积的值为 $S$,表示表面。一条窄条的面积将仅仅是整个面积的一部分,因此可以称为 $dS$。所以我们可写为
\[
\text{一条窄条的面积} = dS = y · dx。
\]
如果我们将所有窄条相加,我们得到
\[
\text{总面积 $S$} = \int dS = \int y\, dx。
\]

因此,我们求 $S$ 的关键在于,当我们知道 $y$ 作为 $x$ 的函数时,是否能够对 $y · dx$ 进行积分。

例如,如果你被告知所讨论的特定曲线的 $y = b + ax^2$,毫无疑问你可以将这个值代入表达式并说:那么我必须求 $\ds\int (b + ax^2)\, dx$。%[ ** \displaystyle]

这很好;但稍加思考就会发现,还需要做更多的工作。因为我们试图找到的面积不是整个曲线下的面积,而只是由 $PM$ 左侧和 $QN$ 右侧限定的面积,因此我们必须做一些工作来定义这个面积在那些“\emph{极限}”之间。\Pagelabel{limits} %[ ** F2: 单引号,其他地方双引号]

这引入了我们一个新的概念,即\emph{在极限之间积分}。我们假设$x$变化,并且为了当前的目的,我们不需要任何低于$x_1$(即$OM$)的$x$值,也不需要任何高于$x_2$(即$ON$)的$x$值。当一个积分被定义在两个极限之间时,我们称这两个值中较小的一个为\emph{下限},较大的一个为\emph{上限}。任何这样有限制的积分,我们称之为\emph{定积分},以区别于没有指定极限的\emph{一般积分}。

在给出积分指令的符号中,极限通过将它们分别放在积分符号的顶部和底部来标记。因此,指令
\[
\int_{x=x_1}^{x=x_2} y · dx
\]
\DPPageSep{221.png}{209}%
将被解读为:求$y · dx$在$x_1$下限和$x_2$上限之间的积分。

有时,这个表达会更简单地写成
\[
\int^{x_2}_{x_1} y · dx.
\]
那么,当你得到这些指令时,\emph{如何}找到在极限之间的积分呢?

再看看\Fig{52} (\Pageref{fig:52})。假设我们能够找到从$A$到$Q$,即从$x = 0$到$x = x_2$的曲线下的较大面积,命名为$AQNO$。然后,假设我们能够找到从$A$到$P$,即从$x = 0$到$x = x_1$的较小面积,即面积$APMO$。然后,如果我们从较大的面积中减去较小的面积,我们将得到剩余的面积$PQNM$,这就是我们想要的。这里我们有了线索,即在两个极限之间的定积分是\emph{上限积分与下限积分之差}。

让我们继续。首先,找到一般积分:
\[
\int y\, dx,
\]
并且由于$y = b + ax^2$是曲线(\Fig{52})的方程,
\[
\int (b + ax^2)\, dx
\]
是我们必须找到的一般积分。

根据规则(\Pageref{section:9})进行积分,我们得到
\[
bx + \frac{a}{3} x^3 + C;
\]
\DPPageSep{222.png}{210}%
并且这将是从$0$到我们可能指定的任何$x$值的整个面积。

因此,较大的面积直到上限$x_2$将是
\[
bx_2 + \frac{a}{3} x_2^3 + C;
\]
较小的面积直到下限$x_1$将是
\[
bx_1 + \frac{a}{3} x_1^3 + C.
\]

现在,从较大的面积中减去较小的面积,我们得到面积$S$的值为
\[
\text{面积~$S$} = b(x_2 - x_1) + \frac{a}{3}(x_2^3 - x_1^3).
\]

这是我们想要的答案。让我们给出一些数值。假设$b = 10$,$a = 0.06$,$x_2 = 8$,$x_1 = 6$。那么面积$S$等于
\begin{gather*}
10(8 - 6) + \frac{0.06}{3} (8^3 - 6^3) \\
\begin{aligned}
&= 20 + 0.02(512 - 216)    \\
&= 20 + 0.02 × 296    \\
&= 20 + 5.92     \\
&= 25.92.
\end{aligned}
\end{gather*}

让我们在这里写下我们关于极限的符号表示:
\[
\int^{x=x_2}_{x=x_1} y\, dx = y_2 - y_1,
\]
其中$y_2$是与$x_2$对应的$y\, dx$的积分值,$y_1$是与$x_1$对应的积分值。
\DPPageSep{223.png}{211}%

所有在极限之间的积分都需要找到两个值的差。还要注意,在进行减法时,附加的常数$C$已经消失了。

\Examples.
(1) 为了熟悉这个过程,让我们取一个我们事先知道答案的例子。让我们求一个底为$x = 12$,高为$y = 4$的三角形(\Fig{53})的面积。我们事先从明显的测量中知道答案将是$24$。

\Figure{223a}{53}

现在,这里我们有一个斜线作为“曲线”,其方程为
\[
y = \frac{x}{3}.
\]

所求的面积将是
\[
\int^{x=12}_{x=0} y · dx = \int^{x=12}_{x=0} \frac{x}{3} · dx.
\]

我们对$\dfrac{x}{3}\, dx$进行积分(见\Pageref{diffrule}页),并将一般积分的值用方括号表示,上下限标记在方括号内,得到
\begin{align*}
\text{面积} %[ ** F2: 如今,+ C 会放在方括号内]
  &= \left[ \frac{1}{3} · \frac{1}{2} x^2 \right]^{x=12}_{x=0} + C \\
  &= \left[ \frac{x^2}{6} \right]^{x=12}_{x=0} + C  \\
  &= \left[ \frac{12^2}{6} \right] - \left[ \frac{0^2}{6} \right] \\
  &= \frac{144}{6} = 24.\quad \textit{答案}.
\end{align*}

让我们通过一个简单的例子来验证这个看似令人惊讶的计算技巧。准备一些方格纸,最好是每边划分成八分之一英寸或十分之一英寸的小方格。在这张方格纸上绘制方程的图像,
\[
y = \frac{x}{3}.
\]

需要绘制的数值为:
\[
\begin{array} {|c|| *{5}{c|}}
\hline
\Strut
\Td[c]{x} & \Td[c]{0} & \Td[c]{3} & \Td[c]{6} & \Td[c]{9} & \Td{12} \\
\hline
\Strut
\Td[c]{y} & \Td[c]{0} & \Td[c]{1} & \Td[c]{2} & \Td[c]{3}  & \Td{4} \\
\hline
\end{array}
\]

图像如图\Fig{54}所示。
\DPPageSep{225.png}{213}%

现在通过计算曲线下的**小方格数量**来估算曲线下的面积,从$x = 0$到$x = 12$。共有$18$个完整的小方格和四个三角形,每个三角形的面积等于$1\frac{1}{2}$个小方格;总计为$24$个小方格。因此,$24$是$\dfrac{x}{3}\, dx$在$x = 0$到$x = 12$之间的积分值。

作为进一步的练习,证明在$x = 3$到$x = 15$之间的相同积分的值为$36$。

\Figure[2.75in]{225a}{55}
(2) 求曲线$y = \dfrac{b}{x + a}$在$x = x_1$和$x = 0$之间的面积。
\begin{align*}
\text{面积}
  &= \int^{x=x_1}_{x=0} y · dx
   = \int^{x=x_1}_{x=0} \frac{b}{x+a}\, dx \displaybreak[1] \\
\DPPageSep{226.png}{214}%
  &= b \bigl[\log_\epsilon(x + a) \bigr]^{x_1} _{0} + C \displaybreak[1] \\
  &= b \bigl[\log_\epsilon(x_1 + a) - \log_\epsilon(0 + a)\bigr] \displaybreak[1] \\
  &= b \log_\epsilon \frac{x_1 + a}{a}.\quad \textit{答案}.
\end{align*}

注意:在处理定积分时,常数$C$总是通过相减而消失。

需要注意的是,这种通过从较大部分中减去一部分来求差值的过程实际上是一种常见做法。如何求平面环(如图\Fig{56})的面积,其外半径为$r_2$,内半径为$r_1$?根据测量学,外圆的面积为$\pi r_2^2$;然后求内圆的面积,$\pi r_1^2$;接着将后者从前者中减去,得到环的面积$=\pi(r_2^2 - r_1^2)$;可以写成
\[
\pi(r_2 + r_1)(r_2 - r_1)
\]
$= \text{环的平均周长} × \text{环的宽度}$。
\DPPageSep{227.png}{215}%

(3) 另一个例子是**衰减曲线**(见\Pageref{section:5}页)。求曲线(如图\Fig{57})在$x = 0$到$x = a$之间的面积,其方程为
\begin{align*}
y &= b\epsilon^{-x}. \\
\text{面积}
  &= b\int^{x=a} _{x=0} \epsilon^{-x} · dx. \displaybreak[1] \\
\intertext{\indent 积分(见\Pageref{differ3}页)给出}
  &= b\left[-\epsilon^{-x}\right]^a _0 \\
  &= b\bigl[-\epsilon^{-a} - (-\epsilon^{-0})\bigr] \\
  &= b(1-\epsilon^{-a}).
\end{align*}

\Figures{227a}{227b}{57}{58}

(4) 另一个例子是由理想气体的绝热曲线提供的,其方程为$pv^n = c$,其中$p$表示压力,$v$表示体积,$n$的值为$1.42$(如图\Fig{58})。

求从体积$v_2$到体积$v_1$的曲线下的面积(该面积与气体突然压缩所做的功成正比)。



这里我们有
\begin{align*}
\text{面积}
  &= \int^{v=v_2}_{v=v_1} cv^{-n} · dv \\
  &= c\left[\frac{1}{1-n} v^{1-n} \right]^{v_2} _{v_1} \\
  &= c \frac{1}{1-n} (v_2^{1-n} - v_1^{1-n}) \\
  &= \frac{-c}{0.42}\left(\frac{1}{v_2^{0.42}} - \frac{1}{v_1^{0.42}}\right).
\end{align*}

\Subsection{一个练习。}
证明普通的几何公式,即半径为~$R$ 的圆的面积~$A$ 等于~$\pi R^2$。

\Figure{228a}{59}

考虑表面上的一个基本环形区域或环带(\Fig{59}),其宽度为~$dr$,位于距中心~$r$ 处。我们可以将整个表面视为由这些狭窄的环带组成,整个面积~$A$ 就是从中心到边缘所有这些基本环带的积分,即从 $r = 0$ 到 $r = R$ 的积分。

因此,我们需要找到一个表示狭窄环带的基本面积~$dA$ 的表达式。将其视为宽度为~$dr$ 的条带,其长度为半径为~$r$ 的圆的周长,即长度为~$2 \pi r$。于是,狭窄环带的面积为
\[
dA = 2 \pi r\, dr.
\]

因此,整个圆的面积为:
\[
A = \int dA
  = \int^{r=R}_{r=0} 2 \pi r · dr
  = 2 \pi \int^{r=R}_{r=0} r · dr.
\]

现在,$r · dr$ 的一般积分为~$\frac{1}{2} r^2$。因此,
\begin{DPalign*}
A &= 2 \pi \bigl[\tfrac{1}{2} r^2 \bigr]^{r=R}_{r=0}; \\
\lintertext{或}
A &= 2 \pi \bigl[\tfrac{1}{2} R^2 - \tfrac{1}{2}(0)^2\bigr]; \\
\lintertext{因此}
A &= \pi R^2.
\end{DPalign*}

\Subsection{另一个练习。}
让我们求曲线 $y = x - x^2$ 的正部分的平均纵坐标,如图 \Fig{60} 所示。
\Figure[3in]{229a}{60}
为了求平均纵坐标,我们需要先求出区域~$OMN$ 的面积,然后将其除以基线~$ON$ 的长度。但在求面积之前,我们必须确定基线的长度,以知道积分的上限。在点~$N$ 处,纵坐标~$y$ 的值为零;因此,我们需要查看方程并找出使 $y = 0$ 的~$x$ 值。显然,如果 $x$ 为~$0$,$y$ 也将为~$0$,曲线通过原点~$O$;但同时,如果 $x=1$,$y=0$;所以 $x=1$ 给出了点~$N$ 的位置。

然后所需的面积为
\begin{align*}
  &= \int^{x=1}_{x=0} (x-x^2)\, dx \\
  &= \left[\tfrac{1}{2} x^2 - \tfrac{1}{3} x^3 \right]^{1}_{0} \\
  &= \left[\tfrac{1}{2} - \tfrac{1}{3} \right] - [0-0] \\
  &= \tfrac{1}{6}.
\end{align*}

但基线的长度为~$1$。

因此,曲线的平均纵坐标 $= \frac{1}{6}$。

[\NB---这是一个关于极值的优美且简单的练习,通过微分求出最大纵坐标的高度。它 \emph{必须} 大于平均值。]

任何曲线在 $x= 0$ 到 $x = x_1$ 范围内的平均纵坐标由以下表达式给出:
\[
\text{平均~$y$} = \frac{1}{x_1} \int^{x=x_1}_{x=0} y · dx.
\]

同样,也可以用相同的方法求旋转体的表面积。
\DPPageSep{231.png}{219}%

%[** TN: 原文包含一个嵌入式标题]
\Example. 曲线 $y=x^2-5$ 绕 $x$ 轴旋转。求曲线在 $x=0$ 到 $x=6$ 之间生成的表面积。

曲线上纵坐标为~$y$ 的点描述了一个长度为~$2\pi y$ 的圆周,对应于该点的狭窄表面带,宽度为~$dx$,其面积为~$2\pi y\, dx$。总面积为
\begin{align*}
2\pi \int^{x=6}_{x=0} y\, dx
  &= 2\pi \int^{x=6}_{x=0} (x^2-5)\, dx
   = 2\pi \left[\frac{x^3}{3} - 5x\right]^6_0 \\
  &= 6.28 × 42=263.76.
\end{align*}

\Section{极坐标下的面积。}

当一个区域的边界方程以某点$O$(见图\ref{fig:61})为极点,并以距离$r$和角度$\theta$表示时,刚刚解释的过程只需稍作修改即可同样容易地应用。我们不再考虑一个面积条,而是考虑一个小三角形$OAB$,其顶角为$d\theta$,并求出所有这些小三角形组成的所需面积之和。

{\loosen%
这样一个小三角形的面积大约为$\dfrac{AB}{2}×r$或$\dfrac{r\, d\theta}{2}×r$;}因此,曲线与$r$在角度$\theta_1$和$\theta_2$之间的两个位置所包含的面积由下式给出:
\[
\tfrac{1}{2} \int^{\theta=\theta_2}_{\theta=\theta_1} r^2\, d\theta.
\]

\tb

\Examples.
(1) 求半径为$a$的圆中$1$弧度的扇形面积。

圆的极坐标方程显然是$r=a$。面积为
\[
\tfrac{1}{2} \int^{\theta=\theta_2}_{\theta=\theta_1} a^2\, d\theta
  = \frac{a^2}{2} \int^{\theta=1}_{\theta=0} d\theta
  = \frac{a^2}{2}.
\]

(2) 求曲线(称为“帕斯卡蜗线”)的第一象限面积,其极坐标方程为$r=a(1+\cos \theta)$。
\begin{align*}
\text{面积}
  &= \tfrac{1}{2}  \int^{\theta=\frac{\pi}{2}}_{\theta=0} a^2(1+\cos \theta)^2\, d\theta  \\
  &= \frac{a^2}{2} \int^{\theta=\frac{\pi}{2}}_{\theta=0} (1+2 \cos \theta + \cos^2 \theta)\, d\theta  \\
  &= \frac{a^2}{2} \left[\theta + 2 \sin \theta + \frac{\theta}{2} + \frac{\sin 2 \theta}{4} \right]^{\efrac{\pi}{2}}_{0} \\
  &= \frac{a^2(3\pi+8)}{8}.
\end{align*}

\Section{通过积分求体积。}

我们用一个小面积条的面积所做的,当然也可以同样容易地用一个小体积条的体积来做。我们可以将组成整个固体的所有小体积条相加,求出其体积,就像我们将组成一个面积的所有小部分相加,以求出最终的图形面积一样。

\tb

\Examples.
(1) 求半径为$r$的球的体积。

一个薄球壳的体积为$4\pi x^2\, dx$(见图\ref{fig:59},第\pageref{fig:59}页);将组成球的所有同心球壳相加,我们得到
\[
\text{球体积}
  = \int^{x=r}_{x=0} 4\pi x^2\, dx
  = 4\pi \left[\frac{x^3}{3} \right]^r_0
  = \tfrac{4}{3} \pi r^3.
\]

我们也可以如下进行:球的一个切片,厚度为$dx$,其体积为$\pi y^2\, dx$(见图\ref{fig:62})。同时,$x$和$y$由以下表达式相关联:
\[
y^2 = r^2 - x^2.
\]
\begin{DPalign*}
\lintertext{\indent 因此}
\text{球体积}
  &= 2 \int^{x=r}_{x=0} \pi(r^2-x^2)\, dx \\
  &= 2 \pi \left[ \int^{x=r}_{x=0} r^2\, dx - \int^{x=r}_{x=0} x^2\, dx \right] \\
  &= 2 \pi \left[r^2x - \frac{x^3}{3} \right]^r_0 = \frac{4\pi}{3} r^3.
\end{DPalign*}

(2) 求曲线$y^2=6x$绕$x$轴旋转生成的固体在$x=0$到$x=4$之间的体积。

固体的条带体积为$\pi y^2\, dx$。
\begin{DPalign*}
\lintertext{\indent 因此}
\text{体积}
  &= \int^{x=4}_{x=0} \pi y^2\, dx = 6\pi \int^{x=4}_{x=0} x\, dx  \\
  &= 6\pi \left[ \frac{x^2}{2} \right]^4_0 = 48\pi = 150.8.
\end{DPalign*}

\Section{关于二次平均值。}

在某些物理学分支,特别是在研究交流电时,有必要计算一个变量的\emph{均方根}。所谓“均方根”是指在考虑的极限之间所有值的平方的平均值的平方根。任何量的均方根的其它名称是它的“有效”值,或其“\textsc{r.m.s.}”(即均方根)值。法语术语是\textit{valeur efficace}。如果~$y$是所考虑的函数,并且均方根是在$x=0$和$x=l$之间取的;那么均方根表示为
\[
\sqrt[2] {\frac{1}{l} \int^l_0 y^2\, dx}.
\]
\DPPageSep{235.png}{223}%

\Examples.
(1) 求函数$y=ax$的均方根(\Fig{63})。

这里积分是$\ds\int^l_0 a^2 x^2\, dx$,即~$\frac{1}{3} a^2 l^3$。

\Figure[2.5in]{235a}{63}

除以~$l$并取平方根,我们得到
\[
\text{均方根} = \frac{1}{\sqrt 3}\, al.
\]

这里算术平均值是~$\frac{1}{2}al$;均方根与算术平均值的比率(这个比率称为\emph{波形因数})是 $\dfrac{2}{\sqrt 3}=1.155$。

(2) 求函数$y=x^a$的均方根。

积分是$\ds\int^{x=l}_{x=0} x^{2a}\, dx$,即 $\dfrac{l^{2a+1}}{2a+1}$。
\begin{DPgather*}
\lintertext{因此}
\text{均方根} = \sqrt[2]{\dfrac{l^{2a}}{2a+1}}.
\end{DPgather*}

(3) 求函数$y=a^{\efrac{x}{2}}$的均方根。

积分是$\ds\int^{x=l}_{x=0} (a^{\efrac{x}{2}})^2\, dx$,即 $\ds\int^{x=l}_{x=0} a^x\, dx$,
\DPPageSep{236.png}{224}%
\begin{DPgather*}
\lintertext{或}
\left[ \frac{a^x}{\log_\epsilon a} \right]^{x=l}_{x=0},
\end{DPgather*}
即 $\dfrac{a^l-1}{\log_\epsilon a}$。

因此,均方根是 $\sqrt[2] {\dfrac{a^l - 1}{l \log_\epsilon a}}$。

\Exercises{XVIII}(答案见\Pageref{AnsEx:XVIII})
\begin{Problems}
\Item{(1)} 求曲线$y=x^2+x-5$在$x=0$和$x=6$之间的面积,以及这些极限之间的平均纵坐标。

\Item{(2)} 求抛物线$y=2a\sqrt x$在$x=0$和$x=a$之间的面积。证明它是极限纵坐标和其横坐标的矩形的三分之二。

\Item{(3)} 求正弦曲线正部分的面积和平均纵坐标。

\Item{(4)} 求曲线$y=\sin^2 x$正部分的面积,并求平均纵坐标。

\Item{(5)} 求曲线$y=x^2 ± x^{\efrac{5}{2}}$在$x=0$到$x=1$之间包含在两条分支之间的面积,以及曲线较低分支正部分的面积(见\Fig{30},\Pageref{fig:30})。

\Item{(6)} 求底半径为~$r$,高为~$h$的圆锥的体积。

\Item{(7)} 求曲线$y=x^3-\log_\epsilon x$在$x=0$和$x=1$之间的面积。

\Item{(8)} 求曲线$y=\sqrt{1+x^2}$绕$x$轴旋转时,在$x=0$和$x=4$之间生成的体积。
\DPPageSep{237.png}{225}%

\Item{(9)} 求正弦曲线绕$x$轴旋转时生成的体积。同时求其表面积。

\Item{(10)} 求曲线$xy=a$在$x=1$和$x = a$之间包含的部分的面积。求这些极限之间的平均纵坐标。

\Item{(11)} 证明函数$y=\sin x$在$0$到$\pi$弧度之间的均方根是~$\dfrac{\sqrt2}{2}$。同时求同一函数在相同极限之间的算术平均值;并证明波形因数是~$=1.11$。

\Item{(12)} 求函数$x^2+3x+2$在$x=0$到$x=3$之间的算术平均值和均方根。

\Item{(13)} 求函数$y=A_1 \sin x + A_1 \sin 3x$的均方根和算术平均值。

\Item{(14)} 某曲线方程为$y=3.42\epsilon^{0.21x}$。求曲线与$x$轴之间在$x=2$到$x = 8$之间的面积。同时求曲线在这些点之间的平均纵坐标的高度。

\Item{(15)} 证明一个圆的半径,其面积等于极坐标图面积的两倍,等于该极坐标图中所有$r$值的二次平均值。

\Item{(16)} 求曲线$y=±\dfrac{x}{6}\sqrt{x(10-x)}$绕$x$轴旋转所生成的体积。
\end{Problems}
\DPPageSep{238.png}{226}%

\Chapter{第廿章}{规避、陷阱与成功}

\Paragraph{规避。} 积分运算的大部分工作在于将它们整理成可以积分的形式。关于积分学的书籍——这里指的是严肃的书籍——充满了各种计划、方法、规避技巧和巧妙手段来进行这种工作。以下是其中的一些。

\Paragraph{分部积分法。}\Pagelabel{intparts} 这种方法的公式为
\[
\int u\, dx = ux - \int x\, du + C.
\]
它在某些无法直接处理的情况下很有用,因为它表明,如果任何情况下$\ds\int x\, du$可以求出,那么$\ds\int u\, dx$也可以求出。该公式可以通过以下方式推导。从\Pageref{differprod},我们有,
\[
d(ux) = u\, dx + x\, du,
\]
可以写成
\[
u(dx) = d(ux) - x\, du,
\]
通过直接积分得到上述表达式。
\DPPageSep{239.png}{227}%

\Examples.
(1) 求$\ds\int w · \sin w\, dw$。

设$u = w$,对于$\sin w · dw$记为$dx$。于是$du = dw$,而$\ds\int \sin w · dw = -\cos w = x$。

将这些代入公式,得到
\begin{align*}
\int w · \sin w\, dw &= w(-\cos w) - \int -\cos w\, dw  \\
                     &=-w \cos w + \sin w + C.
\end{align*}

(2) 求$\ds\int x \epsilon^x\, dx$。
%
\BindMath{\begin{DPalign*}
\lintertext{\indent 设}
 u &=  x, & \epsilon^x\, dx&=dv; \\
\lintertext{则}
du &= dx, & v &=\epsilon^x,
\end{DPalign*}
\begin{DPgather*}
\lintertext{于是}
\int x\epsilon^x\, dx
   = x\epsilon^x - \int \epsilon^x\, dx
      \quad \text{(根据公式)} \\
   = x \epsilon^x - \epsilon^x = \epsilon^x(x-1) + C.
\end{DPgather*}}%

(3) 试求$\ds\int \cos^2 \theta\, d\theta$。\Pagelabel{moreexamples}
\begin{DPalign*}
u &= \cos \theta, &\cos \theta\, d\theta &= dv. \\
\lintertext{\indent 因此}
du&= -\sin \theta\, d\theta, & v &=\sin \theta,
\end{DPalign*}
\begin{align*}
\int \cos^2 \theta\, d\theta
  &= \cos \theta \sin \theta+ \int \sin^2 \theta\, d\theta       \\
  &= \frac{2 \cos\theta \sin\theta}{2} +\int(1-\cos^2 \theta)\, d\theta  \\
  &= \frac{\sin 2\theta}{2} + \int d\theta - \int \cos^2 \theta\, d\theta.
\end{align*}
\begin{DPalign*}
\lintertext{\indent 因此}
2 \int \cos^2 \theta\, d\theta
  &= \frac{\sin 2\theta}{2} + \theta \\
\lintertext{且}
\int \cos^2 \theta\, d\theta
  &= \frac{\sin 2\theta}{4} + \frac{\theta}{2} + C.
\end{DPalign*}
\DPPageSep{240.png}{228}%

(4) 求$\ds\int x^2 \sin x\, dx$。
%
\begin{DPalign*}
\lintertext{\indent 设}
x^2  &= u, & \sin x\, dx &= dv; \\
\lintertext{则}
du &= 2x\, dx, & v &= -\cos x,
\end{DPalign*}
\[
\int x^2 \sin x\, dx = -x^2 \cos x + 2 \int x \cos x\, dx.
\]

现在求$\ds\int x \cos x\, dx$,使用分部积分法(如上例1所示):
\[
\int x \cos x\, dx = x \sin x + \cos x+C.
\]

因此
\begin{align*}
\int x^2 \sin x\, dx
  &= -x^2 \cos x + 2x \sin x + 2 \cos x + C' \\
  &= 2 \left[ x \sin x + \cos x \left(1 - \frac{x^2}{2}\right) \right] +C'.
\end{align*}

(5) 求$\ds\int \sqrt{1-x^2}\, dx$。
\begin{DPalign*}
\lintertext{设}
u &= \sqrt{1-x^2},\quad dx=dv;  \\
\lintertext{则}
du &= -\frac{x\, dx}{\sqrt{1-x^2}}\quad \text{(见第九章,\Pageref{chap:IX})}
\end{DPalign*}
且$x=v$;因此
\[
\int \sqrt{1-x^2}\, dx=x \sqrt{1-x^2} + \int \frac{x^2\, dx}{\sqrt{1-x^2}}.
\]

这里我们可以使用一个小技巧,因为我们可以写成
\[
\int \sqrt{1-x^2}\, dx
  = \int \frac{(1-x^2)\, dx}{\sqrt{1-x^2}}
  = \int \frac{dx}{\sqrt{1-x^2}} - \int \frac{x^2\, dx}{\sqrt{1-x^2}}.
\]

将这两式相加,我们消去了$\ds\int \dfrac{x^2\, dx}{\sqrt{1-x^2}}$,得到
\[
2 \int \sqrt{1-x^2}\, dx = x\sqrt{1-x^2} + \int \frac{dx}{\sqrt{1-x^2}}.
\]
\DPPageSep{241.png}{229}%

你还记得遇到过$\dfrac {dx}{\sqrt{1-x^2}}$吗?它是由对$y=\arcsin x$求导得到的(见\Pageref{intex3});因此它的积分是$\arcsin x$,于是
\[
\int \sqrt{1-x^2}\, dx = \frac{x \sqrt{1-x^2}}{2} + \tfrac{1}{2} \arcsin x +C.
\]

你现在可以尝试自己做些练习;在本章末尾你会找到一些。

\Paragraph{代换法。}这与第九章\Pageref{chap:IX}中解释的技巧相同。让我们通过几个例子来说明它在积分中的应用。

(1) $\ds\int \sqrt{3+x}\, dx$.
\begin{DPalign*}
\lintertext{\indent 设}
3+x &= u,\quad dx = du; \\
\lintertext{代换}
\int u^{\efrac{1}{2}}\, du
  &= \tfrac{2}{3} u^{\efrac{3}{2}} = \tfrac{2}{3}(3+x)^{\efrac{3}{2}}.
\end{DPalign*}

(2) $\ds\int \dfrac{dx}{\epsilon^x+\epsilon^{-x}}$.
\begin{DPgather*}
\lintertext{\indent 设}
\epsilon^x = u,\quad \frac{du}{dx} = \epsilon^x,\quad\text{且}\quad
dx = \frac{du}{\epsilon^x}; \\
\lintertext{因此}
\int \frac{dx}{\epsilon^x+\epsilon^{-x}}
  = \int \frac{du}{\epsilon^x(\epsilon^x+\epsilon^{-x})}
  = \int \frac{du}{u\left(u + \dfrac{1}{u}\right)}
  = \int \frac{du}{u^2+1}.
\end{DPgather*}

$\dfrac{du}{1+u^2}$是$\arctan x$的导数结果。

因此积分结果为$\arctan \epsilon^x$。

(3) $\ds\int \dfrac{dx}{x^2+2x+3} = \ds\int \dfrac{dx}{x^2+2x+1+2} = \ds\int \dfrac{dx}{(x+1)^2+(\sqrt 2)^2}$.
\DPPageSep{242.png}{230}%
\begin{DPgather*}
\lintertext{\indent 设}
x+1=u,\quad dx=du;
\end{DPgather*}
则积分变为$\ds\int \dfrac{du}{u^2+(\sqrt2)^2}$;但$\dfrac{du}{u^2+a^2}$是$u=\dfrac{1}{a} \arctan \dfrac{u}{a}$的导数结果。

因此最终得到给定积分的值为$\dfrac{1}{\sqrt2} \arctan \dfrac{x+1}{\sqrt 2}$。

\emph{归约公式}是主要适用于二项式和三角函数表达式的特殊形式,这些表达式需要被积分并归约为已知积分的形式。

\textit{有理化}和\textit{分母因式分解}是适用于特殊情况的技巧,它们没有简短或通用的解释。需要大量的练习才能熟悉这些预备过程。

下面的例子展示了如何利用我们在第十三章\Pageref{partfracs2}学到的部分分式分解过程来进行积分。

再次考虑$\ds\int \dfrac{dx}{x^2+2x+3}$;如果我们将$\dfrac{1}{x^2+2x+3}$分解为部分分式,这变为(见\Pageref{partfracs3}):
\[
\dfrac{1}{2\sqrt{-2}} \left[\int \dfrac{dx}{x+1-\sqrt{-2}} - \int \dfrac{dx}{x+1+\sqrt{-2}} \right]
\]
\[
=\dfrac{1}{2\sqrt{-2}} \log_\epsilon \dfrac{x+1-\sqrt{-2}}{x+1+\sqrt{-2}}.
\]
注意,同一个积分有时可以用不止一种方式表示(这些方式彼此等价)。

\Paragraph{陷阱。}初学者容易忽略某些要点,而熟练者则会避免;例如使用等于零或无穷大的因子,以及出现诸如$\tfrac{0}{0}$的不定量。没有一条黄金法则能应对所有可能的情况。只有通过练习和细心的关注才能解决问题。在第十八章\Pageref{chap:XVIII}中,当我们遇到积分$x^{-1}\, dx$的问题时,就出现了一个需要规避的陷阱的例子。

\Paragraph{成就.} 所谓成就,指的是微积分在解决其他方法难以处理的难题时所取得的成功。在考虑物理关系时,常常能够构建出一个表达式,用于描述各部分之间的相互作用或支配它们的力的规律,这种表达式自然是以\emph{微分方程}的形式出现,即包含微分系数或带有其他代数量的一种方程。当找到这样的微分方程后,除非将其积分,否则无法继续前进。通常,陈述适当的微分方程比求解它要容易得多:真正的困难始于试图积分时,除非该方程具有某种标准形式,其积分已知,那样的话,成就便轻而易举。通过积分微分方程得到的结果称为其“解”;令人惊讶的是,在许多情况下,解看起来似乎与它所积分的微分方程毫无关系。解往往与原始表达式看起来截然不同,就像蝴蝶与它曾是毛毛虫时的样子。谁能想到,像
\[
\dfrac{dy}{dx} = \dfrac{1}{a^2-x^2}
\]
这样无害的东西,会演变成
\[
y = \dfrac{1}{2a} \log_\epsilon \dfrac{a+x}{a-x} + C?
\]
然而后者正是前者的\textit{解}。

作为最后一个例子,让我们一起求解上述问题。

通过部分分式,\Pagelabel{partfracs3}
\begin{align*}
\frac{1}{a^2-x^2} &= \frac{1}{2a(a+x)} + \frac{1}{2a(a-x)},  \\
dy &= \frac {dx}{2a(a+x)}+ \frac{dx}{2a(a-x)},  \\
y  &= \frac{1}{2a}
       \left( \int \frac{dx}{a+x}
            + \int \frac{dx}{a-x} \right) \displaybreak[1] \\
   &= \frac{1}{2a} \left(\log_\epsilon (a+x) - \log_\epsilon (a-x) \right) \displaybreak[1] \\
   &= \frac{1}{2a} \log_\epsilon \frac{a+x}{a-x} + C.
\end{align*}
\DPPageSep{245.png}{233}%
% [** TN: Hack removes vertical space, giving a better page break below]
\indent 这并不是一个非常困难的蜕变!

有许多专著,如布尔的《微分方程》,专门致力于通过寻找不同原始形式的“解”来解决这一主题。

\Exercises{XIX} (答案见\Pageref{AnsEx:XIX}。)

\begin{Problems}[2]
\Item{(1)} 求 $\ds\int \sqrt {a^2 - x^2}\, dx$。
\Item{(2)} 求 $\ds\int x \log_\epsilon x\, dx$。

\ResetCols{2}
\Item{(3)} 求 $\ds\int x^a \log_\epsilon x\, dx$。
\Item{(4)} 求 $\ds\int \epsilon^x \cos \epsilon^x\, dx$。

\ResetCols{2}
\Item{(5)} 求 $\ds\int \dfrac{1}{x} \cos (\log_\epsilon x)\, dx$。
\Item{(6)} 求 $\ds\int x^2 \epsilon^x\, dx$。

\ResetCols{2}
\Item{(7)} 求 $\ds\int \dfrac{(\log_\epsilon x)^a}{x}\, dx$。
\Item{(8)} 求 $\ds\int \dfrac{dx}{x \log_\epsilon x}$。

\ResetCols{2}
\Item{(9)} 求 $\ds\int \dfrac{5x+1}{x^2 +x-2}\, dx$。
\Item{(10)} 求 $\ds\int \dfrac{(x^2 -3)\, dx}{x^3 - 7x+6}$。

\ResetCols{2}
\Item{(11)} 求 $\ds\int \dfrac{b\, dx}{x^2 -a^2}$。
\Item{(12)} 求 $\ds\int \dfrac{4x\, dx}{x^4 -1}$。

\ResetCols{2}
\Item{(13)} 求 $\ds\int \dfrac{dx}{1-x^4}$。
\Item{(14)} 求 $\ds\int \dfrac{dx}{x \sqrt {a-bx^2}}$。
\end{Problems}
\DPPageSep{246.png}{234}%

\Chapter[求解方法]{第廿一章}{求解一些方程}

\First{本章}我们将着手求解一些重要的微分方程,为此运用前几章展示的过程。

初学者现在已知道,那些过程本身大多很容易,在这里将开始意识到积分是一门艺术。如同所有艺术一样,在这门艺术中,只有通过勤奋和规律的练习才能获得熟练。想要达到这种熟练的人必须解例题,更多的例题,以及更多的例题,正如在所有正规的微积分教材中都能找到丰富的例题一样。我们的目的在这里必须是提供最简短的入门,以进入严肃的工作。

\tb

\Example{1.} 求解微分方程
\[
ay + b \frac{dy}{dx} = 0.
\]

移项后得到
\[
b \frac{dy}{dx} = -ay.
\]
\DPPageSep{247.png}{235}%

现在,仅通过观察这个关系式,我们就可以知道,我们处理的是一个$\dfrac{dy}{dx}$与$y$成比例的情况。如果我们想象一条曲线来表示$y$作为$x$的函数,这条曲线在任意点的斜率将与该点的纵坐标成正比,并且如果$y$为正,斜率将为负。因此,显然这条曲线将是一条衰减曲线(\Pageref{section:5}),解中将包含$\epsilon^{-x}$作为因子。但是,不依赖于这一点的洞察力,让我们开始工作。

由于$y$和$dy$都出现在方程中且位于两侧,我们无法进行任何操作,直到我们将$y$和$dy$移到一侧,将$dx$移到另一侧。为此,我们必须将通常不可分割的伙伴$dy$和$dx$分开。
\[
\frac{dy}{y} = - \frac{a}{b}\, dx.
\]

完成这一操作后,我们现在可以看到,两侧都已变成可积分的形式,因为我们认出$\dfrac{dy}{y}$,即$\dfrac{1}{y}\, dy$,是在对数微分时遇到过的微分(\Pageref{expolo})。因此,我们可以立即写下积分的指令,
\[
\int \frac{dy}{y} = \int -\frac{a}{b}\, dx;
\]
并进行两次积分,我们得到:
\[
\log_\epsilon y = -\frac{a}{b} x + \log_\epsilon C,
\]
\DPPageSep{248.png}{236}%
其中$\log_\epsilon C$是尚未确定的积分常数\footnote
  {我们可以写下任何形式的常数作为“积分常数”,这里选择$\log_\epsilon C$的形式,是因为这一行方程中的其他项,或者被视为对数;如果添加的常数与它们是同一类,可以避免后续的复杂性。}。然后,去对数化,我们得到:
\[
y = C \epsilon^{-\efrac{a}{b} x},
\]
这就是所需的解。现在,这个解看起来与构建它的原始微分方程完全不同:然而,对于一个经验丰富的数学家来说,它们都传达了关于$y$如何依赖于$x$的相同信息。

至于$C$,它的意义取决于$y$的初始值。因为如果我们设$x = 0$以查看$y$此时的值,我们发现这使得$y = C \epsilon^{-0}$;由于$\epsilon^{-0} = 1$,我们看到$C$不过是$y$在起始时的特定值\footnote
  {比较关于“积分常数”的讨论,参考\Fig{48}在\Pageref{constant},以及\Fig{51}在\Pageref{fig:51}。}。我们可以将其称为$y_0$,因此将解写为
\[
y = y_0 \epsilon^{-\efrac{a}{b} x}.
\]

\tb

\Example{2.}

让我们取一个例子来求解
\[
ay + b \frac{dy}{dx} = g,
\]
其中$g$是一个常数。再次,检查方程会提示,(1)无论如何$\epsilon^x$将出现在解中,(2)如果在曲线的某一部分$y$达到最大值或最小值,使得$\dfrac{dy}{dx} = 0$,那么$y$将具有值$=\dfrac{g}{a}$。但让我们像之前一样开始工作,分离微分并尝试将其转化为某种可积分的形式。
\begin{align*}
b\frac{dy}{dx}           &= g -ay; \\
\frac{dy}{dx}            &= \frac{a}{b}\left(\frac{g}{a}-y\right); \\
\frac{dy}{y-\dfrac{g}{a}} &= -\frac{a}{b}\, dx.
\end{align*}

现在我们已经尽力将方程的一边只保留$y$和$dy$,另一边只保留$dx$。
但左边得到的结果是否可积呢?

它与\Pageref{differlog}页的结果形式相同;因此,写出积分指令,我们得到:
\[
\int{\frac{dy}{y-\dfrac{g}{a}}} = - \int{\frac{a}{b}\, dx};
\]
进行积分并加上适当的常数后,
\begin{DPalign*}
\log_\epsilon\left(y-\frac{g}{a}\right) &= -\frac{a}{b}x + \log_\epsilon C; \\
\lintertext{由此可得 }      y-\frac{g}{a} &= C\epsilon^{-\efrac{a}{b}x}; \\
\lintertext{最终得到 }            y &= \frac{g}{a} + C\epsilon^{-\efrac{a}{b}x},
\end{DPalign*}
这就是\emph{解}。
\DPPageSep{250.png}{238}%

如果设定条件为当$x = 0$时$y = 0$,我们可以求出$C$;此时指数项变为$1$;于是有
\begin{DPalign*}
                0 &= \frac{g}{a} + C, \\
\lintertext{即} C &= -\frac{g}{a}.
\end{DPalign*}

代入此值,解变为
\[
y = \frac{g}{a} (1-\epsilon^{-\efrac{a}{b} x}).
\]

此外,如果$x$无限增长,$y$将趋近于一个最大值;当$x=\infty$时,指数项为$0$,得到$y_{\text{max.}} = \dfrac{g}{a}$。代入后,最终得到
\[
y = y_{\text{max.}}(1-\epsilon^{-\efrac{a}{b} x}).
\]

这一结果在物理科学中也具有重要意义。

\tb

\Example{3.}
\begin{DPgather*}
\lintertext{\indent 设} ay+b\frac{dy}{dt} = g · \sin 2\pi nt.
\end{DPgather*}

我们发现这个方程比前面的更难处理。首先两边同时除以$b$。
\[
\frac{dy}{dt} + \frac{a}{b}y = \frac{g}{b} \sin 2\pi nt.
\]

现在,左边直接积分不可行。但可以通过技巧——这也是经验和技巧提示的策略——将所有项乘以$\epsilon^{\efrac{a}{b} t}$,得到:
\[
\frac{dy}{dt} \epsilon^{\efrac{a}{b} t} + \frac{a}{b} y \epsilon^{\efrac{a}{b} t} = \frac{g}{b} \epsilon^{\efrac{a}{b} t} · \sin 2 \pi nt,
\]
这等同于
\[
\frac{dy}{dt} \epsilon^{\efrac{a}{b} t} + y \frac{d(\epsilon^{\efrac{a}{b} t})}{dt} = \frac{g}{b} \epsilon^{\efrac{a}{b} t} · \sin 2 \pi nt;
\]
这是一个完全微分,可以这样积分:因为如果$u = y\epsilon^{\efrac{a}{b} t}$,则$\dfrac{du}{dt} = \dfrac{dy}{dt} \epsilon^{\efrac{a}{b} t} + y \dfrac{d(\epsilon^{\efrac{a}{b} t})}{dt}$,
\begin{DPalign*}
  y \epsilon^{\efrac{a}{b} t}
  &= \frac{g}{b} \int \epsilon^{\efrac{a}{b} t} · \sin 2 \pi nt · dt + C, \\
\lintertext{即}
y &= \frac{g}{b} \epsilon^{-\efrac{a}{b} t}
     \int \epsilon^{ \efrac{a}{b} t} · \sin 2\pi nt · dt
       + C\epsilon^{-\efrac{a}{b} t}.
\tag*{[\textsc{a}]}%[** TN: Omitted dot leaders here, below.]
\end{DPalign*}

最后一项显然会随着$t$的增加而消失,可以忽略。现在的问题在于求解作为因子的积分。为此,我们采用分部积分法(见\Pageref{intparts}),其一般公式为$\ds\int u dv = uv - \int v du$。为此,设
\begin{align*}
&\left\{
\begin{aligned}
 u &= \epsilon^{\efrac{a}{b} t}; \\
dv &= \sin 2\pi nt · dt.
\end{aligned}
\right. \displaybreak[1] \\
\intertext{\indent 则有}
&\left\{
\begin{aligned}
du &= \epsilon^{\efrac{a}{b} t} × \frac{a}{b}\, dt; \\
v &= - \frac{1}{2\pi n} \cos 2\pi nt.
\end{aligned}
\right.
\end{align*}
\DPPageSep{252.png}{240}%

代入这些,所求积分变为:
\begin{align*}
\int \epsilon^{\efrac{a}{b} t} &{} · \sin 2 \pi n t · dt \\
&= -\frac{1}{2 \pi n} · \epsilon^{\efrac{a}{b} t} · \cos 2 \pi nt
   -\int -\frac{1}{2\pi n} \cos 2 \pi nt · \epsilon^{\efrac{a}{b} t} · \frac{a}{b}\, dt \\
&= -\frac{1}{2 \pi n} \epsilon^{\efrac{a}{b} t} \cos 2 \pi nt
   +\frac{a}{2 \pi nb} \int \epsilon^{\efrac{a}{b} t} · \cos 2 \pi nt · dt.
\tag*{[\textsc{b}]}
\end{align*}

最后一个积分仍然是不可约的。为了避开这个困难,重复对左边进行分部积分,但以相反的方式处理,写作:
\begin{DPalign*}
&\left\{
\begin{aligned}
u &= \sin 2 \pi n t ; \\
dv &= \epsilon^{\efrac{a}{b} t} · dt;
\end{aligned}
\right. \\[1ex]
\lintertext{由此}
&\left\{
  \begin{aligned}
  du &= 2 \pi n · \cos 2 \pi n t · dt; \\
 v &= \frac{b}{a} \epsilon ^{\efrac{a}{b} t}
\end{aligned}
\right.
\end{DPalign*}

代入这些,我们得到
\begin{align*}
\int \epsilon^{\efrac{a}{b} t} &{} · \sin 2 \pi n t · dt\\
&= \frac{b}{a} · \epsilon^{\efrac{a}{b} t} · \sin 2 \pi n t -
   \frac{2 \pi n b}{a} \int \epsilon^{\efrac{a}{b} t} · \cos 2 \pi n t · dt. \tag*{[\textsc{c}]}
\end{align*}

注意到[\textsc{c}]中的最终不可约积分与[\textsc{b}]中的相同,我们可以通过将[\textsc{b}]乘以$\dfrac{2 \pi nb}{a}$,将[\textsc{c}]乘以$\dfrac{a}{2 \pi nb}$,并将它们相加来消除它。
\DPPageSep{253.png}{241}%

结果,当简化后,得到:
\begin{align*}
\int \epsilon^{\efrac{a}{b} t} · \sin 2 \pi n t · dt
  &= \epsilon^{\efrac{a}{b} t} \left\{\frac{ ab · \sin 2 \pi nt - 2 \pi n b^2 · \cos 2 \pi n t}{ a^2 + 4 \pi^2 n^2 b^2 } \right\}
\tag*{[\textsc{d}]} &\\
\intertext{\indent 将此值代入[\textsc{a}],我们得到}
y &= g \left\{\frac{ a · \sin 2 \pi n t - 2 \pi n b · \cos 2 \pi nt}{ a^2 + 4 \pi^2  n^2 b^2}\right\}. &
\end{align*}

为了进一步简化,让我们设想一个角度$\phi$,使得$\tan \phi = \dfrac{2 \pi n b}{ a}$。
\begin{DPalign*}
\lintertext{\indent 那么}
\sin \phi &= \frac{2 \pi nb}{\sqrt{a^2 + 4 \pi^2 n^2 b^2}}, \\
\lintertext{并且}
\cos \phi &= \frac{a}{\sqrt{a^2 + 4 \pi^2 n^2 b^2}}. \displaybreak[1] \\
\intertext{\rlap{代入这些,我们得到:}}
y &= g \frac{\cos \phi · \sin 2 \pi nt
  - \sin \phi · \cos 2 \pi nt}{\sqrt{a^2 + 4 \pi^2 n^2 b^2}}, \\
\intertext{\rlap{可以写成}}
y &= g \frac{\sin(2 \pi nt - \phi)}{\sqrt{a^2 + 4 \pi^2 n^2 b^2}},\\
\lintertext{\rlap{这就是所需的\textit{解}。}}
\end{DPalign*}

这确实不是别的,正是交变电流的方程,其中$g$表示电动势的振幅,$n$表示频率,$a$表示电阻,$b$表示电路的自感系数,$\phi$是滞后角。

\tb
\DPPageSep{254.png}{242}%

\Example{4.}
\begin{DPgather*}
\lintertext{\indent 假设}
M\, dx + N\, dy = 0.
\end{DPgather*}

如果$M$只是$x$的函数,$N$只是$y$的函数,我们可以直接积分这个表达式;但如果$M$和$N$都是依赖于$x$和$y$的函数,我们该如何积分呢?它本身是一个全微分吗?也就是说,$M$和$N$是否都是从某个共同的函数$U$通过偏微分形成的?如果是,那么
\[\left\{
  \begin{aligned}
 \frac{\partial U}{\partial x} = M, \\
 \frac{\partial U}{\partial y} = N.
  \end{aligned}
\right.
\]
并且如果存在这样一个共同函数,那么
\[
\frac{\partial U}{\partial x}\, dx + \frac{\partial U}{\partial y}\, dy
\]
就是一个全微分(比较\Pageref{partialdiff})。

%[** TN: 保留原书中使用d而不是\partial的用法]
现在,这个问题的检验方法是:如果表达式是一个全微分,那么必须满足
\begin{DPalign*}
        \frac{dM}{dy} &= \frac{dN}{dx}; \\
\lintertext{因为这样}
\frac{d(dU)}{dx\, dy} &= \frac{d(dU)}{dy\, dx},\\
\lintertext{这是必然成立的。}
\end{DPalign*}

以方程
\[
(1 + 3 xy)\, dx + x^2\, dy = 0.
\]
为例。
\DPPageSep{255.png}{243}%

这是全微分吗?应用检验。
\[\left\{
  \begin{aligned}
 \frac{d(1 + 3xy)}{dy}=3x, \\
  \PadTo{\dfrac{d(1 + 3xy)}{dy}}{\dfrac{d(x^2)}{dx}} = 2x,
  \end{aligned}
\right.
\]
它们并不相等。因此,这不是一个全微分,两个函数$1+3xy$和$x^2$并非来自一个共同的原始函数。

在这种情况下,尽管如此,仍有可能发现\emph{一个积分因子},即一个因子,使得如果两者都乘以这个因子,表达式将变成一个全微分。没有一种规则可以发现这样的积分因子;但经验通常会提供一个。在当前情况下,$2x$将起到这样的作用。乘以~$2x$,我们得到
\[
(2x + 6x^2y)\, dx + 2x^3\, dy = 0.
\]

现在对这个表达式进行检验。
\[
\left\{
  \begin{aligned}
 \frac{d(2x + 6x^2y)}{dy}=6x^2, \\
 \PadTo{\dfrac{d(2x + 6x^2y)}{dy}}{\dfrac{d(2x^3)}{dx}} = 6x^2,
  \end{aligned}
\right.
\]
结果一致。因此,这是一个全微分,可以进行积分。现在,如果 $w = 2x^3y$,
\begin{DPgather*}
dw=6x^2y\, dx + 2x^3\, dy. \\
\lintertext{\indent 因此} \int 6x^2y\, dx + \int 2x^3\, dy=w=2x^3y; \\
\lintertext{所以我们得到}  U = x^2 + 2x^3y + C.
\end{DPgather*}
\DPPageSep{256.png}{244}%

\tb %[** TN: No thought break in orignal]
\Example{5.} 设 $\dfrac{d^2 y}{dt^2} + n^2 y = 0$。

在这种情况下,我们有一个二阶微分方程,其中 $y$ 以二阶导数的形式出现,同时也以自身形式出现。

移项后,我们得到 $\dfrac{d^2 y}{dt^2} = - n^2 y$。

由此可以看出,我们处理的是一个函数,其二阶导数与自身成正比,但符号相反。在第十五章中,我们发现有这样的函数——即\emph{正弦}(或\emph{余弦}),它们具有这一性质。因此,我们可以直接推断出解的形式为 $y = A \sin (pt + q)$。不过,让我们继续推导。

将原方程两边乘以 $2\dfrac{dy}{dt}$ 并积分,得到 $2\dfrac{d^2 y}{dt^2}\, \dfrac{dy}{dt} + 2x^2 y \dfrac{dy}{dt} = 0$,并且,由于
\[
2 \frac{d^2y}{dt^2}\, \frac{dy}{dt}
  = \frac{d \left(\dfrac{dy}{dt}\right)^2}{dt},\quad
\left(\frac{dy}{dt}\right)^2 + n^2 (y^2-C^2) = 0,
\]
$C$ 是一个常数。然后,取平方根,
\[
\frac{dy}{dt} = -n \sqrt{ y^2 - C^2}\quad \text{和}\quad
\frac{dy}{\sqrt{C^2 - y^2}} = n · dt.
\]

但可以证明(见 \Pageref{intex3})
\[
\frac{1}{\sqrt{C^2 - y^2}} = \frac{d (\arcsin \dfrac{y}{C})}{dy};
\]
因此,从角度转换为正弦,
\[
\arcsin \frac{y}{C} = nt + C_1\quad \text{和}\quad y = C \sin (nt + C_1),
\]
\DPPageSep{257.png}{245}%
其中 $C_1$ 是一个通过积分引入的常数角。

或者,更优选地,这可以写成
\[
y = A \sin nt + B \cos nt, \text{ 这就是解。}
\]

\tb

\Example{6.} \hfil $\dfrac{d^2 y}{dt^2} - n^2 y = 0$.\phantom{\indent\textit{Example 6.}}

这里我们显然要处理一个函数 $y$,其二阶导数与自身成正比。我们唯一知道的具有这种性质的函数是指数函数(见 \Pageref{unchanged}),因此我们可以确定方程的解将具有这种形式。

按照之前的方法,通过两边乘以 $2 \dfrac{dy}{dx}$ 并积分,我们得到 $2\dfrac{d^2 y}{dx^2}\, \dfrac{dy}{dx} - 2x^2 y \dfrac{dy}{dx}=0$,
\begin{DPgather*}
\lintertext{并且,由于}
2\frac{d^2 y}{dx^2}\, \frac{dy}{dx}
  = \frac{d \left(\dfrac{dy}{dx}\right)^2}{dx},\quad
\left(\frac{dy}{dx}\right)^2 - n^2 (y^2 + c^2) = 0, \\
\frac{dy}{dx} - n \sqrt{y^2 + c^2} = 0,
\end{DPgather*}
其中 $c$ 是一个常数,且 $\dfrac{dy}{\sqrt{y^2 + c^2}} = n\, dx$。

现在,如果\quad $w = \log_\epsilon ( y+ \sqrt{y^2+ c^2}) = \log_\epsilon u$,
\begin{DPgather*}
\frac{dw}{du} = \frac{1}{u},\quad \frac{du}{dy} = 1 + \frac{y}{\sqrt{y^2 + c^2}} = \frac{y + \sqrt{ y^2 + c^2}}{\sqrt{y^2 + c^2}} \\
\lintertext{并且} \frac{dw}{dy} = \frac{1}{\sqrt{ y^2 + c^2}}.
\end{DPgather*}

因此,积分后得到
\begin{DPgather*}
\log_\epsilon (y + \sqrt{y^2 + c^2} ) = nx + \log_\epsilon C, \\
y + \sqrt{y^2 + c^2} = C \epsilon^{nx}.
\tag*{(1)} \displaybreak[1] \\
\lintertext{\indent 现在} \qquad ( y + \sqrt{y^2 + c^2} ) × ( -y + \sqrt{y^2 + c^2} ) = c^2 ;    \\
\lintertext{由此} \qquad  -y + \sqrt{y^2 + c^2} = \dfrac{c^2}{C} \epsilon^{-nx}.
\tag*{(2)}
\end{DPgather*}
\DPPageSep{258.png}{246}%

从(1)中减去(2)并除以~$2$,我们得到
\[
y = \frac{1}{2} C \epsilon^{nx} - \frac{1}{2}\, \frac{c^2}{C} \epsilon^{-nx},
\]
更方便地写成
\[
y = A \epsilon^{nx} + B \epsilon^{-nx}.
\]
或者,这个解乍看之下似乎与原方程无关,但实际上表明$y$由两个部分组成,其中一个部分随着$x$的增加而对数增长,另一个部分则随着$x$的增加而消失。

\tb

\Example{7.}
\begin{DPalign*}
\lintertext{\indent 设}
b \frac{d^2y}{dt^2} + a \frac{dy}{dt} + gy &= 0.
\end{DPalign*}

检查这个表达式可以发现,如果$b = 0$,它具有例1的形式,其解为负指数函数。另一方面,如果$a = 0$,它的形式与例6相同,其解是正指数和负指数的和。因此,发现当前例子的解为
\begin{DPalign*}
y &= (\epsilon^{-mt})(A \epsilon^{nt} + B \epsilon^{-nt}), \\
\lintertext{其中}
m &= \frac{a}{2b}\quad \text{和}\quad
n  = \sqrt{\frac{a^2}{4b^2}} - \frac{g}{b}.
\end{DPalign*}

达到这个解的步骤在此未给出;它们可以在高级教程中找到。

\tb
\DPPageSep{259.png}{247}%

\Example{8.}
\[
\frac{d^2y}{dt^2} = a^2 \frac{d^2y}{dx^2}.
\]

如(\Pageref{Example4})所示,这个方程是从原始方程
\[
y = F(x+at) + f(x-at),
\]
推导出来的,其中$F$和$f$是$t$的任意函数。

另一种处理方法是将其通过变量变换转化为
\[
\frac{d^2y}{du · dv} = 0,
\]
其中$u = x + at$,$v = x - at$,从而得到相同的一般解。如果我们考虑$F$为零的情况,那么我们简单地得到
\[
y = f(x-at);
\]
这仅仅表明,在时间$t = 0$时,$y$是$x$的特定函数,可以看作表示$y$与$x$的关系曲线具有特定形状。然后,$t$的任何变化都相当于从$x$的起始点进行改变。也就是说,它表明,函数的形状保持不变,它以均匀速度$a$沿$x$方向传播;因此,无论在特定时间$t_0$和特定点$x_0$处$y$的值是多少,在随后的时间$t_1$时,在更远的点上将出现相同的$y$值,该点的横坐标为$x_0 + a(t_1 - t_0)$。在这种情况下,简化的方程表示以均匀速度沿$x$方向传播的波(任意形状)。

如果微分方程写成
\[
m \frac{d^2y}{dt^2} = k\, \frac{d^2y}{dx^2},
\]
解将是相同的,但传播速度将具有值
\[
a = \sqrt{\frac{k}{m}}.
\]

\tb

现在,您已经被亲自引导进入这片神奇的领域。为了方便您参考主要结果,作者在向您告别时,愿意向您提供一份护照,形式为方便的标准形式集合(见\Pagerange{stdforms1}{stdforms2})。中间列出了一些最常见的函数。它们的微分结果列在左侧;积分结果列在右侧。愿您发现它们有用!
\DPPageSep{261.png}{249}%
\backmatter
\phantomsection
\pdfbookmark[-1]{附录}{附录}

\ChapterStar{结语与寓言}

\First{可}以自信地认为,当这本名为“轻松学微积分”的小册子落入专业数学家之手时,他们(如果不至于太懒惰)会一致起身,将其斥为一本彻头彻尾的坏书。从他们的角度来看,对此毫无怀疑的余地。它犯下了几个极其严重且令人遗憾的错误。

首先,它展示了微积分的多数运算实际上是多么荒谬地简单。

其次,它泄露了如此多的行业秘密。通过告诉你“一个傻瓜能做到的,其他傻瓜也能做到”,它让你明白,那些以掌握如此艰深的微积分学科为傲的数学家们,并没有那么充分的理由自满。他们希望你认为这门学科极其困难,并不想让这种迷信被粗暴地打破。

第三,在他们对“如此简单”的可怕评价中,有一条是:作者完全未能以严格且令人满意的方式完整地证明他所简单呈现的若干方法的有效性,甚至敢于在解决问题时使用这些方法!但他为何不能这样做呢?你不会禁止每个不懂如何制造手表的人使用手表吧?你不会反对音乐家演奏他未曾亲手制作的提琴。你不会在孩子已经熟练使用语言之后才教他们语法规则。要求对微积分初学者讲解普遍严格的证明同样荒谬。

专业数学家们还会对这本彻底糟糕且有害的书提出另一项指责:它之所以如此简单,是因为作者省略了所有真正困难的内容。而关于这一指控的可怕事实是——它确实属实!这正是本书的写作目的——为那些迄今为止因微积分教学几乎总是以愚蠢的方式呈现而望而却步的无辜者而写。任何学科都可以通过充满困难的方式呈现而变得令人反感。本书的目的是让初学者能够学习其语言,熟悉其令人喜爱的简单性,并掌握其解决问题的强大方法,而不必被迫通过那些复杂且大多无关的数学体操来劳苦,这些体操深受不切实际的数学家喜爱。

在年轻的工程师中,有一些人听到“一个傻瓜能做到的,另一个也能做到”这句格言时,可能会感到熟悉。恳请他们不要揭露作者的愚蠢,也不要告诉数学家他到底有多傻。

\AltChapter{标准形式表}

\vfil
\begin{center}
\Pagelabel{stdforms1}
\begin{tabular}{|c|c|c|}
\hline
\multicolumn{1}{|c}{\DStrut$\dfrac{dy}{dx}$}
  & \multicolumn{1}{c}{$\longleftarrow\quad y\quad\longrightarrow$}
  & $\ds\int y\, dx$ \\
\hline
\ColumnHead{代数式。} \\
\DStrut$1$ & $x$     & $\frac{1}{2} x^2 + C$ \\
\DStrut$0$ & $a$     & $ax + C $             \\
\DStrut$1$ & $x ± a$ & $\frac{1}{2} x^2 ± ax + C$ \\
\DStrut$a$ & $ax $   & $\frac{1}{2} ax^2 + C $\\
\DStrut$2x$ & $x^2$  & $\frac{1}{3} x^3 + C $ \\
\DStrut$nx^{n-1}$ & $x^n$ &$ \dfrac{1}{n+1} x^{n+1} + C $\\
\DStrut$-x^{-2} $ & $x^{-1}$ & $\log_\epsilon x + C$ \\
\DStrut$\dfrac{du}{dx} ± \dfrac{dv}{dx} ± \dfrac{dw}{dx}$
   & $u ± v ± w$     & $\int u\, dx ± \int v\, dx ± \int w\, dx$ \\
\DStrut$u\, \dfrac{dv}{dx} + v\, \dfrac{du}{dx}$
   & $uv$  & 无通用形式已知 \\
\DStrut$\dfrac{v\, \dfrac{du}{dx} - u\, \dfrac{dv}{dx}}{v^2}$
   & $\dfrac{u}{v}$ & 无通用形式已知 \\
\DStrut$\dfrac{du}{dx}$ & $u$ & $ux - \int x\, du + C$ \\
\hline
\end{tabular}
\vfil\newpage
\null\vfil
%
\begin{tabular}{|c|c|c|}
\hline
\multicolumn{1}{|c}{\DStrut$\dfrac{dy}{dx}$}
  & \multicolumn{1}{c}{$\longleftarrow\quad y\quad\longrightarrow$}
  & $\ds\int y\, dx$ \\
\hline
\ColumnHead{指数与对数。} \\
$\epsilon^x$ & $\epsilon^x$ & $\epsilon^x + C$ \\
$x^{-1}$     & $\log_\epsilon x$ & $ x(\log_\epsilon x - 1) + C$ \\
$0.4343 × x^{-1}$ & $\log_{10} x$ & $0.4343x (\log_\epsilon x - 1) + C$ \\
\DStrut$a^x \log_\epsilon a$ & $a^x$ & $\dfrac{a^x}{\log_\epsilon a} + C$ \\
\hline
%
\ColumnHead{三角函数。} \\
$\cos x$  & $\sin x$ & $-\cos x + C $ \\
$-\sin x$ & $\cos x$ & $\sin x + C $ \\
$\sec^2 x$& $\tan x$ & $-\log_\epsilon \cos x + C $ \\
\hline
%
\ColumnHead{圆函数(反函数)。} \\
$\dfrac{1}{\sqrt{(1-x^2)}}$ & $\arcsin x$ & $x · \arcsin x + \sqrt{1 - x^2} + C$ \\
\DStrut$-\dfrac{1}{\sqrt{(1-x^2)}}$ & $\arccos x$ & $x · \arccos x - \sqrt{1 - x^2} + C$ \\
$\dfrac{1}{1+x^2}$ & $\arctan x$ & $x · \arctan x - \frac{1}{2} \log_\epsilon (1 + x^2) + C$ \\
\hline
%\DPPageSep{265.png}{253}%
\ColumnHead{双曲函数。} \\
$\cosh x   $ & $\sinh x$ & $\cosh x + C$ \\
$\sinh x   $ & $\cosh x$ & $\sinh x + C$ \\
$\sech^2 x $ & $\tanh x$ & $\log_\epsilon \cosh x + C $ \\
\hline
\end{tabular}
\vfil\newpage
%
%[** TN: Manual coaxing to get this portion to fit page horizontally/vertically]
\small\Pagelabel{stdforms2}%
\makebox[0pt][c]{%
\begin{tabular}{|c|c|c|}
\hline
\multicolumn{1}{|c}{\DStrut$\dfrac{dy}{dx}$}
  & \multicolumn{1}{c}{$\longleftarrow\quad y\quad\longrightarrow$}
  & $\ds\int y\, dx$ \\
\hline
\ColumnHead{杂项。} \\
\DStrut$-\dfrac{1}{(x + a)^2}$ & $\dfrac{1}{x + a}$ & $ \log_\epsilon (x+a) + C $ \\
$-\dfrac{x}{(a^2 + x^2)^{\efrac{3}{2}}}$
  & $\dfrac{1}{\sqrt{a^2 + x^2}}$
  & $\log_\epsilon (x + \sqrt{a^2 + x^2}) + C $ \\
\DStrut$\mp \dfrac{b}{(a ± bx)^2}$
  & $\dfrac{1}{a ± bx}$
  & $± \dfrac{1}{b} \log_\epsilon (a ± bx) + C $ \\
$-\dfrac{3a^2x}{(a^2 + x^2)^{\efrac{5}{2}}}$
  & $\dfrac{a^2}{(a^2 + x^2)^{\efrac{3}{2}}}$
  & $\dfrac{x}{\sqrt{a^2 + x^2}} + C $ \\
\DStrut$ a · \cos ax$ & $\sin ax$ & $-\dfrac{1}{a} \cos ax + C $ \\
$-a · \sin ax$ & $\cos ax$ & $ \dfrac{1}{a} \sin ax + C $ \\
\DStrut$ a · \sec^2ax$& $\tan ax$ & $-\dfrac{1}{a} \log_\epsilon \cos ax + C $ \\
$ \sin 2x$ & $\sin^2 x$ & $\dfrac{x}{2} - \dfrac{\sin 2x}{4} + C $ \\
\DStrut$-\sin 2x$ & $\cos^2 x$ & $\dfrac{x}{2} + \dfrac{\sin 2x}{4} + C $ \\
$n · \sin^{n-1} x · \cos x$
  & $ \sin^n x$
  & \footnotesize$\DStrut\ds-\frac{\cos x}{n} \sin^{n-1} x
     + \frac{n-1}{n} \int \sin^{n-2} x\, dx + C$ \\
$-\dfrac{\cos x}{\sin^2 x}$
  & $\dfrac{1}{\sin x}$
  & $\log_\epsilon \tan \dfrac{x}{2} + C$ \\
\DStrut$-\dfrac{\sin 2x}{\sin^4 x}$
  & $\dfrac{1}{\sin^2 x}$
  & $ -\cotan x + C$ \\
$\dfrac{\sin^2 x - \cos^2 x}{\sin^2 x · \cos^2 x}$
  & $ \dfrac{1}{\sin x · \cos x}$
  & $ \log_\epsilon \tan x + C $ \\
\DStrut\parbox{0.2\linewidth}{\raggedright\scriptsize
  $n · \sin mx · \cos nx + m · \sin nx · \cos mx $}
  & $\sin mx · \sin nx$
  & $\frac{1}{2} \cos(m - n)x - \frac{1}{2} \cos(m + n)x + C$ \\
$ 2a·\sin 2ax$ & $\sin^2 ax$ & $\dfrac{x}{2} - \dfrac{\sin 2ax}{4a} + C $ \\
\DStrut$-2a·\sin 2ax$ & $\cos^2 ax$ & $\dfrac{x}{2} + \dfrac{\sin 2ax}{4a} + C $ \\
\hline
\end{tabular}}%
\end{center}
\DPPageSep{266.png}{254}%

\AltChapter[习题解答]{答案}

\begin{Answers}[3]{I}{(\Pageref{Ex:I}.)}{}
\Item{(1)} $\dfrac{dy}{dx} = 13x^{12}$。
\Item{(2)} $\dfrac{dy}{dx} = - \dfrac{3}{2} x^{-\efrac{5}{2}}$。
\Item{(3)} $\dfrac{dy}{dx} = 2ax^{(2a-1)}$。

\ResetCols{3}
\Item{(4)} $\dfrac{du}{dt} = 2.4t^{1.4}$。
\Item{(5)} $\dfrac{dz}{du} = \dfrac{1}{3} u^{-\efrac{2}{3}}$。
\Item{(6)} $\dfrac{dy}{dx} = -\dfrac{5}{3}x^{\DPtypo{-\efrac{8}{5}}{-\efrac{8}{3}}}$。

\ResetCols{2}
\Item{(7)} $\dfrac{du}{dx} = -\dfrac{8}{5}x^{-\efrac{13}{5}}$。
\Item{(8)} $\dfrac{dy}{dx} = 2ax^{a-1}$。

\ResetCols{2}
\Item{(9)} $\dfrac{dy}{dx} = \dfrac{3}{q} x^{\efrac{3-q}{q}}$。
\Item{(10)} $\dfrac{dy}{dx} = -\dfrac{m}{n} x^{-\efrac{m+n}{n}}$。
\end{Answers}

\begin{Answers}[3]{II}{(\Pageref{Ex:II}.)}{}

\Item{(1)} $\dfrac{dy}{dx} = 3ax^2$。
\Item{(2)} $\dfrac{dy}{dx} = 13 \times \frac{3}{2}x^{\efrac{1}{2}}$。
\Item{(3)} $\dfrac{dy}{dx} = 6x^{-\efrac{1}{2}}$。

\ResetCols{3}
\Item{(4)} $\dfrac{dy}{dx} = \dfrac{1}{2}c^{\efrac{1}{2}} x^{-\efrac{1}{2}}$。
\Item{(5)} $\dfrac{du}{dz} = \dfrac{an}{c} z^{n-1}$。
\Item{(6)} $\dfrac{dy}{dt} = 2.36t$。

\ResetCols{1}
\Item{(7)} $\dfrac{dl_t}{dt} = 0.000012 \times l_0$。

\Item{(8)} $\dfrac{dC}{dV} = abV^{b-1}$,分别为 $0.98$、$3.00$ 和 $7.47$~烛光每伏特。

\Item{(9)} $\begin{aligned}[t]
  \dfrac{dn}{dD} &= -\dfrac{1}{LD^2} \sqrt{\dfrac{gT}{\pi \sigma}}, &
  \dfrac{dn}{dL} &= -\dfrac{1}{DL^2} \sqrt{\dfrac{gT}{\pi \sigma}}, \\
%
  \dfrac{dn}{d \sigma}
  &= -\dfrac{1}{2DL} \sqrt{\dfrac{gT}{\pi \sigma^3}}, &
  \dfrac{dn}{dT} &=  \dfrac{1}{2DL} \sqrt{\dfrac{g}{\pi \sigma T}}。
\end{aligned}$
\DPPageSep{267.png}{255}%

\Item{(10)} $\dfrac{\text{当 $t$ 变化时 $P$ 的变化率}}
            {\text{当 $D$ 变化时 $P$ 的变化率}}
  = - \dfrac{D}{t}$。

\Item{(11)} $2\pi$,$2\pi r$,$\pi l$,$\frac{2}{3}\pi rh$,$8\pi r$,$4\pi r^2$。 \hfil
(12)~$\dfrac{dD}{dT} = \dfrac{0.000012l_t}{\pi}$。
\end{Answers}

\begin{Answers}{III}{(\Pageref{Ex:III}.)}{}

\Item{(1)} (\textit{a}) $1 + x + \dfrac{x^2}{2} + \dfrac{x^3}{6} + \dfrac{x^4}{24} + \ldots$ \qquad
    (\textit{b}) $2ax + b$。 \qquad (\textit{c}) $2x + 2a$。

    (\textit{d}) $3x^2 + 6ax + 3a^2$。

\ResetCols{2}
\Item{(2)} $\dfrac{dw}{dt} = a - bt$。
\Item{(3)} $\dfrac{dy}{dx} = 2x$。

\ResetCols{1}
\Item{(4)} $14110x^4 - 65404x^3 - 2244x^2 + 8192x + 1379$。

\ResetCols{2}
\Item{(5)} $\dfrac{dx}{dy} = 2y + 8$。
\Item{(6)} $185.9022654x^2 + 154.36334$。

\ResetCols{2}
\Item{(7)} $\dfrac{-5}{(3x + 2)^2}$。
\Item{(8)} $\dfrac{6x^4 + 6x^3 + 9x^2}{(1 + x + 2x^2)^2}$。

\ResetCols{2}
\Item{(9)} $\dfrac{ad - bc}{(cx + d)^2}$。
\Item{(10)} $\dfrac{anx^{-n-1} + bnx^{n-1} + 2nx^{-1}}{(x^{-n} + b)^2}$。

\ResetCols{1}
\Item{(11)} $b + 2ct$。

\Item{(12)} $R_0(a + 2bt)$,\quad $R_0 \left(a + \dfrac{b}{2\sqrt{t}}\right)$,\quad
  $-\dfrac{R_0(a + 2bt)}{(1 + at + bt^2)^2}$ \quad 或 \quad $\dfrac{R^2 (a + 2bt)}{R_0}$。

\Item{(13)} $1.4340(0.000014t - \DPtypo{0.000828}{0.001024})$,\quad $-0.00117$,\quad $-0.00107$,\quad $-0.00097$。

\Item{(14)} $\dfrac{dE}{dl} = b + \dfrac{k}{i}$,\quad $\dfrac{dE}{di} = -\dfrac{c + kl}{i^2}$。
\end{Answers}

\begin{Answers}[2]{IV}{(\Pageref{Ex:IV}.)}{}

\Item{(1)} $17 + 24x$;\quad $24$。
\Item{(2)} $\dfrac{x^2 + 2ax - a}{(x + a)^2}$;\quad $\dfrac{2a(a + 1)}{(x + a)^3}$。

\ResetCols{1}
\Item{(3)} $1 + x + \dfrac{x^2}{1 \times 2} + \dfrac{x^3}{1 \times 2 \times 3}$;\quad $1 + x + \dfrac{x^2}{1 \times 2}$。

% [** TN: Freely reformatting subitems]
\Item{(4)} (\textit{习题 III.}):
\begin{itemize}
\item[(1)] (\textit{a}) $\dfrac{d^2 y}{dx^2} = \dfrac{d^3 y}{dx^3} = 1 + x + \frac{1}{2}x^2 + \frac{1}{6} x^3 + \ldots$。 \\
  (\textit{b}) $2a$,$0$。\hfil
  (\textit{c}) $2$,$0$。\hfil
  (\textit{d}) $6x + 6a$,$6$。

\DPPageSep{268.png}{256}%
\item[(2)] $-b$,$0$。\hfil (3) $2$,$0$。\hfil

\item[(4)] $\begin{gathered}[t]
    56440x^3 - 196212x^2 - 4488x + 8192。 \\
    169320x^2 - 392424x - 4488。
    \end{gathered}$

\item[(5)] $2$,$0$。 \hfil (6) $371.80453x$,$371.80453$。 \hfil

\item[(7)] $\dfrac{30}{(3x + 2)^3}$,\quad $-\dfrac{270}{(3x + 2)^4}$。
\end{itemize}

\paragraph{\normalfont(\textit{例题}, \Pageref{examples3}):}
\begin{itemize}
\item[(1)] $\dfrac{6a}{b^2} x$,\quad $\dfrac{6a}{b^2}$.\hfil
(2) $\dfrac{3a \sqrt{b}} {2 \sqrt{x}} - \dfrac{6b \sqrt[3]{a}}{x^3}$,\quad
$\dfrac{18b \sqrt[3]{a}}{x^4} - \dfrac{3a \sqrt{b}}{4 \sqrt{x^3}}$\DPtypo{}{.}

\Item{(3)}
$\dfrac{2}{\sqrt[3]{\theta^8}} - \dfrac{1.056}{\sqrt[5]{\theta^{11}}}$,\quad
$\dfrac{2.3232}{\sqrt[5]{\theta^{16}}} - \dfrac{16}{3 \sqrt[3]{\theta^{11}}}$.

\Item{(4)} $\begin{gathered}[t]
  810t^4 - 648t^3 + 479.52t^2 - 139.968t + 26.64. \\
  3240t^3 - 1944t^2 + 959.04t - 139.968.
  \end{gathered}$

\Item{(5)}  $12x + 2$, $12$.\hfil
(6) $6x^2 - 9x$,\quad $12x - 9$.\hfil

\Item{(7)}
$\begin{aligned}[t]
&\dfrac{3}{4} \left(\dfrac{1}{\sqrt{\theta}} + \dfrac{1}{\sqrt{\theta^5}}\right)
+\dfrac{1}{4} \left(\dfrac{15}{\sqrt{\theta^7}} - \dfrac{1}{\sqrt{\theta^3}}\right). \\
&\dfrac{3}{8} \left(\dfrac{1}{\sqrt{\theta^5}} - \dfrac{1}{\sqrt{\theta^3}}\right)
-\dfrac{15}{8}\left(\dfrac{7}{\sqrt{\theta^9}} + \dfrac{1}{\sqrt{\theta^7}}\right).
\end{aligned}$
\end{itemize}
\end{Answers}

\begin{Answers}{V}{(\Pageref{Ex:V}.)}{}

\Item{(2)}  64; 147.2; 和 0.32 英尺每秒。

\ResetCols{2}
\Item{(3)}  $x = a - gt$; $\ddot{x} = -g$.

\Item{(4)}  45.1 英尺每秒。

\ResetCols{1}
\Item{(5)}  12.4 英尺每秒平方。\quad 是。

\Item{(6)} 角速度 ${} = 11.2$ 弧度每秒;角加速度 ${}= 9.6$ 弧度每秒平方。

\Item{(7)}  $v = 20.4t^2 - 10.8$.\quad $a = 40.8t$.\quad 172.8 英寸/秒,122.4 英寸/秒平方。

\Item{(8)}  $v = \dfrac{1}{30 \sqrt[3]{(t - 125)^2}}$,\quad $a = - \dfrac{1}{45 \sqrt[3]{(t - 125)^5}}$.

\Item{(9)}  $v = 0.8 - \dfrac{8t}{(4 + t^2)^2}$,\quad $a = \dfrac{24t^2 - 32}{(4 + t^2)^3}$,\quad 0.7926 和 0.00211。

\Item{(10)}  $n = 2$, $n = 11$.
\end{Answers}
\DPPageSep{269.png}{257}%

\begin{Answers}[3]{VI}{(\Pageref{Ex:VI}.)}{}

\Item{(1)}  $\dfrac{x}{\sqrt{ x^2 + 1}}$.
\Item{(2)}  $\dfrac{x}{\sqrt{ x^2 + a^2}}$.
\Item{(3)}  $- \dfrac{1}{2 \sqrt{(a + x)^3}}$.

\ResetCols{2}
\Item{(4)}  $\dfrac{ax}{\sqrt{(a - x^2)^3}}$.
\Item{(5)}  $\dfrac{2a^2 - x^2}{x^3 \sqrt{ x^2 - a^2}}$.

\ResetCols{2}
\Item{(6)}  $ \dfrac{\frac{3}{2} x^2 \left[ \frac{8}{9} x \left( x^3 + a \right) - \left( x^4 + a \right) \right]}{(x^4 + a)^{\efrac{2}{3}} (x^3 + a)^{\efrac{3}{2}}}$
\Item{(7)}  $\dfrac{2a \left(x - a \right)}{(x + a)^3}$.

\ResetCols{2}
\Item{(8)}  $\frac{5}{2} y^3$.
\Item{(9)}  $\dfrac{1}{(1 - \theta) \sqrt{1 - \theta^2}}$.
\end{Answers}

\begin{Answers}{VII}{(\Pageref{Ex:VII}.)}{}

\Item{(1)}  $\dfrac{dw}{dx} = \dfrac{3x^2 \left( 3 + 3x^3 \right)} {27 \left(\frac{1}{2} x^3 + \frac{1}{4} x^6 \right)^3}$.

\Item{(2)}  $\dfrac{dv}{dx} = - \dfrac{12x}{\sqrt{1 + \sqrt{2} + 3x^2} \left(\sqrt{3} + 4 \sqrt{1 + \sqrt{2} + 3x^2}\right)^2}$.

\Item{(3)}  $\dfrac{du}{dx} = - \dfrac{x^2 \left(\sqrt{3} + x^3 \right)} {\sqrt{ \left[ 1 + \left( 1 + \dfrac{x^3}{\sqrt{3}} \right) ^2 \right]^3}} $
\end{Answers}

\begin{Answers}{VIII}{(\Pageref{Ex:VIII}.)}{}

\Item{(2)}  1.44。

\Item{(4)}  $\dfrac{dy}{dx} = 3x^2 + 3$; 数值为:3,3$\frac{3}{4}$,6,和15。

\Item{(5)}  $ ± \sqrt{2}$。

\Item{(6)}  $ \dfrac{dy}{dx} = - \dfrac{4}{9} \dfrac{x}{y}$。斜率为零处 $x = 0$;斜率为 $\mp \dfrac{1}{3 \sqrt{2}}$ 处 $x = 1$。

\Item{(7)}  $m = 4$, $n = -3$。

\Item{(8)}  交点在 $x = 1$,$x = -3$。角度为 153°26',2°28'。

\Item{(9)}  交点在 $x = 3.57$,$y = 3.50$。角度为 16°16'。

\Item{(10)}  $x = \frac{1}{3}$,$y = 2 \frac{1}{3}$,$b = -\frac{5}{3}$。
\end{Answers}
\DPPageSep{270.png}{258}%

\begin{Answers}{IX}{(\Pageref{Ex:IX}.)}{}

\Item{(1)}  最小值:$x = 0$,$y = 0$;最大值:$x = -2$,$y = -4$。

\ResetCols{2}
\Item{(2)}  $x = a$。

\Item{(4)}  25$\sqrt{3}$ 平方英寸。

\ResetCols{1}
\Item{(5)}  $\dfrac{dy}{dx} = - \dfrac{10}{x^2} + \dfrac{10}{(8 - x)^2}$;$x = 4$;$y = 5$。

\Item{(6)}  最大值为 $x = -1$;最小值为 $x = 1$。

\Item{(7)}  连接四边的中间点。

\Item{(8)}  $r = \frac{2}{3} R$, $r = \dfrac{R}{2}$, 无最大值。

\Item{(9)}  $r = R \sqrt{\dfrac{2}{3}}$, $r = \dfrac{R}{\sqrt{2}}$, $r = 0.8506R$。

\Item{(10)}  以每秒 $\dfrac{8}{r}$ 平方英尺的速度。

\ResetCols{2}
\Item{(11)}  $r = \dfrac{R \sqrt{8}}{3}$。
\Item{(12)}  $n = \sqrt{\dfrac{NR}{r}}$。
\end{Answers}

\begin{Answers}{X}{(\Pageref{Ex:X}.)}{}

\Item{(1)}  最大值:$x = -2.19$, $y = 24.19$;最小值:$x = 1.52$, $y = -1.38$。

\Item{(2)}  $\dfrac{dy}{dx} = \dfrac{b}{a} - 2cx$;$\dfrac{d^2 y}{dx^2} = -2c$;$x = \dfrac{b}{2ac}$(\emph{最大值})。

\Item{(3)}  (\textit{a})一个最大值和两个最小值。  \\
(\textit{b})一个最大值。($x = 0$;其他点不真实。)

\ResetCols{2}
\Item{(4)}  最小值:$x = 1.71$, $y = 6.14$。

\Item{(5)}  最大值:$x = -.5$, $y = 4$。

\ResetCols{1}
\Item{(6)}  最大值:$x = 1.414$, $y = 1.7675$。  \\
最小值:$x = -1.414$, $y = 1.7675$。

\Item{(7)}  最大值:$x = -3.565$, $y = 2.12$。  \\
最小值:$x = +3.565$, $y = 7.88$。

\ResetCols{2}
\Item{(8)}  $0.4N$, $0.6N$。

\Item{(9)}  $x = \sqrt{\dfrac{a}{c}}$。

\ResetCols{1}
\Item{(10)}  速度 $8.66$ 海里每小时。所需时间 $115.47$~小时。 \\
最小成本 £$112$.~$12$\textit{s}。
\DPPageSep{271.png}{259}%

\Item{(11)}  最大值和最小值:$x = 7.5$, $y = ±5.414$。(参见例题
第~10,\Pageref{Example10}。)

\Item{(12)}  最小值:$x = \frac{1}{2}$, $y= 0.25$;最大值:$x = - \frac{1}{3}$, $y= 1.408$。
\end{Answers}

%[** TN: Text block-dependent page break]
\begin{Answers}[3]{XI}{(\Pageref{Ex:XI}.)}{\newpage}

\Item{(1)}  $\dfrac{2}{ x - 3} + \dfrac{1}{ x + 4}$。

\Item{(2)}  $\dfrac{1}{ x - 1} + \dfrac{2}{ x - 2}$。

\Item{(3)}  $\dfrac{2}{ x - 3} + \dfrac{1}{ x + 4}$。

\ResetCols{2}
\Item{(4)}  $\dfrac{5}{ x - 4} - \dfrac{4}{ x - 3}$。

\Item{(5)}  $\dfrac{19}{13(2x + 3)} - \dfrac{22}{13(3x - 2)}$。

\ResetCols{1}
\Item{(6)}  $\dfrac{2}{ x - 2} + \dfrac{4}{ x - 3} - \dfrac{5}{ x - 4}$。

\Item{(7)}  $\dfrac{1}{6(x - 1)} + \dfrac{11}{15(x + 2)} + \dfrac{1}{10(x - 3)}$。

\Item{(8)}  $\dfrac{7}{9(3x + 1)} + \dfrac{71}{63(3x - 2)} - \dfrac{5}{7(2x + 1)}$。

\ResetCols{2}
\Item{(9)}  $\dfrac{1}{3(x - 1)} + \dfrac{2x + 1}{3(x^2 + x + 1)}$。

\Item{(10)} $x + \dfrac{2}{3(x + 1)} + \dfrac{1 - 2x}{3(x^2 - x + 1)}$。

\ResetCols{2}
\Item{(11)} $\dfrac{3}{(x + 1)} + \dfrac{2x + 1}{x^2 + x + 1}$。

\Item{(12)} $\dfrac{1}{ x - 1} - \dfrac{1}{ x - 2} + \dfrac{2}{(x - 2)^2}$。

\ResetCols{1}
\Item{(13)} $\dfrac{1}{4(x - 1)} - \dfrac{1}{4(x + 1)} + \dfrac{1}{2(x + 1)^2}$。

\Item{(14)} $\dfrac{4}{9(x - 1)} - \dfrac{4}{9(x + 2)} - \dfrac{1}{3(x + 2)^2}$。

\Item{(15)} $\dfrac{1}{ x + 2} - \dfrac{x - 1}{ x^2 + x + 1} - \dfrac{1}{(x^2 + x + 1)^2}$。

\Item{(16)} $\dfrac{5}{ x + 4} -\dfrac{32}{(x + 4)^2} + \dfrac{36}{(x + 4)^3}$。

\Item{(17)} $\dfrac{7}{9(3x - 2)^2} + \dfrac{55}{9(3x - 2)^3} + \dfrac{73}{9(3x - 2)^4}$。

\Item{(18)} $\dfrac{1}{6(x - 2)} + \dfrac{1}{3(x - 2)^2} - \dfrac{x}{6(x^2 + 2x + 4)}$。
\end{Answers}
\DPPageSep{272.png}{260}%

\begin{Answers}[3]{XII}{(\Pageref{Ex:XII}.)}{}

\Item{(1)}  $ab(\epsilon^{ax} + \epsilon^{-ax})$。

\Item{(2)}  $2at + \dfrac{2}{t}$。

\Item{(3)}  $\log_\epsilon n$。

\ResetCols{3}
\Item{(5)}  $npv^{n-1}$。

\Item{(6)}  $\dfrac{n}{x}$。

\Item{(7)}  $\dfrac{3\epsilon^{- \frac{x}{x-1}}}{(x - 1)^2}$。

\ResetCols{2}
\Item{(8)}  $6x \epsilon^{-5x} - 5(3x^2 + 1)\epsilon^{-5x}$。

\Item{(9)}  $\dfrac{ax^{a-1}}{x^a + a}$。

\ResetCols{1}
\Item{(10)}  $\left(\dfrac{6x}{3x^2-1} + \dfrac{1}{2\left(\sqrt x + x\right)}\right) \left(3x^2-1\right)\left(\sqrt x + 1\right)$。

\Item{(11)}  $\dfrac{1 - \log_\epsilon \left(x + 3\right)}{\left(x + 3\right)^2}$。

\ResetCols{2}
\Item{(12)}  $a^x\left(ax^{a-1} + x^a \log_\epsilon a\right)$。

\Item{(14)}  最小值:$y = 0.7$ 当 $x = 0.694$。

\ResetCols{2}
\Item{(15)}  $\dfrac{1 + x}{x}$。

\Item{(16)}  $\dfrac{3}{x} (\log_\epsilon ax)^2$。
\end{Answers}

\begin{Answers}{XIII}{(\Pageref{Ex:XIII}.)}{}

\Item{(1)}  设 $\dfrac{t}{T} = x$(因此 $t = 8x$),并使用第\Pageref[page]{littletable}页的表格。

\Item{(2)}  $T = 34.627$;$159.46$ 分钟。

\Item{(3)}  取 $2t = x$;并使用第\Pageref[page]{littletable}页的表格。

\Item{(5)}  (\textit{a}) $x^x \left(1 + \log_\epsilon x\right)$;\quad
(\textit{b}) $2x(\epsilon^x)^x$;\quad
(\textit{c}) $\epsilon^{x^x} × x^x \left(1 + \log_\epsilon x\right)$。

\ResetCols{2}
\Item{(6)}  $0.14$ 秒。

\Item{(7)}  (\textit{a}) $1.642$;\quad (\textit{b}) $15.58$。

\ResetCols{1}
\Item{(8)}  $\mu = 0.00037$,$31^m \frac{1}{4}$。 %[** 时间单位]

\Item{(9)}  $i$ 是 $i_0$ 的 $63.4\%$,$220$ 公里。

\Item{(10)}  $0.133$,$0.145$,$0.155$,平均 $0.144$;$-10.2\%$,$-0.9\%$,$+77.2\%$。

\ResetCols{2}
\Item{(11)}  最小值在 $x = \dfrac{1}{\epsilon}$。

\Item{(12)}  最大值在 $x = \epsilon$。

\ResetCols{1}
\Item{(13)}  最小值在 $x = \log_\epsilon a$。
\end{Answers}
\DPPageSep{273.png}{261}%

\begin{Answers}{XIV}{(\Pageref{Ex:XIV}.)}{}

\Item{(1)} (i) $\dfrac{dy}{d\theta} = A \cos \left( \theta - \dfrac{\pi}{2} \right)$;

(ii) $\dfrac{dy}{d\theta} = 2\sin\theta \cos\theta = \sin2\theta$ 且 $\dfrac{dy}{d\theta} = 2\cos2\theta$;

(iii) $\dfrac{dy}{d\theta} = 3\sin^2 \theta \cos\theta$ 且 $\dfrac{dy}{d\theta} = 3\cos3\theta$。

\ResetCols{2}
\Item{(2)}  $\theta = 45°$ 或 $\dfrac{\pi}{4}$ 弧度。

\Item{(3)}  $\dfrac{dy}{dt} = -n \sin 2\pi nt$。

\ResetCols{2}
\Item{(4)}  $a^x \log_\epsilon a \cos a^x$。

\Item{(5)}  $\dfrac{\cos x}{\sin x} = \cotan x$

\ResetCols{1}
\Item{(6)}  $18.2 \cos \left(x + 26° \right)$。

\Item{(7)} {\loosen 斜率为 $\dfrac{dy}{d\theta} = 100\cos\left(\theta - 15° \right)$,当 $(\theta -15°) = 0$,即 $\theta = 15°$ 时达到最大值,此时斜率为 $100$。当 $\theta = 75°$ 时,斜率为 $100\cos(75°  - 15°) = 100\cos 60° = 100 × \frac{1}{2} = 50$。}

\Item{(8)} $\begin{aligned}[t]
  \cos\theta \sin2\theta + 2\cos2\theta \sin\theta
  &= 2\sin\theta\left(\cos^2 \theta + \cos2\theta\right) \\
  &= 2\sin\theta\left(3\cos^2 \theta - 1\right)。
  \end{aligned}$

\Item{(9)} $amn\theta^{n-1} \tan^{m-1}\left(\theta^n\right)\sec^2 \theta^n$。

\Item{(10)} $\epsilon^x \left(\sin^2 x + \sin2x\right)$;\quad $\epsilon^x \left(\sin^2 x + 2\sin2x + 2\cos2x\right)$。

\Item{(11)} $\left(i\right) \dfrac{dy}{dx} = \dfrac{ab}{\left(x + b\right)^2}$;\quad
(ii)~$\dfrac{a}{b} \epsilon^{-\efrac{x}{b}}$;\quad
(iii)~$\dfrac{1}{90}° × \dfrac{ab}{\left(b^2 + x^2\right)}$。

\Item{(12)} (i) $\dfrac{dy}{dx} = \sec x \tan x$;

(ii) $\dfrac{dy}{dx} = - \dfrac{1}{\sqrt{ 1 - x^2}}$;

(iii) $\dfrac{dy}{dx} = \dfrac{1}{ 1 + x^2}$;

(iv) $\dfrac{dy}{dx} = \dfrac{1}{x \sqrt{ x^2 - 1}}$;

(v) $\dfrac{dy}{dx} = \dfrac{\sqrt{ 3\sec x} \left(3\sec^2 x - 1\right)}{2}$。

\Item{(13)} $\dfrac{dy}{d\theta} = 4.6\left(2\theta + 3\right)^{1.3} \cos\left(2\theta + 3\right)^{2.3}$。
\DPPageSep{274.png}{262}%

\Item{(14)}  $\dfrac{dy}{d\theta} = 3\theta^2 + 3\cos \left( \theta + 3 \right) - \log_\epsilon 3 \left( \cos\theta × 3^{\sin\theta} + 3\theta \right)$。

\Item{(15)}  $\theta = \cot\theta;\theta = ±0.86$;对于 $+\theta$ 为最大值,对于 $-\theta$ 为最小值。
\end{Answers}

\begin{Answers}{XV}{(\Pageref{Ex:XV}.)}{}

\Item{(1)}  $x^3 - 6x^2 y - 2y^2;\quad \frac{1}{3} - 2x^3 - 4xy$。

\Item{(2)}  $2xyz + y^2 z + z^2 y + 2xy^2 z^2$;\\
     $2xyz + x^2 z + xz^2 + 2x^2 yz^2$;\\
     $2xyz + x^2 y + xy^2 + 2x^2 y^2 z$。

\Item{(3)}  $\dfrac{1}{r} \{ \left(x - a\right) + \left( y - b \right) + \left( z - c \right) \} = \dfrac{ \left( x + y + z \right) - \left( a + b + c \right) }{r}$;$\dfrac{3}{r}$。

\Item{(4)}  $dy = vu^{v-1}\, du + u^v \log_\epsilon u\, dv$。

\Item{(5)}  $dy = 3\sin v u^2\, du + u^3 \cos v\, dv$,\\
     $dy = u \sin x^{u-1} \cos x\, dx + (\sin x)^u \log_\epsilon \sin x du$,\\
     $dy = \dfrac{1}{v}\, \dfrac{1}{u}\, du - \log_\epsilon u \dfrac{1}{v^2}\, dv$。

\Item{(7)}  最小值在 $x = y = -\frac{1}{2}$。

\Item{(8)}  (\textit{a}) 长度 $2$ 英尺,宽度和深度 $1$ 英尺,体积 $2$ 立方英尺。

 (\textit{b}) 半径 $=\dfrac{2}{\pi}$ 英尺 $= 7.46$ 英寸,长度 $2$ 英尺,体积 $2.54$。

\Item{(9)}  三部分相等;乘积为最大值。

\Item{(10)}  最小值在 $x = y = 1$。

\Item{(11)}  最小值:$x = \frac{1}{2}$ 且 $y = 2$。

\Item{(12)} 顶角 $= 90°$;等边长度 $= \sqrt[3]{2V}$。
\end{Answers}

%[** TN: Text block-dependent page break]
\begin{Answers}[3]{XVI}{(\Pageref{Ex:XVI}.)}{\newpage}

\Item{(1)}  $1\frac{1}{3}$。

\Item{(2)}  $0.6344$。

\Item{(3)}  $0.2624$。

\ResetCols{1}
\Item{(4)}  (\textit{a}) $y = \frac{1}{8} x^2 + C$;\quad
     (\textit{b}) $y = \sin x + C$。

\Item{(5)}  $y = x^2 + 3x + C$。
\end{Answers}
\DPPageSep{275.png}{263}%

\begin{Answers}[3]{XVII}{(\Pageref{Ex:XVII}.)}{}

\Item{(1)} $\dfrac{4\sqrt{a} x^{\efrac{3}{2}}}{3} + C$。

\Item{(2)} $-\dfrac{1}{x^3} + C$。

\Item{(3)} $\dfrac{x^4}{4a} + C$。

\ResetCols{2}
\Item{(4)} $\tfrac{1}{3} x^3 + ax + C$。

\Item{(5)} $-2x^{-\efrac{5}{2}} + C$。

\ResetCols{2}
\Item{(6)} $x^4 + x^3 + x^2 + x + C$。

\Item{(7)} $\dfrac{ax^2}{4} + \dfrac{bx^3}{9} + \dfrac{cx^4}{16} + C$。

\ResetCols{1}
\Item{(8)} {\loosen $\dfrac{x^2 + a}{x + a} = x - a + \dfrac{a^2 + a}{x + a}$ 通过除法得到。因此答案是 $\dfrac{x^2}{2} - ax + (a^2 + a)\log_\epsilon (x + a) + C$。}
  (参见第 \pageref{intex1} 和~\pageref{intex2} 页。)

\ResetCols{2}
\Item{(9)} $\dfrac{x^4}{4} + 3x^3 + \dfrac{27}{2} x^2 + 27x + C$。

\Item{(10)} $\dfrac{x^3}{3} + \dfrac{2 - a}{2} x^2 - 2ax + C$。

\ResetCols{2}
\Item{(11)} $a^2(2x^{\efrac{3}{2}} + \tfrac{9}{4} x^{\efrac{4}{3}}) + C$。

\Item{(12)} $-\tfrac{1}{3} \cos\theta - \tfrac{1}{6} \theta + C$。

\ResetCols{2}
\Item{(13)} $\dfrac{\theta}{2} + \dfrac{\sin 2a\theta}{4a} + C$。

\Item{(14)} $\dfrac{\theta}{2} - \dfrac{\sin 2\theta}{4} + C$。

\ResetCols{2}
\Item{(15)} $\dfrac{\theta}{2} - \dfrac{\sin 2a\theta}{4a} + C$。

\Item{(16)} $\tfrac{1}{3} \epsilon^{3x}$。 % [F1: +C?]

\ResetCols{2}
\Item{(17)} $\log(1 + x) + C$。

\Item{(18)} $-\log_\epsilon (1 - x) + C$。
\end{Answers}

\begin{Answers}{XVIII}{(\Pageref{Ex:XVIII}.)}{}

\Item{(1)} 面积 $= 60$;平均纵坐标 $= 10$。

\Item{(2)} 面积 $= \frac{2}{3}$ 的 $a × 2a \sqrt{a}$。

\Item{(3)} 面积 $= 2$;平均纵坐标 $= \dfrac{2}{\pi} = 0.637$。

\Item{(4)} 面积 $= 1.57$;平均纵坐标 $= 0.5$。

\ResetCols{2}
\Item{(5)} $0.572$,$0.0476$。

\Item{(6)} 体积 $= \pi r^2 \dfrac{h}{3}$。

\ResetCols{2}
\Item{(7)} $1.25$。

\Item{(8)} $79.4$。

\ResetCols{1}
\Item{(9)} 体积 $= 4.9348$;表面积 $= 12.57$(从 $0$ 到~$\pi$)。

\Item{(10)} $a\log_\epsilon a$,\quad $\dfrac{a}{a - 1} \log_\epsilon a$。

\Item{(12)} 算术平均数 $= 9.5$;二次平均数 $= 10.85$。
\DPPageSep{276.png}{264}%

\Item{(13)} 二次平均数 $= \dfrac{1}{\sqrt{2}} \sqrt{A_1^2 + A_3^2}$;算术平均数 $= 0$。

第一个涉及一个较为困难的积分,可以这样表述:根据定义,二次平均数为
\[
\sqrt{\dfrac{1}{2\pi} \int_0^{2\pi} (A_1 \sin x + A_3 \sin 3x)^2\, dx}. %[** TN: Moved period out of radicand]
\]
现在,由
\[
\int (A_1^2 \sin^2 x + 2A_1 A_3 \sin x \sin 3x + A_3^2 \sin^2 3x)\, dx
\]
表示的积分更容易获得,如果我们将 $\sin^2 x$ 写成
\[
\dfrac{1 - \cos 2x}{2}。
\]
对于 $2\sin x \sin 3x$,我们写成 $\cos 2x - \cos 4x$;而对于 $\sin^2 3x$,
\[
\dfrac{1 - \cos 6x}{2}。
\]

进行这些替换并积分,我们得到(参见第 \Pageref{cosax} 页)
\[
\dfrac{A_1^2}{2} \left( x - \dfrac{\sin 2x}{2} \right)
 + A_1 A_3 \left( \dfrac{\sin 2x}{2} - \dfrac{\sin 4x}{4} \right)
 + \dfrac{A_3^2}{2} \left( x - \dfrac{\sin 6x}{6} \right)。
\]

在下限处,将 $0$ 代入 $x$ 使得所有项都消失,而在上限处,将 $2\pi$ 代入 $x$ 得到 $A_1^2 \pi + A_3^2 \pi$。因此答案得出。

\Item{(14)} 面积为 $62.6$ 平方单位。平均纵坐标为 $10.42$。

\Item{(16)} $436.3$。(此立体形状为梨形。)
\end{Answers}

\begin{Answers}[2]{XIX}{(\Pageref{Ex:XIX}.)}{}

\Item{(1)} $\dfrac{x\sqrt{a^2 - x^2}}{2} + \dfrac{a^2}{2} \sin^{-1} \dfrac{x}{a} + C$。

\Item{(2)} $\dfrac{x^2}{2}(\log_\epsilon x - \tfrac{1}{2}) + C$。

\ResetCols{2}
\Item{(3)} $\dfrac{x^{a+1}}{a + 1} \left(\log_\epsilon x - \dfrac{1}{a + 1}\right) + C$。

\Item{(4)} $\sin \DPtypo{e}{\epsilon}^x + C$。

\ResetCols{2}
\Item{(5)} $\sin(\log_\epsilon x) + C$。

\Item{(6)} $\DPtypo{e}{\epsilon}^x (x^2 - 2x + 2) + C$。

\ResetCols{2}
\Item{(7)} $\dfrac{1}{a + 1} (\log_\epsilon x)^{a+1} + C$。
\DPPageSep{277.png}{265}%

\Item{(8)} $\log_\epsilon(\log_\epsilon x) + C$。

\ResetCols{1}
\Item{(9)} $2\log_\epsilon(x - 1) + 3\log_\epsilon(x + 2) + C$。

\Item{(10)} $\frac{1}{2} \log_\epsilon(x - 1) + \frac{1}{5} \log_\epsilon(x - 2) + \frac{3}{10} \log_\epsilon(x + 3) + C$。

\ResetCols{2}
\Item{(11)} $\dfrac{b}{2a} \log_\epsilon \dfrac{x - a}{x + a} + C$。

\Item{(12)} $\log_\epsilon \dfrac{x^2 - 1}{x^2 + 1} + C$。

\ResetCols{1}
\Item{(13)} $\frac{1}{4} \log_\epsilon \dfrac{1 + x}{1 - x} + \frac{1}{2} \arctan x + C$。

\Item{(14)} $\dfrac{1}{\sqrt{a}} \log_\epsilon \dfrac{\sqrt{a} - \sqrt{a - bx^2}}{x\sqrt{a}}$。\quad (令 $\dfrac{1}{x} = v$;然后,在结果中,令 $\sqrt{v^2 - \dfrac{b}{a}} = v - u$。) %[** TN: Large () in the original]

你最好现在对答案求导,并反推回给定的表达式以进行检查。
\end{Answers}

每一位认真的学生都应被鼓励在每个阶段为自己制造更多的例子,以测试自己的能力。在积分时,他总是可以通过对答案求导来验证其正确性,看看是否能回到他开始时的表达式。

有许多书籍提供练习题。这里只需提及两本:R.~G.~Blaine 的《微积分及其应用》和 F.~M.~Saxelby 的《实用数学教程》。

\vfill
\begin{center}
  \tiny 格拉斯哥:由罗伯特·麦克莱霍斯及有限公司在大学出版社印刷
\end{center}
\DPPageSep{278.png}{266}%
% [Blank Page]
\DPPageSep{279.png}{267}%

% [** TN: Macro prints the text below]
%A SELECTION OF
%Mathematical Works
\Catalogue

\Book{\Title{微积分导论} 基于图形方法。作者:\Author{G.~A.~Gibson} 教授,文学硕士,法学博士。3先令6便士。}

\Book{\Title{微积分基础教程} 附几何、力学和物理学的例证。作者:\Author{G.~A.~Gibson} 教授,文学硕士,法学博士。7先令6便士。}

\Book{\Title{初学者微分学} 作者:\Author{J.~Edwards},文学硕士。4先令6便士。}

\Book{\Title{初学者积分学} 附微分方程研究导论。作者:\Author{Joseph Edwards},文学硕士。4先令6便士。}

\Book{\Title{微积分轻松学} 作为那些通常被称为微分学和积分学的美丽计算方法的非常简单的入门。作者:\Author{F.~R.~S}。2先令净价。新版,附许多例题。}

\Book{\Title{微积分初步教程} 作者:\Author{W.~F.~Osgood} 教授,哲学博士。8先令6便士净价。}

\Book{\Title{工程师等实用积分学} 作者:\Author{A.~S.~Percival},文学硕士。2先令6便士净价。}

\Book{\Title{微分学} 附应用及众多例题。初等教程,作者:\Author{Joseph Edwards},文学硕士。14先令。}

\Book{\Title{技术学校和学院的微分与积分学} 作者:\Author{P.~A.~Lambert},文学硕士。7先令6便士。}

\Book{\Title{微分与积分学及其应用} 作者:\Author{A.~G.~Greenhill} 爵士,皇家学会会员。10先令6便士。}
%\vfill
%\begin{center}LONDON: MACMILLAN AND CO., LTD.\end{center}
\DPPageSep{280.png}{268}%

\Book{\Title{积分学及其应用专论} 作者:\Author{I.~Todhunter},皇家学会会员。10先令6便士。答案。作者:\Author{H.~St.~J.~Hunter},文学硕士。10先令6便士。}

\Book{\Title{微分学与积分学基础专论} 附众多例题。作者:\Author{I.~Todhunter},皇家学会会员。10先令6便士。答案。作者:\Author{H.~St.~J.~Hunter},文学硕士。10先令6便士。}

\Book{\Title{常微分方程} 一本基础教材。作者:\Author{James Morris Page},哲学博士。6先令6便士。}

\Book{\Title{方程现代理论导论} 作者:\Author{F.~Cajori} 教授,哲学博士。7先令6便士净价。}

\Book{\Title{微分方程论} 作者:\Author{安德鲁·拉塞尔·福赛斯},科学博士,法学博士。第四版。14先令。净价。}

\Book{\Title{微分方程简明教程} 作者:\Author{唐纳德·F·坎贝尔} 教授,哲学博士。4先令。净价。}

\Book{\Title{四元数手册} 作者:\Author{C·J·乔利},文学硕士,科学博士,皇家学会会员。10先令。净价。}

\Book{\Title{行列式理论的历史发展顺序} 卷一。第一部分。一般行列式,至1841年。第二部分。特殊行列式,至1841年。17先令。净价。卷二。1841年至1860年时期。17先令。净价。作者:\Author{T·缪尔},文学硕士,法学博士,皇家学会会员。}

\Book{\Title{无穷级数理论导论} 作者:\Author{T·J·I'a 布罗姆维奇},\DPnote{** [原文如此;托马斯·约翰·艾森·布罗姆维奇]} 文学硕士,皇家学会会员。15先令。净价。}

\Book{\Title{傅里叶级数与积分理论导论,以及热传导的数学理论} 作者:\Author{H·S·卡斯劳} 教授,文学硕士,科学博士,皇家苏格兰学会会员。14先令。净价。}

\vfill

\begin{center}
  伦敦:麦克米伦有限公司。
\end{center}
\DPPageSep{281.png}{269}%
%[空白页]


\end{document}